%--------------------------------------------------------
% *******begin section***************
\section{\DbgSecSt{\StPart}{Вспомогательные функции}}
%--------------------------------------------------------

% *******begin subsection***************
\subsection{\DbgSecSt{\StPart}{detectEdgeRise}}
\index{Программный интерфейс ПЛК!Вспомогательные функции!Функция detectEdgeRise}
\label{sec:detectEdgeRise}

\begin{pHeader}
    Синтаксис:      & \RightHandText{int detectEdgeRise(int \&detector, int input);}\\
    Аргумент(ы):    & \RightHandText{int \&detector ~-- предыдущее входное значение} \\    
    & \RightHandText{int input ~--  входное значение} \\ 
    Файл объявления:             & \RightHandText{include/func/misc.h} \\       
\end{pHeader}

Функция служит для детектирования изменения с 0 на 1 (детектирования фронта) входной величины. \killoverfullbefore

Функция возвращает 0, если входное значение не изменилось и осталось равным 0, и 1, если входное значение стало отличным от 0. \killoverfullbefore

Является системной.
% *******end subsection*****************
%-------------------------------------------------------------------
% *******begin subsection***************
\subsection{\DbgSecSt{\StPart}{detectEdgeFall}}
\index{Программный интерфейс ПЛК!Вспомогательные функции!Функция detectEdgeFall}
\label{sec:detectEdgeFall}

\begin{pHeader}
    Синтаксис:      & \RightHandText{int detectEdgeFall(int \&detector, int input);}\\
    Аргумент(ы):    & \RightHandText{int \&detector ~-- предыдущее входное значение} \\    
    & \RightHandText{int input ~--  входное значение} \\ 
    Файл объявления:             & \RightHandText{include/func/misc.h} \\       
\end{pHeader}

Функция служит для детектирования изменения с 1 на 0 (детектирования спада) входной величины. \killoverfullbefore

Функция возвращает 0, если входное значение не изменилось и осталось равным 1, и 1, если входное значение стало равным 0. \killoverfullbefore

Является системной.
% *******end subsection*****************
%-------------------------------------------------------------------
% *******begin subsection***************
\subsection{\DbgSecSt{\StPart}{initPulsedTimer}}
\index{Программный интерфейс ПЛК!Вспомогательные функции!Функция initPulsedTimer}
\label{sec:initPulsedTimer}

\begin{pHeader}
    Синтаксис:      & \RightHandText{void initPulsedTimer();}\\
    Аргумент(ы):    & \RightHandText{Нет} \\  
%    Возвращаемое значение:       & \RightHandText{Целое знаковое число} \\ 
    Файл объявления:             & \RightHandText{include/func/misc.h} \\       
\end{pHeader}

Функция инициализации периодического (импульсного) таймера. \killoverfullbefore
%который срабатывает (возвращает 1) через заданный интервал. 

Является системной.
% *******end subsection*****************
%-------------------------------------------------------------------
% *******begin subsection***************
\subsection{\DbgSecSt{\StPart}{timerSc}}
\index{Программный интерфейс ПЛК!Вспомогательные функции!Функция timerSc}
\label{sec:timerSc}

\begin{pHeader}
    Синтаксис:      & \RightHandText{int timerSc(int period);}\\
    Аргумент(ы):    & \RightHandText{int period ~-- период таймера} \\  
%    Возвращаемое значение:       & \RightHandText{Целое знаковое число} \\ 
    Файл объявления:             & \RightHandText{include/func/misc.h} \\
\end{pHeader}

Функция периодического (импульсного) таймера ~-- таймера, выходное значение которого периодически переключается с 0 на 1 и обратно через интервал, равный половине периода таймера. Период таймера задаётся в периодах сервоцикла (1 период сервоцикла равен 400 мс). Так, например, интервал 1 c соответствует значению периода таймера равному 2500. \killoverfullbefore
%сброса таймера (установки 0)

Функция возвращает 1, если с момента переключения таймера с 1 на 0 истёк интервал, больший или равный половине периода, и 0 в противном случае. \killoverfullbefore 

Является системной.
% *******end subsection*****************
%-------------------------------------------------------------------
% *******end section*****************

