\etocsettocdepth.toc {section}

\chapterimage{chapter_head_0} 
\chapter{\DbgSecSt{\StPart}{Особенности реализации программ ПЛК}}
\label{sec:Elements}
\index{Особенности реализации программ ПЛК|(}

%--------------------------------------------------------
% *******begin section***************
\section{\DbgSecSt{\StPart}{Встроенный логический контроллер}}
\index{Особенности реализации программ ПЛК!Встроенный логический контроллер}

Системы \mbox{ЧПУ} серии \textbf{IntNC PRO} имеют встроенный механизм выполнения логических программ управления ~-- программно реализованный встроенный логический контроллер.\killoverfullbefore

Интегрированный в системное программное обеспечение (рис. ~\ref{fig:PLC}) логический контроллер гарантирует:
\begin{itemize}
\item одно адресное пространство для выполнения системных задач и программ логического управления;
\item синхронизацию между различными задачами УЧПУ;
\item выполнение до 4-х программ ПЛК в режиме реального времени;
\item выполнение до 32-х программ ПЛК в фоновом режиме. \killoverfullbefore 
\end{itemize}

%\centering{\includegraphics[scale=0.7]{./Pictures/eps/buttons/1.eps}}

\DrawPictEpsFromSvg[0.7\textwidth]{./Pictures/svg/PLC}{Функциональная схема системного программного обеспечения}{PLC}
% *******end section*****************
%--------------------------------------------------------
\begin{comment}
% *******begin section***************
\section{\DbgSecSt{\StPart}{Программы ПЛК реального времени и фонового режима}}
\index{Программы ПЛК реального времени и фонового режима}

Константа ~-- число, символ или строка символов. Константы используются в программе для задания постоянных величин. Различают четыре типа констант: целые, с плавающей точкой, символьные константы и cтроковые литералы.\BL

\index{Элементы языка!Константы!Целые константы}
\mylbl{Целые константы}{IntegerConstant} \BL

Строковые литералы имеют тип массива char, то есть строка ~-- массив элементов типа char. Число элементов массива равно числу символов в строке плюс один для заканчивающего пустого символа. \killoverfullbefore
% *******end section*****************
\end{comment}
%--------------------------------------------------------

% *******begin section***************
\section{\DbgSecSt{\StPart}{Язык программ ПЛК}}
\index{Особенности реализации программ ПЛК!Язык программ ПЛК}

Для создания программ ПЛК используется процедурный язык программирования IntLang, разработанный на основе стандарта ANSI C.

Язык программирования IntLang имеет следующие особенности:
\begin{itemize}
\item простую языковую базу;
\item минимальное число ключевых слов;
\item систему типов;
\item области действия имён;
\item определяемые пользователем собирательные типы данных ~-- структуры и объединения;
\item передачу параметров в функцию по значению;
\item препроцессор для определения макросов и включения файлов с исходным кодом;
\item математические функции и функции работы с массивами. \killoverfullbefore 
\end{itemize}
% *******end section*****************
%--------------------------------------------------------

% *******begin section***************
\section{\DbgSecSt{\StPart}{Организация программ ПЛК}}
\index{Особенности реализации программ ПЛК!Организация программ ПЛК}

Программы ПЛК реализуются в виде текстовых файлов с расширением \texttt{cfg} и входят 
в состав конфигурационных файлов УЧПУ для станка. \killoverfullbefore 

Программы ПЛК размещаются в директории пользовательских файлов \texttt{<<source/platform/имя\_проекта>>} и их имена включаются в файл \texttt{<<source/platform/имя\_проекта/target.cfg>>}. \killoverfullbefore 

\DrawPictEpsFromSvg[0.6\textwidth]{./Pictures/svg/Struct_1}{Организация конфигурационных файлов проекта <<stanok>>}{Struct_1}
% *******end section*****************
%--------------------------------------------------------

\index{Особенности реализации программ ПЛК|)}

\clearpage