%--------------------------------------------------------
% *******begin section***************
\section{\DbgSecSt{\StPart}{Управление осями}}
%--------------------------------------------------------
\subsection{\DbgSecSt{\StPart}{Типы данных}}

% *******begin subsection***************
\subsubsection{\DbgSecSt{\StPart}{AxisStates}}
\index{Программный интерфейс ПЛК!Управление осями!Перечисление AxisStates}
\label{sec:AxisStates}

\begin{fHeader}
    Тип данных:            & \RightHandText{Перечисление AxisStates}\\
    Файл объявления:             & \RightHandText{include/func/axis.h} \\
\end{fHeader}

Перечисление определяет идентификаторы состояний оси.

\begin{MyTableTwoColAllCntr}{Перечисление AxisStates}{tbl:AxisStates}{|m{0.38\linewidth}|m{0.57\linewidth}|}{Идентификатор}{Описание}
\hline axisInactive &  Ось выключена  \\
\hline axisActive &  Ось находится в слежении \\
\hline axisJoggingPlus & Толчковое движение в положительном направлении \\
\hline axisJoggingMinus & Толчковое движение в отрицательном направлении \\
\hline axisJoggingTo & Толчковое движение в заданное положение или на заданное расстояние \\
\hline axisStopping & Останов оси \\
\hline axisHomeWaitHW & Ожидание применения аппаратных настроек выезда в нулевую точку \\
\hline axisHoming & Выезд в нулевую точку \\
\hline axisIndexWaitHW & Ожидание применения аппаратных настроек поиска индексной метки \\
\hline axisIndexing & Поиск индексной метки \\
\hline axisAborting & Аварийное торможение \\
\hline axisWaitActivate & Ожидание включения \\
\hline axisWaitDeactivate & Ожидание выключения \\
\hline axisWaitPhaseRef  & Ожидание фазировки \\
\end{MyTableTwoColAllCntr}
% *******end subsection***************
%--------------------------------------------------------
% *******begin subsection***************
\subsubsection{\DbgSecSt{\StPart}{AxisCommands}}
\index{Программный интерфейс ПЛК!Управление осями!Перечисление AxisCommands}
\label{sec:AxisCommands}

\begin{fHeader}
    Тип данных:            & \RightHandText{Перечисление AxisCommands}\\
    Файл объявления:             & \RightHandText{include/func/axis.h} \\
\end{fHeader}

Перечисление определяет идентификаторы команд управления осями. 

\begin{MyTableTwoColAllCntr}{Перечисление AxisCommands}{tbl:AxisCommands}{|m{0.38\linewidth}|m{0.57\linewidth}|}{Идентификатор}{Описание}
\hline axisCmdIdle &  Нет команды  \\
\hline axisCmdKill  &  Выключить ось \\
\hline axisCmdActivate  &  Включить ось в слежение \\
\hline axisCmdDeactivate  &  Выключить ось  \\
\hline axisCmdJogPlus  &  Выполнить толчковое движение в положительном направлении \\
\hline axisCmdJogMinus  & Выполнить толчковое движение в отрицательном направлении \\
\hline axisCmdJogStop & Выполнить останов \\
\hline axisCmdJogRet & Вернуться в сохраненную позицию \\
\hline axisCmdInc  & Выполнить толчковое движение на заданное расстояние \\
\hline axisCmdAbs & Выполнить толчковое движение в заданное положение \\
\hline axisCmdHome & Выполнить движение в нулевую точку \\
\hline axisCmdIndex & Выполнить движение до индексной метки \\
\hline axisCmdPhaseRef & Выполнить фазировку \\
\hline axisCmdAbort & Выполнить аварийное выключение \\
\end{MyTableTwoColAllCntr}
% *******end subsection***************

\clearpage

%--------------------------------------------------------
% *******begin subsection***************
\subsubsection{\DbgSecSt{\StPart}{AxisAbortMode}}
\index{Программный интерфейс ПЛК!Управление осями!Перечисление AxisAbortMode}
\label{sec:AxisAbortMode}

\begin{fHeader}
    Тип данных:            & \RightHandText{Перечисление AxisAbortMode}\\
    Файл объявления:             & \RightHandText{include/func/axis.h} \\
\end{fHeader}

Перечисление определяет идентификаторы действий по команде аварийного останова (ABORT).

\begin{MyTableTwoColAllCntr}{Перечисление AxisAbortMode}{tbl:AxisAbortMode}{|m{0.38\linewidth}|m{0.57\linewidth}|}{Идентификатор}{Описание}
\hline axisAbortStop & Останов категории 2 (аварийно затормозить и оставаться в слежении) \\
\hline axisAbortAndKill  & Останов категории 1 (аварийно затормозить и выключить) \\
\hline axisAbortKill & Останов категории 0 (выключить) \\
\end{MyTableTwoColAllCntr}
% *******end subsection***************
%--------------------------------------------------------
% *******begin subsection***************
\subsubsection{\DbgSecSt{\StPart}{AxisConfig}}
\index{Программный интерфейс ПЛК!Управление осями!Структура AxisConfig}
\label{sec:AxisConfig}

\begin{fHeader}
    Тип данных:            & \RightHandText{Структура AxisConfig}\\
    Файл объявления:             & \RightHandText{include/func/axis.h} \\
\end{fHeader}

Структура определяет настройки оси.

\begin{MyTableThreeColAllCntr}{Структура AxisConfig}{tbl:AxisConfig}{|m{0.33\linewidth}|m{0.22\linewidth}|m{0.45\linewidth}|}{Элемент}{Тип}{Описание}
\hline servo & \centering{unsigned} & Номер платы управления ($0\div3$)\\
\hline chan & \centering{unsigned} & Номер канала ($0\div7$)\\
\hline motor & \centering{unsigned} & Номер связанного с осью двигателя ($0\div31$)\\
\hline homeOrder & \centering{unsigned} & Порядок выезда в нулевую точку \\
\hline needDKill & \centering{unsigned:1} & Требуется задержка перед отключением \\
\hline needPhaseRef & \centering{unsigned:1} & Требуется фазировка \\

\hline needHome & \centering{unsigned:1} & Требуется выезд в нулевую точку \\
\hline needIndex & \centering{unsigned:1} & Требуется позиционирование по индексной метке \\
\hline hasAbsPos & \centering{unsigned:1} & Установлен абсолютный датчик \\

\hline abortMode & \centering{unsigned:2} & Реакции на команду аварийного выключения (см. \myreftosec{AxisAbortMode}) \\
\hline homeCaptCtrl & \centering{unsigned:4} & Настройка захвата положения для выезда в нулевую точку по входу (флагу) \\
\hline indexCaptCtrl & \centering{unsigned:4} & Настройка захвата положения для выезда в нулевую точку по индексной метке ДОС \\
\hline needPosRef & \centering{unsigned:1} & Требуется позиционирование при включении станка \\
\hline refAxis & \centering{unsigned:5} & Координата для оси \\
\hline rotaryAxis & \centering{unsigned:1} & Вращающаяся ось с периодом 360 \\

\hline reserved & \centering{unsigned:10} & Резерв \\
\hline homeVel & \centering{double} & Скорость и направление выезда в нулевую точку \\
\hline indexVel & \centering{double} & Скорость и направление поиска индексной метки \\
\hline homeOffset & \centering{double} & Смещение нулевой точки относительно позиции ДОС \\
\hline indexOffset & \centering{double} & Смещение индексной метки относительно позиции ДОС \\
\hline homeOfsVel & \centering{double} & Скорость движения в позицию смещения нулевой точки (не используется) \\
\hline indexOfsVel & \centering{double} & Скорость движения в позицию смещения индексной метки (не используется) \\
\hline minPos & \centering{double} & Программное ограничение в отрицательном направлении (для абсолютного ДОС настраивается в дискретах датчика, определённых в энкодерной таблице) \\
\hline maxPos & \centering{double} &  Программное ограничение в положительном направлении (для абсолютного ДОС настраивается в дискретах датчика, определённых в энкодерной таблице) \\
\hline defaultTa & \centering{double} & Время в мс ускорения/замедления (при значении больше 0) или коэффициент, обратный величине амплитуды ускорения/замедления (при значении меньше 0) по умолчанию \\
\hline defaultTs & \centering{double} & Время в мс (при значении больше 0) или коэффициент, обратный значению амплитуды рывка (при значении меньше 0), для каждой половины S-кривой профиля ускорения по умолчанию \\
\hline manualTa & \centering{double} & Время в мс ускорения/замедления (при значении больше 0) или коэффициент, обратный величине амплитуды ускорения/замедления (при значении меньше 0) в ручном режиме \\
\hline manualTs & \centering{double} & Время в мс (при значении больше 0) или коэффициент, обратный значению амплитуды рывка (при значении меньше 0), для каждой половины S-кривой профиля ускорения в ручном режиме \\
\hline hwlTa & \centering{double} & Время в мс ускорения/замедления (при значении больше 0) или коэффициент, обратный величине амплитуды ускорения/замедления (при значении меньше 0) в режиме дискретных перемещений \\
\hline hwlTs & \centering{double} & Время в мс (при значении больше 0) или коэффициент, обратный значению амплитуды рывка (при значении меньше 0), для каждой половины S-кривой профиля ускорения в режиме дискретных перемещений \\
\hline homeTa & \centering{double} & Время в мс ускорения/замедления (при значении больше 0) или коэффициент, обратный величине амплитуды ускорения/замедления (при значении меньше 0) в режиме выезда в нулевую точку \\
\hline homeTs & \centering{double} &  Время в мс (при значении больше 0) или коэффициент, обратный значению амплитуды рывка (при значении меньше 0), для каждой половины S-кривой профиля ускорения в режиме выезда в нулевую точку \\
\hline autoTa & \centering{double} & Время в мс ускорения/замедления (при значении больше 0) или коэффициент, обратный величине амплитуды ускорения/замедления (при значении меньше 0) в автоматическом режиме \\
\hline autoTs & \centering{double} &  Время в мс (при значении больше 0) или коэффициент, обратный значению амплитуды рывка (при значении меньше 0), для каждой половины S-кривой профиля ускорения в автоматическом режиме \\
\hline encRes & \centering{double} &  Число дискрет датчика на оборот\\
\end{MyTableThreeColAllCntr}
% *******end subsection***************

Поля \texttt{homeCaptCtrl} и \texttt{indexCaptCtrl} являются 4-битными и содержат настройки захвата положения для выезда в нулевую точку:
\begin{itemize}
\item биты 0 и 1 определяют тип захвата положения (0 ~-- непосредственный захват, 1 ~--  по индексному сигналу датчика, 2 ~-- захват по флагу, 3 ~-- по флагу и индексному сигналу); \killoverfullbefore
\item бит 2 управляет инверсией индексного сигнала ДОС (0 ~-- не инвертировать, 1 ~-- инвертировать); \killoverfullbefore
\item бит 3 управляет инверсией флага при захвате положения (0 ~-- не инвертировать, 1 ~-- инвертировать). \killoverfullbefore \BL
\end{itemize} 

%--------------------------------------------------------
% *******begin subsection***************
\subsubsection{\DbgSecSt{\StPart}{Axis}}
\index{Программный интерфейс ПЛК!Управление осями!Структура Axis}
\label{sec:Axis}

\begin{fHeader}
    Тип данных:            & \RightHandText{Структура Axis}\\
    Файл объявления:             & \RightHandText{include/func/axis.h} \\
\end{fHeader}

Структура определяет состояние, параметры и данные оси.

\begin{MyTableThreeColAllCntr}{Структура Axis}{tbl:Axis}{|m{0.3\linewidth}|m{0.25\linewidth}|m{0.45\linewidth}|}{Элемент}{Тип}{Описание}
\hline state & \centering{\myreftosec{AxisStates}} & Текущее состояние \\
\hline command & \centering{\myreftosec{AxisCommands}} & Текущая команда \\
\hline statePreHome & \centering{\myreftosec{AxisStates}} & Состояние перед выездом в нулевую точку \\
\hline followup & \centering{unsigned:1} & Восстановление после ошибки \\
\hline phaseRefComplete & \centering{unsigned:1} & Фазировка выполнена \\
\hline phaseRefError & \centering{unsigned:1} & Ошибка фазировки \\
\hline homeComplete & \centering{unsigned:1} & Выполнен выезд в нулевую точку \\
\hline homeErrorFlag & \centering{unsigned:1} & Ошибка выезда в нулевую точку \\
\hline posRefComplete & \centering{unsigned:1} & Позиционирование при включении станка выполнено \\
\hline timer & \centering{\myreftosec{Timer}} & Таймер \\
\hline JogValue & \centering{double} & Значение заданной позиции для толчкового перемещения \\
\hline IncStep & \centering{double} & Значение заданного расстояния для толчкового перемещения \\
\hline platform & \centering{\hyperlink{Axis_Platform_Control}{AxisPlatformControl}} & Пользовательские параметры и переменные оси \\
\end{MyTableThreeColAllCntr}

\hypertarget{Axis_Platform_Control}{Структура} \texttt{AxisPlatformControl} является пользовательской и служит для введения дополнительных параметров и переменных оси.\killoverfullbefore

Если структура \texttt{AxisPlatformControl} задана пользователем, то должен быть определён идентификатор \texttt{AXES\_PLATFORM\_CONTROL\_DEFINED}: \texttt{\#define AXES\_PLATFORM\_CONTROL\_DEFINED}. \killoverfullbefore
% *******end subsection***************
%--------------------------------------------------------
% *******begin subsection***************
\subsubsection{\DbgSecSt{\StPart}{AxesControl}}
\index{Программный интерфейс ПЛК!Управление осями!Структура AxesControl}
\label{sec:AxesControl}

\begin{fHeader}
    Тип данных:            & \RightHandText{Структура AxesControl}\\
    Файл объявления:             & \RightHandText{include/func/axis.h} \\
\end{fHeader}

Структура определяет состояние, параметры и данные осей.

\begin{MyTableThreeColAllCntr}{Структура AxesControl}{tbl:AxesControl}{|m{0.3\linewidth}|m{0.25\linewidth}|m{0.45\linewidth}|}{Элемент}{Тип}{Описание}
\hline homeState & \centering{\myreftosec{HomeStates}} & Состояние выезда в нулевую точку\\
\hline homeComplete & \centering{unsigned:1} & Выполнен выезд в нулевую точку \\
\hline homeErrorFlag & \centering{unsigned:1} & Ошибка выезда в нулевую точку \\
\hline homeStage & \centering{int} & Этап выезда в нулевую точку \\
\hline axis[ЧИСЛО\_ОСЕЙ] & \centering{\myreftosec{Axis}} & Данные осей \\
\hline timerHome & \centering{\myreftosec{Timer}} & Таймер для задержек переключений состояний в режиме выезда в нулевую точку\\
\hline platform & \centering{\hyperlink{Axes_Platform_Control}{AxesPlatformControl}} & Пользовательские параметры и переменные осей \\
\hline saveSpeed & \centering{int} & Сохранённая скорость с пульта оператора \\
\hline activeAxis & \centering{int} & Номер активной оси \\
\end{MyTableThreeColAllCntr}

\hypertarget{Axes_Platform_Control}{Структура} \texttt{AxesPlatformControl} является пользовательской и служит для введения дополнительных параметров и переменных осей.\killoverfullbefore

Если структура \texttt{AxesPlatformControl} задана пользователем, то должен быть определён идентификатор \texttt{AXES\_PLATFORM\_CONTROL\_DEFINED}: \texttt{\#define AXES\_PLATFORM\_CONTROL\_DEFINED}.\killoverfullbefore
% *******end subsection***************
%-------------------------------------------------------------------
% *******begin subsection***************
\subsection{\DbgSecSt{\StPart}{Функции}}

% *******begin subsection***************
\subsubsection{\DbgSecSt{\StPart}{axesForceKill}}
\index{Программный интерфейс ПЛК!Управление осями!Функция axesForceKill}
\label{sec:axesForceKill}

\begin{pHeader}
    Синтаксис:      & \RightHandText{void axesForceKill();}\\
    Аргумент(ы):    & \RightHandText{Нет} \\    
%    Возвращаемое значение:       & \RightHandText{Нет} \\ 
    Файл объявления:             & \RightHandText{include/func/axis.h} \\       
\end{pHeader}

Функция вызывает принудительное выключение всех осей.

Является системной.
% *******end section*****************
%-------------------------------------------------------------------
% *******begin subsection***************
\subsubsection{\DbgSecSt{\StPart}{axisForceKill}}
\index{Программный интерфейс ПЛК!Управление осями!Функция axisForceKill}
\label{sec:axisForceKill}

\begin{pHeader}
    Синтаксис:      & \RightHandText{void axisForceKill(unsigned axis);}\\
    Аргумент(ы):    & \RightHandText{unsigned axis ~-- номер оси} \\    
%    Возвращаемое значение:       & \RightHandText{Нет} \\ 
    Файл объявления:             & \RightHandText{include/func/axis.h} \\       
\end{pHeader}

Функция вызывает принудительное выключение оси, номер которой является аргументом функции.

Является системной.
% *******end section*****************

%-------------------------------------------------------------------
% *******begin subsection***************
\subsubsection{\DbgSecSt{\StPart}{axesDeactivate}}
\index{Программный интерфейс ПЛК!Управление осями!Функция axesDeactivate}
\label{sec:axesDeactivate}

\begin{pHeader}
    Синтаксис:      & \RightHandText{void axesDeactivate();}\\
    Аргумент(ы):    & \RightHandText{Нет} \\    
%    Возвращаемое значение:       & \RightHandText{Нет} \\ 
    Файл объявления:             & \RightHandText{include/func/axis.h} \\       
\end{pHeader}

Функция вызывает выключение всех осей.

Является системной.
% *******end section*****************
%-------------------------------------------------------------------
% *******begin subsection***************
\subsubsection{\DbgSecSt{\StPart}{axesActivate}}
\index{Программный интерфейс ПЛК!Управление осями!Функция axesActivate}
\label{sec:axesActivate}

\begin{pHeader}
    Синтаксис:      & \RightHandText{void axesActivate();}\\
    Аргумент(ы):    & \RightHandText{Нет} \\    
%    Возвращаемое значение:       & \RightHandText{Нет} \\ 
    Файл объявления:             & \RightHandText{include/func/axis.h} \\
\end{pHeader}

Функция вызывает включение в слежение всех осей.

Является системной.
% *******end section*****************

%-------------------------------------------------------------------
% *******begin subsection***************
\subsubsection{\DbgSecSt{\StPart}{axesInactive}}
\index{Программный интерфейс ПЛК!Управление осями!Функция axesInactive}
\label{sec:axesInactive}

\begin{pHeader}
    Синтаксис:      & \RightHandText{int axesInactive();}\\
    Аргумент(ы):    & \RightHandText{Нет} \\    
%    Возвращаемое значение:       & \RightHandText{Целое знаковое число} \\ 
    Файл объявления:             & \RightHandText{include/func/axis.h} \\       
\end{pHeader}

Функция возвращает 1, если хотя бы одна ось не находится в слежении, и 0 в противном случае.

Является системной.
% *******end section*****************
%-------------------------------------------------------------------
% *******begin subsection***************
\subsubsection{\DbgSecSt{\StPart}{axesActive}}
\index{Программный интерфейс ПЛК!Управление осями!Функция axesActive}
\label{sec:axesActive}

\begin{pHeader}
    Синтаксис:      & \RightHandText{int axesActive();}\\
    Аргумент(ы):    & \RightHandText{Нет} \\    
%    Возвращаемое значение:       & \RightHandText{Целое знаковое число} \\ 
    Файл объявления:             & \RightHandText{include/func/axis.h} \\       
\end{pHeader}

Функция возвращает 1, если все оси находятся в слежении, и 0 в противном случае.

Является системной.
% *******end section*****************
%-------------------------------------------------------------------
% *******begin subsection***************
\subsubsection{\DbgSecSt{\StPart}{axesPhaseRefComplete}}
\index{Программный интерфейс ПЛК!Управление осями!Функция axesPhaseRefComplete}
\label{sec:axesPhaseRefComplete}

\begin{pHeader}
    Синтаксис:      & \RightHandText{int axesPhaseRefComplete();}\\
    Аргумент(ы):    & \RightHandText{Нет} \\    
%    Возвращаемое значение:       & \RightHandText{Целое знаковое число} \\ 
    Файл объявления:             & \RightHandText{include/func/axis.h} \\       
\end{pHeader}

Функция возвращает 1, если фазировка выполнена для всех осей, и 0 в противном случае.

Является системной.
% *******end section*****************
%-------------------------------------------------------------------
% *******begin subsection***************
\subsubsection{\DbgSecSt{\StPart}{axesPhaseRef}}
\index{Программный интерфейс ПЛК!Управление осями!Функция axesPhaseRef}
\label{sec:axesPhaseRef}

\begin{pHeader}
    Синтаксис:      & \RightHandText{int axesPhaseRef();}\\
    Аргумент(ы):    & \RightHandText{Нет} \\    
%    Возвращаемое значение:       & \RightHandText{Целое знаковое число} \\ 
    Файл объявления:             & \RightHandText{include/func/axis.h} \\
\end{pHeader}

Функция возвращает 1, если хотя бы одна ось требует фазировки, и 0 в противном случае. Для оси, фазировка которой не выполнена, даётся команда фазировки.

Является системной.
% *******end section*****************
%-------------------------------------------------------------------
% *******begin subsection***************
\subsubsection{\DbgSecSt{\StPart}{axesAborted}}
\index{Программный интерфейс ПЛК!Управление осями!Функция axesAborted}
\label{sec:axesAborted}

\begin{pHeader}
    Синтаксис:      & \RightHandText{int axesAborted();}\\
    Аргумент(ы):    & \RightHandText{Нет} \\    
%    Возвращаемое значение:       & \RightHandText{Целое знаковое число} \\ 
    Файл объявления:             & \RightHandText{include/func/axis.h} \\       
\end{pHeader}

Функция возвращает 1, если все оси аварийно остановлены, и 0 в противном случае.

Является системной.
% *******end section*****************
%-------------------------------------------------------------------
% *******begin subsection***************
\subsubsection{\DbgSecSt{\StPart}{axisStopped}}
\index{Программный интерфейс ПЛК!Управление осями!Функция axisStopped}
\label{sec:axisStopped}

\begin{pHeader}
    Синтаксис:      & \RightHandText{int axisStopped(unsigned axis);}\\
    Аргумент(ы):    & \RightHandText{unsigned axis ~-- номер оси} \\    
%    Возвращаемое значение:       & \RightHandText{Целое знаковое число} \\
    Файл объявления:             & \RightHandText{include/func/axis.h} \\
\end{pHeader}

Функция возвращает 1, если ось остановлена (ось в слежении имеет равную нулю заданную скорость и находится в позиции), и 0 в противном случае.  

Является системной.
% *******end section*****************
%-------------------------------------------------------------------
% *******begin subsection***************
\subsubsection{\DbgSecSt{\StPart}{axesStopped}}
\index{Программный интерфейс ПЛК!Управление осями!Функция axesStopped}
\label{sec:axesStopped}

\begin{pHeader}
    Синтаксис:      & \RightHandText{int axesStopped();}\\
    Аргумент(ы):    & \RightHandText{Нет} \\    
%    Возвращаемое значение:       & \RightHandText{Целое знаковое число} \\ 
    Файл объявления:             & \RightHandText{include/func/axis.h} \\       
\end{pHeader}

Функция возвращает 1, если все оси остановлены (оси в слежении имеют равную нулю заданную скорость и находятся в позиции), и 0 в противном случае.

Является системной.
% *******end section*****************
%-------------------------------------------------------------------
% *******begin subsection***************
\subsubsection{\DbgSecSt{\StPart}{axesAbortAll}}
\index{Программный интерфейс ПЛК!Управление осями!Функция axesAbortAll}
\label{sec:axesAbortAll}

\begin{pHeader}
    Синтаксис:      & \RightHandText{void axesAbortAll();}\\
    Аргумент(ы):    & \RightHandText{Нет} \\    
%    Возвращаемое значение:       & \RightHandText{Нет} \\ 
    Файл объявления:             & \RightHandText{include/func/axis.h} \\
\end{pHeader}

Функция вызывает аварийное выключение всех осей.

Является системной.
% *******end section*****************
%-------------------------------------------------------------------
% *******begin subsection***************
\subsubsection{\DbgSecSt{\StPart}{axesStopAll}}
\index{Программный интерфейс ПЛК!Управление осями!Функция axesStopAll}
\label{sec:axesStopAll}

\begin{pHeader}
    Синтаксис:      & \RightHandText{void axesStopAll();}\\
    Аргумент(ы):    & \RightHandText{Нет} \\    
%    Возвращаемое значение:       & \RightHandText{Нет} \\ 
    Файл объявления:             & \RightHandText{include/func/axis.h} \\
\end{pHeader}

Функция вызывает останов всех осей при толчковых перемещениях.

Является системной.
% *******end section*****************
%--------------------------------------------------------
% *******begin subsection***************
\subsubsection{\DbgSecSt{\StPart}{axesRet}}
\index{Программный интерфейс ПЛК!Управление осями!Функция axesRet}
\label{sec:axesRet}

\begin{pHeader}
    Синтаксис:      & \RightHandText{void axesRet();}\\
    Аргумент(ы):    & \RightHandText{Нет} \\    
%    Возвращаемое значение:       & \RightHandText{Нет} \\ 
    Файл объявления:             & \RightHandText{include/func/axis.h} \\
\end{pHeader}

Функция вызывает перемещение осей в сохраненную позицию при толчковых перемещениях.

Является системной.
% *******end section*****************

%--------------------------------------------------------
% *******begin subsection***************
\subsubsection{\DbgSecSt{\StPart}{axisIndexInit}}
\index{Программный интерфейс ПЛК!Управление осями!Функция axisIndexInit}
\label{sec:axisIndexInit}

\begin{pHeader}
    Синтаксис:      & \RightHandText{void axisIndexInit(unsigned axis);}\\
    Аргумент(ы):    & \RightHandText{unsigned axis ~-- номер оси} \\    
%    Возвращаемое значение:       & \RightHandText{Нет} \\ 
    Файл объявления:             & \RightHandText{include/func/axis.h} \\
\end{pHeader}

%Функция выполняет запрос выезда оси, номер которой является аргументом функции, в нулевую точку по индексной метке. 

Функция выполняет инициализацию параметров поиска индексной метки для оси, номер которой является аргументом функции.

Является системной.
% *******end section*****************

%--------------------------------------------------------
% *******begin subsection***************
\subsubsection{\DbgSecSt{\StPart}{axisPosition}}
\index{Программный интерфейс ПЛК!Управление осями!Функция axisPosition}
\label{sec:axisPosition}

\begin{pHeader}
    Синтаксис:      & \RightHandText{double axisPosition(unsigned axis);}\\
    Аргумент(ы):    & \RightHandText{unsigned axis ~-- номер оси} \\    
%    Возвращаемое значение:       & \RightHandText{Число с плавающей запятой двойной точности} \\ 
    Файл объявления:             & \RightHandText{include/func/axis.h} \\
\end{pHeader}

Функция возвращает заданную позицию оси, номер которой является аргументом функции.

Возвращаемое значение измеряется в единицах \texttt{encRes} (см. структуру \myreftosec{AxisConfig}). Для вращающихся осей возвращаемое значение ~-- остаток от деления на 360. \killoverfullbefore

Является системной.
% *******end section*****************
%--------------------------------------------------------
% *******begin subsection***************
\subsubsection{\DbgSecSt{\StPart}{axesFollowup}}
\index{Программный интерфейс ПЛК!Управление осями!Функция axesFollowup}
\label{sec:axesFollowup}

\begin{pHeader}
    Синтаксис:      & \RightHandText{void axesFollowup();}\\
    Аргумент(ы):    & \RightHandText{Нет} \\    
%    Возвращаемое значение:       & \RightHandText{Нет} \\ 
    Файл объявления:             & \RightHandText{include/func/axis.h} \\
\end{pHeader}

Функция устанавливает для всех осей флаг <<Восстановление после ошибки>> (см. структуру \myreftosec{Axis}).\killoverfullbefore

Является системной.
% *******end section*****************
%--------------------------------------------------------
% *******begin subsection***************
\subsubsection{\DbgSecSt{\StPart}{initAxis}}
\index{Программный интерфейс ПЛК!Управление осями!Функция initAxis}
\label{sec:initAxis}

\begin{pHeader}
    Синтаксис:      & \RightHandText{void initAxis(int axis);}\\
    Аргумент(ы):    & \RightHandText{int axis ~-- номер оси} \\    
%    Возвращаемое значение:       & \RightHandText{Нет} \\ 
    Файл объявления:             & \RightHandText{include/func/axis.h} \\
\end{pHeader}

Функция выполняет инициализацию оси, номер которой является аргументом функции, параметрами по умолчанию. 

Является системной.
% *******end section*****************
%--------------------------------------------------------
% *******begin subsection***************
\subsubsection{\DbgSecSt{\StPart}{initAxes}}
\index{Программный интерфейс ПЛК!Управление осями!Функция initAxes}
\label{sec:initAxes}

\begin{pHeader}
    Синтаксис:      & \RightHandText{void initAxes();}\\
    Аргумент(ы):    & \RightHandText{Нет} \\    
%    Возвращаемое значение:       & \RightHandText{Нет} \\ 
    Файл объявления:             & \RightHandText{include/func/axis.h} \\
\end{pHeader}

Функция выполняет инициализацию осей параметрами по умолчанию. 

Является системной.
% *******end section*****************
%--------------------------------------------------------
% *******begin subsection***************
\subsubsection{\DbgSecSt{\StPart}{axisInitPlatform}}
\index{Программный интерфейс ПЛК!Управление осями!Функция axisInitPlatform}
\label{sec:axisInitPlatform}

\begin{pHeader}
    Синтаксис:      & \RightHandText{void axisInitPlatform(int axis);}\\
    Аргумент(ы):    & \RightHandText{int axis ~-- номер оси} \\    
%    Возвращаемое значение:       & \RightHandText{Нет} \\ 
    Файл объявления:             & \RightHandText{include/func/axis.h} \\
\end{pHeader}

Функция выполняет инициализацию параметров оси, номер которой является аргументом функции, пользовательскими значениями, в том числе структуры \hyperlink{Axis_Platform_Control}{AxisPlatformControl}. \killoverfullbefore

Реализуется пользователем.
% *******end section*****************
%--------------------------------------------------------
% *******begin subsection***************
\subsubsection{\DbgSecSt{\StPart}{axesInitPlatform}}
\index{Программный интерфейс ПЛК!Управление осями!Функция axesInitPlatform}
\label{sec:axesInitPlatform}

\begin{pHeader}
    Синтаксис:      & \RightHandText{void axesInitPlatform();}\\
    Аргумент(ы):    & \RightHandText{Нет} \\    
%    Возвращаемое значение:       & \RightHandText{Нет} \\ 
    Файл объявления:             & \RightHandText{include/func/axis.h} \\
\end{pHeader}

Функция выполняет инициализацию параметров осей пользовательскими значениями, в том числе структуры \hyperlink{Axes_Platform_Control}{AxesPlatformControl}. \killoverfullbefore

Реализуется пользователем.
% *******end section*****************
%--------------------------------------------------------
% *******begin subsection***************
\subsubsection{\DbgSecSt{\StPart}{axesAbsPosRead}}
\index{Программный интерфейс ПЛК!Управление осями!Функция axesAbsPosRead}
\label{sec:axesAbsPosRead}

\begin{pHeader}
    Синтаксис:      & \RightHandText{void axesAbsPosRead();}\\
    Аргумент(ы):    & \RightHandText{Нет} \\    
%    Возвращаемое значение:       & \RightHandText{Нет} \\ 
    Файл объявления:             & \RightHandText{include/func/axis.h} \\
\end{pHeader}

Функция выполняет запрос чтения данных абсолютных ДОС для осей.

Является системной.
% *******end section*****************
%-------------------------------------------------------------------
% *******begin subsection***************
\subsubsection{\DbgSecSt{\StPart}{axesAbsPosReadComplete}}
\index{Программный интерфейс ПЛК!Управление осями!Функция axesAbsPosReadComplete}
\label{sec:axesAbsPosReadComplete}

\begin{pHeader}
    Синтаксис:      & \RightHandText{int axesAbsPosReadComplete();}\\
    Аргумент(ы):    & \RightHandText{Нет} \\    
%    Возвращаемое значение:       & \RightHandText{Целое знаковое число} \\ 
    Файл объявления:             & \RightHandText{include/func/axis.h} \\       
\end{pHeader}

Функция возвращает 1, если чтение данных абсолютных ДОС для осей завершено, и 0 в противном случае.

Является системной.
% *******end section*****************
%-------------------------------------------------------------------
% *******begin subsection***************
\subsubsection{\DbgSecSt{\StPart}{axisRefPosComplete}}
\index{Программный интерфейс ПЛК!Управление осями!Функция axisRefPosComplete}
\label{sec:axisRefPosComplete}

\begin{pHeader}
    Синтаксис:      & \RightHandText{int axisRefPosComplete(unsigned axis);}\\
    Аргумент(ы):   & \RightHandText{unsigned axis ~-- номер оси}\\
%    Возвращаемое значение:       & \RightHandText{Целое знаковое число} \\ 
    Файл объявления:             & \RightHandText{include/func/axis.h} \\       
\end{pHeader}

Функция возвращает 1, если требуется и выполнено позиционирование оси, номер которой является аргументом функции, при включении станка или запуске программы, и 0 в противном случае. \killoverfullbefore

Является системной.
% *******end section*****************
%-------------------------------------------------------------------
% *******begin subsection***************
\subsubsection{\DbgSecSt{\StPart}{axesRefPosComplete}}
\index{Программный интерфейс ПЛК!Управление осями!Функция axesRefPosComplete}
\label{sec:axesRefPosComplete}

\begin{pHeader}
    Синтаксис:      & \RightHandText{int axesRefPosComplete();}\\
    Аргумент(ы):    & \RightHandText{Нет} \\    
%    Возвращаемое значение:       & \RightHandText{Целое знаковое число} \\ 
    Файл объявления:             & \RightHandText{include/func/axis.h} \\       
\end{pHeader}

Функция возвращает 1, если требуется и выполнено позиционирование всех осей при включении станка или запуске программы, и 0 в противном случае. \killoverfullbefore

Является системной.
% *******end section*****************
%-------------------------------------------------------------------
% *******begin subsection***************
\subsubsection{\DbgSecSt{\StPart}{axisAtRefPos}}
\index{Программный интерфейс ПЛК!Управление осями!Функция axisAtRefPos}
\label{sec:axisAtRefPos}

\begin{pHeader}
    Синтаксис:      & \RightHandText{int axisAtRefPos(unsigned axis);}\\
    Аргумент(ы):    & \RightHandText{unsigned axis ~-- номер оси} \\    
%%    Возвращаемое значение:       & \RightHandText{Целое знаковое число} \\ 
    Файл объявления:             & \RightHandText{include/func/axis.h} \\       
\end{pHeader}

Функция возвращает 1, если ось, номер которой является аргументом функции, находится в первой референтной позиции, и 0 в противном случае. \killoverfullbefore

Является системной.
% *******end section*****************
%-------------------------------------------------------------------
% *******begin subsection***************
\subsubsection{\DbgSecSt{\StPart}{axesAtRefPos}}
\index{Программный интерфейс ПЛК!Управление осями!Функция axesAtRefPos}
\label{sec:axesAtRefPos}

\begin{pHeader}
    Синтаксис:      & \RightHandText{int axesAtRefPos();}\\
    Аргумент(ы):    & \RightHandText{Нет} \\    
%%    Возвращаемое значение:       & \RightHandText{Целое знаковое число} \\ 
    Файл объявления:             & \RightHandText{include/func/axis.h} \\       
\end{pHeader}

Функция возвращает 1, если все оси находятся в первой референтной позиции, и 0 в противном случае. 

Является системной.
% *******end section*****************
%--------------------------------------------------------
