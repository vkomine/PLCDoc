%--------------------------------------------------------
% *******begin section***************
\section{\DbgSecSt{\StPart}{Управление движением при фрезерной обработке}}
%--------------------------------------------------------

\subsection{\DbgSecSt{\StPart}{Типы данных}}
%--------------------------------------------------------
% *******begin subsection***************
\subsubsection{\DbgSecSt{\StPart}{Tool}}
\index{Программный интерфейс ПЛК!Управление движением при фрезерной  обработке!Структура Tool}
\label{sec:Tool}

\begin{fHeader}
    Тип данных:            & \RightHandText{Структура Tool}\\
    Файл объявления:             & \RightHandText{include/motion/mill/motion.h} \\
\end{fHeader}

Структура определяет параметры инструмента.

\begin{MyTableThreeColAllCntr}{Структура Tool}{tbl:Tool}{|m{0.3\linewidth}|m{0.25\linewidth}|m{0.45\linewidth}|}{Элемент}{Тип}{Описание}
\hline X & \centering{double} &   \\
\hline Y & \centering{double} &  \\
\hline L & \centering{double} & Длина \\
\hline R & \centering{double} & Радиус \\
\end{MyTableThreeColAllCntr}
% *******end subsection***************
%--------------------------------------------------------
% *******begin subsection***************
\subsubsection{\DbgSecSt{\StPart}{CSTransform}}
\index{Программный интерфейс ПЛК!Управление движением при фрезерной  обработке!Структура CSTransform}
\label{sec:CSTransform}

\begin{fHeader}
    Тип данных:            & \RightHandText{Структура CSTransform}\\
    Файл объявления:             & \RightHandText{include/motion/mill/motion.h} \\
\end{fHeader}

Структура определяет параметры преобразования системы координат.

\begin{MyTableThreeColAllCntr}{Структура CSTransform}{tbl:CSTransform}{|m{0.41\linewidth}|m{0.24\linewidth}|m{0.35\linewidth}|}{Элемент}{Тип}{Описание}
\hline ofs & \centering{\myreftosec{XYZ}} & Смещение \\
\hline scale & \centering{\myreftosec{XYZ}} & Коэффициенты масштабирования \\
\hline scaleOfs & \centering{\myreftosec{XYZ}} & Координаты центра масштабирования \\
\hline mirror & \centering{\myreftosec{int}} & Оси для зеркального отображения (MIRROR\_X | MIRROR\_Y | MIRROR\_Z) \\
\hline mirrorOfs & \centering{\myreftosec{XYZ}} & Координаты центра зеркального отображения \\

\hline rotVec & \centering{\myreftosec{XYZ}} & Компоненты вектора поворота  \\
\hline rotOfs & \centering{\myreftosec{XYZ}} & Координаты центра поворота \\

\hline rotAngle & \centering{\myreftosec{double}} & Угол поворота \\
\hline rot[9] & \centering{\myreftosec{double}} &   \\
\hline local & \centering{\myreftosec{XYZ}} &   \\

\hline Bias[32] & \centering{\myreftosec{double}} &   \\
\hline rot[9] & \centering{\myreftosec{double}} &   \\
\hline XYZ[6] & \centering{\myreftosec{double}} &   \\
\end{MyTableThreeColAllCntr}
% *******end subsection***************
%--------------------------------------------------------
% *******begin subsection***************
\subsubsection{\DbgSecSt{\StPart}{MoveMode}}
\index{Программный интерфейс ПЛК!Управление движением при фрезерной  обработке!Перечисление MoveMode}
\label{sec:MoveMode}

\begin{fHeader}
    Тип данных:            & \RightHandText{Перечисление MoveMode}\\
    Файл объявления:             & \RightHandText{include/motion/mill/motion.h} \\
\end{fHeader}

Перечисление определяет идентификаторы режимов движения.

\begin{MyTableTwoColAllCntr}{Перечисление MoveMode}{tbl:MoveMode}{|m{0.38\linewidth}|m{0.57\linewidth}|}{Идентификатор}{Описание}
\hline moveRapid &   Быстрые перемещения  \\
\hline moveLinear  &  Линейная интерполяция \\
\hline moveCircleCW  &  Круговая интерполяция по часовой стрелке \\
\hline moveCircleCCW &  Круговая интерполяция против часовой стрелки \\
\end{MyTableTwoColAllCntr}
% *******end subsection***************
%--------------------------------------------------------
% *******begin subsection***************
\subsubsection{\DbgSecSt{\StPart}{PositionMode}}
\index{Программный интерфейс ПЛК!Управление движением при фрезерной  обработке!Перечисление PositionMode}
\label{sec:PositionMode}

\begin{fHeader}
    Тип данных:            & \RightHandText{Перечисление PositionMode}\\
    Файл объявления:             & \RightHandText{include/motion/mill/motion.h} \\
\end{fHeader}

Перечисление определяет идентификаторы режимов позиционирования.

\begin{MyTableTwoColAllCntr}{Перечисление PositionMode}{tbl:PositionMode}{|m{0.38\linewidth}|m{0.57\linewidth}|}{Идентификатор}{Описание}
\hline posAbsolute &  В абсолютных координатах \\
\hline posIncremental  & В относительных координатах (в приращениях)\\
\end{MyTableTwoColAllCntr}
% *******end subsection***************
%--------------------------------------------------------
% *******begin subsection***************
\subsubsection{\DbgSecSt{\StPart}{FeedMode}}
\index{Программный интерфейс ПЛК!Управление движением при фрезерной  обработке!Перечисление FeedMode}
\label{sec:FeedMode}

\begin{fHeader}
    Тип данных:            & \RightHandText{Перечисление FeedMode}\\
    Файл объявления:             & \RightHandText{include/motion/mill/motion.h} \\
\end{fHeader}

Перечисление определяет идентификаторы режимов управления подачей.

\begin{MyTableTwoColAllCntr}{Перечисление FeedMode}{tbl:FeedMode}{|m{0.38\linewidth}|m{0.57\linewidth}|}{Идентификатор}{Описание}
\hline feedUnitsMin &  Минутная подача (асинхронная подача) \\
\hline feedUnitsRev & Оборотная подача (синхронная подача) \\
\hline feedInverse & Подача с обратно зависимым временем \\
\end{MyTableTwoColAllCntr}
% *******end subsection***************
%--------------------------------------------------------
% *******begin subsection***************
\subsubsection{\DbgSecSt{\StPart}{SpindleMode}}
\index{Программный интерфейс ПЛК!Управление движением при фрезерной  обработке!Перечисление SpindleMode}
\label{sec:SpindleMode}

\begin{fHeader}
    Тип данных:            & \RightHandText{Перечисление SpindleMode}\\
    Файл объявления:             & \RightHandText{include/motion/mill/motion.h} \\
\end{fHeader}

Перечисление определяет идентификаторы режимов управления скоростью шпинделя.

\begin{MyTableTwoColAllCntr}{Перечисление SpindleMode}{tbl:SpindleMode}{|m{0.38\linewidth}|m{0.57\linewidth}|}{Идентификатор}{Описание}
\hline spindleRevMin & Скорость в об/мин \\
\hline spindleUnitsMin & Постоянная скорость резания \\
\end{MyTableTwoColAllCntr}
% *******end subsection***************
%--------------------------------------------------------
% *******begin subsection***************
\subsubsection{\DbgSecSt{\StPart}{FeedOverMode}}
\index{Программный интерфейс ПЛК!Управление движением при фрезерной  обработке!Перечисление FeedOverMode}
\label{sec:FeedOverMode}

\begin{fHeader}
    Тип данных:            & \RightHandText{Перечисление FeedOverMode}\\
    Файл объявления:             & \RightHandText{include/motion/mill/motion.h} \\
\end{fHeader}

Перечисление определяет идентификаторы режимов коррекции величины подачи.

\begin{MyTableTwoColAllCntr}{Перечисление FeedOverMode}{tbl:FeedOverMode}{|m{0.38\linewidth}|m{0.57\linewidth}|}{Идентификатор}{Описание}
\hline feedPercent & Коррекция величины подачи в процентном отношении \\
\hline feedFixed100 & Фиксированная величина подачи 100\% \\
\end{MyTableTwoColAllCntr}
% *******end subsection***************
%--------------------------------------------------------
% *******begin subsection***************
\subsubsection{\DbgSecSt{\StPart}{SpindleOverMode}}
\index{Программный интерфейс ПЛК!Управление движением при фрезерной  обработке!Перечисление SpindleOverMode}
\label{sec:SpindleOverMode}

\begin{fHeader}
    Тип данных:            & \RightHandText{Перечисление SpindleOverMode}\\
    Файл объявления:             & \RightHandText{include/motion/mill/motion.h} \\
\end{fHeader}

Перечисление определяет идентификаторы режимов коррекции скорости шпинделя.

\begin{MyTableTwoColAllCntr}{Перечисление SpindleOverMode}{tbl:SpindleOverMode}{|m{0.38\linewidth}|m{0.57\linewidth}|}{Идентификатор}{Описание}
\hline spinPercent & Коррекция величины скорости в процентном отношении  \\
\hline spinFixed100 & Фиксированная величина скорости 100\% \\
\end{MyTableTwoColAllCntr}
% *******end subsection***************
%--------------------------------------------------------
% *******begin subsection***************
\subsubsection{\DbgSecSt{\StPart}{BlendMode}}
\index{Программный интерфейс ПЛК!Управление движением при фрезерной  обработке!Перечисление BlendMode}
\label{sec:BlendMode}

\begin{fHeader}
    Тип данных:            & \RightHandText{Перечисление BlendMode}\\
    Файл объявления:             & \RightHandText{include/motion/mill/motion.h} \\
\end{fHeader}

Перечисление определяет идентификаторы режимов сопряжения кадров (последовательных программных перемещений).

\begin{MyTableTwoColAllCntr}{Перечисление BlendMode}{tbl:BlendMode}{|m{0.38\linewidth}|m{0.57\linewidth}|}{Идентификатор}{Описание}
\hline blendExactStop & Режим точного останова \\
\hline blendCornerOverride & Режим угловой коррекции \\
\hline blendCutting & Режим резания \\
\hline blendTapping & Режим резьбонарезания \\
\end{MyTableTwoColAllCntr}
% *******end subsection***************
%--------------------------------------------------------
% *******begin subsection***************
\subsubsection{\DbgSecSt{\StPart}{ProgramState}}
\index{Программный интерфейс ПЛК!Управление движением при фрезерной  обработке!Перечисление ProgramState}
\label{sec:ProgramState}

\begin{fHeader}
    Тип данных:            & \RightHandText{Перечисление ProgramState}\\
    Файл объявления:             & \RightHandText{include/motion/mill/motion.h} \\
\end{fHeader}

Перечисление определяет идентификаторы состояния УП.

\begin{MyTableTwoColAllCntr}{Перечисление ProgramState}{tbl:ProgramState}{|m{0.38\linewidth}|m{0.57\linewidth}|}{Идентификатор}{Описание}
\hline progReset & УП готова к выполнению \\
\hline progStart & Начало выполнения УП  \\
\hline progRunning &  УП выполняется \\
\hline progHold & УП приостановлена \\
\hline progStop &  УП остановлена \\
\end{MyTableTwoColAllCntr}
% *******end subsection***************
%--------------------------------------------------------
% *******begin subsection***************
\subsubsection{\DbgSecSt{\StPart}{SpindleDirection}}
\index{Программный интерфейс ПЛК!Управление движением при фрезерной  обработке!Перечисление SpindleDirection}
\label{sec:SpindleDirection}

\begin{fHeader}
    Тип данных:            & \RightHandText{Перечисление SpindleDirection}\\
    Файл объявления:             & \RightHandText{include/motion/mill/motion.h} \\
\end{fHeader}

Перечисление определяет идентификаторы состояния движения шпинделя.

\begin{MyTableTwoColAllCntr}{Перечисление SpindleDirection}{tbl:SpindleDirection}{|m{0.38\linewidth}|m{0.57\linewidth}|}{Идентификатор}{Описание}
\hline spinStopped & Шпиндель остановлен \\
\hline spinForward & Шпиндель вращается в прямом направлении \\
\hline spinReverse & Шпиндель вращается в обратном направлении \\
\end{MyTableTwoColAllCntr}
% *******end subsection***************
%--------------------------------------------------------
% *******begin subsection***************
\subsubsection{\DbgSecSt{\StPart}{CycleReturnMode}}
\index{Программный интерфейс ПЛК!Управление движением при фрезерной  обработке!Перечисление CycleReturnMode}
\label{sec:CycleReturnMode}

\begin{fHeader}
    Тип данных:            & \RightHandText{Перечисление CycleReturnMode}\\
    Файл объявления:             & \RightHandText{include/motion/mill/motion.h} \\
\end{fHeader}

Перечисление определяет идентификаторы типов возврата инструмента при выполнении постоянных циклов.

\begin{MyTableTwoColAllCntr}{Перечисление CycleReturnMode}{tbl:CycleReturnMode}{|m{0.38\linewidth}|m{0.57\linewidth}|}{Идентификатор}{Описание}
\hline cycleReturnInitial & Возврат на исходный уровень \\
\hline cycleReturnR & Возврат на опорный уровень (уровень точки R) \\
\end{MyTableTwoColAllCntr}
% *******end subsection***************
%--------------------------------------------------------
% *******begin subsection***************
\subsubsection{\DbgSecSt{\StPart}{ProgramRuntime}}
\index{Программный интерфейс ПЛК!Управление движением при фрезерной  обработке!Структура ProgramRuntime}
\label{sec:ProgramRuntime}

\begin{fHeader}
    Тип данных:            & \RightHandText{Структура ProgramRuntime}\\
    Файл объявления:             & \RightHandText{include/motion/mill/motion.h} \\
\end{fHeader}

Структура определяет параметры и данные канала обработки.

\begin{MyTableThreeColAllCntr}{Структура ProgramRuntime}{tbl:ProgramRuntime}{|m{0.3\linewidth}|m{0.25\linewidth}|m{0.45\linewidth}|}{Элемент}{Тип}{Описание}
\hline cs & \centering{int} & Координатная система \\
\hline spindle & \centering{int} &  \\
\hline programState & \centering{int} & Идентификатор состояния УП (см. \myreftosec{ProgramState}) \\
\hline optionalStop & \centering{int} & Опциональный останов \\
\hline spindleDirection & \centering{int} & Идентификатор состояния движения шпинделя (см. \myreftosec{SpindleDirection}) \\

\hline baseWCS & \centering{\myreftosec{XYZ}} & Координаты центра системы координат заготовки  \\
\hline baseWCSStore & \centering{\myreftosec{XYZ}} &    \\
\hline baseWCSSync & \centering{\myreftosec{XYZ}} &    \\

\hline transform & \centering{\myreftosec{CSTransform}} & Параметры преобразования системы координат \\
\hline transformStore & \centering{\myreftosec{CSTransform}} & Параметры преобразования системы координат \\
\hline transformSync & \centering{\myreftosec{CSTransform}} & Параметры преобразования системы координат \\

\hline gcodes[ЧИСЛО\_G\_КОДОВ] & \centering{int} & Массив идентификаторов G-кодов \\
\hline mcodes[ЧИСЛО\_M\_КОДОВ] & \centering{int} & Массив идентификаторов М-кодов \\

\hline moveMode & \centering{int} & Идентификатор типа перемещения (см. \myreftosec{MoveMode}) \\
\hline posMode & \centering{int} & Идентификатор режима перемещения (см. \myreftosec{PositionMode}\\

\hline scaleUnits & \centering{double} & Единицы измерения перемещений (1.0 ~-- мм)\\

\hline blendMode & \centering{int} & Идентификатор режима сопряжения последовательных программных перемещений (см. \myreftosec{BlendMode}\\
\hline blendModeStore & \centering{int} &  \\

\hline F & \centering{double} & Величина заданной подачи \\
\hline Fact & \centering{double} & Величина фактической подачи \\
\hline feedMode & \centering{int} & Идентификатор режима управления подачей (см. \myreftosec{FeedMode}\\
\hline feedOverMode & \centering{int} & Идентификатор режима коррекции величины подачи (см. \myreftosec{FeedOverMode}\\
\hline feedOverride & \centering{double} & Величина коррекции подачи \\
\hline feedSlewRate & \centering{double} &   \\

\hline S & \centering{double} & Величина заданной скорости шпинделя \\
\hline Sact & \centering{double} & Величина фактической скорости шпинделя \\
\hline spinMode & \centering{int} & Идентификатор режима управления шпинделем (см. \myreftosec{SpindleMode}\\
\hline spinOverMode & \centering{int} & Идентификатор режима коррекции скорости шпинделя (см. \myreftosec{SpindleOverMode}\\
\hline spinOverride & \centering{double} & Величина коррекции скорости шпинделя \\
\hline spinSlewRate & \centering{double} &   \\
\hline Srun & \centering{double} &   \\

\hline T & \centering{int} & Номер инструмента (ячейки) \\
\hline TatM6 & \centering{int} &  \\
\hline D & \centering{int} & Номер корректора на радиус инструмента \\
\hline H & \centering{int} & Корректор корректора на длину инструмента \\
\hline Dvalue & \centering{double} &   \\
\hline Dwear & \centering{double} &   \\
\hline Hvalue & \centering{double} &   \\
\hline Hwear & \centering{double} &   \\

\hline tool & \centering{\myreftosec{Tool}} &  Параметры инструмента  \\
\hline toolWear & \centering{\myreftosec{Tool}} &    \\
\hline prevInterPoint & \centering{\myreftosec{Pos}} &    \\

\hline activeG17 & \centering{int} & Активна рабочая плоскость XY, продольная ось Z \\

\hline cycleReturnMode & \centering{int} & Идентификатор типа возврата инструмента при выполнении постоянных циклов (см. \myreftosec{CycleReturnMode} \\

\hline cycleZ & \centering{double} & Координата основания отверстия \\
\hline cycleR & \centering{double} & Координата опорного уровня (точки R)\\
\hline cycleP & \centering{double} & Длительность паузы в миллисекундах у основания отверстия \\
\hline cycleQ & \centering{double} & Глубина обработки для каждого прохода\\
\hline cycleI & \centering{double} & Величина сдвига инструмента по оси X \\
\hline cycleJ & \centering{double} & Величина сдвига инструмента по оси Y  \\
\hline cycleK & \centering{double} & Величина сдвига инструмента по оси Z  \\
\end{MyTableThreeColAllCntr}
% *******end subsection***************
%--------------------------------------------------------
% *******begin subsection***************
\subsubsection{\DbgSecSt{\StPart}{WorkCS}}
\index{Программный интерфейс ПЛК!Управление движением при фрезерной  обработке!Структура WorkCS}
\label{sec:WorkCS}

\begin{fHeader}
    Тип данных:            & \RightHandText{Структура WorkCS}\\
    Файл объявления:             & \RightHandText{include/motion/mill/motion.h} \\
\end{fHeader}

Структура определяет параметры системы координат заготовки (рабочей системы координат).

\begin{MyTableThreeColAllCntr}{Структура WorkCS}{tbl:WorkCS}{|m{0.3\linewidth}|m{0.25\linewidth}|m{0.45\linewidth}|}{Элемент}{Тип}{Описание}
\hline offset & \centering{\myreftosec{XYZ}} &   \\
\hline rot & \centering{\myreftosec{XYZ}} &   \\
\end{MyTableThreeColAllCntr}
% *******end subsection***************
%--------------------------------------------------------
% *******begin subsection***************
\subsection{\DbgSecSt{\StPart}{Функции}}

% *******begin subsection***************
\subsubsection{\DbgSecSt{\StPart}{initChannel}}
\index{Программный интерфейс ПЛК!Управление движением при фрезерной обработке!Функция initChannel}
\label{sec:initChannel}

\begin{pHeader}
    Синтаксис:      & \RightHandText{void initChannel(int chan, int cs, ProgramRuntime \&channel);}\\
   Аргумент(ы):    & \RightHandText{int chan ~-- номер канала,} \\    
       & \RightHandText{int cs ~-- номер координатной системы,} \\  
       & \RightHandText{\myreftosec{ProgramRuntime} \&channel ~-- параметры и данные канала} \\ 
%    Возвращаемое значение:       & \RightHandText{Нет} \\ 
    Файл объявления:             & \RightHandText{include/motion/mill/motion.h} \\       
\end{pHeader}

Функция инициализирует канал обработки параметрами по умолчанию. 

Является системной.
% *******end section*****************
%--------------------------------------------------------
% *******begin subsection***************
\subsubsection{\DbgSecSt{\StPart}{resetChannel}}
\index{Программный интерфейс ПЛК!Управление движением при фрезерной обработке!Функция resetChannel}
\label{sec:resetChannel}

\begin{pHeader}
    Синтаксис:      & \RightHandText{void resetChannel(int chan, int cs, ProgramRuntime \&channel);}\\
   Аргумент(ы):    & \RightHandText{int chan ~-- номер канала,} \\    
       & \RightHandText{int cs ~-- номер координатной системы,} \\  
       & \RightHandText{\myreftosec{ProgramRuntime} \&channel ~-- параметры и данные канала} \\ 
%    Возвращаемое значение:       & \RightHandText{Нет} \\ 
    Файл объявления:             & \RightHandText{include/motion/mill/motion.h} \\       
\end{pHeader}

Функция инициализирует канал обработки.

Является системной.
% *******end section*****************
%--------------------------------------------------------
% *******begin subsection***************
\subsubsection{\DbgSecSt{\StPart}{reapplyTransform}}
\index{Программный интерфейс ПЛК!Управление движением при фрезерной обработке!Функция reapplyTransform}
\label{sec:reapplyTransform}

\begin{pHeader}
    Синтаксис:      & \RightHandText{void reapplyTransform(int chan, int cs, ProgramRuntime \&channel);}\\
   Аргумент(ы):    & \RightHandText{int chan ~-- номер канала,} \\    
       & \RightHandText{int cs ~-- номер координатной системы,} \\  
       & \RightHandText{\myreftosec{ProgramRuntime} \&channel ~-- параметры и данные канала} \\ 
%    Возвращаемое значение:       & \RightHandText{Нет} \\ 
    Файл объявления:             & \RightHandText{include/motion/mill/motion.h} \\       
\end{pHeader}



Является системной.
% *******end section*****************
%--------------------------------------------------------
% *******begin subsection***************
\subsubsection{\DbgSecSt{\StPart}{csApplyTransform}}
\index{Программный интерфейс ПЛК!Управление движением при фрезерной обработке!Функция csApplyTransform}
\label{sec:csApplyTransform}

\begin{pHeader}
    Синтаксис:      & \RightHandText{void csApplyTransform(ProgramRuntime \&channel);}\\
   Аргумент(ы):   & \RightHandText{\myreftosec{ProgramRuntime} \&channel ~-- параметры и данные канала} \\ 
%    Возвращаемое значение:       & \RightHandText{Нет} \\ 
    Файл объявления:             & \RightHandText{include/motion/mill/motion.h} \\       
\end{pHeader}

Функция выполняет пространственные преобразования координатной системы.

Является системной.
% *******end section*****************
%--------------------------------------------------------
% *******begin subsection***************
\subsubsection{\DbgSecSt{\StPart}{csResetTransform}}
\index{Программный интерфейс ПЛК!Управление движением при фрезерной обработке!Функция csResetTransform}
\label{sec:csResetTransform}

\begin{pHeader}
    Синтаксис:      & \RightHandText{void csResetTransform(ProgramRuntime \&channel);}\\
   Аргумент(ы):   & \RightHandText{\myreftosec{ProgramRuntime} \&channel ~-- параметры и данные канала} \\ 
%    Возвращаемое значение:       & \RightHandText{Нет} \\ 
    Файл объявления:             & \RightHandText{include/motion/mill/motion.h} \\       
\end{pHeader}

Функция отменяет пространственные преобразования координатной системы.

Является системной.
% *******end section*****************
%--------------------------------------------------------
% *******begin subsection***************
\subsubsection{\DbgSecSt{\StPart}{csBase}}
\index{Программный интерфейс ПЛК!Управление движением при фрезерной обработке!Функция csBase}
\label{sec:csBase}

\begin{pHeader}
    Синтаксис:      & \RightHandText{void csBase(ProgramRuntime \&channel, const XYZ \&base);}\\
   Аргумент(ы):   & \RightHandText{\myreftosec{ProgramRuntime} \&channel ~-- параметры и данные канала, } \\ 
 & \RightHandText{const \myreftosec{XYZ} \&base ~-- базовое смещение системы координат} \\ 
  & \RightHandText{заготовки} \\
%    Возвращаемое значение:       & \RightHandText{Нет} \\ 
    Файл объявления:             & \RightHandText{include/motion/mill/motion.h} \\       
\end{pHeader}

Функция задаёт базовое смещение системы координат заготовки (G92).

Является системной.
% *******end section*****************
%--------------------------------------------------------
% *******begin subsection***************
\subsubsection{\DbgSecSt{\StPart}{csOffset}}
\index{Программный интерфейс ПЛК!Управление движением при фрезерной обработке!Функция csOffset}
\label{sec:csOffset}

\begin{pHeader}
    Синтаксис:      & \RightHandText{void csOffset(ProgramRuntime \&channel, const XYZ \&offset);}\\
   Аргумент(ы):   & \RightHandText{\myreftosec{ProgramRuntime} \&channel ~-- параметры и данные канала, } \\ 
   & \RightHandText{const \myreftosec{XYZ} \&offset ~-- смещение рабочей системы координат} \\ 
%    Возвращаемое значение:       & \RightHandText{Нет} \\ 
    Файл объявления:             & \RightHandText{include/motion/mill/motion.h} \\       
\end{pHeader}

Функция задаёт смещение рабочей системы координат (G54).

Является системной.
% *******end section*****************
%--------------------------------------------------------
% *******begin subsection***************
\subsubsection{\DbgSecSt{\StPart}{csScale}}
\index{Программный интерфейс ПЛК!Управление движением при фрезерной обработке!Функция csScale}
\label{sec:csScale}

\begin{pHeader}
    Синтаксис:      & \RightHandText{void csScale(ProgramRuntime \&channel, const XYZ \&center, }\\
    & \RightHandText{const XYZ \&scale);}\\
   Аргумент(ы):   & \RightHandText{\myreftosec{ProgramRuntime} \&channel ~-- параметры и данные канала, } \\ 
   & \RightHandText{const \myreftosec{XYZ} \&center ~-- координаты центра масштабирования,} \\ 
   & \RightHandText{const \myreftosec{XYZ} \&scale ~-- масштабы по осям} \\ 
%    Возвращаемое значение:       & \RightHandText{Нет} \\ 
    Файл объявления:             & \RightHandText{include/motion/mill/motion.h} \\       
\end{pHeader}

Функция задаёт масштабирование рабочей системы координат (G51).

Является системной.
% *******end section*****************
%--------------------------------------------------------
% *******begin subsection***************
\subsubsection{\DbgSecSt{\StPart}{csScaleReset}}
\index{Программный интерфейс ПЛК!Управление движением при фрезерной обработке!Функция csScaleReset}
\label{sec:csScaleReset}

\begin{pHeader}
    Синтаксис:      & \RightHandText{void csScaleReset(ProgramRuntime \&channel);}\\
   Аргумент(ы):   & \RightHandText{\myreftosec{ProgramRuntime} \&channel ~-- параметры и данные канала} \\ 
%    Возвращаемое значение:       & \RightHandText{Нет} \\ 
    Файл объявления:             & \RightHandText{include/motion/mill/motion.h} \\       
\end{pHeader}

Функция отменяет масштабирование рабочей системы координат.

Является системной.
% *******end section*****************
%--------------------------------------------------------
% *******begin subsection***************
\subsubsection{\DbgSecSt{\StPart}{csMirror}}
\index{Программный интерфейс ПЛК!Управление движением при фрезерной обработке!Функция csMirror}
\label{sec:csMirror}

\begin{pHeader}
    Синтаксис:      & \RightHandText{void csMirror(ProgramRuntime \&channel, const XYZ \&center, int mirror);}\\
   Аргумент(ы):   & \RightHandText{\myreftosec{ProgramRuntime} \&channel ~-- параметры и данные канала, } \\ 
   & \RightHandText{const \myreftosec{XYZ} \&center ~-- координаты центра зеркалирования,} \\ 
   & \RightHandText{int mirror ~-- оси для зеркалирования } \\ 
   & \RightHandText{(MIRROR\_X | MIRROR\_Y | MIRROR\_Z)} \\ 
%    Возвращаемое значение:       & \RightHandText{Нет} \\ 
    Файл объявления:             & \RightHandText{include/motion/mill/motion.h} \\       
\end{pHeader}

Функция задаёт зеркалирование рабочей системы координат (G51.1).

Является системной.
% *******end section*****************
%--------------------------------------------------------
% *******begin subsection***************
\subsubsection{\DbgSecSt{\StPart}{csMirrorReset}}
\index{Программный интерфейс ПЛК!Управление движением при фрезерной обработке!Функция csMirrorReset}
\label{sec:csMirrorReset}

\begin{pHeader}
    Синтаксис:      & \RightHandText{void csMirrorReset(ProgramRuntime \&channel);}\\
   Аргумент(ы):   & \RightHandText{\myreftosec{ProgramRuntime} \&channel ~-- параметры и данные канала} \\ 
%    Возвращаемое значение:       & \RightHandText{Нет} \\ 
    Файл объявления:             & \RightHandText{include/motion/mill/motion.h} \\       
\end{pHeader}

Функция отменяет зеркалирование рабочей системы координат.

Является системной.
% *******end section*****************
%--------------------------------------------------------
% *******begin subsection***************
\subsubsection{\DbgSecSt{\StPart}{csRotate}}
\index{Программный интерфейс ПЛК!Управление движением при фрезерной обработке!Функция csRotate}
\label{sec:csRotate}

\begin{pHeader}
    Синтаксис:      & \RightHandText{void csRotate(ProgramRuntime \&channel, const XYZ \&center, }\\
   & \RightHandText{const XYZ \&vector, double angle);}\\
   Аргумент(ы):   & \RightHandText{\myreftosec{ProgramRuntime} \&channel ~-- параметры и данные канала, } \\ 
   & \RightHandText{const \myreftosec{XYZ} \&center ~-- координаты центра поворота,} \\ 
   & \RightHandText{const \myreftosec{XYZ} \&vector ~-- вектор оси поворота,} \\
   & \RightHandText{double angle ~-- угол поворота} \\ 
%    Возвращаемое значение:       & \RightHandText{Нет} \\ 
    Файл объявления:             & \RightHandText{include/motion/mill/motion.h} \\       
\end{pHeader}

Функция задаёт поворот рабочей системы координат (G68).

Является системной.
% *******end section*****************
%--------------------------------------------------------
% *******begin subsection***************
\subsubsection{\DbgSecSt{\StPart}{csRotateReset}}
\index{Программный интерфейс ПЛК!Управление движением при фрезерной обработке!Функция csRotateReset}
\label{sec:csRotateReset}

\begin{pHeader}
    Синтаксис:      & \RightHandText{void csRotateReset(ProgramRuntime \&channel);}\\
   Аргумент(ы):   & \RightHandText{\myreftosec{ProgramRuntime} \&channel ~-- параметры и данные канала} \\ 
%    Возвращаемое значение:       & \RightHandText{Нет} \\ 
    Файл объявления:             & \RightHandText{include/motion/mill/motion.h} \\       
\end{pHeader}

Функция отменяет поворот рабочей системы координат.

Является системной.
% *******end section*****************
%--------------------------------------------------------
% *******begin subsection***************
\subsubsection{\DbgSecSt{\StPart}{csLocal}}
\index{Программный интерфейс ПЛК!Управление движением при фрезерной обработке!Функция csLocal}
\label{sec:csLocal}

\begin{pHeader}
    Синтаксис:      & \RightHandText{void csLocal(ProgramRuntime \&channel, const XYZ \&offset);}\\
   Аргумент(ы):   & \RightHandText{\myreftosec{ProgramRuntime} \&channel ~-- параметры и данные канала, } \\ 
   & \RightHandText{const \myreftosec{XYZ} \&offset ~-- смещение локалькой системы координат} \\ 
%    Возвращаемое значение:       & \RightHandText{Нет} \\ 
    Файл объявления:             & \RightHandText{include/motion/mill/motion.h} \\       
\end{pHeader}

Функция задаёт задание локальную систему координат (G52).

Является системной.
% *******end section*****************
%--------------------------------------------------------
% *******begin subsection***************
\subsubsection{\DbgSecSt{\StPart}{csFeedMode}}
\index{Программный интерфейс ПЛК!Управление движением при фрезерной обработке!Функция csFeedMode}
\label{sec:csFeedMode}

\begin{pHeader}
    Синтаксис:      & \RightHandText{void csFeedMode(ProgramRuntime \&channel, int mode);}\\
   Аргумент(ы):   & \RightHandText{\myreftosec{ProgramRuntime} \&channel ~-- параметры и данные канала, } \\ 
   & \RightHandText{int mode ~-- идентификатор режима управления подачей} \\ 
%    Возвращаемое значение:       & \RightHandText{Нет} \\ 
    Файл объявления:             & \RightHandText{include/motion/mill/motion.h} \\       
\end{pHeader}

Функция задаёт режим подачи, принимая в качестве аргумента идентификатор из перечисления \myreftosec{FeedMode}.

Является системной.
% *******end section*****************
%--------------------------------------------------------
% *******begin subsection***************
\subsubsection{\DbgSecSt{\StPart}{csFeedOverride}}
\index{Программный интерфейс ПЛК!Управление движением при фрезерной обработке!Функция csFeedOverride}
\label{sec:csFeedOverride}

\begin{pHeader}
    Синтаксис:      & \RightHandText{void csFeedOverride(ProgramRuntime \&channel, double value);}\\
   Аргумент(ы):   & \RightHandText{\myreftosec{ProgramRuntime} \&channel ~-- параметры и данные канала, } \\ 
   & \RightHandText{double value ~-- величина коррекции подачи} \\ 
%    Возвращаемое значение:       & \RightHandText{Нет} \\ 
    Файл объявления:             & \RightHandText{include/motion/mill/motion.h} \\       
\end{pHeader}

Функция задаёт корректор подачи, принимая в качестве аргумента величину коррекции подачи (1.0 = 100\%).

Является системной.
% *******end section*****************
%--------------------------------------------------------
% *******begin subsection***************
\subsubsection{\DbgSecSt{\StPart}{csFeedOverrideMode}}
\index{Программный интерфейс ПЛК!Управление движением при фрезерной обработке!Функция csFeedOverrideMode}
\label{sec:csFeedOverrideMode}

\begin{pHeader}
    Синтаксис:      & \RightHandText{void csFeedOverrideMode(ProgramRuntime \&channel, int mode);}\\
   Аргумент(ы):   & \RightHandText{\myreftosec{ProgramRuntime} \&channel ~-- параметры и данные канала, } \\ 
   & \RightHandText{int mode ~-- идентификатор режима коррекции величины подачи} \\ 
%    Возвращаемое значение:       & \RightHandText{Нет} \\ 
    Файл объявления:             & \RightHandText{include/motion/mill/motion.h} \\       
\end{pHeader}

Функция задаёт режим коррекции величины подачи, принимая в качестве аргумента идентификатор из перечисления \myreftosec{FeedOverMode}.

Является системной.
% *******end section*****************
%--------------------------------------------------------
% *******begin subsection***************
\subsubsection{\DbgSecSt{\StPart}{csSpindleMode}}
\index{Программный интерфейс ПЛК!Управление движением при фрезерной обработке!Функция csSpindleMode}
\label{sec:csSpindleMode}

\begin{pHeader}
    Синтаксис:      & \RightHandText{void csSpindleMode(ProgramRuntime \&channel, int mode);}\\
   Аргумент(ы):   & \RightHandText{\myreftosec{ProgramRuntime} \&channel ~-- параметры и данные канала, } \\ 
   & \RightHandText{int mode ~-- идентификатор режима управления скоростью  } \\ 
   & \RightHandText{ шпинделя} \\   
%    Возвращаемое значение:       & \RightHandText{Нет} \\ 
    Файл объявления:             & \RightHandText{include/motion/mill/motion.h} \\       
\end{pHeader}

Функция задаёт режим подачи, принимая в качестве аргумента идентификатор из перечисления \myreftosec{SpindleMode}.

Является системной.
% *******end section*****************
%--------------------------------------------------------
% *******begin subsection***************
\subsubsection{\DbgSecSt{\StPart}{csSpindleOverride}}
\index{Программный интерфейс ПЛК!Управление движением при фрезерной обработке!Функция csSpindleOverride}
\label{sec:csSpindleOverride}

\begin{pHeader}
    Синтаксис:      & \RightHandText{void csSpindleOverride(ProgramRuntime \&channel, double value);}\\
   Аргумент(ы):   & \RightHandText{\myreftosec{ProgramRuntime} \&channel ~-- параметры и данные канала, } \\ 
   & \RightHandText{double value ~-- величина коррекции скорости шпинделя} \\ 
%    Возвращаемое значение:       & \RightHandText{Нет} \\ 
    Файл объявления:             & \RightHandText{include/motion/mill/motion.h} \\       
\end{pHeader}

Функция задаёт корректор подачи, принимая в качестве аргумента величину коррекции скорости шпинделя (1.0 = 100\%).

Является системной.
% *******end section*****************
%--------------------------------------------------------
% *******begin subsection***************
\subsubsection{\DbgSecSt{\StPart}{csSpindleOverrideMode}}
\index{Программный интерфейс ПЛК!Управление движением при фрезерной обработке!Функция csSpindleOverrideMode}
\label{sec:csSpindleOverrideMode}

\begin{pHeader}
    Синтаксис:      & \RightHandText{void csSpindleOverrideMode(ProgramRuntime \&channel, int mode);}\\
   Аргумент(ы):   & \RightHandText{\myreftosec{ProgramRuntime} \&channel ~-- параметры и данные канала, } \\ 
   & \RightHandText{int mode ~-- идентификатор режима коррекции скорости шпинделя} \\ 
%    Возвращаемое значение:       & \RightHandText{Нет} \\ 
    Файл объявления:             & \RightHandText{include/motion/mill/motion.h} \\       
\end{pHeader}

Функция задаёт режим коррекции скорости шпинделя, принимая в качестве аргумента идентификатор из перечисления \myreftosec{SpindleOverMode}.

Является системной.
% *******end section*****************
%--------------------------------------------------------
% *******begin subsection***************
\subsubsection{\DbgSecSt{\StPart}{csBlendMode}}
\index{Программный интерфейс ПЛК!Управление движением при фрезерной обработке!Функция csBlendMode}
\label{sec:csBlendMode}

\begin{pHeader}
    Синтаксис:      & \RightHandText{void csBlendMode(ProgramRuntime \&channel, int mode);}\\
   Аргумент(ы):   & \RightHandText{\myreftosec{ProgramRuntime} \&channel ~-- параметры и данные канала, } \\ 
   & \RightHandText{int mode ~-- идентификатор режима сопряжения кадров} \\ 
%    Возвращаемое значение:       & \RightHandText{Нет} \\ 
    Файл объявления:             & \RightHandText{include/motion/mill/motion.h} \\       
\end{pHeader}

Функция задаёт режим сопряжения кадров, принимая в качестве аргумента идентификатор из перечисления \myreftosec{BlendMode}.

Является системной.
% *******end section*****************
%--------------------------------------------------------
% *******begin subsection***************
\subsubsection{\DbgSecSt{\StPart}{csMoveMode}}
\index{Программный интерфейс ПЛК!Управление движением при фрезерной обработке!Функция csMoveMode}
\label{sec:csMoveMode}

\begin{pHeader}
    Синтаксис:      & \RightHandText{void csMoveMode(ProgramRuntime \&channel, int mode);}\\
   Аргумент(ы):   & \RightHandText{\myreftosec{ProgramRuntime} \&channel ~-- параметры и данные канала, } \\ 
   & \RightHandText{int mode ~-- идентификатор режима движения} \\ 
%    Возвращаемое значение:       & \RightHandText{Нет} \\ 
    Файл объявления:             & \RightHandText{include/motion/mill/motion.h} \\       
\end{pHeader}

Функция задаёт режим движения, принимая в качестве аргумента идентификатор из перечисления \myreftosec{MoveMode}.

Является системной.
% *******end section*****************
%--------------------------------------------------------
% *******begin subsection***************
\subsubsection{\DbgSecSt{\StPart}{csPosMode}}
\index{Программный интерфейс ПЛК!Управление движением при фрезерной обработке!Функция csPosMode}
\label{sec:csPosMode}

\begin{pHeader}
    Синтаксис:      & \RightHandText{void csPosMode(ProgramRuntime \&channel, int mode);}\\
   Аргумент(ы):   & \RightHandText{\myreftosec{ProgramRuntime} \&channel ~-- параметры и данные канала, } \\ 
   & \RightHandText{int mode ~-- идентификатор режима позиционирования} \\ 
%    Возвращаемое значение:       & \RightHandText{Нет} \\ 
    Файл объявления:             & \RightHandText{include/motion/mill/motion.h} \\       
\end{pHeader}

Функция задаёт режим позиционирования, принимая в качестве аргумента идентификатор из перечисления \myreftosec{PositionMode}.

Является системной.
% *******end section*****************
%--------------------------------------------------------
% *******begin subsection***************
\subsubsection{\DbgSecSt{\StPart}{csSetUnits}}
\index{Программный интерфейс ПЛК!Управление движением при фрезерной обработке!Функция csSetUnits}
\label{sec:csSetUnits}

\begin{pHeader}
    Синтаксис:      & \RightHandText{void csSetUnits(ProgramRuntime \&channel, double units);}\\
   Аргумент(ы):   & \RightHandText{\myreftosec{ProgramRuntime} \&channel ~-- параметры и данные канала, } \\ 
   & \RightHandText{double units ~-- значение единиц измерения в долях мм} \\ 
%    Возвращаемое значение:       & \RightHandText{Нет} \\ 
    Файл объявления:             & \RightHandText{include/motion/mill/motion.h} \\       
\end{pHeader}

Функция задаёт единицы измерения перемещений, принимая в качестве аргумента значение единиц измерения в долях мм (1.0 = 1.0 мм).

Является системной.
% *******end section*****************
%--------------------------------------------------------
% *******begin subsection***************
\subsubsection{\DbgSecSt{\StPart}{csToolSelect}}
\index{Программный интерфейс ПЛК!Управление движением при фрезерной обработке!Функция csToolSelect}
\label{sec:csToolSelect}

\begin{pHeader}
    Синтаксис:      & \RightHandText{void csToolSelect(ProgramRuntime \&channel, Tool \&tool);}\\
   Аргумент(ы):   & \RightHandText{\myreftosec{ProgramRuntime} \&channel ~-- параметры и данные канала, } \\ 
   & \RightHandText{\myreftosec{Tool} \&tool ~-- параметры выбранного инструмента} \\ 
%    Возвращаемое значение:       & \RightHandText{Нет} \\ 
    Файл объявления:             & \RightHandText{include/motion/mill/motion.h} \\       
\end{pHeader}

Функция задаёт инструмент, параметры которого являются аргументом функции.

Является системной.
% *******end section*****************
%--------------------------------------------------------
% *******begin subsection***************
\subsubsection{\DbgSecSt{\StPart}{csToolReset}}
\index{Программный интерфейс ПЛК!Управление движением при фрезерной обработке!Функция csToolReset}
\label{sec:csToolReset}

\begin{pHeader}
    Синтаксис:      & \RightHandText{void csToolReset(ProgramRuntime \&channel);}\\
   Аргумент(ы):   & \RightHandText{\myreftosec{ProgramRuntime} \&channel ~-- параметры и данные канала} \\ 
%    Возвращаемое значение:       & \RightHandText{Нет} \\ 
    Файл объявления:             & \RightHandText{include/motion/mill/motion.h} \\       
\end{pHeader}

Функция отменяет выбор инструмента.

Является системной.
% *******end section*****************
%--------------------------------------------------------
% *******begin subsection***************
\subsubsection{\DbgSecSt{\StPart}{csToolWear}}
\index{Программный интерфейс ПЛК!Управление движением при фрезерной обработке!Функция csToolWear}
\label{sec:csToolWear}

\begin{pHeader}
    Синтаксис:      & \RightHandText{void csToolWear(ProgramRuntime \&channel, Tool \&tool);}\\
   Аргумент(ы):   & \RightHandText{\myreftosec{ProgramRuntime} \&channel ~-- параметры и данные канала, } \\ 
   & \RightHandText{\myreftosec{Tool} \&tool ~-- параметры выбранного инструмента} \\ 
%    Возвращаемое значение:       & \RightHandText{Нет} \\ 
    Файл объявления:             & \RightHandText{include/motion/mill/motion.h} \\       
\end{pHeader}



Является системной.
% *******end section*****************
%--------------------------------------------------------
% *******begin subsection***************
\subsubsection{\DbgSecSt{\StPart}{csToolWearReset}}
\index{Программный интерфейс ПЛК!Управление движением при фрезерной обработке!Функция csToolWearReset}
\label{sec:csToolWearReset}

\begin{pHeader}
    Синтаксис:      & \RightHandText{void csToolWearReset(ProgramRuntime \&channel);}\\
   Аргумент(ы):   & \RightHandText{\myreftosec{ProgramRuntime} \&channel ~-- параметры и данные канала} \\ 
%    Возвращаемое значение:       & \RightHandText{Нет} \\ 
    Файл объявления:             & \RightHandText{include/motion/mill/motion.h} \\       
\end{pHeader}



Является системной.
% *******end section*****************
%--------------------------------------------------------
% *******begin subsection***************
\subsubsection{\DbgSecSt{\StPart}{csToolXY}}
\index{Программный интерфейс ПЛК!Управление движением при фрезерной обработке!Функция csToolXY}
\label{sec:csToolXY}

\begin{pHeader}
    Синтаксис:      & \RightHandText{void csToolXY(ProgramRuntime \&channel, double X, double Y);}\\
   Аргумент(ы):   & \RightHandText{\myreftosec{ProgramRuntime} \&channel ~-- параметры и данные канала,} \\ 
   & \RightHandText{double X ~--  ,} \\    
   & \RightHandText{double Y ~--  } \\       
%    Возвращаемое значение:       & \RightHandText{Нет} \\ 
    Файл объявления:             & \RightHandText{include/motion/mill/motion.h} \\       
\end{pHeader}



Является системной.
% *******end section*****************
%--------------------------------------------------------
% *******begin subsection***************
\subsubsection{\DbgSecSt{\StPart}{csToolX}}
\index{Программный интерфейс ПЛК!Управление движением при фрезерной обработке!Функция csToolX}
\label{sec:csToolX}

\begin{pHeader}
    Синтаксис:      & \RightHandText{void csToolX(ProgramRuntime \&channel, double X);}\\
   Аргумент(ы):   & \RightHandText{\myreftosec{ProgramRuntime} \&channel ~-- параметры и данные канала,} \\ 
   & \RightHandText{double X ~--  } \\    
%    Возвращаемое значение:       & \RightHandText{Нет} \\ 
    Файл объявления:             & \RightHandText{include/motion/mill/motion.h} \\       
\end{pHeader}



Является системной.
% *******end section*****************
%--------------------------------------------------------
% *******begin subsection***************
\subsubsection{\DbgSecSt{\StPart}{csToolY}}
\index{Программный интерфейс ПЛК!Управление движением при фрезерной обработке!Функция csToolY}
\label{sec:csToolY}

\begin{pHeader}
    Синтаксис:      & \RightHandText{void csToolY(ProgramRuntime \&channel, double Y);}\\
   Аргумент(ы):   & \RightHandText{\myreftosec{ProgramRuntime} \&channel ~-- параметры и данные канала,} \\ 
   & \RightHandText{double Y ~--  } \\    
%    Возвращаемое значение:       & \RightHandText{Нет} \\ 
    Файл объявления:             & \RightHandText{include/motion/mill/motion.h} \\       
\end{pHeader}



Является системной.
% *******end section*****************
%--------------------------------------------------------
% *******begin subsection***************
\subsubsection{\DbgSecSt{\StPart}{csToolLength}}
\index{Программный интерфейс ПЛК!Управление движением при фрезерной обработке!Функция csToolLength}
\label{sec:csToolLength}

\begin{pHeader}
    Синтаксис:      & \RightHandText{void csToolLength(ProgramRuntime \&channel, double L);}\\
   Аргумент(ы):   & \RightHandText{\myreftosec{ProgramRuntime} \&channel ~-- параметры и данные канала,} \\ 
   & \RightHandText{double L ~--  длина инструмента} \\    
%    Возвращаемое значение:       & \RightHandText{Нет} \\ 
    Файл объявления:             & \RightHandText{include/motion/mill/motion.h} \\       
\end{pHeader}

Функция задаёт длину инструмента, значение которой является аргументом функции.

Является системной.
% *******end section*****************
%--------------------------------------------------------
% *******begin subsection***************
\subsubsection{\DbgSecSt{\StPart}{cycleReset}}
\index{Программный интерфейс ПЛК!Управление движением при фрезерной обработке!Функция cycleReset}
\label{sec:cycleReset}

\begin{pHeader}
    Синтаксис:      & \RightHandText{void cycleReset(ProgramRuntime \&channel);}\\
   Аргумент(ы):   & \RightHandText{\myreftosec{ProgramRuntime} \&channel ~-- параметры и данные канала} \\    
%    Возвращаемое значение:       & \RightHandText{Нет} \\ 
    Файл объявления:             & \RightHandText{include/motion/mill/motion.h} \\       
\end{pHeader}


Является системной.
% *******end section*****************
%--------------------------------------------------------
% *******begin subsection***************
\subsubsection{\DbgSecSt{\StPart}{stepBlock}}
\index{Программный интерфейс ПЛК!Управление движением при фрезерной обработке!Функция stepBlock}
\label{sec:stepBlock}

\begin{pHeader}
    Синтаксис:      & \RightHandText{void stepBlock(ProgramRuntime \&channel);}\\
   Аргумент(ы):   & \RightHandText{\myreftosec{ProgramRuntime} \&channel ~-- параметры и данные канала} \\    
%    Возвращаемое значение:       & \RightHandText{Нет} \\ 
    Файл объявления:             & \RightHandText{include/motion/mill/motion.h} \\       
\end{pHeader}


Является системной.
% *******end section*****************
%--------------------------------------------------------
% *******begin subsection***************
\subsubsection{\DbgSecSt{\StPart}{csSetGCode}}
\index{Программный интерфейс ПЛК!Управление движением при фрезерной обработке!Функция csSetGCode}
\label{sec:csSetGCode}

\begin{pHeader}
    Синтаксис:      & \RightHandText{void csSetGCode(ProgramRuntime \&channel, int group, int value);}\\
   Аргумент(ы):   & \RightHandText{\myreftosec{ProgramRuntime} \&channel ~-- параметры и данные канала,} \\    длина инструмента
   & \RightHandText{int group ~--  идентификатор группы,} \\    
   & \RightHandText{int value ~--  номер G-кода в группе} \\          
%    Возвращаемое значение:       & \RightHandText{Нет} \\ 
    Файл объявления:             & \RightHandText{include/motion/mill/motion.h} \\       
\end{pHeader}


Является системной.
% *******end section*****************
%--------------------------------------------------------
% *******begin subsection***************
\subsubsection{\DbgSecSt{\StPart}{csSetMCode}}
\index{Программный интерфейс ПЛК!Управление движением при фрезерной обработке!Функция csSetMCode}
\label{sec:csSetMCode}

\begin{pHeader}
    Синтаксис:      & \RightHandText{void csSetMCode(ProgramRuntime \&channel, int group, int value);}\\
   Аргумент(ы):   & \RightHandText{\myreftosec{ProgramRuntime} \&channel ~-- параметры и данные канала,} \\    длина инструмента
   & \RightHandText{int group ~--  идентификатор группы,} \\    
   & \RightHandText{int value ~--  номер G-кода в группе} \\          
%    Возвращаемое значение:       & \RightHandText{Нет} \\ 
    Файл объявления:             & \RightHandText{include/motion/mill/motion.h} \\       
\end{pHeader}


Является системной.
% *******end section*****************

%--------------------------------------------------------
\index{Программный интерфейс ПЛК|)}

\clearpage
