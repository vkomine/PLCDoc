%--------------------------------------------------------
% *******begin section***************
\section{\DbgSecSt{\StPart}{Управление программами ПЛК}}
%--------------------------------------------------------

% *******begin subsection***************
\subsection{\DbgSecSt{\StPart}{Функции}}

%--------------------------------------------------------
% *******begin subsection***************
\subsubsection{\DbgSecSt{\StPart}{enablePLC}}
\index{Программный интерфейс ПЛК!Управление программами ПЛК!Функция enablePLC}
\label{sec:enablePLC}

\begin{pHeader}
    Синтаксис:      & \RightHandText{int enablePLC(int plc);}\\
    Аргумент(ы):    & \RightHandText {int plc ~-- номер программы ПЛК} \\  
%    Возвращаемое значение:       & \RightHandText{Нет} \\
    Файл объявления:             & \RightHandText{sys/sys.h} \\      
\end{pHeader}

Функция вызывает выполнение программы ПЛК, номер которой (от 0 до 31) определяется аргументом функции. Выполнение стартует с начала программы. \killoverfullbefore

Возвращаемое значение равно 0 при отсутствии ошибок и отлично от 0 в противном случае. \killoverfullbefore

Является системной.
% *******end subsection*****************
%--------------------------------------------------------
% *******begin subsection***************
\subsubsection{\DbgSecSt{\StPart}{enablePLCs}}
\index{Программный интерфейс ПЛК!Управление программами ПЛК!Функция enablePLCs}
\label{sec:enablePLCs}

\begin{pHeader}
    Синтаксис:      & \RightHandText{int enablePLCs(int plc);}\\
    Аргумент(ы):    & \RightHandText {int plc ~-- номера программ ПЛК} \\  
%    Возвращаемое значение:       & \RightHandText{Нет} \\
    Файл объявления:             & \RightHandText{sys/sys.h} \\      
\end{pHeader}

Функция вызывает выполнение программ ПЛК, номера которых (от 0 до 31) определяются аргументом функции. Аргумент функции – битовое поле, в котором номера установленных битов (значения которых равны 1) соответствуют номерам программ ПЛК. Выполнение стартует с начала программы. \killoverfullbefore

Возвращаемое значение равно 0 при отсутствии ошибок и отлично от 0 в противном случае. \killoverfullbefore

Является системной.
% *******end subsection*****************
%--------------------------------------------------------
% *******begin subsection***************
\subsubsection{\DbgSecSt{\StPart}{pausePLC}}
\index{Программный интерфейс ПЛК!Управление программами ПЛК!Функция pausePLC}
\label{sec:pausePLC}

\begin{pHeader}
    Синтаксис:      & \RightHandText{int pausePLC(int plc);}\\
    Аргумент(ы):    & \RightHandText {int plc ~-- номер программы ПЛК} \\  
%    Возвращаемое значение:       & \RightHandText{Нет} \\
    Файл объявления:             & \RightHandText{sys/sys.h} \\      
\end{pHeader}

Функция вызывает временный останов программы ПЛК, номер которой (от 0 до 31) определяется аргументом функции. \killoverfullbefore

Возвращаемое значение равно 0 при отсутствии ошибок и отлично от 0 в противном случае. \killoverfullbefore

Является системной.
% *******end subsection*****************
%--------------------------------------------------------
% *******begin subsection***************
\subsubsection{\DbgSecSt{\StPart}{pausePLCs}}
\index{Программный интерфейс ПЛК!Управление программами ПЛК!Функция pausePLCs}
\label{sec:pausePLCs}

\begin{pHeader}
    Синтаксис:      & \RightHandText{int pausePLCs(int plc);}\\
    Аргумент(ы):    & \RightHandText {int plc ~-- номера программ ПЛК} \\  
%    Возвращаемое значение:       & \RightHandText{Нет} \\
    Файл объявления:             & \RightHandText{sys/sys.h} \\      
\end{pHeader}

Функция вызывает временный останов программ ПЛК, номера которых (от 0 до 31) определяются аргументом функции. Аргумент функции – битовое поле, в котором номера установленных битов (значения которых равны 1) соответствуют номерам программ ПЛК.\killoverfullbefore

 Возвращаемое значение равно 0 при отсутствии ошибок и отлично от 0 в противном случае. \killoverfullbefore

Является системной.
% *******end subsection*****************
%--------------------------------------------------------
% *******begin subsection***************
\subsubsection{\DbgSecSt{\StPart}{resumePLC}}
\index{Программный интерфейс ПЛК!Управление программами ПЛК!Функция resumePLC}
\label{sec:resumePLC}

\begin{pHeader}
    Синтаксис:      & \RightHandText{int resumePLC(int plc);}\\
    Аргумент(ы):    & \RightHandText {int plc ~-- номер программы ПЛК} \\  
%    Возвращаемое значение:       & \RightHandText{Нет} \\
    Файл объявления:             & \RightHandText{sys/sys.h} \\      
\end{pHeader}

Функция вызывает возобновление выполнения программы ПЛК, номер которой (от 0 до 31) определяется аргументом функции. \killoverfullbefore

Возвращаемое значение равно 0 при отсутствии ошибок и отлично от 0 в противном случае. \killoverfullbefore

Является системной.
% *******end subsection*****************

%--------------------------------------------------------
% *******begin subsection***************
\subsubsection{\DbgSecSt{\StPart}{resumePLCs}}
\index{Программный интерфейс ПЛК!Управление программами ПЛК!Функция resumePLCs}
\label{sec:resumePLCs}

\begin{pHeader}
    Синтаксис:      & \RightHandText{int resumePLCs(int plc);}\\
    Аргумент(ы):    & \RightHandText {int plc ~-- номера программ ПЛК} \\  
%    Возвращаемое значение:       & \RightHandText{Нет} \\
    Файл объявления:             & \RightHandText{sys/sys.h} \\      
\end{pHeader}

Функция вызывает возобновление выполнения программ ПЛК, номера которых (от 0 до 31) определяются аргументом функции. Аргумент функции – битовое поле, в котором номера установленных битов (значения которых равны 1) соответствуют номерам программ ПЛК.\killoverfullbefore

 Возвращаемое значение равно 0 при отсутствии ошибок и отлично от 0 в противном случае. \killoverfullbefore

Является системной.
% *******end subsection*****************
%--------------------------------------------------------
% *******begin subsection***************
\subsubsection{\DbgSecSt{\StPart}{disablePLC}}
\index{Программный интерфейс ПЛК!Управление программами ПЛК!Функция disablePLC}
\label{sec:disablePLC}

\begin{pHeader}
    Синтаксис:      & \RightHandText{int disablePLC(int plc);}\\
    Аргумент(ы):    & \RightHandText {int plc ~-- номер программы ПЛК} \\  
%    Возвращаемое значение:       & \RightHandText{Нет} \\
    Файл объявления:             & \RightHandText{sys/sys.h} \\      
\end{pHeader}

Функция вызывает отмену выполнения программы ПЛК, номер которой (от 0 до 31) определяется аргументом функции. Возвращаемое значение равно 0 при отсутствии ошибок и отлично от 0 в противном случае. \killoverfullbefore

Является системной.
% *******end subsection*****************
%--------------------------------------------------------
% *******begin subsection***************
\subsubsection{\DbgSecSt{\StPart}{disablePLCs}}
\index{Программный интерфейс ПЛК!Управление программами ПЛК!Функция disablePLCs}
\label{sec:disablePLCs}

\begin{pHeader}
    Синтаксис:      & \RightHandText{int disablePLCs(int plc);}\\
    Аргумент(ы):    & \RightHandText {int plc ~-- номера программ ПЛК} \\  
%    Возвращаемое значение:       & \RightHandText{Нет} \\
    Файл объявления:             & \RightHandText{sys/sys.h} \\      
\end{pHeader}

Функция вызывает отмену выполнения программ ПЛК, номера которых (от 0 до 31) определяются аргументом функции. Аргумент функции – битовое поле, в котором номера установленных битов (значения которых равны 1) соответствуют номерам программ ПЛК.\killoverfullbefore

 Возвращаемое значение равно 0 при отсутствии ошибок и отлично от 0 в противном случае. \killoverfullbefore

Является системной.
% *******end subsection*****************
%--------------------------------------------------------
% *******begin subsection***************
\subsubsection{\DbgSecSt{\StPart}{stepPLC}}
\index{Программный интерфейс ПЛК!Управление программами ПЛК!Функция stepPLC}
\label{sec:stepPLC}

\begin{pHeader}
    Синтаксис:      & \RightHandText{int stepPLC(int plc);}\\
    Аргумент(ы):    & \RightHandText {int plc ~-- номер программы ПЛК} \\  
%    Возвращаемое значение:       & \RightHandText{Нет} \\
    Файл объявления:             & \RightHandText{sys/sys.h} \\      
\end{pHeader}

Функция вызывает пошаговое выполнение программы ПЛК, номер которой (от 0 до 31) определяется аргументом функции. \killoverfullbefore

Возвращаемое значение равно 0 при отсутствии ошибок и отлично от 0 в противном случае. \killoverfullbefore

Является системной.
% *******end subsection*****************

%--------------------------------------------------------
% *******begin subsection***************
\subsubsection{\DbgSecSt{\StPart}{stepPLCs}}
\index{Программный интерфейс ПЛК!Управление программами ПЛК!Функция stepPLCs}
\label{sec:stepPLCs}

\begin{pHeader}
    Синтаксис:      & \RightHandText{int stepPLCs(int plc);}\\
    Аргумент(ы):    & \RightHandText {int plc ~-- номера программ ПЛК} \\  
%    Возвращаемое значение:       & \RightHandText{Нет} \\
    Файл объявления:             & \RightHandText{sys/sys.h} \\      
\end{pHeader}

Функция вызывает пошаговое выполнение программ ПЛК, номера которых (от 0 до 31) определяются аргументом функции. Аргумент функции – битовое поле, в котором номера установленных битов (значения которых равны 1) соответствуют номерам программ ПЛК.\killoverfullbefore

 Возвращаемое значение равно 0 при отсутствии ошибок и отлично от 0 в противном случае. \killoverfullbefore

Является системной.
% *******end subsection*****************
%--------------------------------------------------------