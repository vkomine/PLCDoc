%--------------------------------------------------------
% *******begin section***************
\section{\DbgSecSt{\StPart}{Переменные и буфера}}

% *******begin subsection***************
\subsection{\DbgSecSt{\StPart}{Функции}}
%--------------------------------------------------------
% *******begin subsection***************
\subsubsection{\DbgSecSt{\StPart}{detectEdgeRise}}
\index{Программный интерфейс ПЛК!Переменные и буфера!Функция detectEdgeRise}
\label{sec:detectEdgeRise}

\begin{pHeader}
    Синтаксис:      & \RightHandText{int detectEdgeRise(int \&detector, int input);}\\
    Аргумент(ы):    & \RightHandText{int \&detector ~-- сохранённое входное значение} \\    
    & \RightHandText{int input ~--  текущее входное значение} \\ 
    Файл объявления:             & \RightHandText{include/func/misc.h} \\       
\end{pHeader}

Функция служит для детектирования изменения с 0 на 1 (детектирования фронта) входной величины. \killoverfullbefore

Функция возвращает 0, если входное значение не изменилось и осталось равным 0, и 1, если входное значение стало отличным от 0. \killoverfullbefore

Является системной.
% *******end subsection*****************
%-------------------------------------------------------------------
% *******begin subsection***************
\subsubsection{\DbgSecSt{\StPart}{detectEdgeFall}}
\index{Программный интерфейс ПЛК!Переменные и буфера!Функция detectEdgeFall}
\label{sec:detectEdgeFall}

\begin{pHeader}
    Синтаксис:      & \RightHandText{int detectEdgeFall(int \&detector, int input);}\\
    Аргумент(ы):    & \RightHandText{int \&detector ~-- сохранённое входное значение} \\    
    & \RightHandText{int input ~--  текущее входное значение} \\ 
    Файл объявления:             & \RightHandText{include/func/misc.h} \\       
\end{pHeader}

Функция служит для детектирования изменения с 1 на 0 (детектирования спада) входной величины. \killoverfullbefore

Функция возвращает 0, если входное значение не изменилось и осталось равным 1, и 1, если входное значение стало равным 0. \killoverfullbefore

Является системной.
% *******end subsection*****************
%--------------------------------------------------------
% *******begin subsection***************
\subsubsection{\DbgSecSt{\StPart}{syncset}}
\index{Программный интерфейс ПЛК!Переменные и буфера!Функция syncset}
\label{sec:syncset}

\begin{pHeader}
    Синтаксис:      & \RightHandText{void syncset(void *ptr, int value);}\\
   Аргумент(ы):    & \RightHandText {void *ptr ~-- указатель на переменную} \\  
  & \RightHandText{int value ~-- присваиваемое значение} \\
    Файл объявления:             & \RightHandText{sys/sys.h} \\      
\end{pHeader}

Функция выполняет синхронное присваивание значения типа \texttt{int}, адресуемой указателем переменной. Синхронное присваивание осуществляется в момент начала следующего перемещения.

Первый аргумент функции \texttt{*ptr} ~-- указатель, ссылающийся на переменную. Второй аргумент \texttt{value} ~-– присваиваемое значение.\killoverfullbefore

Является системной.
% *******end subsection*****************
%--------------------------------------------------------
% *******begin subsection***************
\subsubsection{\DbgSecSt{\StPart}{syncsetf}}
\index{Программный интерфейс ПЛК!Переменные и буфера!Функция syncsetf}
\label{sec:syncsetf}

\begin{pHeader}
    Синтаксис:      & \RightHandText{void syncsetf(void *ptr, float value);}\\
   Аргумент(ы):    & \RightHandText {void *ptr ~-- указатель на переменную} \\  
  & \RightHandText{float value ~-- присваиваемое значение} \\
    Файл объявления:             & \RightHandText{sys/sys.h} \\      
\end{pHeader}

Синхронное присваивание значения типа \texttt{float}, адресуемой указателем переменной. Синхронное присваивание осуществляется в момент начала следующего перемещения.

Первый аргумент функции \texttt{*ptr} ~-- указатель, ссылающийся на переменную. Второй аргумент \texttt{value} ~-– присваиваемое значение.\killoverfullbefore

Является системной.
% *******end subsection*****************
%--------------------------------------------------------
% *******begin subsection***************
\subsubsection{\DbgSecSt{\StPart}{syncsetd}}
\index{Программный интерфейс ПЛК!Переменные и буфера!Функция syncsetd}
\label{sec:syncsetd}

\begin{pHeader}
    Синтаксис:      & \RightHandText{void syncsetd(void *ptr, double value);}\\
   Аргумент(ы):    & \RightHandText {void *ptr ~-- указатель на переменную} \\  
  & \RightHandText{double value ~-- присваиваемое значение} \\
    Файл объявления:             & \RightHandText{sys/sys.h} \\      
\end{pHeader}

Синхронное присваивание значения типа \texttt{double}, адресуемой указателем переменной. Синхронное присваивание осуществляется в момент начала следующего перемещения.

Первый аргумент функции \texttt{*ptr} ~-- указатель, ссылающийся на переменную. Второй аргумент \texttt{value} ~-– присваиваемое значение.\killoverfullbefore

Является системной.
% *******end subsection*****************
%--------------------------------------------------------
% *******begin subsection***************
\subsubsection{\DbgSecSt{\StPart}{usave}}
\index{Программный интерфейс ПЛК!Переменные и буфера!Функция usave}
\label{sec:usave}

\begin{pHeader}
    Синтаксис:      & \RightHandText{void usave(void *ptr);}\\
   Аргумент(ы):    & \RightHandText {void *ptr ~-- указатель на переменную} \\  
%    Возвращаемое значение:       & \RightHandText{Нет} \\
    Файл объявления:             & \RightHandText{sys/sys.h} \\      
\end{pHeader}

Функция выполняет сохранение пользовательской переменной, на которую ссылается указатель. Пользовательская переменная должна быть объявлена посредством макроса \texttt{\#define USER\_SAVE(name)}. \killoverfullbefore

Является системной.
% *******end subsection*****************
%--------------------------------------------------------
% *******begin subsection***************
\subsubsection{\DbgSecSt{\StPart}{clearGather}}
\index{Программный интерфейс ПЛК!Переменные и буфера!Функция clearGather}
\label{sec:clearGather}

\begin{pHeader}
    Синтаксис:      & \RightHandText{int clearGather();}\\
   Аргумент(ы):    & \RightHandText {нет} \\  
%    Возвращаемое значение:       & \RightHandText{Нет} \\
    Файл объявления:             & \RightHandText{sys/sys.h} \\
\end{pHeader}

Функция очищает буфер данных сервопрерываний. \killoverfullbefore

Является системной.
% *******end subsection*****************
%--------------------------------------------------------
% *******begin subsection***************
\subsubsection{\DbgSecSt{\StPart}{clearPhaseGather}}
\index{Программный интерфейс ПЛК!Переменные и буфера!Функция clearPhaseGather}
\label{sec:clearPhaseGather}

\begin{pHeader}
    Синтаксис:      & \RightHandText{int clearPhaseGather();}\\
   Аргумент(ы):    & \RightHandText {нет} \\  
%    Возвращаемое значение:       & \RightHandText{Нет} \\
    Файл объявления:             & \RightHandText{sys/sys.h} \\
\end{pHeader}

Функция очищает буфер данных фазных прерываний. \killoverfullbefore

Является системной.
% *******end subsection*****************
%--------------------------------------------------------