%--------------------------------------------------------
% *******begin section***************
\section{\DbgSecSt{\StPart}{Таймеры}}
%--------------------------------------------------------
\subsection{\DbgSecSt{\StPart}{Типы данных}}
%--------------------------------------------------------

% *******begin subsection***************
\subsubsection{\DbgSecSt{\StPart}{Timer}}
\index{Программный интерфейс ПЛК!Таймеры!Структура Timer}
\label{sec:Timer}

\begin{fHeader}
    Тип данных:            & \RightHandText{Структура Timer}\\
    Файл объявления:             & \RightHandText{sys/sys.h} \\
\end{fHeader}

Структура определяет параметры таймера.

\begin{MyTableThreeColAllCntr}{Структура Timer}{tbl:Timer}{|m{0.41\linewidth}|m{0.24\linewidth}|m{0.35\linewidth}|}{Элемент}{Тип}{Описание}
\hline start & \centering{int} & Начальное значение счётчика таймера \\
\hline timeout & \centering{int} & Интервал \\
\end{MyTableThreeColAllCntr}
% *******end subsection***************

% *******begin subsection***************
\subsection{\DbgSecSt{\StPart}{Функции и макросы}}

%--------------------------------------------------------
% *******begin subsection***************
\subsubsection{\DbgSecSt{\StPart}{timerStart}}
\index{Программный интерфейс ПЛК!Таймеры!Макрос timerStart}
\label{sec:timerStart}

\begin{pHeader}
    Синтаксис:      & \RightHandText{timerStart(timer, timeoutVal);}\\
   Аргумент(ы):    & \RightHandText {timer ~-- переменная типа \myreftosec{Timer}} \\  
    & \RightHandText{timeoutVal ~-- интервал срабатывания} \\
    Файл объявления:             & \RightHandText{sys/sys.h} \\      
\end{pHeader}

Макрос запускает таймер, инициализируя переменную \texttt{timer}: полю timer.start присваивается текущее значение системного счётчика, полю timer.timeout ~-- значение интервала срабатывания.\killoverfullbefore

Интервал срабатывания таймера задаётся в периодах сервоцикла (1 период сервоцикла равен 400 мс). Так, например, 1 c соответствует значению интервала равному 2500. \killoverfullbefore

Является системной.
% *******end subsection*****************
%--------------------------------------------------------
% *******begin subsection***************
\subsubsection{\DbgSecSt{\StPart}{timerTimeout}}
\index{Программный интерфейс ПЛК!Таймеры!Макрос timerTimeout}
\label{sec:timerTimeout}

\begin{pHeader}
    Синтаксис:      & \RightHandText{timerTimeout(timer);}\\
   Аргумент(ы):    & \RightHandText {timer ~-- переменная типа \myreftosec{Timer}} \\  
%    Возвращаемое значение:       & \RightHandText{Нет} \\
    Файл объявления:             & \RightHandText{sys/sys.h} \\      
\end{pHeader}

Макрос возвращает 0, если не истёк заданный интервал срабатывания, и значение, отличное от 0, в противном случае.\killoverfullbefore

Является системной.
%--------------------------------------------------------
% *******begin subsection***************
\subsubsection{\DbgSecSt{\StPart}{timerLeft}}
\index{Программный интерфейс ПЛК!Таймеры!Макрос timerLeft}
\label{sec:timerLeft}

\begin{pHeader}
    Синтаксис:      & \RightHandText{timerLeft(timer);}\\
   Аргумент(ы):    & \RightHandText {timer ~-- переменная типа \myreftosec{Timer}} \\  
%    Возвращаемое значение:       & \RightHandText{Нет} \\
    Файл объявления:             & \RightHandText{sys/sys.h} \\      
\end{pHeader}

Макрос возвращает число периодов сервоцикла, оставшихся до срабатывания таймера.\killoverfullbefore

Является системной.
%--------------------------------------------------------
% *******begin subsection***************
\subsubsection{\DbgSecSt{\StPart}{timerPassed}}
\index{Программный интерфейс ПЛК!Таймеры!Макрос timerPassed}
\label{sec:timerPassed}

\begin{pHeader}
    Синтаксис:      & \RightHandText{timerPassed(timer);}\\
   Аргумент(ы):    & \RightHandText {timer ~-- переменная типа \myreftosec{Timer}} \\  
%    Возвращаемое значение:       & \RightHandText{Нет} \\
    Файл объявления:             & \RightHandText{sys/sys.h} \\      
\end{pHeader}

Макрос возвращает число периодов сервоцикла, прошедших с момента запуска таймера.\killoverfullbefore

Является системной.
% *******end subsection*****************

%-------------------------------------------------------------------
% *******begin subsection***************
\subsubsection{\DbgSecSt{\StPart}{initPulsedTimer}}
\index{Программный интерфейс ПЛК!Таймеры!Функция initPulsedTimer}
\label{sec:initPulsedTimer}

\begin{pHeader}
    Синтаксис:      & \RightHandText{void initPulsedTimer();}\\
    Аргумент(ы):    & \RightHandText{Нет} \\  
%    Возвращаемое значение:       & \RightHandText{Целое знаковое число} \\ 
    Файл объявления:             & \RightHandText{include/func/misc.h} \\       
\end{pHeader}

Функция инициализации периодического (импульсного) таймера. \killoverfullbefore
%который срабатывает (возвращает 1) через заданный интервал. 

Является системной.
% *******end subsection*****************
%-------------------------------------------------------------------
% *******begin subsection***************
\subsubsection{\DbgSecSt{\StPart}{timerSc}}
\index{Программный интерфейс ПЛК!Таймеры!Функция timerSc}
\label{sec:timerSc}

\begin{pHeader}
    Синтаксис:      & \RightHandText{int timerSc(int period);}\\
    Аргумент(ы):    & \RightHandText{int period ~-- период таймера} \\  
%    Возвращаемое значение:       & \RightHandText{Целое знаковое число} \\ 
    Файл объявления:             & \RightHandText{include/func/misc.h} \\
\end{pHeader}

Функция периодического (импульсного) таймера ~-- таймера, выходное значение которого периодически переключается с 0 на 1 и обратно через интервал, равный половине периода таймера. Период таймера задаётся в периодах сервоцикла (1 период сервоцикла равен 400 мс). Так, например, интервал 1 c соответствует значению периода таймера равному 2500. \killoverfullbefore
%сброса таймера (установки 0)

Функция возвращает 1, если с момента переключения таймера с 1 на 0 истёк интервал, больший или равный половине периода, и 0 в противном случае. \killoverfullbefore 

Является системной.
% *******end subsection*****************
%-------------------------------------------------------------------
