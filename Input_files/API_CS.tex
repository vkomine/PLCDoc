%--------------------------------------------------------
% *******begin section***************
\section{\DbgSecSt{\StPart}{Состояние управляющей программы}}
%--------------------------------------------------------
% *******begin subsection***************
\subsection{\DbgSecSt{\StPart}{Функции}}

% *******begin subsection***************
\subsubsection{\DbgSecSt{\StPart}{csProgramRunning}}
\index{Программный интерфейс ПЛК!Состояние управляющей программы!Функция csProgramRunning}
\label{sec:csProgramRunning}

\begin{pHeader}
    Синтаксис:      & \RightHandText{int csProgramRunning(int cs);}\\
    Аргумент(ы):    & \RightHandText{int cs ~-- номер координатной системы} \\    
%    Возвращаемое значение:       & \RightHandText{Целое знаковое число} \\ 
    Файл объявления:             & \RightHandText{include/func/cs.h} \\       
\end{pHeader}

Функция возвращает 1, если выполняется УП, и 0 в противном случае. 

Является системной.
% *******end section*****************
%-------------------------------------------------------------------
% *******begin subsection***************
\subsubsection{\DbgSecSt{\StPart}{csProgramHolding}}
\index{Программный интерфейс ПЛК!Состояние управляющей программы!Функция csProgramHolding}
\label{sec:csProgramHolding}

\begin{pHeader}
    Синтаксис:      & \RightHandText{int csProgramHolding(int cs);}\\
    Аргумент(ы):    & \RightHandText{int cs ~-- номер координатной системы} \\
%    Возвращаемое значение:       & \RightHandText{Целое знаковое число} \\ 
    Файл объявления:             & \RightHandText{include/func/cs.h} \\       
\end{pHeader}

Функция возвращает 1, если произведён приостанов подачи, и 0 в противном случае. 

Является системной.
% *******end section*****************
%-------------------------------------------------------------------
% *******begin subsection***************
\subsubsection{\DbgSecSt{\StPart}{csProgramStarting}}
\index{Программный интерфейс ПЛК!Состояние управляющей программы!Функция csProgramStarting}
\label{sec:csProgramStarting}

\begin{pHeader}
    Синтаксис:      & \RightHandText{int csProgramStarting(int cs);}\\
    Аргумент(ы):    & \RightHandText{int cs ~-- номер координатной системы} \\  
%    Возвращаемое значение:       & \RightHandText{Целое знаковое число} \\ 
    Файл объявления:             & \RightHandText{include/func/cs.h} \\       
\end{pHeader}

Функция возвращает 1, если УП начинает выполняться, и 0 в противном случае. 

Является системной.
% *******end section*****************
%-------------------------------------------------------------------
% *******begin subsection***************
\subsubsection{\DbgSecSt{\StPart}{csProgramPaused}}
\index{Программный интерфейс ПЛК!Состояние управляющей программы!Функция csProgramPaused}
\label{sec:csProgramPaused}

\begin{pHeader}
    Синтаксис:      & \RightHandText{int csProgramPaused(int cs);}\\
    Аргумент(ы):    & \RightHandText{int cs ~-- номер координатной системы} \\  
%    Возвращаемое значение:       & \RightHandText{Целое знаковое число} \\ 
    Файл объявления:             & \RightHandText{include/func/cs.h} \\       
\end{pHeader}

Функция возвращает 1, если УП временно приостановлена, и 0 в противном случае. 

Является системной.
% *******end section*****************
%-------------------------------------------------------------------
% *******begin subsection***************
\subsubsection{\DbgSecSt{\StPart}{csProgramStopped}}
\index{Программный интерфейс ПЛК!Состояние управляющей программы!Функция csProgramStopped}
\label{sec:csProgramStopped}

\begin{pHeader}
    Синтаксис:      & \RightHandText{int csProgramStopped(int cs);}\\
    Аргумент(ы):    & \RightHandText{int cs ~-- номер координатной системы} \\    
%    Возвращаемое значение:       & \RightHandText{Целое знаковое число} \\ 
    Файл объявления:             & \RightHandText{include/func/cs.h} \\       
\end{pHeader}

Функция возвращает 1, если УП остановлена, и 0 в противном случае. 

Является системной.
% *******end section*****************

