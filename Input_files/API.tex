\etocsettocdepth.toc {chapter}

\renewcommand{\arraystretch}{1.0} %% increase table row spacing
\renewcommand{\tabcolsep}{0.1cm}   %% increase table column spacing

\chapterimage{chapter_head_0} 
\chapter{\DbgSecSt{\StPart}{Программный интерфейс ПЛК}}
\label{sec:Functions}
\index{Программный интерфейс ПЛК|(}

%--------------------------------------------------------
% *******begin section***************
\section{\DbgSecSt{\StPart}{Управление УЧПУ}}

\subsection{\DbgSecSt{\StPart}{Типы данных}}

% *******begin subsection***************
\subsubsection{\DbgSecSt{\StPart}{CNCMode}}
\index{Программный интерфейс ПЛК!Управление УЧПУ!Типы данных!Перечисление CNCMode}
\label{sec:CNCMode}

\begin{fHeader}
    Тип данных:            & \RightHandText{Перечисление CNCMode}\\
    Файл объявления:             & \RightHandText{include/cnc/cnc.h} \\
\end{fHeader}

Перечисление определяет идентификаторы режимов работы УЧПУ.

\begin{MyTableTwoColAllCntr}{Перечисление CNCMode}{tbl:CNCMode}{|m{0.38\linewidth}|m{0.57\linewidth}|}{Идентификатор}{Описание}
\hline cncNull &   Режим не определён  \\
\hline cncOff &  УЧПУ не активно \\
\hline cncManual  & Ручной режим \\
\hline cncHome &  Режим выезда в нулевую точку \\
\hline cncHWL &  Режим дискретных перемещений \\
\hline cncAuto &  Автоматический режим \\
\hline cncStep &  Пошаговый режим \\
\hline cncMDI &  Режим преднабора \\
\hline cncVirtual &  Виртуальный режим \\
\hline cncReset &  Режим сброса \\
\hline cncRepos &  Режим возврата на контур \\
\hline cncWaitChangeMode &  Ожидание смены режима \\
\end{MyTableTwoColAllCntr}
% *******end subsection***************

% *******begin subsection***************
\subsubsection{\DbgSecSt{\StPart}{ProgramSeekMode}}
\index{Программный интерфейс ПЛК!Управление УЧПУ!Типы данных!Перечисление ProgramSeekMode}
\label{sec:ProgramSeekMode}

\begin{fHeader}
    Тип данных:            & \RightHandText{Перечисление ProgramSeekMode}\\
    Файл объявления:             & \RightHandText{include/cnc/cnc.h} \\
\end{fHeader}

Перечисление определяет идентификаторы режимов выполнения УП с произвольного кадра.

\begin{MyTableTwoColAllCntr}{Перечисление ProgramSeekMode}{tbl:ProgramSeekMode}{|m{0.38\linewidth}|m{0.57\linewidth}|}{Идентификатор}{Описание}
\hline seekNone &  Режим не активен  \\
\hline seekApproach &  Выполнение УП с начала выбранного кадра  \\
\hline seekWithoutApproach  & Выполнение УП с конца выбранного кадра \\
\hline seekWithoutCalc &  Выполнение УП без расчёта фрагмента программы до выбранного кадра \\
\end{MyTableTwoColAllCntr}
% *******end subsection***************

% *******begin subsection***************
\subsubsection{\DbgSecSt{\StPart}{ProgramStatus}}
\index{Программный интерфейс ПЛК!Управление УЧПУ!Типы данных!Перечисление ProgramStatus}
\label{sec:ProgramStatus}

\begin{fHeader}
    Тип данных:            & \RightHandText{Перечисление ProgramStatus}\\
    Файл объявления:             & \RightHandText{include/cnc/cnc.h} \\
\end{fHeader}

Перечисление определяет идентификаторы состояний УП.

\begin{MyTableTwoColAllCntr}{Перечисление ProgramStatus}{tbl:ProgramStatus}{|m{0.38\linewidth}|m{0.57\linewidth}|}{Идентификатор}{Описание}
\hline programAborted &  Выполнение УП прервано и завершено  \\
\hline programInterrupted  & Выполнение УП временно прервано для  какой-либо операции\\
\hline programStopped  & Выполнение УП остановлено \\
\hline programRunning  &  УП выполняется \\
\end{MyTableTwoColAllCntr}
% *******end subsection***************

% *******begin subsection***************
\subsubsection{\DbgSecSt{\StPart}{ChannelStatus}}
\index{Программный интерфейс ПЛК!Управление УЧПУ!Типы данных!Перечисление ChannelStatus}
\label{sec:ChannelStatus}

\begin{fHeader}
    Тип данных:            & \RightHandText{Перечисление ChannelStatus}\\
    Файл объявления:             & \RightHandText{include/cnc/cnc.h} \\
\end{fHeader}

Перечисление определяет идентификаторы состояний канала управления.

\begin{MyTableTwoColAllCntr}{Перечисление ChannelStatus}{tbl:ChannelStatus}{|m{0.38\linewidth}|m{0.57\linewidth}|}{Идентификатор}
{Описание}
\hline channelReset &   Готовность  \\
\hline channelInterrupted &   Работа прервана \\
\hline channelActive &   Активен \\
\end{MyTableTwoColAllCntr}
% *******end subsection***************

% *******begin subsection***************
\subsubsection{\DbgSecSt{\StPart}{ModeState}}
\index{Программный интерфейс ПЛК!Управление УЧПУ!Типы данных!Перечисление ModeState}
\label{sec:ModeState}

\begin{fHeader}
    Тип данных:            & \RightHandText{Перечисление ModeState}\\
    Файл объявления:             & \RightHandText{include/cnc/cnc.h} \\
\end{fHeader}

Перечисление определяет идентификаторы состояний текущего режима УЧПУ.

\begin{MyTableTwoColAllCntr}{Перечисление ModeState}{tbl:ModeState}{|m{0.38\linewidth}|m{0.57\linewidth}|}{Идентификатор}{Описание}
\hline modeReset &   Готовность  \\
\hline modeRunning  &  Выполнение \\
\hline modeStopped  &  Останов \\
\end{MyTableTwoColAllCntr}
% *******end subsection***************

% *******begin subsection***************
\subsubsection{\DbgSecSt{\StPart}{ShutdownState}}
\index{Программный интерфейс ПЛК!Управление УЧПУ!Типы данных!Перечисление ShutdownState}
\label{sec:ShutdownState}

\begin{fHeader}
    Тип данных:            & \RightHandText{Перечисление ShutdownState}\\
    Файл объявления:             & \RightHandText{include/cnc/cnc.h} \\
\end{fHeader}

Перечисление определяет идентификаторы состояний автомата выключения УЧПУ и станка.

\begin{MyTableTwoColAllCntr}{Перечисление ShutdownState}{tbl:ShutdownState}{|m{0.38\linewidth}|m{0.57\linewidth}|}{Идентификатор}{Описание}
\hline shutdownWaitCommand &  Ожидание команды выключения \\
\hline shutdownWaitAck &  Ожидание подтверждения команды выключения \\
\end{MyTableTwoColAllCntr}
% *******end subsection***************

% *******begin subsection***************
\subsubsection{\DbgSecSt{\StPart}{ChannelInfo}}
\index{Программный интерфейс ПЛК!Управление УЧПУ!Типы данных!Структура ChannelInfo}
\label{sec:ChannelInfo}

\begin{fHeader}
    Тип данных:            & \RightHandText{Структура ChannelInfo}\\
    Файл объявления:             & \RightHandText{include/cnc/cnc.h} \\
\end{fHeader}

Структура определяет данные канала управления.

\begin{MyTableThreeColAllCntr}{Структура ChannelInfo}{tbl:ChannelInfo}{|m{0.33\linewidth}|m{0.22\linewidth}|m{0.45\linewidth}|}{Элемент}{Тип}{Описание}
\hline canLoad & \centering{Битовое поле:1} &  Разрешение загрузки УП  \\
\hline starting & \centering{Битовое поле:1} &  Подготовка к выполнению УП \\
\hline running & \centering{Битовое поле:1} & Выполнение УП \\
\hline holding & \centering{Битовое поле:1} & УП в процессе останова или возобновления  \\
\hline stopped & \centering{Битовое поле:1} & УП не выполняется \\
\hline waitingBlock & \centering{Битовое поле:1} & Запрос поиска кадра  \\
\hline seekingBlock & \centering{Битовое поле:1} & Выполнение поиска кадра \\
\hline virtualStart & \centering{Битовое поле:1} & Подготовка к выполнению УП в виртуальном режиме \\
\hline virtualRun & \centering{Битовое поле:1} & Выполнение УП в виртуальном режиме \\
\hline canLoadMDI & \centering{Битовое поле:1} & Разрешение загрузки УП в режиме преднабора \\
\hline startingMDI & \centering{Битовое поле:1} & Подготовка к выполнению УП в режиме преднабора \\
\hline runningMDI & \centering{Битовое поле:1} &  Выполнение УП в режиме преднабора \\
\hline holdingMDI & \centering{Битовое поле:1} &  УП в режиме преднабора в процессе останова или возобновления \\
\hline waitingMDI & \centering{Битовое поле:2} &  0 ~-- УП загружена для выполнения в режиме преднабора \newline 1 ~-- запрос загрузки УП для выполнения в режиме преднабора \newline 2 ~-- ошибка загрузки УП для выполнения в режиме преднабора \\
\hline mdiReady & \centering{Битовое поле:1} &  УП в режиме преднабора готова к выполнению \\
\hline switchToRepos & \centering{Битовое поле:1} &  Разрешение перехода в режим возврата на контур \\
\hline setActual & \centering{Битовое поле:8} & Младшие 4 бита ~-- команда: \newline 1 ~-- текущая позиция = 0; \newline 2 ~-- текущая позиция = машинная позиция; \newline  3 ~-- текущая позиция = программная позиция \newline   
Старшие 4 бита ~-- область применения: 0 ~-- все оси; другие значения определяются конфигурацией станка \\
\hline res & \centering{Битовое поле:7} &  Резерв \\
\hline Pos[ЧИСЛО\_ОСЕЙ] & \centering{double} &  Программная позиция \\
\hline WorkPos[ЧИСЛО\_ОСЕЙ] & \centering{double} & Программная позиция относительно базового смещения \\
\hline MachPos[ЧИСЛО\_ОСЕЙ] & \centering{double} &  Машинная позиция \\
\hline TargetPos[ЧИСЛО\_ОСЕЙ] & \centering{double} &  Конечная позиция текущего кадра \\
\hline DistToGo[ЧИСЛО\_ОСЕЙ] & \centering{double} &  Остаток пути \\
\hline ActualPos[ЧИСЛО\_ОСЕЙ] & \centering{double} &  Текущая позиция \\
\hline ActualBase[ЧИСЛО\_ОСЕЙ] & \centering{double} &  Базовое смещение текущей позиции \\
\hline state & \centering{\myreftosec{ChannelStatus}} & Состояние канала управления \\
\hline modeState & \centering{\myreftosec{ModeState}} &  Состояние текущего режима УЧПУ \\
\hline runtime & \centering{\myreftosec{ProgramRuntime}} & Данные УП \\
\hline startBlock & \centering{unsigned} & Начальный блок поиска кадра при выполнении УП с произвольного кадра\\
\hline blockMode & \centering{unsigned} &  Режим выполнения УП с произвольного кадра \\
\hline seekCount & \centering{unsigned} &  Номер итерации поиска кадра \\
\end{MyTableThreeColAllCntr}
% *******end subsection***************

% *******begin subsection***************
\subsubsection{\DbgSecSt{\StPart}{CNCDesc}}
\index{Программный интерфейс ПЛК!Управление УЧПУ!Типы данных!Структура CNCDesc}
\label{sec:CNCDesc}

\begin{fHeader}
    Тип данных:            & \RightHandText{Структура CNCDesc}\\
    Файл объявления:             & \RightHandText{include/cnc/cnc.h} \\
\end{fHeader}

Структура определяет данные УЧПУ.

\begin{MyTableThreeColAllCntr}{Структура CNCDesc}{tbl:CNCDesc}{|m{0.33\linewidth}|m{0.22\linewidth}|m{0.45\linewidth}|}{Элемент}{Тип}{Описание}
\hline mode & \centering{\myreftosec{CNCMode}} &  Текущий режим работы УЧПУ  \\
\hline prevMode & \centering{\myreftosec{CNCMode}} & Предыдущий режим работы УЧПУ \\
\hline nextMode & \centering{\myreftosec{CNCMode}} & Следующий режим работы УЧПУ \\
\hline Watchdog & \centering{int} & Счётчик сторожевого таймера \\
\hline HMIFeedback & \centering{int} &  Флаг обратной связи пульта оператора \\
\hline HMIFirstStart & \centering{int} &  Флаг включения пульта оператора (до включения пульта оператора равен 1) \\
\hline hmiTripped & \centering{int} & Флаг срабатывания сторожевого таймера \\
\hline HMIWatchdog & \centering{\myreftosec{Timer}} &  Таймер сторожевого таймера \\
\hline shutdown & \centering{\myreftosec{Timer}} &  Таймер выключения УЧПУ и станка \\
\hline modeAutoStep & \centering{unsigned} & Флаг покадровой отработки УП \\
\hline modeAutoVirtual & \centering{unsigned} & Флаг отработки УП в виртуальном режиме \\
\hline modeAutoSkip & \centering{unsigned} &  Флаг программного пропуска кадров при отработке УП \\
\hline modeAutoOptStop & \centering{unsigned} & Флаг опционального останова при отработке УП  \\
\hline modeAutoRepos & \centering{unsigned} &  Флаг возврата на контур при возобновлении выполнения УП \\
\hline alarmCancel & \centering{unsigned} &  Запрос сброса ошибок \\
\hline modeDryRun & \centering{unsigned} &  Флаг пробной подачи при отработке УП  \\
\hline modeReducedG0 & \centering{unsigned} &  Флаг уменьшенной подачи быстрого хода при отработке УП  \\
\hline nodeNoMovement & \centering{unsigned} &  Флаг отработки УП с блокировкой движения \\
\hline request & \centering{\myreftosec{MTCNCRequests}} & Текущая исполняемая команда УЧПУ  \\
\hline channel[ЧИСЛО\_КАНАЛОВ] & \centering{\myreftosec{ChannelInfo}} &  Данные канала управления \\
\hline notReadyReq & \centering{Битовое поле:1} &  УЧПУ не готово \\
\hline startDisableReq & \centering{Битовое поле:1} &  Запрет запуска УП \\
\hline enablePortablePult & \centering{Битовое поле:1} &  Разрешение работы переносного пульта \\
\hline ShutdownHMI & \centering{int} &  Переменная выключения УЧПУ и станка принимает значения: \newline 0x5A при включении УЧПУ, \newline 0xA5 ~-- при получении команды, выключения, \newline 0x55 ~-- при подтверждении команды выключения\\
\hline ShutdownState & \centering{int} & Состояние автомата выключения УЧПУ и станка \\
\hline commands & \centering{\myreftosec{CommandQueue}} &  Очередь команд \\
\end{MyTableThreeColAllCntr}
% *******end subsection***************

% *******begin subsection***************
\subsubsection{\DbgSecSt{\StPart}{CNCSettings}}
\index{Программный интерфейс ПЛК!Управление УЧПУ!Типы данных!Структура CNCSettings}
\label{sec:CNCSettings}

\begin{fHeader}
    Тип данных:            & \RightHandText{Структура CNCSettings}\\
    Файл объявления:             & \RightHandText{include/cnc/cnc.h} \\
\end{fHeader}

Структура определяет значения подачи для различных режимов.

\begin{MyTableThreeColAllCntr}{Структура CNCSettings}{tbl:CNCSettings}{|m{0.33\linewidth}|m{0.22\linewidth}|m{0.45\linewidth}|}{Элемент}{Тип}{Описание}
\hline Frapid & \centering{double} &  Значение подачи быстрого хода \\
\hline Fdry & \centering{double} & Значение пробной подачи \\
\hline FrapidReduced & \centering{double} & Значение уменьшенной подачи быстрого хода \\
\end{MyTableThreeColAllCntr}
% *******end subsection***************
%-------------------------------------------------------------------
% *******begin subsection***************
\subsection{\DbgSecSt{\StPart}{Функции}}
\begin{comment}
% *******begin subsection***************
\subsubsection{\DbgSecSt{\StPart}{void InitCnc()}}
\index{Программный интерфейс ПЛК!Управление УЧПУ!void InitCnc()}
\label{sec:InitCnc}

\begin{pHeader}
%    Синтаксис:      & \RightHandText{void InitCnc();}\\
    Аргумент(ы):    & \RightHandText{Нет} \\    
    Возвращаемое значение:       & \RightHandText{Нет} \\ 
    Файл объявления:             & \RightHandText{include/cnc/cnc.h} \\
\end{pHeader}

Функция инициализации УЧПУ. 

Является системной.
% *******end section*****************
\end{comment}
% *******begin subsection***************
\subsubsection{\DbgSecSt{\StPart}{InitCnc}}
\index{Программный интерфейс ПЛК!Управление УЧПУ!Функции!InitCnc}
\label{sec:InitCnc}

\begin{pHeader}
    Синтаксис:      & \RightHandText{void InitCnc();}\\
    Аргумент(ы):    & \RightHandText{Нет} \\    
    Возвращаемое значение:       & \RightHandText{Нет} \\ 
    Файл объявления:             & \RightHandText{include/cnc/cnc.h} \\
\end{pHeader}

Функция инициализации УЧПУ. 

Является системной.
% *******end section*****************
%--------------------------------------------------------
% *******begin subsection***************
\subsubsection{\DbgSecSt{\StPart}{mtIsReady}}
\index{Программный интерфейс ПЛК!Управление УЧПУ!Функции!mtIsReady}
\label{sec:mtIsReady}

\begin{pHeader}
    Синтаксис:      & \RightHandText{int mtIsReady();}\\
    Аргумент(ы):    & \RightHandText{Нет} \\   
    Возвращаемое значение:       & \RightHandText{Целое знаковое число} \\
    Файл объявления:             & \RightHandText{include/cnc/cnc.h} \\      
\end{pHeader}

Функция проверки готовности станка к работе. Функция возвращает 1, если станок готов, и 0 в противном случае.  

Реализуется пользователем. 
% *******end subsection*****************
%--------------------------------------------------------
% *******begin subsection***************
\subsubsection{\DbgSecSt{\StPart}{cncSetMode}}
\index{Программный интерфейс ПЛК!Управление УЧПУ!Функции!cncSetMode}
\label{sec:cncSetMode}

\begin{pHeader}
    Синтаксис:      & \RightHandText{void cncSetMode(CNCMode mode);}\\
    Аргумент(ы):    & \RightHandText{Идентификатор перечисления \myreftosec{CNCMode}} \\   
    Возвращаемое значение:       & \RightHandText{Нет} \\    
    Файл объявления:             & \RightHandText{include/cnc/cnc.h} \\
\end{pHeader}

Функция устанавливает режим работы УЧПУ, принимая в качестве аргумента значение одного из идентификаторов перечисления \myreftosec{CNCMode}. 

Является системной.
% *******end subsection*****************
%--------------------------------------------------------
% *******begin subsection***************
\subsubsection{\DbgSecSt{\StPart}{cncRequest}}
\index{Программный интерфейс ПЛК!Управление УЧПУ!Функции!cncRequest}
\label{sec:cncRequest}

\begin{pHeader}
    Синтаксис:      & \RightHandText{void cncRequest (MTCNCRequests request);}\\
    Аргумент(ы):    & \RightHandText{Идентификатор перечисления \myreftosec{MTCNCRequests}} \\
    Возвращаемое значение:       & \RightHandText{Нет} \\    
    Файл объявления:             & \RightHandText{include/cnc/cnc.h} \\
\end{pHeader}

Функция посылает команду УЧПУ, принимая в качестве аргумента значение одного из идентификаторов перечисления \myreftosec{MTCNCRequests}. 

Является системной.

% *******end subsection*****************
\begin{comment}
%--------------------------------------------------------
% *******begin subsection***************
\subsubsection{\DbgSecSt{\StPart}{void cncCustomRequest (MTCNCRequests request)}}
\index{Программный интерфейс ПЛК!Управление УЧПУ!void cncCustomRequest (MTCNCRequests request)}
\label{sec:cncCustomRequest}

\begin{pHeader}
%    Синтаксис:      & \RightHandText{void cncSetMode(CNCMode mode);}\\
    Аргумент(ы):    & \RightHandText{Идентификатор перечисления \myreftosec{MTCNCRequests}} \\
    Возвращаемое значение:       & \RightHandText{Нет} \\    
    Файл объявления:             & \RightHandText{include/cnc/cnc.h} \\
\end{pHeader}

Функция посылает команду УЧПУ, принимая в качестве аргумента значение одного из идентификаторов перечисления \myreftosec{MTCNCRequests}. 
% *******end subsection*****************
\end{comment}
%--------------------------------------------------------
% *******begin subsection***************
\subsubsection{\DbgSecSt{\StPart}{cncChangeMode}}
\index{Программный интерфейс ПЛК!Управление УЧПУ!Функции!cncChangeMode}
\label{sec:cncChangeMode}

\begin{pHeader}
    Синтаксис:      & \RightHandText{void cncChangeMode (int newMode);}\\
    Аргумент(ы):    & \RightHandText{Целое знаковое число} \\
    Возвращаемое значение:       & \RightHandText{Нет} \\    
    Файл объявления:             & \RightHandText{include/cnc/cnc.h} \\
\end{pHeader}

Функция выполняет запрос изменения режима работы УЧПУ. 

Является системной.
% *******end subsection*****************
%--------------------------------------------------------
% *******begin subsection***************
\subsubsection{\DbgSecSt{\StPart}{channelUpdate}}
\index{Программный интерфейс ПЛК!Управление УЧПУ!Функции!channelUpdate}
\label{sec:channelUpdate}

\begin{pHeader}
    Синтаксис:      & \RightHandText{void channelUpdate (int channel);}\\
    Аргумент(ы):    & \RightHandText{Целое знаковое число} \\
    Возвращаемое значение:       & \RightHandText{Нет} \\    
    Файл объявления:             & \RightHandText{include/cnc/cnc.h} \\
\end{pHeader}

Функция обновляет данные канала, номер которого задаётся в качестве аргумента. 

Является системной.
% *******end subsection*****************
%--------------------------------------------------------
% *******begin subsection***************
\subsubsection{\DbgSecSt{\StPart}{cncModeManual}}
\index{Программный интерфейс ПЛК!Управление УЧПУ!Функции!cncModeManual}
\label{sec:cncModeManual}

\begin{pHeader}
    Синтаксис:      & \RightHandText{void cncModeManual();}\\
    Аргумент(ы):    & \RightHandText{Нет} \\
    Возвращаемое значение:       & \RightHandText{Нет} \\    
    Файл объявления:             & \RightHandText{include/cnc/cnc.h} \\
\end{pHeader}

Функция обработки команд в ручном режиме работы УЧПУ. 

Является системной.
% *******end subsection*****************
%--------------------------------------------------------
% *******begin subsection***************
\subsubsection{\DbgSecSt{\StPart}{cncModeHome}}
\index{Программный интерфейс ПЛК!Управление УЧПУ!Функции!cncModeHome}
\label{sec:cncModeHome}

\begin{pHeader}
    Синтаксис:      & \RightHandText{void cncModeHome();}\\
    Аргумент(ы):    & \RightHandText{Нет} \\
    Возвращаемое значение:       & \RightHandText{Нет} \\    
    Файл объявления:             & \RightHandText{include/cnc/cnc.h} \\
\end{pHeader}

Функция обработки команд в режиме выезда в нулевую точку УЧПУ. 

Является системной.
% *******end subsection*****************

%--------------------------------------------------------
% *******begin subsection***************
\subsubsection{\DbgSecSt{\StPart}{cncModeHandwheel}}
\index{Программный интерфейс ПЛК!Управление УЧПУ!Функции!cncModeHandwheel}
\label{sec:cncModeHandwheel}

\begin{pHeader}
    Синтаксис:      & \RightHandText{void cncModeHandwheel();}\\
    Аргумент(ы):    & \RightHandText{Нет} \\
    Возвращаемое значение:       & \RightHandText{Нет} \\    
    Файл объявления:             & \RightHandText{include/cnc/cnc.h} \\
\end{pHeader}

Функция обработки команд в режиме дискретных перемещений УЧПУ. 

Является системной.
% *******end subsection*****************
%--------------------------------------------------------
% *******begin subsection***************
\subsubsection{\DbgSecSt{\StPart}{cncModeAuto}}
\index{Программный интерфейс ПЛК!Управление УЧПУ!Функции!cncModeAuto}
\label{sec:cncModeAuto}

\begin{pHeader}
    Синтаксис:      & \RightHandText{void cncModeAuto();}\\
    Аргумент(ы):    & \RightHandText{Нет} \\
    Возвращаемое значение:       & \RightHandText{Нет} \\    
    Файл объявления:             & \RightHandText{include/cnc/cnc.h} \\
\end{pHeader}

Функция обработки команд в автоматическом режиме работы УЧПУ. 

Является системной.
% *******end subsection*****************
%--------------------------------------------------------
% *******begin subsection***************
\subsubsection{\DbgSecSt{\StPart}{cncModeMDI}}
\index{Программный интерфейс ПЛК!Управление УЧПУ!Функции!cncModeMDI}
\label{sec:cncModeMDI}

\begin{pHeader}
    Синтаксис:      & \RightHandText{void cncModeMDI();}\\
    Аргумент(ы):    & \RightHandText{Нет} \\
    Возвращаемое значение:       & \RightHandText{Нет} \\    
    Файл объявления:             & \RightHandText{include/cnc/cnc.h} \\
\end{pHeader}

Функция обработки команд в режиме преднабора УЧПУ. 

Является системной.
% *******end subsection*****************
%--------------------------------------------------------
% *******begin subsection***************
\subsubsection{\DbgSecSt{\StPart}{cncModeRepos}}
\index{Программный интерфейс ПЛК!Управление УЧПУ!Функции!cncModeRepos}
\label{sec:cncModeRepos}

\begin{pHeader}
    Синтаксис:      & \RightHandText{void cncModeRepos();}\\
    Аргумент(ы):    & \RightHandText{Нет} \\
    Возвращаемое значение:       & \RightHandText{Нет} \\    
    Файл объявления:             & \RightHandText{include/cnc/cnc.h} \\
\end{pHeader}

Функция обработки команд в режиме возврата на контур УЧПУ. 

Является системной.
% *******end subsection*****************
%--------------------------------------------------------



% *******begin subsection***************
\subsubsection{\DbgSecSt{\StPart}{cncManualEnter}}
\index{Программный интерфейс ПЛК!Управление УЧПУ!Функции!cncManualEnter}
\label{sec:cncManualEnter}

\begin{pHeader}
    Синтаксис:      & \RightHandText{void cncManualEnter());}\\
    Аргумент(ы):    & \RightHandText{Нет} \\
    Возвращаемое значение:       & \RightHandText{Нет} \\    
    Файл объявления:             & \RightHandText{include/cnc/cnc.h} \\
\end{pHeader}

Функция вызывается при установке ручного режима работы УЧПУ. В ней должны определяться действия, выполняемые при входе в данный режим. \killoverfullbefore

Реализуется пользователем. 
% *******end subsection*****************
%--------------------------------------------------------
% *******begin subsection***************
\subsubsection{\DbgSecSt{\StPart}{cncManualLeave}}
\index{Программный интерфейс ПЛК!Управление УЧПУ!Функции!cncManualLeave}
\label{sec:cncManualLeave}

\begin{pHeader}
    Синтаксис:      & \RightHandText{int cncManualLeave (CNCMode newMode);}\\
    Аргумент(ы):    & \RightHandText{Идентификатор перечисления \myreftosec{CNCMode}} \\
    Возвращаемое значение:       & \RightHandText{Целое знаковое число} \\    
    Файл объявления:             & \RightHandText{include/cnc/cnc.h} \\
\end{pHeader}

Функция вызывается при выходе из ручного режима работы УЧПУ. В ней должны определяться действия, выполняемые при выходе из данного режима, а также осуществляться проверка возможности установки нового режима работы УЧПУ, который задаётся аргументом ~-- значением одного из идентификаторов перечисления \myreftosec{CNCMode}. Возвращаемое значение должно быть отлично от 0 для разрешения нового режима работы. \killoverfullbefore

Реализуется пользователем.  
% *******end subsection*****************
%--------------------------------------------------------
% *******begin subsection***************
\subsubsection{\DbgSecSt{\StPart}{cncHwlEnter}}
\index{Программный интерфейс ПЛК!Управление УЧПУ!Функции!cncHwlEnter}
\label{sec:cncHwlEnter}

\begin{pHeader}
    Синтаксис:      & \RightHandText{void cncHwlEnter();}\\
    Аргумент(ы):    & \RightHandText{Нет} \\
    Возвращаемое значение:       & \RightHandText{Нет} \\    
    Файл объявления:             & \RightHandText{include/cnc/cnc.h} \\
\end{pHeader}

Функция вызывается при установке режима дискретных перемещений УЧПУ. В ней должны определяться действия, выполняемые при входе в данный режим. \killoverfullbefore

Реализуется пользователем. 
% *******end subsection*****************
%--------------------------------------------------------
% *******begin subsection***************
\subsubsection{\DbgSecSt{\StPart}{cncHwlLeave}}
\index{Программный интерфейс ПЛК!Управление УЧПУ!Функции!cncHwlLeave}
\label{sec:cncHwlLeave}

\begin{pHeader}
    Синтаксис:      & \RightHandText{int cncHwlLeave (CNCMode newMode);}\\
    Аргумент(ы):    & \RightHandText{Идентификатор перечисления \myreftosec{CNCMode}} \\
    Возвращаемое значение:       & \RightHandText{Целое знаковое число} \\    
    Файл объявления:             & \RightHandText{include/cnc/cnc.h} \\
\end{pHeader}

Функция вызывается при выходе из режима дискретных перемещений УЧПУ. В ней должны определяться действия, выполняемые при выходе из данного режима, а также осуществляться проверка возможности установки нового режима работы УЧПУ, который задаётся аргументом ~-- значением одного из идентификаторов перечисления \myreftosec{CNCMode}. Возвращаемое значение должно быть отлично от 0 для разрешения нового режима работы. \killoverfullbefore

Реализуется пользователем. 
% *******end subsection*****************
%--------------------------------------------------------
% *******begin subsection***************
\subsubsection{\DbgSecSt{\StPart}{cncHomeEnter}}
\index{Программный интерфейс ПЛК!Управление УЧПУ!Функции!cncHomeEnter}
\label{sec:cncHomeEnter}

\begin{pHeader}
    Синтаксис:      & \RightHandText{void cncHomeEnter();}\\
    Аргумент(ы):    & \RightHandText{Нет} \\
    Возвращаемое значение:       & \RightHandText{Нет} \\    
    Файл объявления:             & \RightHandText{include/cnc/cnc.h} \\
\end{pHeader}

Функция вызывается при установке режима выезда в нулевую точку УЧПУ. В ней должны определяться действия, выполняемые при входе в данный режим. \killoverfullbefore

Реализуется пользователем. 
% *******end subsection*****************
%--------------------------------------------------------
% *******begin subsection***************
\subsubsection{\DbgSecSt{\StPart}{cncHomeLeave}}
\index{Программный интерфейс ПЛК!Управление УЧПУ!Функции!cncHomeLeave}
\label{sec:cncHomeLeave}

\begin{pHeader}
    Синтаксис:      & \RightHandText{int cncHomeLeave (CNCMode newMode);}\\
    Аргумент(ы):    & \RightHandText{Идентификатор перечисления \myreftosec{CNCMode}} \\
    Возвращаемое значение:       & \RightHandText{Целое знаковое число} \\    
    Файл объявления:             & \RightHandText{include/cnc/cnc.h} \\
\end{pHeader}

Функция вызывается при выходе из режима выезда в нулевую точку УЧПУ. В ней должны определяться действия, выполняемые при выходе из данного режима, а также осуществляться проверка возможности установки нового режима работы УЧПУ, который задаётся аргументом ~-- значением одного из идентификаторов перечисления \myreftosec{CNCMode}. Возвращаемое значение должно быть отлично от 0 для разрешения нового режима работы. \killoverfullbefore

Реализуется пользователем.
% *******end subsection*****************
%--------------------------------------------------------
% *******begin subsection***************
\subsubsection{\DbgSecSt{\StPart}{cncAutoEnter}}
\index{Программный интерфейс ПЛК!Управление УЧПУ!Функции!cncAutoEnter}
\label{sec:cncAutoEnter}

\begin{pHeader}
    Синтаксис:      & \RightHandText{void cncAutoEnter();}\\
    Аргумент(ы):    & \RightHandText{Нет} \\
    Возвращаемое значение:       & \RightHandText{Нет} \\    
    Файл объявления:             & \RightHandText{include/cnc/cnc.h} \\
\end{pHeader}

Функция вызывается при установке автоматического режима УЧПУ. В ней должны определяться действия, выполняемые при входе в данный режим. \killoverfullbefore

Реализуется пользователем. 
% *******end subsection*****************
%--------------------------------------------------------
% *******begin subsection***************
\subsubsection{\DbgSecSt{\StPart}{cncAutoLeave}}
\index{Программный интерфейс ПЛК!Управление УЧПУ!Функции!cncAutoLeave}
\label{sec:cncAutoLeave}

\begin{pHeader}
    Синтаксис:      & \RightHandText{int cncAutoLeave (CNCMode newMode);}\\
    Аргумент(ы):    & \RightHandText{Идентификатор перечисления \myreftosec{CNCMode}} \\
    Возвращаемое значение:       & \RightHandText{Целое знаковое число} \\    
    Файл объявления:             & \RightHandText{include/cnc/cnc.h} \\
\end{pHeader}

Функция вызывается при выходе из автоматического режима УЧПУ. В ней должны определяться действия, выполняемые при выходе из данного режима, а также осуществляться проверка возможности установки нового режима работы УЧПУ, который задаётся аргументом ~-- значением одного из идентификаторов перечисления \myreftosec{CNCMode}. Возвращаемое значение должно быть отлично от 0 для разрешения нового режима работы. \killoverfullbefore

Реализуется пользователем.
% *******end subsection*****************
%--------------------------------------------------------
% *******begin subsection***************
\subsubsection{\DbgSecSt{\StPart}{cncMDIEnter}}
\index{Программный интерфейс ПЛК!Управление УЧПУ!Функции!cncMDIEnter}
\label{sec:cncMDIEnter}

\begin{pHeader}
    Синтаксис:      & \RightHandText{void cncMDIEnter();}\\
    Аргумент(ы):    & \RightHandText{Нет} \\
    Возвращаемое значение:       & \RightHandText{Нет} \\    
    Файл объявления:             & \RightHandText{include/cnc/cnc.h} \\
\end{pHeader}

Функция вызывается при установке режима преднабора УЧПУ. В ней должны определяться действия, выполняемые при входе в данный режим. \killoverfullbefore

Реализуется пользователем. 
% *******end subsection*****************
%--------------------------------------------------------
% *******begin subsection***************
\subsubsection{\DbgSecSt{\StPart}{cncMDILeave}}
\index{Программный интерфейс ПЛК!Управление УЧПУ!Функции!cncMDILeave}
\label{sec:cncMDILeave}

\begin{pHeader}
    Синтаксис:      & \RightHandText{int cncMDILeave (CNCMode newMode);}\\
    Аргумент(ы):    & \RightHandText{Идентификатор перечисления \myreftosec{CNCMode}} \\
    Возвращаемое значение:       & \RightHandText{Целое знаковое число} \\    
    Файл объявления:             & \RightHandText{include/cnc/cnc.h} \\
\end{pHeader}

Функция вызывается при выходе из режима преднабора УЧПУ. В ней должны определяться действия, выполняемые при выходе из данного режима, а также осуществляться проверка возможности установки нового режима работы УЧПУ, который задаётся аргументом ~-- значением одного из идентификаторов перечисления \myreftosec{CNCMode}. Возвращаемое значение должно быть отлично от 0 для разрешения нового режима работы. \killoverfullbefore

Реализуется пользователем.
% *******end subsection*****************
%--------------------------------------------------------
% *******begin subsection***************
\subsubsection{\DbgSecSt{\StPart}{cncReposEnter}}
\index{Программный интерфейс ПЛК!Управление УЧПУ!Функции!cncReposEnter}
\label{sec:cncReposEnter}

\begin{pHeader}
    Синтаксис:      & \RightHandText{void cncReposEnter();}\\
    Аргумент(ы):    & \RightHandText{Нет} \\
    Возвращаемое значение:       & \RightHandText{Нет} \\    
    Файл объявления:             & \RightHandText{include/cnc/cnc.h} \\
\end{pHeader}

Функция вызывается при установке режима возврата на контур УЧПУ. В ней должны определяться действия, выполняемые при входе в данный режим. \killoverfullbefore

Реализуется пользователем. 
% *******end subsection*****************
%--------------------------------------------------------
% *******begin subsection***************
\subsubsection{\DbgSecSt{\StPart}{cncReposLeave}}
\index{Программный интерфейс ПЛК!Управление УЧПУ!Функции!cncReposLeave}
\label{sec:cncReposLeave}

\begin{pHeader}
    Синтаксис:      & \RightHandText{int cncReposLeave (CNCMode newMode);}\\
    Аргумент(ы):    & \RightHandText{Идентификатор перечисления \myreftosec{CNCMode}} \\
    Возвращаемое значение:       & \RightHandText{Целое знаковое число} \\    
    Файл объявления:             & \RightHandText{include/cnc/cnc.h} \\
\end{pHeader}

Функция вызывается при выходе из режима возврата на контур УЧПУ. В ней должны определяться действия, выполняемые при выходе из данного режима, а также осуществляться проверка возможности установки нового режима работы УЧПУ, который задаётся аргументом ~-- значением одного из идентификаторов перечисления \myreftosec{CNCMode}. Возвращаемое значение должно быть отлично от 0 для разрешения нового режима работы. \killoverfullbefore

Реализуется пользователем.
% *******end subsection*****************
%--------------------------------------------------------
% *******begin subsection***************
\subsubsection{\DbgSecSt{\StPart}{controlPowerCNC}}
\index{Программный интерфейс ПЛК!Управление УЧПУ!Функции!controlPowerCNC}
\label{sec: controlPowerCNC}

\begin{pHeader}
    Синтаксис:      & \RightHandText{void controlPowerCNC (int request);}\\
    Аргумент(ы):    & \RightHandText{Целое знаковое число} \\
    Возвращаемое значение:       & \RightHandText{Нет} \\    
    Файл объявления:             & \RightHandText{include/cnc/cnc.h} \\
\end{pHeader}

Функция обработки запроса выключения УЧПУ и станка.  Аргументом функции является  значение одного из идентификаторов перечисления \myreftosec{MTCNCRequests}.

Является системной.
% *******end subsection*****************
%--------------------------------------------------------
% *******begin subsection***************
\subsubsection{\DbgSecSt{\StPart}{cncAutoOnProgramExit}}
\index{Программный интерфейс ПЛК!Управление УЧПУ!Функции!cncAutoOnProgramExit}
\label{sec: cncAutoOnProgramExit}

\begin{pHeader}
    Синтаксис:      & \RightHandText{void cncAutoOnProgramExit (int channel);}\\
    Аргумент(ы):    & \RightHandText{Целое знаковое число} \\
    Возвращаемое значение:       & \RightHandText{Нет} \\    
    Файл объявления:             & \RightHandText{include/cnc/cnc.h} \\
\end{pHeader}

Функция вызывается при выходе из автоматического режима УЧПУ. В ней должны определяться действия, выполняемые при выходе из данного режима для канала, номер которого является аргументом функции. \killoverfullbefore

Реализуется пользователем.
% *******end subsection*****************
%--------------------------------------------------------
% *******begin subsection***************
\subsubsection{\DbgSecSt{\StPart}{cncCustomRequestManual}}
\index{Программный интерфейс ПЛК!Управление УЧПУ!Функции!cncCustomRequestManual}
\label{sec: cncCustomRequestManual}

\begin{pHeader}
    Синтаксис:      & \RightHandText{void cncCustomRequestManual (int request);}\\
    Аргумент(ы):    & \RightHandText{Целое знаковое число} \\
    Возвращаемое значение:       & \RightHandText{Нет} \\    
    Файл объявления:             & \RightHandText{include/cnc/cnc.h} \\
\end{pHeader}

Функция обработки пользовательских команд в ручном режиме УЧПУ. Аргументом функции является команда пользователя.

Реализуется пользователем.
% *******end subsection*****************
%--------------------------------------------------------
% *******begin subsection***************
\subsubsection{\DbgSecSt{\StPart}{cncCustomRequestHome}}
\index{Программный интерфейс ПЛК!Управление УЧПУ!Функции!cncCustomRequestHome}
\label{sec: cncCustomRequestHome}

\begin{pHeader}
    Синтаксис:      & \RightHandText{void cncCustomRequestHome (int request);}\\
    Аргумент(ы):    & \RightHandText{Целое знаковое число} \\
    Возвращаемое значение:       & \RightHandText{Нет} \\    
    Файл объявления:             & \RightHandText{include/cnc/cnc.h} \\
\end{pHeader}

Функция обработки пользовательских команд в режиме выезда в нулевую точку УЧПУ.  Аргументом функции является команда пользователя.

Реализуется пользователем.
% *******end subsection*****************
%--------------------------------------------------------
% *******begin subsection***************
\subsubsection{\DbgSecSt{\StPart}{cncCustomRequestAuto}}
\index{Программный интерфейс ПЛК!Управление УЧПУ!Функции!cncCustomRequestAuto}
\label{sec: cncCustomRequestAuto}

\begin{pHeader}
    Синтаксис:      & \RightHandText{void cncCustomRequestAuto (int request);}\\
    Аргумент(ы):    & \RightHandText{Целое знаковое число} \\
    Возвращаемое значение:       & \RightHandText{Нет} \\    
    Файл объявления:             & \RightHandText{include/cnc/cnc.h} \\
\end{pHeader}

Функция обработки пользовательских команд в автоматическом режиме УЧПУ.  Аргументом функции является команда пользователя.

Реализуется пользователем.
% *******end subsection*****************
%--------------------------------------------------------
% *******begin subsection***************
\subsubsection{\DbgSecSt{\StPart}{cncCustomRequestMDI}}
\index{Программный интерфейс ПЛК!Управление УЧПУ!Функции!cncCustomRequestMDI}
\label{sec: cncCustomRequestMDI}

\begin{pHeader}
    Синтаксис:      & \RightHandText{void cncCustomRequestMDI (int request);}\\
    Аргумент(ы):    & \RightHandText{Целое знаковое число} \\
    Возвращаемое значение:       & \RightHandText{Нет} \\    
    Файл объявления:             & \RightHandText{include/cnc/cnc.h} \\
\end{pHeader}

Функция обработки пользовательских команд в режиме преднабора УЧПУ.  Аргументом функции является команда пользователя.

Реализуется пользователем.
% *******end subsection*****************
%--------------------------------------------------------
% *******begin subsection***************
\subsubsection{\DbgSecSt{\StPart}{cncCustomRequestHwl}}
\index{Программный интерфейс ПЛК!Управление УЧПУ!Функции cncCustomRequestHwl}
\label{sec: cncCustomRequestHwl}

\begin{pHeader}
   Синтаксис:      & \RightHandText{void cncCustomRequestHwl (int request);}\\
    Аргумент(ы):    & \RightHandText{Целое знаковое число} \\
    Возвращаемое значение:       & \RightHandText{Нет} \\    
    Файл объявления:             & \RightHandText{include/cnc/cnc.h} \\
\end{pHeader}

Функция обработки пользовательских команд в режиме дискретных перемещений УЧПУ.  Аргументом функции является команда пользователя.

Реализуется пользователем.
% *******end subsection*****************
%--------------------------------------------------------
% *******begin subsection***************
\subsubsection{\DbgSecSt{\StPart}{cncCustomRequestRepos}}
\index{Программный интерфейс ПЛК!Управление УЧПУ!Функции!cncCustomRequestRepos}
\label{sec: cncCustomRequestRepos}

\begin{pHeader}
    Синтаксис:      & \RightHandText{void cncCustomRequestRepos (int request);}\\
    Аргумент(ы):    & \RightHandText{Целое знаковое число} \\
    Возвращаемое значение:       & \RightHandText{Нет} \\    
    Файл объявления:             & \RightHandText{include/cnc/cnc.h} \\
\end{pHeader}

Функция обработки пользовательских команд в режиме возврата на контур УЧПУ.  Аргументом функции является команда пользователя. 

Реализуется пользователем.
% *******end subsection*****************
%--------------------------------------------------------
% *******begin subsection***************
\subsubsection{\DbgSecSt{\StPart}{cncManualCanChangeOverride}}
\index{Программный интерфейс ПЛК!Управление УЧПУ!Функции!cncManualCanChangeOverride}
\label{sec: cncManualCanChangeOverride}

\begin{pHeader}
    Синтаксис:      & \RightHandText{int cncManualCanChangeOverride();}\\
    Аргумент(ы):    & \RightHandText{Нет} \\
    Возвращаемое значение:       & \RightHandText{Целое знаковое число} \\
    Файл объявления:             & \RightHandText{include/cnc/cnc.h} \\
\end{pHeader}

Функция выполняет запрос на разрешение применения коррекции подачи. Возвращает 1, если коррекция разрешена, и 0 в противном случае.

Реализуется пользователем.
% *******end subsection*****************
% *******end section*****************

%--------------------------------------------------------
% *******begin section***************
\section{\DbgSecSt{\StPart}{Управление станком}}
%--------------------------------------------------------
\subsection{\DbgSecSt{\StPart}{Типы данных}}

% *******begin subsection***************
\subsubsection{\DbgSecSt{\StPart}{MTState}}
\index{Программный интерфейс ПЛК!Управление станком!Перечисление MTState}
\label{sec:MTState}

\begin{fHeader}
    Тип данных:            & \RightHandText{Перечисление MTState}\\
    Файл объявления:             & \RightHandText{include/cnc/mt.h} \\
\end{fHeader}

Перечисление определяет идентификаторы состояний станка.

\begin{MyTableTwoColAllCntr}{Перечисление MTState}{tbl:MTState}{|m{0.38\linewidth}|m{0.57\linewidth}|}{Идентификатор}{Описание}
\hline mtNotReady &  Станок выключен  \\
\hline mtStartOn &  Начало включения \\
\hline mtDriveOn &  Включение приводов \\
\hline mtWaitDriveOn &  Ожидание включения приводов \\
\hline mtOthersMotorOn & Включение вспомогательных моторов \\
\hline mtReady & Станок включен \\
\hline mtStartOff & Начало выключения \\
\hline mtOthersMotorOff & Выключение вспомогательных моторов \\
\hline mtAxisStop & Останов осей и шпинделя \\
\hline mtAxisWaitStop & Ожидание останова осей и шпинделя \\
\hline mtDriveOff & Выключение приводов \\
\hline mtAbort & Аварийное торможение \\
\hline mtPhaseRef & Фазировка \\
\hline mtWaitPhaseRef  & Ожидание фазировки \\
\hline mtWaitOff & Ожидание выключения питания станка \\
\hline mtWaitAbsPos & Ожидание данных от абсолютного ДОС \\
\end{MyTableTwoColAllCntr}
% *******end subsection***************
%--------------------------------------------------------
% *******begin subsection***************
\subsubsection{\DbgSecSt{\StPart}{MTCNCRequests}}
\index{Программный интерфейс ПЛК!Управление станком!Перечисление MTCNCRequests}
\label{sec:MTCNCRequests}

\begin{fHeader}
    Тип данных:            & \RightHandText{Перечисление MTCNCRequests}\\
    Файл объявления:             & \RightHandText{include/cnc/mt.h} \\
\end{fHeader}

Перечисление определяет идентификаторы команд управления станком. 

Начальный номер блока пользовательских команд  (mtcncCommandStart) равен 1000, конечный (mtcncCommandEnd) ~-- 1999. \killoverfullbefore

Начальный номер блока пользовательских команд движения (mtcncMoveCommandStart) равен 2000, конечный (mtcncMoveCommandEnd) ~-- 2999. \killoverfullbefore

\begin{MyTableTwoColAllCntr}{Перечисление MTCNCRequests}{tbl:MTCNCRequests}{|m{0.38\linewidth}|m{0.57\linewidth}|}{Идентификатор}{Описание}
\hline mtcncNone &  Нет команды  \\
\hline mtcncPowerOn  &  Включение станка \\
\hline mtcncPowerOff  &  Выключение станка \\
\hline mtcncEmergencyStop  &  Аварийный останов  \\
\hline mtcncReset  &  Сброс в начальное состояние \\
\hline mtcncStart  & Запуск операции в текущем режиме \\
\hline mtcncStop &  Останов операции в текущем режиме  \\
\hline mtcncCncOff &  Выключение УЧПУ \\

\hline mtcncActivateManual  & Включение ручного режима  \\
\hline mtcncActivateHandwheel  & Включение режима дискретных перемещений \\
\hline mtcncActivateRef &  Включение режима выезда в нулевую точку \\
\hline mtcncActivateMDI &  Включение режима преднабора \\
\hline mtcncActivateAuto & Включение  автоматического режима \\
\hline mtcncActivateRepos &  Включение режима возврата на контур \\

\hline mtcncToggleStep &  Покадровая отработка УП \\
\hline mtcncToggleRepos &  Возврат на контур \\
\hline mtcncToggleVirtual &  Отработка УП в виртуальном режиме \\
\hline mtcncToggleOptionalSkip & Отработка УП с программным пропуском кадров \\
\hline mtcncToggleOptionalStop & Отработка УП с опциональным остановом \\
\hline mtcncSelectSpeed1 & Выбор первой скорости/дискреты \newline безразмерных/дискретных
перемещений  \\
\hline mtcncSelectSpeed2 & Выбор второй скорости/дискреты \newline безразмерных/дискретных
перемещений \\
\hline mtcncSelectSpeed3 & Выбор третьей скорости/дискреты \newline безразмерных/дискретных перемещений  \\
\hline mtcncSelectSpeed4 & Выбор четвёртой скорости/дискреты \newline безразмерных/дискретных перемещений \\
\hline mtcncSelectRapid & Перемещение на скорости быстрого хода \\
\hline mtcncDryRun & Пробная подача   \\
\hline mtcncReducedRapid &  Уменьшенная подача быстрого хода \\
\hline mtcncMoveLock & Отработка УП с блокировкой движения \\
\hline mtcncAlarmCancel & Сброс ошибок \\

\hline mtcncCommandStart & Начальный номер блока пользовательских команд \\
\hline mtcncCommandEnd &  Конечный номер блока пользовательских команд \\

\hline mtcncMoveCommandStart & Начальный номер блока пользовательских команд движения \\
\hline mtcncMoveCommandEnd & Конечный номер блока пользовательских команд движения  \\

\end{MyTableTwoColAllCntr}
% *******end subsection***************
%--------------------------------------------------------
\begin{comment}
% *******begin subsection***************
\subsubsection{\DbgSecSt{\StPart}{UsedPult}}
\index{Программный интерфейс ПЛК!Управление станком!Перечисление UsedPult}
\label{sec:UsedPult}

\begin{fHeader}
    Тип данных:            & \RightHandText{Перечисление UsedPult}\\
    Файл объявления:             & \RightHandText{include/cnc/mt.h} \\
\end{fHeader}

Перечисление определяет идентификаторы терминальных устройств.

\begin{MyTableThreeColAllCntr}{Перечисление UsedPult}{tbl:UsedPult}{|m{0.33\linewidth}|m{0.22\linewidth}|m{0.45\linewidth}|}{Идентификатор}{Значение}{Описание}
\hline opertor & \centering{0} & Пульт оператора  \\
\hline portable & \centering{1} & Переносной пульт \\
\end{MyTableThreeColAllCntr}
% *******end subsection***************
\end{comment}
%--------------------------------------------------------
% *******begin subsection***************
\subsubsection{\DbgSecSt{\StPart}{MTDesc}}
\index{Программный интерфейс ПЛК!Управление станком!Структура MTDesc}
\label{sec:MTDesc}

\begin{fHeader}
    Тип данных:            & \RightHandText{Структура MTDesc}\\
    Файл объявления:             & \RightHandText{include/cnc/mt.h} \\
\end{fHeader}

Структура определяет данные станка.

\begin{MyTableThreeColAllCntr}{Структура MTDesc}{tbl:MTDesc}{|m{0.33\linewidth}|m{0.222\linewidth}|m{0.45\linewidth}|}{Элемент}{Тип}{Описание}
\hline State & \centering{int} &  Состояние автомата включения/выключения станка  \\
\hline IN & \centering{\myreftosec{MTInputs}} & Входы плат входов\\
\hline OUT & \centering{\myreftosec{MTOutputs}} & Выходы плат реле\\
\hline PultIn & \centering{\myreftosec{PultInputs}} & Входы пульта оператора \\
\hline PultOut & \centering{\myreftosec{PultOutputs}} & Выходы пульта оператора \\
\hline PortablePultIn & \centering{\myreftosec{PortablePultInputs}} &  Входы переносного пульта \\

\hline timerState & \centering{\myreftosec{Timer}} & Таймер состояния \\
\hline timerReset & \centering{\myreftosec{Timer}} & Таймер сброса \\
\hline timerScan & \centering{\myreftosec{Timer}} & Таймер выполнения операции \\

\hline ncNotReadyReq & \centering{Битовое поле:1} & Запрос готовности системы \\
\hline ncFollowUpReq & \centering{Битовое поле:1} & Запрос восстановления после ошибки  \\
\hline ncStopReq & \centering{Битовое поле:1} & Запрос немедленного останова УП или движения  \\
\hline ncStopAtEndReq & \centering{Битовое поле:1} & Запрос останова в конце текущего кадра\\
%\hline usedPult & \centering{int} &   \\
%\hline corrFTemp & \centering{double} &   \\
\end{MyTableThreeColAllCntr}
% *******end subsection***************
%-------------------------------------------------------------------
% *******begin subsection***************
\subsection{\DbgSecSt{\StPart}{Функции}}

% *******begin subsection***************
\subsubsection{\DbgSecSt{\StPart}{int systemPlcActive()}}
\index{Программный интерфейс ПЛК!Управление станком!int systemPlcActive()}
\label{sec:systemPlcActive}

\begin{pHeader}
%    Синтаксис:      & \RightHandText{void InitCnc();}\\
    Аргумент(ы):    & \RightHandText{Нет} \\    
    Возвращаемое значение:       & \RightHandText{Целое знаковое число} \\ 
    Файл объявления:             & \RightHandText{include/cnc/mt.h} \\       
\end{pHeader}

Функция возвращает 1, если нет ошибок программ ПЛК, и 0 в противном случае.

Реализуется пользователем.
% *******end section*****************
%-------------------------------------------------------------------
% *******begin subsection***************
\subsubsection{\DbgSecSt{\StPart}{int hasEmergencyStopRequest()}}
\index{Программный интерфейс ПЛК!Управление станком!int hasEmergencyStopRequest()}
\label{sec:hasEmergencyStopRequest}

\begin{pHeader}
%    Синтаксис:      & \RightHandText{void InitCnc();}\\
    Аргумент(ы):    & \RightHandText{Нет} \\    
    Возвращаемое значение:       & \RightHandText{Целое знаковое число} \\ 
    Файл объявления:             & \RightHandText{include/cnc/mt.h} \\       
\end{pHeader}

Функция возвращает 1, если есть запрос аварийного останова, и 0 в противном случае.

Реализуется пользователем.
% *******end section*****************
\begin{comment}
%-------------------------------------------------------------------
% *******begin subsection***************
\subsubsection{\DbgSecSt{\StPart}{int hasEmergencyStopMt()}}
\index{Программный интерфейс ПЛК!Управление станком!int hasEmergencyStopMt()}
\label{sec:hasEmergencyStopMt}

\begin{pHeader}
%    Синтаксис:      & \RightHandText{void InitCnc();}\\
    Аргумент(ы):    & \RightHandText{Нет} \\    
    Возвращаемое значение:       & \RightHandText{Целое знаковое число} \\ 
    Файл объявления:             & \RightHandText{include/cnc/mt.h} \\       
\end{pHeader}


% *******end section*****************
\end{comment}
%-------------------------------------------------------------------
% *******begin subsection***************
\subsubsection{\DbgSecSt{\StPart}{void mtControlRequest()}}
\index{Программный интерфейс ПЛК!Управление станком!void mtControlRequest()}
\label{sec:mtControlRequest}

\begin{pHeader}
%    Синтаксис:      & \RightHandText{void InitCnc();}\\
    Аргумент(ы):    & \RightHandText{Нет} \\    
    Возвращаемое значение:       & \RightHandText{Нет} \\ 
    Файл объявления:             & \RightHandText{include/cnc/mt.h} \\       
\end{pHeader}

Функция добавления команд в очередь.

Реализуется пользователем.
% *******end section*****************
%-------------------------------------------------------------------
% *******begin subsection***************
\subsubsection{\DbgSecSt{\StPart}{void mtUpdateCNCIndication()}}
\index{Программный интерфейс ПЛК!Управление станком!void mtUpdateCNCIndication()}
\label{sec:mtUpdateCNCIndication}

\begin{pHeader}
%    Синтаксис:      & \RightHandText{void InitCnc();}\\
    Аргумент(ы):    & \RightHandText{Нет} \\    
    Возвращаемое значение:       & \RightHandText{Нет} \\ 
    Файл объявления:             & \RightHandText{include/cnc/mt.h} \\       
\end{pHeader}

Функция обновления индикации пульта оператора.

Реализуется пользователем.
% *******end section*****************
%-------------------------------------------------------------------
% *******begin section***************
\section{\DbgSecSt{\StPart}{Обработка ошибок}}
%--------------------------------------------------------
\subsection{\DbgSecSt{\StPart}{Типы данных}}

% *******begin subsection***************
\subsubsection{\DbgSecSt{\StPart}{DriveErrors}}
\index{Программный интерфейс ПЛК!Обработка ошибок!Объединение DriveErrors}
\label{sec:DriveErrors}

\begin{fHeader}
    Тип данных:            & \RightHandText{Объединение DriveErrors}\\
    Файл объявления:             & \RightHandText{include/cnc/errors.h} \\
\end{fHeader}

Объединение определяет ошибки сервоусилителя.

\begin{MyTableThreeColAllCntr}{Объединение DriveErrors}{tbl:DriveErrors}{|m{0.33\linewidth}|m{0.22\linewidth}|m{0.45\linewidth}|}{Элемент}{Тип}{Описание}
\hline protocol & \centering{Битовое поле:1} & Ошибка протокола  \\
\hline ampNotReady & \centering{Битовое поле:1} & Нет готовности \\
\hline ampFault & \centering{Битовое поле:1} & Сервоусилитель в состоянии ошибки \\
\hline i2tFault & \centering{Битовое поле:1} &  Ошибка i2t \\
\hline crc & \centering{Битовое поле:1} & Ошибка контрольной суммы \\
\hline igbtFault & \centering{Битовое поле:1} & Ошибка IGBT модуля \\
\hline igbtTempFault & \centering{Битовое поле:1} & Превышение температуры IGBT модуля \\
\hline highDCFault & \centering{Битовое поле:1} & Повышенное напряжение в ЗПТ \\
\hline lowDCFault & \centering{Битовое поле:1} & Пониженное напряжение в ЗПТ \\
\hline linkFault & \centering{Битовое поле:1} & Ошибка связи \\
\hline brakeOnLowFault & \centering{Битовое поле:1} & Сигнал на открытие тормозного транзистора в состоянии L (не в слежении) \\
\hline brakeOnHighFault & \centering{Битовое поле:1} & Сигнал на открытие тормозного транзистора в состоянии H (не в слежении) \\
\hline brakeFault & \centering{Битовое поле:1} & Недостаточная мощность тормозного резистора \\
\hline currentOutFault & \centering{Битовое поле:1} & Измеренный ток в фазе в отсечке \\
\hline adcFault & \centering{Битовое поле:1} & Ошибка АЦП \\
\hline pwmShortFault & \centering{Битовое поле:1} & Период ШИМ сигнала меньше 50 мкс \\
\hline pwmLongFault & \centering{Битовое поле:1} & Период ШИМ сигнала больше 400 мкс \\
\hline reserved & \centering{Битовое поле:8} & Резерв \\
\hline ampState & \centering{Битовое поле:2} & Состояние сервоусилителя \\
\hline errorCode & \centering{Битовое поле:4} & Текущий код ошибки \\
\end{MyTableThreeColAllCntr}
% *******end subsection***************
%--------------------------------------------------------
% *******begin subsection***************
\subsubsection{\DbgSecSt{\StPart}{EncoderErrors}}
\index{Программный интерфейс ПЛК!Обработка ошибок!Объединение EncoderErrors}
\label{sec:EncoderErrors}

\begin{fHeader}
    Тип данных:            & \RightHandText{Объединение EncoderErrors}\\
    Файл объявления:             & \RightHandText{include/cnc/errors.h} \\
\end{fHeader}

Объединение определяет ошибки ДОС.

\begin{MyTableThreeColAllCntr}{Объединение EncoderErrors}{tbl:EncoderErrors}{|m{0.33\linewidth}|m{0.22\linewidth}|m{0.45\linewidth}|}{Элемент}{Тип}{Описание}
\hline encFault & \centering{Битовое поле:1} & Комбинированная ошибка датчика  \\
\hline decode & \centering{Битовое поле:1} & Ошибка декодирования \\
\hline sumOfSqr & \centering{Битовое поле:1} & Неверная сумма квадратов каналов синусно-косинусного датчика \\
\hline faultN & \centering{Битовое поле:1} &  Сигнал FAULT\_N \\
\hline adc & \centering{Битовое поле:1} & Ошибка АЦП \\
\hline lineA & \centering{Битовое поле:1} & Ошибка канала A \\
\hline lineB & \centering{Битовое поле:1} & Ошибка канала B \\
\hline lineC & \centering{Битовое поле:1} & Ошибка канала C \\
\hline power & \centering{Битовое поле:1} & Ошибка питания \\
\hline serialDataNotReady & \centering{Битовое поле:1} & Ошибка последовательного ДОС \\
\hline warningBiSS & \centering{Битовое поле:1} & Предупреждение ДОС BiSS \\
\hline faultBiSS & \centering{Битовое поле:1} & Ошибка ДОС BiSS \\
\hline statusEnDat & \centering{Битовое поле:1} & Ошибка статуса ДОС с протоколом EnDat \\
\end{MyTableThreeColAllCntr}
% *******end subsection***************
%--------------------------------------------------------
% *******begin subsection***************
\subsubsection{\DbgSecSt{\StPart}{IOErrors}}
\index{Программный интерфейс ПЛК!Обработка ошибок!Объединение IOErrors}
\label{sec:IOErrors}

\begin{fHeader}
    Тип данных:            & \RightHandText{Объединение IOErrors}\\
    Файл объявления:             & \RightHandText{include/cnc/errors.h} \\
\end{fHeader}

Объединение определяет ошибки последовательного интерфейса плат входов/выходов.

\begin{MyTableThreeColAllCntr}{Объединение IOErrors}{tbl:IOErrors}{|m{0.33\linewidth}|m{0.22\linewidth}|m{0.45\linewidth}|}{Элемент}{Тип}{Описание}
\hline parity & \centering{Битовое поле:1} & Ошибка четности \\
\hline protocol & \centering{Битовое поле:1} & Ошибка протокола \\
\hline crc & \centering{Битовое поле:1} & Ошибка контрольной суммы \\
\hline watchdog & \centering{Битовое поле:1} & Срабатывание сторожевого таймера \\
\end{MyTableThreeColAllCntr}
% *******end subsection***************
%--------------------------------------------------------
% *******begin subsection***************
\subsubsection{\DbgSecSt{\StPart}{MotorErrors}}
\index{Программный интерфейс ПЛК!Обработка ошибок!Объединение MotorErrors}
\label{sec:MotorErrors}

\begin{fHeader}
    Тип данных:            & \RightHandText{Объединение MotorErrors}\\
    Файл объявления:             & \RightHandText{include/cnc/errors.h} \\
\end{fHeader}

Объединение определяет ошибки приводов.

\begin{MyTableThreeColAllCntr}{Объединение MotorErrors}{tbl:MotorErrors}{|m{0.33\linewidth}|m{0.22\linewidth}|m{0.45\linewidth}|}{Элемент}{Тип}{Описание}
\hline phaseref & \centering{Битовое поле:1} & Не выполнена фазировка \\
\hline home & \centering{Битовое поле:1} & Не выполнен поиск нулевой точки \\
\hline homeError & \centering{Битовое поле:1} & Произошла ошибка при поиске нулевой точки \\
\hline openLoop & \centering{Битовое поле:1} & Двигатель не в слежении \\

\hline encoder & \centering{Битовое поле:1} & Ошибка ДОС \\
\hline plusLimit & \centering{Битовое поле:1} & Срабатывание аппаратного ограничителя в положительном направлении \\
\hline minusLimit & \centering{Битовое поле:1} & Срабатывание аппаратного ограничителя в отрицательном направлении\\
\hline swPlusLimit & \centering{Битовое поле:1} & Срабатывание программного ограничителя в положительном направлении \\
\hline swMinusLimit & \centering{Битовое поле:1} & Срабатывание программного ограничителя в отрицательном направлении \\
\hline folError & \centering{Битовое поле:1} & Критическая ошибка слежения \\
\hline folErrorWarning & \centering{Битовое поле:1} & Предупредительная ошибка слежения \\
\hline temperature & \centering{Битовое поле:1} & Перегрев двигателя \\
\hline tempWarning & \centering{Битовое поле:1} & Предупреждение о перегреве двигателя \\
\hline auxFault & \centering{Битовое поле:1} & Внешняя ошибка \\
\hline pos2Error & \centering{Битовое поле:1} & Ошибка рассогласования датчика положения и скорости \\
\hline pos2Warning & \centering{Битовое поле:1} & Предупреждение рассогласования датчика положения и скорости \\
\hline phasePosError & \centering{Битовое поле:1} & Ошибка рассогласования датчика положения и коммутации \\
\hline phasePosWarning & \centering{Битовое поле:1} & Предупреждение рассогласования датчика положения и коммутации \\
\end{MyTableThreeColAllCntr}
% *******end subsection***************
%--------------------------------------------------------
% *******begin subsection***************
\subsubsection{\DbgSecSt{\StPart}{AxisErrors}}
\index{Программный интерфейс ПЛК!Обработка ошибок!Объединение AxisErrors}
\label{sec:AxisErrors}

\begin{fHeader}
    Тип данных:            & \RightHandText{Объединение AxisErrors}\\
    Файл объявления:             & \RightHandText{include/cnc/errors.h} \\
\end{fHeader}

Объединение определяет ошибки оси.

\begin{MyTableThreeColAllCntr}{Объединение AxisErrors}{tbl:AxisErrors}{|m{0.33\linewidth}|m{0.22\linewidth}|m{0.45\linewidth}|}{Элемент}{Тип}{Описание}
\hline abortTimeout & \centering{Битовое поле:1} &   \\
\hline activateTimeout & \centering{Битовое поле:1} &   \\
\hline phaseRefTimeout & \centering{Битовое поле:1} &   \\
\hline deactivateTimeout & \centering{Битовое поле:1} &   \\
\end{MyTableThreeColAllCntr}
% *******end subsection***************
%--------------------------------------------------------
% *******begin subsection***************
\subsubsection{\DbgSecSt{\StPart}{SpindleErrors}}
\index{Программный интерфейс ПЛК!Обработка ошибок!Объединение SpindleErrors}
\label{sec:SpindleErrors}

\begin{fHeader}
    Тип данных:            & \RightHandText{Объединение SpindleErrors}\\
    Файл объявления:             & \RightHandText{include/cnc/errors.h} \\
\end{fHeader}

Объединение определяет ошибки шпинделя.

\begin{MyTableThreeColAllCntr}{Объединение SpindleErrors}{tbl:SpindleErrors}{|m{0.33\linewidth}|m{0.22\linewidth}|m{0.45\linewidth}|}{Элемент}{Тип}{Описание}
\hline abortTimeout & \centering{Битовое поле:1} &   \\
\hline activateTimeout & \centering{Битовое поле:1} &   \\
\hline phaseRefTimeout & \centering{Битовое поле:1} &   \\
\hline deactivateTimeout & \centering{Битовое поле:1} &   \\
\hline speedTimeout & \centering{Битовое поле:1} &   \\
\hline stopTimeout & \centering{Битовое поле:1} &   \\
\hline homeTimeout & \centering{Битовое поле:1} &   \\
\hline positionTimeout & \centering{Битовое поле:1} &   \\
\end{MyTableThreeColAllCntr}
% *******end subsection***************
%--------------------------------------------------------
% *******begin subsection***************
\subsubsection{\DbgSecSt{\StPart}{ChannelErrors}}
\index{Программный интерфейс ПЛК!Обработка ошибок!Объединение ChannelErrors}
\label{sec:ChannelErrors}

\begin{fHeader}
    Тип данных:            & \RightHandText{Объединение ChannelErrors}\\
    Файл объявления:             & \RightHandText{include/cnc/errors.h} \\
\end{fHeader}

Объединение определяет ошибки канала управления.

\begin{MyTableThreeColAllCntr}{Объединение ChannelErrors}{tbl:ChannelErrors}{|m{0.33\linewidth}|m{0.22\linewidth}|m{0.45\linewidth}|}{Элемент}{Тип}{Описание}
\hline phaseRefTimeout  & \centering{Битовое поле:1} &   \\
\hline driveOnTimeout  & \centering{Битовое поле:1} &   \\
\hline driveOffTimeout  & \centering{Битовое поле:1} &   \\
\hline abortTimeout & \centering{Битовое поле:1} &   \\
\hline stopTimeout & \centering{Битовое поле:1} &   \\
\hline homeTimeout & \centering{Битовое поле:1} &   \\
\hline homeError & \centering{Битовое поле:1} &   \\

\hline startWithoutHome & \centering{Битовое поле:1} &   \\
\hline cannotStart & \centering{Битовое поле:1} &   \\

\hline progStopOk & \centering{Битовое поле:1} &   \\
\hline progStopAbort & \centering{Битовое поле:1} &   \\
\hline progStopSyncError & \centering{Битовое поле:1} &   \\
\hline progStopBufferError & \centering{Битовое поле:1} &   \\
\hline progStopCCMove & \centering{Битовое поле:1} &   \\
\hline progStopLinToPvt & \centering{Битовое поле:1} &   \\
\hline progStopCCLeadOut & \centering{Битовое поле:1} &   \\
\hline progStopCCLeadIn & \centering{Битовое поле:1} &   \\
\hline progStopCCBufSize & \centering{Битовое поле:1} &   \\
\hline progStopPvt & \centering{Битовое поле:1} &   \\
\hline progStopCCFeed & \centering{Битовое поле:1} &   \\
\hline progStopCCDir & \centering{Битовое поле:1} &   \\
\hline progStopAbort & \centering{Битовое поле:1} &   \\
\hline progStopNoSolve & \centering{Битовое поле:1} &   \\
\hline progStopCC3NdotT & \centering{Битовое поле:1} &   \\
\hline progStopCCDist & \centering{Битовое поле:1} &   \\
\hline progStopCCNoIntersect & \centering{Битовое поле:1} &   \\
\hline progStopCCNoMoves & \centering{Битовое поле:1} &   \\

\hline progStopRunTime & \centering{Битовое поле:1} &   \\
\hline progStopInPos & \centering{Битовое поле:1} &   \\
\hline progStopSoftLimit & \centering{Битовое поле:1} &   \\
\hline progStopRadiusX & \centering{Битовое поле:1} &   \\
\hline progStopRadiusXX & \centering{Битовое поле:1} &   \\

\hline progPausedM00 & \centering{Битовое поле:1} &   \\

\hline cycleInvalidArgs & \centering{Битовое поле:1} &   \\

\hline seekingBlock & \centering{Битовое поле:1} &   \\
\hline seekBlockFound & \centering{Битовое поле:1} &   \\
\hline seekBlockNotFound & \centering{Битовое поле:1} &   \\
\end{MyTableThreeColAllCntr}
% *******end subsection***************
%--------------------------------------------------------
% *******begin subsection***************
\subsubsection{\DbgSecSt{\StPart}{NCErrors}}
\index{Программный интерфейс ПЛК!Обработка ошибок!Объединение NCErrors}
\label{sec:NCErrors}

\begin{fHeader}
    Тип данных:            & \RightHandText{Объединение NCErrors}\\
    Файл объявления:             & \RightHandText{include/cnc/errors.h} \\
\end{fHeader}

Объединение определяет системные ошибки.

\begin{MyTableThreeColAllCntr}{Объединение NCErrors}{tbl:NCErrors}{|m{0.33\linewidth}|m{0.22\linewidth}|m{0.45\linewidth}|}{Элемент}{Тип}{Описание}
\hline factory & \centering{Битовое поле:1} &  Ошибка загрузки системных параметров \\
\hline userFactory & \centering{Битовое поле:1} &  Ошибка загрузки параметров пользователя \\
\hline swClock & \centering{Битовое поле:1} &   \\
\hline bgWdt & \centering{Битовое поле:1} & Срабатывание сторожевого таймера фонового режима \\
\hline rtWdt & \centering{Битовое поле:1} &  Срабатывание сторожевого таймера реального времени \\
\hline sysPlcFault & \centering{Битовое поле:1} &   \\
\hline hmiWatchdog & \centering{Битовое поле:1} &   \\
\end{MyTableThreeColAllCntr}
% *******end subsection***************
%--------------------------------------------------------
% *******begin subsection***************
\subsubsection{\DbgSecSt{\StPart}{MachineErrors}}
\index{Программный интерфейс ПЛК!Обработка ошибок!Структура MachineErrors}
\label{sec:MachineErrors}

\begin{fHeader}
    Тип данных:            & \RightHandText{Структура MachineErrors}\\
    Файл объявления:             & \RightHandText{include/cnc/errors.h} \\
\end{fHeader}

Структура определяет ошибки станка.

% *******end subsection***************
%--------------------------------------------------------
% *******begin subsection***************
\subsubsection{\DbgSecSt{\StPart}{Errors}}
\index{Программный интерфейс ПЛК!Обработка ошибок!Структура Errors}
\label{sec:Errors}

\begin{fHeader}
    Тип данных:            & \RightHandText{Структура Errors}\\
    Файл объявления:             & \RightHandText{include/cnc/errors.h} \\
\end{fHeader}

Структура содержит данные о системных ошибках, об ошибках станка, каналов управления, осей, шпинделей, приводов, ДОС, сервоусилителей и плат входов/выходов.

\begin{MyTableThreeColAllCntr}{Структура Errors}{tbl:Errors}{|m{0.38\linewidth}|m{0.22\linewidth}|m{0.4\linewidth}|}{Элемент}{Тип}{Описание}
\hline machine & \centering{\myreftosec{MachineErrors}} & Ошибки станка \\
\hline nc & \centering{\myreftosec{NCErrors}} & Системные ошибки  \\
\hline channel [ЧИСЛО\_КАНАЛОВ] & \centering{\myreftosec{ChannelErrors}} & Ошибки каналов управления \\
\hline axes [ЧИСЛО\_ОСЕЙ] & \centering{\myreftosec{AxisErrors}} & Ошибки осей \\
\hline spindles [ЧИСЛО\_ШПИНДЕЛЕЙ] & \centering{\myreftosec{SpindleErrors}} & Ошибки шпинделей \\
\hline motors [ЧИСЛО\_ДВИГАТЕЛЕЙ+1] & \centering{\myreftosec{MotorErrors}} & Ошибки приводов \\
\hline encoders [ЧИСЛО\_ДОС] & \centering{\myreftosec{EncoderErrors}} & Ошибки ДОС \\
\hline drive [ЧИСЛО\_ДВИГАТЕЛЕЙ+1] & \centering{\myreftosec{DriveErrors}} & Ошибки сервоусилителей \\
\hline io [ЧИСЛО\_ПЛАТ\_ВХ/ВЫХ] & \centering{\myreftosec{IOErrors}} & Ошибки плат входов/выходов \\
\end{MyTableThreeColAllCntr}
% *******end subsection***************
%--------------------------------------------------------
% *******begin subsection***************
\subsubsection{\DbgSecSt{\StPart}{ErrorReaction}}
\index{Программный интерфейс ПЛК!Обработка ошибок!Перечисление ErrorReaction}
\label{sec:ErrorReaction}

\begin{fHeader}
    Тип данных:            & \RightHandText{Перечисление ErrorReaction}\\
    Файл объявления:             & \RightHandText{include/cnc/errors.h} \\
\end{fHeader}

Перечисление определяет идентификаторы типов реакций на ошибки.

\begin{MyTableTwoColAllCntr}{Перечисление ErrorReaction}{tbl:ErrorReaction}{|m{0.38\linewidth}|m{0.57\linewidth}|}{Идентификатор}{Описание}
\hline reactNone &  Нет реакции \\
\hline reactLocal &    \\
\hline reactFollowUp  &  Восстановление после ошибки \\
\hline reactStopProgram  & Прервано выполнение программы \\
\hline reactNCNotReady  & Нет готовности системы \\
\hline reactChannelNotReady  & Нет готовности канала \\
\hline reactStartDisable  & Запрет запуска программы в канале \\
\hline reactNeedHome  & Требуется повторный выезд в 0 для осей в канале \\
\hline reactShowAlarm  & Показать в ЧПУ \\
\hline reactStop &  Остановить оси \\
\hline reactStopAtEnd  & Остановить в конце блока \\
\hline reactAutoOnly  & Только в автоматическом режиме \\
\hline reactWarning  & Отображение предупреждения \\
\end{MyTableTwoColAllCntr}
% *******end subsection***************
%--------------------------------------------------------
% *******begin subsection***************
\subsubsection{\DbgSecSt{\StPart}{ErrorClear}}
\index{Программный интерфейс ПЛК!Обработка ошибок!Перечисление ErrorClear}
\label{sec:ErrorClear}

\begin{fHeader}
    Тип данных:            & \RightHandText{Перечисление ErrorClear}\\
    Файл объявления:             & \RightHandText{include/cnc/errors.h} \\
\end{fHeader}

Перечисление определяет идентификаторы типов сброса ошибок.

\begin{MyTableTwoColAllCntr}{Перечисление ErrorClear}{tbl:ErrorClear}{|m{0.38\linewidth}|m{0.57\linewidth}|}{Идентификатор}{Описание}
\hline clearSelf &  Автоматический сброс \\
\hline clearCancel &  Сброс по ALARM CANCEL, CYCLE START или RESET \\
\hline clearNCStart &   Сброс по CYCLE START, RESET с возобновлением программы \\
\hline clearReset &  Сброс по RESET для канала \\
\hline clearNCReset &  Сброс по RESET для системы \\
\hline clearPowerOn &  Сброс по включению питания \\
\end{MyTableTwoColAllCntr}
% *******end subsection***************
%--------------------------------------------------------
% *******begin subsection***************
\subsubsection{\DbgSecSt{\StPart}{DriveErrorReaction}}
\index{Программный интерфейс ПЛК!Обработка ошибок!Перечисление DriveErrorReaction}
\label{sec:DriveErrorReaction}

\begin{fHeader}
    Тип данных:            & \RightHandText{Перечисление DriveErrorReaction}\\
    Файл объявления:             & \RightHandText{include/cnc/errors.h} \\
\end{fHeader}

Перечисление определяет идентификаторы типов реакции на ошибки сервоусилителя.

\begin{MyTableTwoColAllCntr}{Перечисление DriveErrorReaction}{tbl:DriveErrorReaction}{|m{0.38\linewidth}|m{0.57\linewidth}|}{Идентификатор}{Описание}
\hline dreactNone &  Нет реакции \\
\hline dreactOFF1 &  STOP -> DKILL в слежении, иначе KILL \\
\hline dreactOFF1delayed  &  DELAY -> STOP -> KILL в слежении, иначе DELAY -> KILL \\
\hline dreactOFF2 &  KILL \\
\hline dreactOFF3 &  ABORT -> DKILL в слежении, иначе KILL \\
\hline dreactSTOP2 &  ABORT \\
\hline dreactIASC\_DCBRK  & Для синхронного - закоротить обмотки, для асинхронного - торможение постоянным током \\
\hline dreactENC & Настраивается (по умолчанию OFF2) \\
\end{MyTableTwoColAllCntr}
% *******end subsection***************
%--------------------------------------------------------
% *******begin subsection***************
\subsubsection{\DbgSecSt{\StPart}{DriveErrorAcknowledge}}
\index{Программный интерфейс ПЛК!Обработка ошибок!Перечисление DriveErrorAcknowledge}
\label{sec:DriveErrorAcknowledge}

\begin{fHeader}
    Тип данных:            & \RightHandText{Перечисление DriveErrorAcknowledge}\\
    Файл объявления:             & \RightHandText{include/cnc/errors.h} \\
\end{fHeader}

Перечисление определяет идентификаторы подтверждений ошибок сервоусилителя.

\begin{MyTableTwoColAllCntr}{Перечисление DriveErrorAcknowledge}{tbl:DriveErrorAcknowledge}{|m{0.38\linewidth}|m{0.57\linewidth}|}{Идентификатор}{Описание}
\hline dackPowerOn &    \\
\hline dackImmediately  &   \\
\hline dackDisable &    \\
\end{MyTableTwoColAllCntr}
% *******end subsection***************
%--------------------------------------------------------
% *******begin subsection***************
\subsubsection{\DbgSecSt{\StPart}{ErrorDescription}}
\index{Программный интерфейс ПЛК!Обработка ошибок!Структура ErrorDescription}
\label{sec:ErrorDescription}

\begin{fHeader}
    Тип данных:            & \RightHandText{Структура ErrorDescription}\\
    Файл объявления:             & \RightHandText{include/cnc/errors.h} \\
\end{fHeader}

Структура определяет описание ошибки.

\begin{MyTableThreeColAllCntr}{Структура ErrorDescription}{tbl:ErrorDescription}{|m{0.33\linewidth}|m{0.22\linewidth}|m{0.45\linewidth}|}{Элемент}{Тип}{Описание}
\hline id & \centering{double} &  Код ошибки \\
\hline reaction & \centering{double} & Тип реакции \\
\hline clear & \centering{double} & Тип сброса \\
\end{MyTableThreeColAllCntr}
% *******end subsection***************
%--------------------------------------------------------
% *******begin subsection***************
\subsection{\DbgSecSt{\StPart}{Функции}}

% *******begin subsection***************
\subsubsection{\DbgSecSt{\StPart}{errorScanRequest}}
\index{Программный интерфейс ПЛК!Обработка ошибок!Функции!errorScanRequest}
\label{sec:errorScanRequest}

\begin{pHeader}
    Синтаксис:      & \RightHandText{void errorScanRequest (ErrorClear request);}\\
    Аргумент(ы):    & \RightHandText{Идентификатор перечисления \myreftosec{ErrorClear}} \\    
    Возвращаемое значение:       & \RightHandText{Нет} \\ 
    Файл объявления:             & \RightHandText{include/cnc/errors.h} \\
\end{pHeader}


Является системной.
% *******end section*****************
%--------------------------------------------------------
% *******begin subsection***************
\subsubsection{\DbgSecSt{\StPart}{void errorScan()}}
\index{Программный интерфейс ПЛК!Обработка ошибок!void errorScan()}
\label{sec:errorScan}

\begin{pHeader}
%    Синтаксис:      & \RightHandText{int mtIsReady();}\\
    Аргумент(ы):    & \RightHandText{Нет} \\   
    Возвращаемое значение:       & \RightHandText{Нет} \\
    Файл объявления:             & \RightHandText{include/cnc/errors.h} \\      
\end{pHeader}


Является системной.
% *******end subsection*****************
%--------------------------------------------------------
% *******begin subsection***************
\subsubsection{\DbgSecSt{\StPart}{errorSetScan}}
\index{Программный интерфейс ПЛК!Обработка ошибок!Функции!errorSetScan}
\label{sec:errorSetScan}

\begin{pHeader}
    Синтаксис:      & \RightHandText{int errorSetScan (unsigned curInput, unsigned input, const ErrorDescription \&desc, ErrorClear request)}\\
    Аргумент(ы):    & \RightHandText{Целое беззнаковое число, целое беззнаковое число,} \\   
    & \RightHandText {экземпляр структуры \myreftosec{ErrorDescription},} \\
    & \RightHandText {идентификатор перечисления \myreftosec{ErrorClear}} \\
    Возвращаемое значение:       & \RightHandText{Целое знаковое число} \\
    Файл объявления:             & \RightHandText{include/cnc/errors.h} \\      
\end{pHeader}


Является системной.
% *******end subsection*****************
%--------------------------------------------------------
% *******begin subsection***************
\subsubsection{\DbgSecSt{\StPart}{errorReaction}}
\index{Программный интерфейс ПЛК!Обработка ошибок!Функции!errorReaction}
\label{sec:errorReaction}

\begin{pHeader}
    Синтаксис:      & \RightHandText{void errorReaction (unsigned input, const ErrorDescription \&desc);}\\
    Аргумент(ы):    & \RightHandText{Целое беззнаковое число, экземпляр структуры \myreftosec{ErrorDescription}} \\   
    Возвращаемое значение:       & \RightHandText{Нет} \\
    Файл объявления:             & \RightHandText{include/cnc/errors.h} \\      
\end{pHeader}


Является системной.
% *******end subsection*****************
%--------------------------------------------------------
% *******begin subsection***************
\subsubsection{\DbgSecSt{\StPart}{void errorsMachineScan (int request)}}
\index{Программный интерфейс ПЛК!Обработка ошибок!void errorsMachineScan (int request)}
\label{sec:errorsMachineScan}

\begin{pHeader}
%    Синтаксис:      & \RightHandText{int mtIsReady();}\\
    Аргумент(ы):    & \RightHandText{Целое знаковое число} \\   
    Возвращаемое значение:       & \RightHandText{Нет} \\
    Файл объявления:             & \RightHandText{include/cnc/errors.h} \\      
\end{pHeader}


Реализуется пользователем.
% *******end subsection*****************
%--------------------------------------------------------
% *******begin subsection***************
\subsubsection{\DbgSecSt{\StPart}{void errorsMachineReaction()}}
\index{Программный интерфейс ПЛК!Обработка ошибок!void errorsMachineReaction()}
\label{sec:errorsMachineReaction}

\begin{pHeader}
%    Синтаксис:      & \RightHandText{int mtIsReady();}\\
    Аргумент(ы):    & \RightHandText{Нет} \\   
    Возвращаемое значение:       & \RightHandText{Нет} \\
    Файл объявления:             & \RightHandText{include/cnc/errors.h} \\      
\end{pHeader}


Реализуется пользователем. 
% *******end subsection*****************

%--------------------------------------------------------
\index{Программный интерфейс ПЛК|)}

\clearpage