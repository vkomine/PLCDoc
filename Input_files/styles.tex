\newif\ifShowAllVarI        % Старые переменные (I, Q и пр)

\newif\ifbiblatex            % Библиография

\newif\ifKeyWordUpCase        % Keyword - Upcase/LOcase

\newif\ifTempDescript        % временное комментирование

\newif\ifDebugSectState        % показать состояние готовности пункта


%%%******************************************%%%
%%%***    Настройка условий компиляции    ***%%%
%%%******************************************%%%

% Старые переменные (I, Q и пр)
    %\ShowAllVarItrue          % показать все недоделанное
    \ShowAllVarIfalse        % скрыть все недоделанное

% Библиография
    %\biblatextrue 
    \biblatexfalse

% Keyword - Upcase/LOcase
    %\KeyWordUpCasetrue 
    \KeyWordUpCasefalse

% временное комментирование
    \TempDescripttrue
%    \TempDescriptfalse

% показать состояние готовности пунктов
    \DebugSectStatetrue
%\DebugSectStatefalse
    


%%%*** ---------------------------------- ***%%%

% для борьбы с висячими строками
    \clubpenalty = 10000
    \widowpenalty = 10000


 \setcounter{tocdepth}{3}            % чтобы добавить подпараграфы...
 \setcounter{secnumdepth}{4}
 \setcounter{totalnumber}{10}
 \setcounter{topnumber}{10}
 \setcounter{bottomnumber}{10}
 \renewcommand{\topfraction}{1}
 \renewcommand{\bottomfraction}{1}
 \renewcommand{\textfraction}{0}
 
 
% **********************************************************


\graphicspath{{./}{./Pictures/eps/}{./Pictures/svg/}}


% **********************************************************
%                 ЦВЕТА
% **********************************************************
% Цвета для кода
\definecolor{string}{HTML}{B40000}          % цвет строк в коде
\definecolor{comment}{HTML}{228B22}         % цвет комментариев в коде
\definecolor{keyword}{HTML}{1A00FF}         % цвет ключевых слов в коде

\definecolor{key_word_color}{HTML}{0000FF}

\definecolor{key_word_1_color}{HTML}{3A2FCD}

%\definecolor{key_word_1_color}{HTML}{0000FF}

\definecolor{key_word_2_color}{HTML}{BF0000}

\definecolor{key_word_3_color}{HTML}{000070}

\definecolor{key_word_4_color}{HTML}{700070}

\definecolor{morecomment}{HTML}{8000FF}     % цвет include и других элементов в коде

%\definecolor{сaptiontext}{HTML}{FFFFFF}     % цвет текста заголовка в коде
\definecolor{сaptiontext}{HTML}{000000}     % цвет текста заголовка в коде


%\definecolor{сaptionbk}{HTML}{999999}       % цвет фона заголовка в коде
\definecolor{сaptionbk}{rgb}{.9, .9, .9}       % цвет фона заголовка в коде

\definecolor{bk}{HTML}{F8F8FF}              % цвет фона в коде

%\definecolor{frame}{HTML}{999999}           % цвет рамки в коде

\definecolor{frame}{rgb}{.9, .9, .9}           % цвет рамки в коде
%\definecolor{frame}{RGB}{10,175,15}           % цвет рамки в коде
 

\definecolor{brackets}{HTML}{B40000}        % цвет скобок в коде

\definecolor{digits}{HTML}{FF0000}        % цвет скобок в коде
\definecolor{digits_color}{HTML}{FF0000}        % цвет скобок в коде


%\definecolor{shadecolor}{HTML}{999999}
\definecolor{shadecolor}{rgb}{.9, .9, .9}
%\definecolor{shadecolor}{RGB}{10,175,15}
% **********************************************************



%\begin{comment}
\makeatletter
\setlength{\@fptop}{2pt}
\setlength{\@fpbot}{0pt plus 1fil}
\makeatother
%\end{comment}

% **********************************************************
\begin{comment}
\makeatletter
   \def\relativepath{\import@path}
\makeatother
\end{comment}

\makeatletter
\long\def\@makecaption#1#2{%
  \vskip\abovecaptionskip
  \hbox to\textwidth{\hfill\parbox{0.9\textwidth}{\begin{center}#1 #2\end{center}}\hfill}
  \vskip\belowcaptionskip}
\makeatother

%**************************************************

\IfFileExists{/dev/null}{%
  \newcommand{\Inkscape}{inkscape}%
  }{%
  \newcommand{\Inkscape}{"C:/Program Files (x86)/Inkscape/inkscape.exe"}%
}

\newcommand{\executeiffilenewer}[3]{%
 \ifnum\pdfstrcmp{\pdffilemoddate{#1}}%
 {\pdffilemoddate{#2}}>0%
 {\immediate\write18{#3}}\fi%
}

\newcommand{\DrawPictEpsFromSvg}[4][0.9\textwidth]{%
    \executeiffilenewer{#2.svg}{#2.eps}%
    {inkscape -z --file=#2.svg --export-eps=#2.eps --export-text-to-path}%
    \begin{figure}[htb]%
    \noindent\centering\includegraphics[keepaspectratio=true, width=#1]{#2.eps}%
    \caption{#3}%
    \label{fig:#4}%
    \end{figure}%
    \FloatBarrier%
    \afterpage{\FloatBarrier}%
}

\newcommand{\DrawOnlyEpsFromSvg}[2][0.9\textwidth]{%
    \executeiffilenewer{#2.svg}{#2.eps}%
    {inkscape -z --file=#2.svg --export-eps=#2.eps --export-text-to-path}%
    \begin{figure}[htb]%
    \noindent\centering\includegraphics[width=#1]{#2.eps}%
    \end{figure}%
    \FloatBarrier%
    \afterpage{\FloatBarrier}%
}

\newcommand{\IncludeEpsFromSvg}[2][\textwidth]{%
    \executeiffilenewer{#2.svg}{#2.eps}%
    {inkscape -z --file=#2.svg --export-eps=#2.eps --export-text-to-path}%
    \includegraphics[width=#1]{#2.eps}%
}
    
\begin{comment}
\newcommand{\includesvg}[1]{%
    \executeiffilenewer{#1.svg}{#1.eps}%
    {inkscape -z --file=#1.svg --export-eps=#1.eps --export-text-to-path}%
    \includegraphics{#1.eps}
}
\end{comment}




\newenvironment{MyItemize}[1]%
{%
    %\begin{center}%
        \begin{longtable}[h]{#1}
}%
{%
        \end{longtable}    \addtocounter{table}{-1}%
    %\end{center}%
%    \FloatBarrier%
%    \afterpage{\FloatBarrier}%
}    


\newcommand*{\TblLeftHeaderCenter}[1]%
{%
\multicolumn{1}{c|}{\textbf{#1}}%
}%


\newcommand{\TblLeftHeaderCenterNew}[1]%
{%
\multicolumn{1}{c|}{\textbf{#1}}%
}%

\newcommand{\TBHdr}[1]%
{%
\textbf{#1}%
}%

\newcommand{\TBHdrCntr}[1]%
{%
\centering\textbf{#1}%
}%

\newcommand{\TBRightHdrCntr}[1]%
{%
\multicolumn{1}{c|}{\textbf{#1}}%
}%

\newenvironment{MyTableTwoCol}[5]%
{%
    \begin{center}%
        \begin{longtable}[h]{#3}%
            \caption{#1}\label{#2}\\
            \hline                  &                \\
                    \TBHdrCntr{#4}      & \TBHdr{#5}   \\
                                     &               \\
            \hline
            \endfirsthead
            \caption*{Продолжение таблицы \ref{#2}.}\\
            \hline                     &                \\
                    \TBHdrCntr{#4}      & \TBHdr{#5}       \\
                                     &                \\
            \hline
            \endhead
            \hline 
            \endfoot
            \hline
            \endlastfoot
}%
{%
        \end{longtable}%
    \end{center}%
    \FloatBarrier%
    \afterpage{\FloatBarrier}%
}    

\newenvironment{MyTableTwoColCntr}[5]%
{%
    \begin{center}%
        \begin{longtable}[h]{#3}%
            \caption{#1}\label{#2}\\
            \hline                  &                \\
                    \TBHdrCntr{#4}      & \TBRightHdrCntr{#5}   \\
                                     &               \\
            \hline
            \endfirsthead
            \caption*{Продолжение таблицы \ref{#2}.}\\
            \hline                     &                \\
                    \TBHdrCntr{#4}      & \TBHdr{#5}       \\
                                     &                \\
            \hline
            \endhead
            \hline 
            \endfoot
            \hline
            \endlastfoot
}%
{%
        \end{longtable}%
    \end{center}%
    \FloatBarrier%
    \afterpage{\FloatBarrier}%
} 

\newenvironment{MyTableTwoColAllCntr}[5]%
{%
    \begin{center}%
        \begin{longtable}[h]{#3}%
            \caption{#1}\label{#2}\\
            \hline                  &                \\
                    \TBHdrCntr{#4}      & \TBRightHdrCntr{#5}   \\
                                     &               \\
            \hline
            \endfirsthead
            \caption*{Продолжение таблицы \ref{#2}.}\\
            \hline                     &                \\
                    \TBHdrCntr{#4}      & \TBRightHdrCntr{#5}       \\
                                     &                \\
            \hline
            \endhead
            \hline 
            \endfoot
            \hline
            \endlastfoot
}%
{%
        \end{longtable}%
    \end{center}%
    \FloatBarrier%
    \afterpage{\FloatBarrier}%
}    



%\centering\bfseries
\newenvironment{MyTableThreeCol}[6]%
{%
    \begin{center}%
        \begin{longtable}[h]{#3}%
            \caption{#1}\label{#2}\\
            \hline                  &                & \\
                    \TBHdrCntr{#4}      &  \TBHdrCntr{#5}     & \TBHdr{#6}     \\
                                     &                & \\
            \hline
            \endfirsthead
            \caption*{Продолжение таблицы \ref{#2}.}\\
            \hline                     &                & \\
                    \TBHdrCntr{#4}      & \TBHdrCntr{#5}     & \TBHdr{#6}     \\
                   &                & \\
            \hline
            \endhead
            \hline 
            \endfoot
            \hline
            \endlastfoot
}%
{%
        \end{longtable}%
    \end{center}%
    \FloatBarrier%
    \afterpage{\FloatBarrier}%
}    

\newenvironment{MyTableThreeColCntr}[6]%
{%
    \begin{center}%
        \begin{longtable}{#3}%
            \caption{#1}\label{#2}\\
            \hline                  &                & \\
                    \TBHdrCntr{#4}      &  \TBHdrCntr{#5}     & \TBRightHdrCntr{#6}     \\
                                     &                & \\
            \hline
            \endfirsthead
            \caption*{Продолжение таблицы \ref{#2}.}\\
            \hline                     &                & \\
                    \TBHdrCntr{#4}      & \TBHdr{#5}     & \TBHdr{#6}     \\
                   &                & \\
            \hline
            \endhead
            \hline 
            \endfoot
            \hline
            \endlastfoot
}%
{%
        \end{longtable}%
    \end{center}%
    \FloatBarrier%
    \afterpage{\FloatBarrier}%
} 

%\centering\bfseries
\newenvironment{MyTableThreeColAllCntr}[6]%
{%
    \begin{center}%
        \begin{longtable}{#3}%
            \caption{#1}\label{#2}\\
            \hline                  &                & \\
                    \TBHdrCntr{#4}      &  \TBHdrCntr{#5}     & \TBRightHdrCntr{#6}     \\
                                     &                & \\
            \hline
            \endfirsthead
            \caption*{Продолжение таблицы \ref{#2}.}\\
            \hline                     &                & \\
                    \TBHdrCntr{#4}      & \TBHdrCntr{#5}     & \TBRightHdrCntr{#6}     \\
                   &                & \\
            \hline
            \endhead
            \hline 
            \endfoot
            \hline
            \endlastfoot
}%
{%
        \end{longtable}%
    \end{center}%
    \FloatBarrier%
    \afterpage{\FloatBarrier}%
}   

\newenvironment{MyTableFourCol}[7]%
{%
    \begin{center}%
        \begin{longtable}{#3}
            \caption{#1}\label{#2}\\
            \hline                  &                &              & \\
                    \TBHdrCntr{#4}      & \TBHdrCntr{#5}     & \TBHdrCntr{#6} & \TBHdr{#7}     \\
                                    &                &              & \\
            \hline
            \endfirsthead
            \caption*{Продолжение таблицы \ref{#2}.}\\
            \hline                  &                &              & \\
                    \TBHdrCntr{#4}      & \TBHdrCntr{#5}     & \TBHdrCntr{#6} & \TBHdr{#7}     \\
                                    &                &              & \\
            \hline
            \endhead
            \hline 
            \endfoot
            \hline
            \endlastfoot
}%
{%
        \end{longtable}%
    \end{center}%
    \FloatBarrier%
    \afterpage{\FloatBarrier}%
}


\newenvironment{MyTableFourColAllCntr}[7]%
{%
    \begin{center}%
        \begin{longtable}{#3}
            \caption{#1}\label{#2}\\
            \hline                  &                &              & \\
                    \TBHdrCntr{#4}      & \TBHdrCntr{#5}     & \TBHdrCntr{#6} & \TBRightHdrCntr{#7}     \\
                                    &                &              & \\
            \hline
            \endfirsthead
            \caption*{Продолжение таблицы \ref{#2}.}\\
            \hline                  &                &              & \\
                    \TBHdrCntr{#4}      & \TBHdrCntr{#5}     & \TBHdrCntr{#6} & \TBRightHdrCntr{#7}     \\
                                    &                &              & \\
            \hline
            \endhead
            \hline 
            \endfoot
            \hline
            \endlastfoot
}%
{%
        \end{longtable}%
    \end{center}%
    \FloatBarrier%
    \afterpage{\FloatBarrier}%
}

\newenvironment{MyTableFiveCol}[8]%
{%
    \begin{center}%
        \begin{longtable}{#3}
            \caption{#1}\label{#2}\\
            \hline                  &                &              &             &\\
                    \TBHdrCntr{#4}      & \TBHdrCntr{#5}     & \TBHdrCntr{#6} & \TBHdrCntr{#7} & \TBHdr{#8}     \\
                                    &                &              &             &\\
            \hline
            \endfirsthead
            \caption*{Продолжение таблицы \ref{#2}.}\\
            \hline  &                &              &             &\\
                    \TBHdrCntr{#4}      & \TBHdrCntr{#5}     & \TBHdrCntr{#6} & \TBHdrCntr{#7} & \TBHdr{#8}     \\
                    &                &              &             &\\
            \hline
            \endhead
            \hline 
            \endfoot
            \hline
            \endlastfoot
}%
{%
        \end{longtable}%
    \end{center}%
    \FloatBarrier%
    \afterpage{\FloatBarrier}%
}    

\newenvironment{MyTableFiveColAllCntr}[8]%
{%
    \begin{center}%
        \begin{longtable}{#3}
            \caption{#1}\label{#2}\\
            \hline                  &                &              &             &\\
                    \TBHdrCntr{#4}      & \TBHdrCntr{#5}     & \TBHdrCntr{#6} & \TBHdrCntr{#7} & \TBRightHdrCntr{#8}     \\
                                    &                &              &             &\\
            \hline
            \endfirsthead
            \caption*{Продолжение таблицы \ref{#2}.}\\
            \hline  &                &              &             &\\
                    \TBHdrCntr{#4}      & \TBHdrCntr{#5}     & \TBHdrCntr{#6} & \TBHdrCntr{#7} & \TBRightHdrCntr{#8}     \\
                    &                &              &             &\\
            \hline
            \endhead
            \hline 
            \endfoot
            \hline
            \endlastfoot
}%
{%
        \end{longtable}%
    \end{center}%
    \FloatBarrier%
    \afterpage{\FloatBarrier}%
}    

\newenvironment{MyTableSixCol}[9]%
{%
    \begin{center}%
        \begin{longtable}{#3}
            \caption{#1}\label{#2}\\
            \hline  &                &              &             &                &\\
                    \TBHdrCntr{#4}      & \TBHdrCntr{#5}     & \TBHdrCntr{#6} & \TBHdrCntr{#7} & \TBHdrCntr{#8} & \TBRightHdrCntr{#9}     \\
                    &                &              &             &                &\\
            \hline
            \endfirsthead
            \caption*{Продолжение таблицы \ref{#2}.}\\
            \hline  &                &              &             &                &\\
                    \TBHdrCntr{#4}      & \TBHdrCntr{#5}     & \TBHdrCntr{#6} & \TBHdrCntr{#7} & \TBHdrCntr{#8} & \TBRightHdrCntr{#9}    \\
                    &                &              &             &                &\\
            \hline
            \endhead
            \hline 
            \endfoot
            \hline
            \endlastfoot
}%
{%
        \end{longtable}%
    \end{center}%
    \FloatBarrier%
    \afterpage{\FloatBarrier}%
}    

\newenvironment{MyTableSixColAllCntr}[9]%
{%
    \begin{center}%
        \begin{longtable}{#3}
            \caption{#1}\label{#2}\\
            \hline  &                &              &             &                &\\
                    \TBHdrCntr{#4}      & \TBHdrCntr{#5}     & \TBHdrCntr{#6} & \TBHdrCntr{#7} & \TBHdrCntr{#8} & \TBHdr{#9}     \\
                    &                &              &             &                &\\
            \hline
            \endfirsthead
            \caption*{Продолжение таблицы \ref{#2}.}\\
            \hline  &                &              &             &                &\\
                    \TBHdrCntr{#4}      & \TBHdrCntr{#5}     & \TBHdrCntr{#6} & \TBHdrCntr{#7} & \TBHdrCntr{#8} & \TBHdr{#9}    \\
                    &                &              &             &                &\\
            \hline
            \endhead
            \hline 
            \endfoot
            \hline
            \endlastfoot
}%
{%
        \end{longtable}%
    \end{center}%
    \FloatBarrier%
    \afterpage{\FloatBarrier}%
}    






\begin{comment}

\newenvironment{MyTableSevenCol}[10]%
{%
    \begin{center}%
        \begin{longtable}{#3}
            \caption{#1}\label{#2}\\
            \hline  &        &        &         &        &        &\\
                    \textbf{#4}      & \textbf{#5}     & \textbf{#6} & \textbf{#7} & \textbf{#8} & \textbf{#9} & \textbf{#10}    \\
                    &        &        &        &         &        &\\
            \hline
            \endfirsthead
            \caption*{Продолжение таблицы \ref{#2}.}\\
            \hline  &                &              &     &        &                &\\
                    \textbf{#4}      & \textbf{#5}     & \textbf{#6} & \textbf{#7} & \textbf{#8} & \textbf{#9}    & \textbf{#10} \\
                    &                &              &     &        &                &\\
            \hline
            \endhead
            \hline 
            \endfoot
            \hline
            \endlastfoot
}%
{%
        \end{longtable}%
    \end{center}%
    \FloatBarrier%
    \afterpage{\FloatBarrier}%
}    

\end{comment}

%----------------------------------------------------------------------------------------
%    MINI TABLE OF CONTENTS IN CHAPTER HEADS
%----------------------------------------------------------------------------------------

% Section text styling
\titlecontents{lsection}[0em] % Indendating
%{\footnotesize\sffamily} % Font settings
{\footnotesize} % Font settings
{}
{}
{}

% Subsection text styling
\titlecontents{lsubsection}[.5em] % Indentation
%{\normalfont\footnotesize\sffamily} % Font settings
{\normalfont\footnotesize} % Font settings
{}
{}
{}
 
%----------------------------------------------------------------------------------------
%    PAGE HEADERS
%----------------------------------------------------------------------------------------
%\newdateformat{specialdate}{\THEYEAR-\twodigit{\THEMONTH}-\twodigit{\THEDAY}}

\newdateformat{monthyeardate}{\monthname[\THEMONTH], \THEYEAR}

% Для печати на одной странице
\pagestyle{fancy}
\fancyhead{}
\fancyfoot{}
\renewcommand{\chaptermark}[1]{\markboth{\normalsize\bfseries #1}{}} % Chapter text font settings
\renewcommand{\sectionmark}[1]{\markright{\normalsize\thesection\hspace{5pt}#1}{}} % Section text font settings
\fancyhf{} \fancyfoot[CE,CO]{\normalsize\thepage} % Font setting for the page number in the header
\rhead{\rightmark}
%\fancyhead[RE,RO]{\rightmark} % Print the nearest section name on the left side of odd pages
%\fancyhead[LE,LO]{\leftmark} % Print the current chapter name on the right side of even pages
\renewcommand{\headrulewidth}{0.4pt} % Width of the rule under the header
\addtolength{\headheight}{12pt} % Increase the spacing around the header slightly
\setlength{\headsep}{13pt} %
\renewcommand{\footrulewidth}{0.0pt} % Removes the rule in the footer

\fancypagestyle{plain}{ 
    \fancyhf{}
    \fancyfoot[C]{\thepage}}

\begin{comment}
\pagestyle{fancy}
\renewcommand{\chaptermark}[1]{\markboth{\normalsize\bfseries #1}{}} % Chapter text font settings
\renewcommand{\sectionmark}[1]{\markright{\normalsize\thesection\hspace{5pt}#1}{}} % Section text font settings
\fancyhf{} \fancyhead[LE,RO]{\normalsize\thepage} % Font setting for the page number in the header
\fancyhead[LO]{\rightmark} % Print the nearest section name on the left side of odd pages
\fancyhead[RE]{\leftmark} % Print the current chapter name on the right side of even pages
\renewcommand{\headrulewidth}{0.5pt} % Width of the rule under the header
\addtolength{\headheight}{12pt} % Increase the spacing around the header slightly
\setlength{\headsep}{13pt} %
\renewcommand{\footrulewidth}{0pt} % Removes the rule in the footer
\fancypagestyle{plain}{\fancyhead{}\renewcommand{\headrulewidth}{0pt}} % Style for when a plain pagestyle is specified
\fancyfoot[LE]{\copyright INELSY 2017}
\fancyfoot[RO]{Ревизия 1.0 от \today}
\fancyfoot[C]{}
 \end{comment}
% Removes the header from odd empty pages at the end of chapters
\makeatletter
\renewcommand{\cleardoublepage}{
\afterpage{\clearpage}\ifodd\c@page\else
\hbox{}
\vspace*{\fill}
\thispagestyle{empty}
\newpage
\fi}



%----------------------------------------------------------------------------------------
%    THEOREM STYLES
%----------------------------------------------------------------------------------------

\newcommand{\intoo}[2]{\mathopen{]}#1\,;#2\mathclose{[}}
\newcommand{\ud}{\mathop{\mathrm{{}d}}\mathopen{}}
\newcommand{\intff}[2]{\mathopen{[}#1\,;#2\mathclose{]}}
\newtheorem{notation}{Notation}[chapter]

%%%%%%%%%%%%%%%%%%%%%%%%%%%%%%%%%%%%%%%%%%%%%%%%%%%%%%%%%%%%%%%%%%%%%%%%%%%
%%%%%%%%%%%%%%%%%%%% dedicated to boxed/framed environements %%%%%%%%%%%%%%
%%%%%%%%%%%%%%%%%%%%%%%%%%%%%%%%%%%%%%%%%%%%%%%%%%%%%%%%%%%%%%%%%%%%%%%%%%%
\newtheoremstyle{ocrenumbox}% % Theorem style name
{0pt}% Space above
{0pt}% Space below
{\normalfont}% % Body font
{}% Indent amount
{\small\bf\sffamily\color{ocre}}% % Theorem head font
{\;}% Punctuation after theorem head
{0.25em}% Space after theorem head
{\small\sffamily\color{ocre}\thmname{#1}\nobreakspace\thmnumber{\@ifnotempty{#1}{}\@upn{#2}}% Theorem text (e.g. Theorem 2.1)
%\thmnote{\nobreakspace\the\thm@notefont\sffamily\bfseries\color{black}---\nobreakspace#3.}} % Optional theorem note
\thmnote{\nobreakspace\the\thm@notefont\bfseries\color{black}---\nobreakspace#3.}} % Optional theorem note
\renewcommand{\qedsymbol}{$\blacksquare$}% Optional qed square

\newtheoremstyle{blacknumex}% Theorem style name
{5pt}% Space above
{5pt}% Space below
{\normalfont}% Body font
{} % Indent amount
%{\small\bf\sffamily}% Theorem head font
{\small\bf}% Theorem head font
{\;}% Punctuation after theorem head
{0.25em}% Space after theorem head
%{\small\sffamily{\tiny\ensuremath{\blacksquare}}\nobreakspace\thmname{#1}\nobreakspace\thmnumber{\@ifnotempty{#1}{}\@upn{#2}}% Theorem text (e.g. Theorem 2.1)
{\small{\tiny\ensuremath{\blacksquare}}\nobreakspace\thmname{#1}\nobreakspace\thmnumber{\@ifnotempty{#1}{}\@upn{#2}}% Theorem text (e.g. Theorem 2.1)
%\thmnote{\nobreakspace\the\thm@notefont\sffamily\bfseries---\nobreakspace#3.}}% Optional theorem note
\thmnote{\nobreakspace\the\thm@notefont\bfseries---\nobreakspace#3.}}% Optional theorem note

\newtheoremstyle{blacknumbox} % Theorem style name
{0pt}% Space above
{0pt}% Space below
{\normalfont}% Body font
{}% Indent amount
{\small\bf}% Theorem head font
%{\small\bf\sffamily}% Theorem head font
{\;}% Punctuation after theorem head
{0.25em}% Space after theorem head
%{\small\sffamily\thmname{#1}\nobreakspace\thmnumber{\@ifnotempty{#1}{}\@upn{#2}}% Theorem text (e.g. Theorem 2.1)
{\small\thmname{#1}\nobreakspace\thmnumber{\@ifnotempty{#1}{}\@upn{#2}}% Theorem text (e.g. Theorem 2.1)
%\thmnote{\nobreakspace\the\thm@notefont\sffamily\bfseries---\nobreakspace#3.}}% Optional theorem note
\thmnote{\nobreakspace\the\thm@notefont\bfseries---\nobreakspace#3.}}% Optional theorem note

%%%%%%%%%%%%%%%%%%%%%%%%%%%%%%%%%%%%%%%%%%%%%%%%%%%%%%%%%%%%%%%%%%%%%%%%%%%
%%%%%%%%%%%%% dedicated to non-boxed/non-framed environements %%%%%%%%%%%%%
%%%%%%%%%%%%%%%%%%%%%%%%%%%%%%%%%%%%%%%%%%%%%%%%%%%%%%%%%%%%%%%%%%%%%%%%%%%
\newtheoremstyle{ocrenum}% % Theorem style name
{5pt}% Space above
{5pt}% Space below
{\normalfont}% % Body font
{}% Indent amount
%{\small\bf\sffamily\color{ocre}}% % Theorem head font
{\small\bf\color{ocre}}% % Theorem head font
{\;}% Punctuation after theorem head
{0.25em}% Space after theorem head
{\small\sffamily\color{ocre}\thmname{#1}\nobreakspace\thmnumber{\@ifnotempty{#1}{}\@upn{#2}}% Theorem text (e.g. Theorem 2.1)
%\thmnote{\nobreakspace\the\thm@notefont\sffamily\bfseries\color{black}---\nobreakspace#3.}} % Optional theorem note
\thmnote{\nobreakspace\the\thm@notefont\bfseries\color{black}---\nobreakspace#3.}} % Optional theorem note
\renewcommand{\qedsymbol}{$\blacksquare$}% Optional qed square
\makeatother

% Defines the theorem text style for each type of theorem to one of the three styles above
\newcounter{dummy} 
\numberwithin{dummy}{section}
\theoremstyle{ocrenumbox}
\newtheorem{theoremeT}[dummy]{Theorem}
\newtheorem{problem}{Problem}[chapter]
\newtheorem{exerciseT}{Exercise}[chapter]
\theoremstyle{blacknumex}
\newtheorem{exampleT}{Example}[chapter]
\theoremstyle{blacknumbox}
\newtheorem{vocabulary}{Vocabulary}[chapter]
\newtheorem{definitionT}{Definition}[section]
\newtheorem{corollaryT}[dummy]{Corollary}
\theoremstyle{ocrenum}
\newtheorem{proposition}[dummy]{Proposition}


%******************************************************
%******************************************************

%----------------------------------------------------------------------------------------
%    DEFINITION OF COLORED BOXES
%----------------------------------------------------------------------------------------

\RequirePackage[framemethod=default]{mdframed} % Required for creating the theorem, definition, exercise and corollary boxes

% Theorem box
\newmdenv[skipabove=7pt,
skipbelow=7pt,
backgroundcolor=black!5,
linecolor=ocre,
innerleftmargin=5pt,
innerrightmargin=5pt,
innertopmargin=5pt,
leftmargin=0cm,
rightmargin=0cm,
innerbottommargin=5pt]{tBox}

% Exercise box      
\newmdenv[skipabove=7pt,
skipbelow=7pt,
rightline=false,
leftline=true,
topline=false,
bottomline=false,
backgroundcolor=ocre!10,
linecolor=ocre,
innerleftmargin=5pt,
innerrightmargin=5pt,
innertopmargin=5pt,
innerbottommargin=5pt,
leftmargin=0cm,
rightmargin=0cm,
linewidth=4pt]{eBox}    

% Definition box
\newmdenv[skipabove=7pt,
skipbelow=7pt,
rightline=false,
leftline=true,
topline=false,
bottomline=false,
linecolor=ocre,
backgroundcolor=bk,%
innerleftmargin=5pt,
innerrightmargin=5pt,
innertopmargin=0pt,
leftmargin=0cm,
rightmargin=0cm,
linewidth=4pt,
innerbottommargin=0pt]{dBox}    

% Corollary box
\newmdenv[skipabove=7pt,
skipbelow=7pt,
rightline=false,
leftline=true,
topline=false,
bottomline=false,
linecolor=gray,
backgroundcolor=black!5,
innerleftmargin=5pt,
innerrightmargin=5pt,
innertopmargin=5pt,
leftmargin=0cm,
rightmargin=0cm,
linewidth=4pt,
innerbottommargin=5pt]{cBox}                  
          

% Corollary box
\newmdenv[skipabove=7pt,%
skipbelow=7pt,%
rightline=false,%
leftline=true,%
topline=false,%
bottomline=false,%
linecolor= orange,% %warning_color,%
backgroundcolor=bk,%
innerleftmargin=5pt,%
innerrightmargin=5pt,%
innertopmargin=0pt,%
leftmargin=0cm,%
rightmargin=0cm,%
linewidth=4pt,%
innerbottommargin=0pt]{zBox}                  
          
% G box
\newmdenv[skipabove=7pt,%
skipbelow=7pt,%
rightline=false,%
leftline=true,%
topline=false,%
bottomline=false,%
linecolor=ashgrey,% %warning_color,%
backgroundcolor=lightgray,%
innerleftmargin=5pt,%
innerrightmargin=5pt,%
innertopmargin=0pt,%
leftmargin=0cm,%
rightmargin=0cm,%
linewidth=8pt,%
innerbottommargin=0pt]{gBox} 

\newmdenv[skipabove=7pt,%
skipbelow=7pt,%
rightline=false,%
leftline=true,%
topline=false,%
bottomline=false,%
linecolor=title_color,
backgroundcolor=bk,%
innerleftmargin=5pt,%
innerrightmargin=4pt,%
innertopmargin=0pt,%
leftmargin=0cm,%
rightmargin=0cm,%
linewidth=12pt,%
innerbottommargin=0pt]{listingBox}
          
% Definition box
\newmdenv[skipabove=7pt,
skipbelow=7pt,
rightline=false,
leftline=true,
topline=false,
bottomline=false,
linecolor=linkcolor,
innerleftmargin=5pt,
innerrightmargin=5pt,
innertopmargin=0pt,
leftmargin=0cm,
rightmargin=0cm,
linewidth=4pt,
innerbottommargin=0pt]{SeeAlsoBox}    

% Creates an environment for each type of theorem and assigns it a theorem text style from the "Theorem Styles" section above and a colored box from above
\newenvironment{theorem}{\begin{tBox}\begin{theoremeT}}{\end{theoremeT}\end{tBox}}
\newenvironment{exercise}{\begin{eBox}\begin{exerciseT}}{\hfill{\color{ocre}\tiny\ensuremath{\blacksquare}}\end{exerciseT}\end{eBox}}        
        
\newenvironment{definition}{\begin{dBox}\begin{definitionT}}{\end{definitionT}\end{dBox}}    
\newenvironment{pHeader}{\begin{dBox}\begin{tabular}[h]{lp{34em}}}{\end{tabular}\end{dBox}}    

\newenvironment{fHeader}{\begin{zBox}\begin{tabular}[h]{lp{28em}}}{\end{tabular}\end{zBox}}  

\newenvironment{cHeader}{\begin{gBox}\begin{tabular}[h]{lp{48em}}}{\end{tabular}\end{gBox}} 

\newenvironment{smallTblBits}{\begin{tabular}[h]{llp{31em}}}{\end{tabular}}  
%\newenvironment{pHeader}{\begin{dBox}\begin{longtable}[h]{lp{26em}}}{\addtocounter{table}{-1}\end{longtable}\end{dBox}}    

%\newenvironment{pHeader}{\begin{dBox}}{\end{dBox}}    
\newenvironment{pExample}{\begin{zBox}}{\end{zBox}}
\newenvironment{listingExample}{\begin{listingBox}}{\end{listingBox}}
\newenvironment{gExample}{\begin{cBox}}{\end{cBox}}
\newenvironment{example}{\begin{exampleT}}{\hfill{\tiny\ensuremath{\blacksquare}}\end{exampleT}}        
\newenvironment{corollary}{\begin{cBox}\begin{corollaryT}}{\end{corollaryT}\end{cBox}}    

%\newenvironment{MyDescription}{\begin{tabular}[h]{p{18em}p{14em}}}{\end{tabular}}    

%\newenvironment{SeeAlsoList}{\begin{SeeAlsoBox}\begin{tabular}[t]{lp{23em}}}{\end{tabular}\end{SeeAlsoBox}}    
\newenvironment{SeeAlsoList}{\begin{SeeAlsoBox}\begin{longtable}[h]{lp{23em}}}{\addtocounter{table}{-1}\end{longtable}\end{SeeAlsoBox}}    

\newenvironment{SeeAlsoLstAcc24}{\begin{SeeAlsoBox}\begin{longtable}[h]{m{7em}m{28em}}}{ \addtocounter{table}{-1}\end{longtable}\end{SeeAlsoBox}
%\newenvironment{SeeAlsoLstAcc24}{\begin{SeeAlsoBox}\begin{tabular}[h]{p{10em}p{35em}}}{\end{tabular}\end{SeeAlsoBox}
}    


% форматирование нумерованных списков
\renewcommand{\labelenumii}{\arabic{enumi}.\arabic{enumii}.}


%----------------------------------------------------------------------------------------
%    REMARK ENVIRONMENT
%----------------------------------------------------------------------------------------

\newenvironment{remark}{%
    \par\vskip10pt\small % Vertical white space above the remark and smaller font size
    \begin{list}{}{
        \leftmargin=35pt % Indentation on the left
        \rightmargin=25pt}\item\ignorespaces % Indentation on the right
        \makebox[-2.5pt]{%
            \begin{tikzpicture}[overlay]%
                \node[draw=remark_color,line width=1pt,circle,%
                fill=remark_color,font=\bfseries,inner sep=2pt,%
%                fill=remark_color,font=\sffamily\bfseries,inner sep=2pt,%
                outer sep=0pt] at (-15pt,0pt){\normalfont\bfseries\large\textcolor{white}{i}};%
            \end{tikzpicture}} % Orange R in a circle
        \advance\baselineskip -1pt}
    {\end{list}\vskip5pt%
} % Tighter line spacing and white space after remark

%----------------------------------------------------------------------------------------
%    WARNING ENVIRONMENT
%----------------------------------------------------------------------------------------

\newenvironment{warning}{%
    \par\vskip10pt\small % Vertical white space above the warning and smaller font size
    \begin{list}{}{%
        \leftmargin=35pt % Indentation on the left
        \rightmargin=25pt}\item\ignorespaces % Indentation on the right
        \makebox[-2.5pt]{%
            \begin{tikzpicture}[overlay]%
                \node[draw=warning_color,line width=1pt,circle,%
%                fill=white,font=\sffamily\bfseries,inner sep=2pt,%
                fill=white,font=\bfseries,inner sep=2pt,%
                outer sep=0pt] at (-15pt,0pt){\normalfont\bfseries\large\textcolor{warning_color}{!}};%
            \end{tikzpicture}} % Orange R in a circle
        \advance\baselineskip -1pt}% 
    {\end{list}\vskip5pt%
} % Tighter line spacing and white space after warning

%----------------------------------------------------------------------------------------
%    SECTION NUMBERING IN THE MARGIN
%----------------------------------------------------------------------------------------

\makeatletter

\begin{comment}

    \renewcommand{\@seccntformat}[1]{%
        \llap{%
            \textcolor{ocre}%
            {\csname the#1\endcsname}%
            \hspace{1em}%
        }%
    }%                    
\end{comment}

\begin{comment}

    \renewcommand{\@seccntformat}[1]{%
     \csname the#1\endcsname.\quad
    }


% настройка нумерации

    \renewcommand{\@seccntformat}[1]{%
        \llap{%
            %\textcolor{ocre}%
            {\csname the#1\endcsname}%            %.\quad%
            \hspace{5mm}%
        }%    
    }%
\end{comment}

% 
\begin{comment}

%    Стиль оформления section
    \renewcommand{\section}{\@startsection
        {section}%                                    name
        {1}%                                        level    
        {\z@}%                                        indent
        {-4ex \@plus -1ex \@minus -.4ex}%            space above header
        {1ex \@plus.2ex }%                            space under header
        {\normalfont\LARGE\bfseries}%                style
    }%
\end{comment}

%    Стиль оформления section
    \renewcommand{\section}{\@startsection%
        {section}%                                     name
           {1}%                                           level
%        {\z@}%                                        indent
           {0mm}%                                         indent
%        {-4ex \@plus -1ex \@minus -.4ex}%            space above header
        {-5.5ex \@plus -1.5ex \@minus -.3ex}%                space above header
%           {-3\baselineskip}%                           space above header
%        {1ex \@plus.2ex }%                            space under header
        {3ex \@plus1.5ex }%                            space under header
%          {1.5\baselineskip}%                            space under header
           {\normalfont\Large\bfseries}%                 style
    }%

%\begin{comment}
    \renewcommand{\subsection}{\@startsection%
        {subsection}%                                name
        {2}%                                        level    
%        {\z@}%                                        indent
           {0mm}%                                         indent
%        {-3ex \@plus -0.1ex \@minus -.4ex}%            space above header
        {-4ex \@plus -1.5ex \@minus -0.2ex}%            space above header
%       {-4ex \@plus -1.5ex \@minus -0.2ex}%       ace above header  last settings      
%        {-3.5ex \@plus -1ex \@minus -.2ex}%            space above header
%        {-5ex \@plus -0.1ex \@minus -.8ex}%            space above header
%        {-1.5\baselineskip}%                           space above header
%        {1.5ex}%                        space under header
         {2ex \@plus1.5ex }%                        space under header
%        {1ex \@plus 0.5ex \@minus -1ex}%                        space under header
%        {1ex \@plus.2ex }%                            space under header
%        {1ex \@plus -0.8ex \@minus-.8ex }%            space under header
%          {0.5\baselineskip}%                            space under header
        {\normalfont\large\bfseries}%                        style
    }%
%\end{comment}

\begin{comment}
    \renewcommand{\subsection}{\@startsection%
        {subsection}%                                name
        {2}%                                        level    
%        {\z@}%                                        indent
           {0mm}%                                         indent
        {-5ex \@plus -20ex \@minus -2.5ex}%            space above header
        {-1.5ex \@minus -1ex}%                        space under header
        {\normalfont\bfseries}%                        style
    }%
\end{comment}



    \renewcommand{\subsubsection}{\@startsection%    
        {subsubsection}%                              name
        {3}%                                          level
        {0mm}%                                        indent
        {-2ex \@plus -1ex \@minus -.2ex}%           space above header
        {2ex \@plus.2ex }%                          space under header
        {\normalfont\bfseries}%                       style
    }                    
    
%\begin{comment}
    
    \titleformat{\subsubsection} [runin]
      {}
      {\thesubsubsection}
      {1ex}{}[.]
    \titlespacing*{\subsubsection}{\parindent}{*4}{1ex}
    
%\end{comment}
    
    
    \renewcommand\paragraph{%
        \@startsection{paragraph}{4}{\z@}%
        {-2ex \@plus-.2ex \@minus .2ex}%
        {0.1ex}%
%        {\normalfont\small\sffamily\bfseries}%
        {\normalfont\small\bfseries}%
    }%

\makeatother
    
    

    
    
    %----------------------------------------------------------------------------------------
    %    CHAPTER HEADINGS
    %----------------------------------------------------------------------------------------



\begin{comment}
\makeatletter
    
    \newcommand{\thechapterimage}{}
    \newcommand{\chapterimage}[1]{\renewcommand{\thechapterimage}{#1}}
    \def\thechapter{\arabic{chapter}}
    \def\@makechapterhead#1{
    \thispagestyle{empty}
%    {\centering \normalfont\sffamily
    {\centering \normalfont
    \ifnum \c@secnumdepth >\m@ne
    \if@mainmatter
    \startcontents
    \begin{tikzpicture}[remember picture,overlay]
    \node at (current page.north west)
    {\begin{tikzpicture}[remember picture,overlay]
    
    \node[anchor=north west,inner sep=0pt] at (0,0) {\includegraphics[width=\paperwidth]{\thechapterimage}};
    %\node[anchor=north west,inner sep=0pt] at (0,0) {\IncludeEpsFromSvg[width=\paperwidth]{\thechapterimage}};
    %Commenting the 3 lines below removes the small contents box in the chapter heading
    %\draw[fill=white,opacity=.6] (1cm,0) rectangle (8cm,-7cm);
    %\node[anchor=north west] at (1cm,.25cm) {\parbox[t][8cm][t]{6.5cm}{\huge\bfseries\flushleft \printcontents{l}{1}{\setcounter{tocdepth}{2}}}};
%\normalfont\Large\bfseries    
    \draw[anchor=west] (1.5cm,-8.4cm) node [rounded corners=25pt,fill=white,fill opacity=.6,text opacity=1,draw=ocre,draw opacity=1,line width=2pt,inner sep=15pt]{\normalfont\Large\bfseries\textcolor{black}{\thechapter\ ---\ #1\vphantom{plPQq}\makebox[22cm]{}}};    
    \end{tikzpicture}};
    \end{tikzpicture}}\par\vspace*{230\p@}
    \fi
    \fi
    }
    \def\@makeschapterhead#1{
    \thispagestyle{empty}
%    {\centering \normalfont\sffamily
    {\centering \normalfont
    \ifnum \c@secnumdepth >\m@ne
    \if@mainmatter
    \startcontents
    \begin{tikzpicture}[remember picture,overlay]
    \node at (current page.north west)
    {\begin{tikzpicture}[remember picture,overlay]
    \node[anchor=north west] at (-4pt,4pt) {\includegraphics[width=\paperwidth]{\thechapterimage}};
    %\node[anchor=north west] at (-4pt,4pt) {\IncludeEpsFromSvg[width=\paperwidth]{\thechapterimage}};
%    \draw[anchor=west] (5cm,-9cm) node [rounded corners=25pt,fill=white,opacity=.7,inner sep=15.5pt]{\huge\sffamily\bfseries\textcolor{black}{\vphantom{plPQq}\makebox[22cm]{}}};
    \draw[anchor=west] (5cm,-9cm) node [rounded corners=25pt,fill=white,opacity=.7,inner sep=15.5pt]{\huge\bfseries\textcolor{black}{\vphantom{plPQq}\makebox[22cm]{}}};    
%    \draw[anchor=west] (5cm,-9cm) node [rounded corners=25pt,draw=ocre,line width=2pt,inner sep=15pt]{\huge\sffamily\bfseries\textcolor{black}{#1\vphantom{plPQq}\makebox[22cm]{}}};
    \draw[anchor=west] (5cm,-9cm) node [rounded corners=25pt,draw=ocre,line width=2pt,inner sep=15pt]{\huge\bfseries\textcolor{black}{#1\vphantom{plPQq}\makebox[22cm]{}}};    
    \end{tikzpicture}};
    \end{tikzpicture}}\par\vspace*{230\p@}
    \fi
    \fi
    }
\makeatother



\definecolor{gray75}{gray}{0.75} % определяем цвет
\newcommand{\hsp}{\hspace{10pt}} % длина линии в 20pt
% titleformat определяет стиль
\titleformat{\chapter}[hang]{\Huge\bfseries}{\thechapter\hsp\textcolor{gray75}{|}\hsp}{0pt}{\Huge\bfseries}
\end{comment}

\makeatletter
    
    \newcommand{\thechapterimage}{}
    \newcommand{\chapterimage}[1]{\renewcommand{\thechapterimage}{#1}}
    \def\thechapter{\arabic{chapter}}
    \def\@makechapterhead#1{
    \thispagestyle{empty}
%    {\centering \normalfont\sffamily
    {\centering \normalfont
    \ifnum \c@secnumdepth >\m@ne
    \if@mainmatter
    \startcontents
    \begin{tikzpicture}[remember picture,overlay]
    \node at (current page.north west)
    {\begin{tikzpicture}[remember picture,overlay]
    
    \node[anchor=north west,inner sep=0pt] at (0,0) {\includegraphics[width=\paperwidth]{\thechapterimage}};
    %\node[anchor=north west,inner sep=0pt] at (0,0) {\IncludeEpsFromSvg[width=\paperwidth]{\thechapterimage}};
    %Commenting the 3 lines below removes the small contents box in the chapter heading
    %\draw[fill=white,opacity=.6] (1cm,0) rectangle (8cm,-7cm);
    %\node[anchor=north west] at (1cm,.25cm) {\parbox[t][8cm][t]{6.5cm}{\huge\bfseries\flushleft \printcontents{l}{1}{\setcounter{tocdepth}{2}}}};
%\normalfont\Large\bfseries    
    \draw[anchor=west] (1.5cm,-6.3cm) node [rounded corners=25pt,fill=white,fill opacity=.8,text opacity=1,draw=ocre, draw opacity=1,line width=2pt,inner sep=15pt]{\normalfont\LARGE\bfseries\textcolor{black}{\thechapter\ .\ #1\vphantom{plPQq}\makebox[22cm]{}}};    
    \end{tikzpicture}};
    \end{tikzpicture}}\par\vspace*{150\p@}
    \fi
    \fi
    }
    \def\@makeschapterhead#1{
    \thispagestyle{empty}
%    {\centering \normalfont\sffamily
    {\centering \normalfont
    \ifnum \c@secnumdepth >\m@ne
    \if@mainmatter
    \startcontents
    \begin{tikzpicture}[remember picture,overlay]
    \node at (current page.north west)
    {\begin{tikzpicture}[remember picture,overlay]
    \node[anchor=north west] at (-4pt,4pt) {\includegraphics[width=\paperwidth]{\thechapterimage}};
    %\node[anchor=north west] at (-4pt,4pt) {\IncludeEpsFromSvg[width=\paperwidth]{\thechapterimage}};
%    \draw[anchor=west] (5cm,-9cm) node [rounded corners=25pt,fill=white,opacity=.7,inner sep=15.5pt]{\huge\sffamily\bfseries\textcolor{black}{\vphantom{plPQq}\makebox[22cm]{}}};
    \draw[anchor=west] (5cm,-6.3cm) node [rounded corners=25pt,fill=white,opacity=.8,inner sep=15.5pt]{\normalfont\LARGE\bfseries\textcolor{black}{\vphantom{plPQq}\makebox[22cm]{}}};    
%    \draw[anchor=west] (5cm,-9cm) node [rounded corners=25pt,draw=ocre,line width=2pt,inner sep=15pt]{\huge\sffamily\bfseries\textcolor{black}{#1\vphantom{plPQq}\makebox[22cm]{}}};
    \draw[anchor=west] (5cm,-6.3cm) node [rounded corners=25pt,draw=ocre,line width=2pt,inner sep=15pt]{\normalfont\LARGE\bfseries\textcolor{black}{#1\vphantom{plPQq}\makebox[22cm]{}}};    
    \end{tikzpicture}};
    \end{tikzpicture}}\par\vspace*{150\p@}
    \fi
    \fi
    }
\makeatother

%----------------------------------------------------------------------------------------
%    Оформление ОГЛАВЛЕНИя
%----------------------------------------------------------------------------------------

% точки после названий частей !!!
\renewcommand\cftchapaftersnum{.}

%все номера частей, секций и т.п. и точки - одним цветом !
\renewcommand\cftchappagefont{\color{my_color}}

\renewcommand\cftchappagefont{\color{my_color}}
\renewcommand\cftchapleader{\color{my_color}\cftdotfill{\cftdotsep}}

\renewcommand\cftsecpagefont{\color{my_color}}
\renewcommand\cftsecleader{\color{my_color}\cftdotfill{\cftsecdotsep}}

\renewcommand\cftsubsecpagefont{\color{my_color}}
\renewcommand\cftsubsecleader{\color{my_color}\cftdotfill{\cftsubsecdotsep}}


% отступы между частями!!!
%\preto\section{\ifnum\value{section}=0\addtocontents{toc}{\vskip10pt}\fi}
\makeatletter
\pretocmd{\chapter}{\addtocontents{toc}{\protect\addvspace{7\p@}}}{}{}
\pretocmd{\section}{\addtocontents{toc}{\protect\addvspace{7\p@}}}{}{}
\pretocmd{\subsection}{\addtocontents{toc}{\protect\addvspace{5\p@}}}{}{}
\pretocmd{\subsubsection}{\addtocontents{toc}{\protect\addvspace{5\p@}}}{}{}
\makeatother


%\preto\subsection{\ifnum\value{subsection}=0\addtocontents{toc}{\vskip7pt}\fi}

%\preto\subsection{\addtocontents{toc}{\vskip7pt}}

% отступы!!! (для subsection - обязательно оставить!!!)

    \setlength{\cftsecindent}{7pt}  % Indent of section No.
    \setlength{\cftsecnumwidth}{34pt}  % Width of section No.

    \setlength{\cftsubsecindent}{21pt}  % Indent of subsection No.
    \setlength{\cftsubsecnumwidth}{52pt}  % Width of subsection No.

    \setlength{\cftsubsubsecindent}{35pt}  % Indent of subsubsection No.
    \setlength{\cftsubsubsecnumwidth}{62pt}  % Width of subsubsection No.

\begin{comment}
\contentsmargin{0cm} % Removes the default margin
% Chapter text styling
\titlecontents{chapter}[1.25cm] % Indentation
{\addvspace{15pt}\large\sffamily\bfseries} % Spacing and font options for chapters
{\color{ocre!60}\contentslabel[\Large\thecontentslabel]{1.25cm}\color{ocre}} % Chapter number
{}  
{\color{ocre!60}\normalsize\sffamily\bfseries\;\titlerule*[.5pc]{.}\;\thecontentspage} % Page number
% Section text styling
\titlecontents{section}[1.25cm] % Indentation
{\addvspace{5pt}\sffamily\bfseries} % Spacing and font options for sections
{\contentslabel[\thecontentslabel]{1.25cm}} % Section number
{}
{\sffamily\hfill\color{black}\thecontentspage} % Page number
[]
% Subsection text styling
\titlecontents{subsection}[1.25cm] % Indentation
{\addvspace{1pt}\sffamily\small} % Spacing and font options for subsections
{\contentslabel[\thecontentslabel]{1.25cm}} % Subsection number
{}
{\sffamily\;\titlerule*[.5pc]{.}\;\thecontentspage} % Page number
[] 

\end{comment}


% % % % % % % % % % % Настройка параметров и переопределение предметного указателя % % % % %
\makeatletter
% 2-й параметр - отступ от левого края основного текста
% 1-й параметр - сдвиг (от левого края) при переносе текста на новую строку
\begin{comment}
\renewcommand{\@idxitem}{\par\hangindent=7pt\hspace*{0pt}}

%\renewcommand{\subitem}{\par\hangindent=70pt\hspace*{12pt}}
\renewcommand{\subitem}{\par\hangindent=21pt\hspace*{14pt}}

%\renewcommand{\subsubitem}{\par\hangindent=35pt\hspace*{24pt}}
\renewcommand{\subsubitem}{\par\hangindent=35pt\hspace*{28pt}}
\end{comment}

\renewcommand{\@idxitem}{\par\hangindent=5pt\hspace*{0pt}}

%\renewcommand{\subitem}{\par\hangindent=70pt\hspace*{12pt}}
\renewcommand{\subitem}{\par\hangindent=17pt\hspace*{10pt}}

%\renewcommand{\subsubitem}{\par\hangindent=35pt\hspace*{24pt}}
\renewcommand{\subsubitem}{\par\hangindent=28pt\hspace*{21pt}}


\renewenvironment*{theindex}{\columnseprule=0pt\columnsep=15pt
\@makeschapterhead{\indexname}%
\@mkboth{\uppercase{\indexname}}{\uppercase{\indexname}}%
\thispagestyle{plain}\parindent=0pt
\setlength{\parskip}{0pt plus .3pt}%
\let\item=\@idxitem
\begin{multicols}{2}}%
{\end{multicols}}

\makeatother



%%%***************************%%%
%%%***    Листинги         ***%%%
%%%***************************%%%


%\color{frame}

\begin{comment}
\makeatletter

\renewenvironment{snugshade}{%
 \def\FrameCommand##1{\hskip\@totalleftmargin \hskip-\fboxsep
 \colorbox{shadecolor}{##1}\hskip-\fboxsep
     % There is no \@totalrightmargin, so:
     \hskip-\linewidth \hskip-\@totalleftmargin \hskip\columnwidth}%
 \MakeFramed {\advance\hsize-\width
   \@totalleftmargin\z@ \linewidth\hsize
   \@setminipage}}%
 {\par\unskip\endMakeFramed}

\makeatother
\end{comment}

%!!! определение стиля для служебных слов

\newcommand{\KeyWordUpcaseBold}{\normalfont\bfseries}%  нужен знак *
\newcommand{\KeyWordLocaseBold}{\normalfont\bfseries}%    без знака *
\newcommand{\KeyWordUpcase}{\bfseries}%                   нужен знак *
\newcommand{\KeyWordLocase}{\bfseries}%                      без знака *

%\newcommand{\KeyWordSt}{\bfseries}%

%    *\normalfont   - для UPcase
%     \normalfont\bfseries   - для LOcase

\newcommand{\KeyWordSt}{\KeyWordLocaseBold}%

%   \bfseries - lo case
%  *\bfseries - up case
% просто \bfseries не дает нужного эффекта, нужен \bf, но он только на аргумент
% хорошая альтернатива для нормального болда: \normalfont\bfseries
% 

\ifKeyWordUpCase

    \newcommand{\MyKeyWordSt}{%
        *\normalfont}%
        
    }%

\else
    \newcommand{\MyKeyWordSt}{%
        \normalfont\bfseries%
    }%
\fi


\lstdefinelanguage{cnc_lang}
{%
    classoffset=0,
    morekeywords={abort, abs, adisable, begin, bstart, bstop,% break,
    cclr, ccmode0, ccmode1, ccmode2, ccmode3, ccr,  cexec,% 
    cmd, continue, cout, cset, cskip, D, ddisable,  delay,%default,
    disable, bgcplc, rticplc, dkill, do, dread, dtogread, dwell,%
    enable, F, frax, fread, G, hold, home, homez, inc, jog, jogret,%
%    kill, lh, lhpurge, M, N, nofrax, normal, nxyz, pause, pclear, pload,% 
    kill, lh, lhpurge, M, nofrax, normal, nxyz, pause, pclear, pload,% 
    pmatch, pread, pset, pstore, read, resume, return, run, s, S, send,% 
    sendall, sendallcmds, sendallsystemsmds, start, step, system, %stop,
    struct,union, void, __attribute__%  
    T, td, tm, tread, tsel, tsel1, tsel2, tsel3, txyz,%ts,
%    rot, lookahead, plc, plcc, undefine, r, R, A, b, B, c, C, S,% 
    rot, lookahead, plc, plcc, undefine, r, R, b, B, c, C, S,% 
    p, q, v, h, u, U, % 
%    a, j, p, q, v, k, h, u, U, I, i, % 
    open, inverse, forward, prog, prog1000, close, clear, delete, gather,%
    define, all, rotary, ecat, echo, buffer, buffers, hmz, hm, free,%
    program, list, kinematic, backup, bpclear, bpclearall, bpset, %
    brickacver, bricklvver, cpu, cpx, cx, alias, assign, config, slaves,%
%    apc, pc, subprog, out, reboot, rotfree, save, size, string, type, date,%  
    apc, pc, subprog, out, reboot, rotfree, save, size, string, date,%  
    G00, G01, G02, G03, G04, G05, G06, G07, G08, G09,%
    G10, G11, G12, G13, G14, G15, G16, G17, G18, G19,%
    G20, G21, G22, G23, G24, G25, G26, G27, G28, G29,%
    G30, G31, G32, G33, G34, G35, G36, G37, G38, G39,%
    G40, G41, G42, G43, G44, G45, G46, G47, G48, G49,%
    G50, G51, G52, G53, G54, G55, G56, G57, G58, G59,%
    G60, G61, G62, G63, G64, G65, G66, G67, G68, G69,%
    G70, G71, G72, G73, G74, G75, G76, G77, G78, G79,%
    G80, G81, G82, G83, G84, G85, G86, G87, G88, G89,%
    G90, G91, G92, G93, G94, G95, G96, G97, G98, G99,%
    G100,Gnn, G115, F, tm, Global, csglobal,%ts, ta,
    },%
%    keywordstyle=*\ttfamily\color{key_word_1_color},%
    keywordstyle=\ttfamily\normalfont\bfseries\color{key_word_1_color},%
    classoffset=1,%
    morekeywords={call, callsub, case, if, else, endif, and, or, endwhile, while,%
    , wait, goto, gosub, switch, break, default,},%
%    keywordstyle=*\ttfamily\color{key_word_2_color},%
    keywordstyle=\ttfamily\normalfont\bfseries\color{key_word_2_color},%
    classoffset=2,%
    morekeywords={circle, circle1, circle2, circle3, circle4, linear,pvt,%
    rapid,spline, spline1, spline2, circlen, splinen,},%
%    keywordstyle=*\ttfamily\color{key_word_3_color},%
    keywordstyle=\ttfamily\normalfont\bfseries\color{key_word_3_color},%
    classoffset=3,
    morekeywords={cos, sin, tan, asin, acos, atan2,%
%    morekeywords={cos, sin, tan, asin, acos, atan, atan2,%
    int, double, int32_t, uint32_t, exp, ln, sqrt},%
%    keywordstyle=*\ttfamily\color{key_word_4_color},%
    keywordstyle=\ttfamily\normalfont\bfseries\color{key_word_4_color},%
    classoffset=0,
    %otherkeywords={$, &, $},
    %otherkeywords={&,\#},
%    otherkeywords={\#,\$,&,\%},
    otherkeywords={\#,\%},    %&,
     morecomment=[l]{//},% l is for line comment
    morecomment=[l]{;},%
    morecomment=[s]{/*}{*/}, % s is for start and end delimiter
    %morestring=[b]" % defines that strings are enclosed in double quotes
% Цвета
    stringstyle=\color{string}%
    commentstyle=\ttfamily\normalfont\footnotesize\bfseries\itshape\color{comment},%
        sensitive=false,%
%    commentstyle=\sffamily\footnotesize\itshape\bfseries\color{comment},%
%    commentstyle=\ttfamily\normalfont\bfseries\itshape\color{comment},%
    commentstyle=\ttfamily\normalfont\footnotesize\bfseries\itshape\color{comment},%
    %numberstyle=\color{kew_word_2_color},
    postbreak=\space, 
    breakindent=5pt, 
    breaklines,                            % Перенос длинных строк
    breakatwhitespace,
%    xleftmargin = 12pt,                % отступ слева
    lineskip = 5pt,                        % расстояние между строками
    emptylines =*1,                        % максимальное количество "пустых" линий между строками - остальные просто режутся
}

%breaklines=true,                        % Automatic line breaking?
%   breakatwhitespace=true,                % Automatic breaks only at whitespace
   

% Настройки отображения кода
\lstset{%
    language=cnc_lang,                      % Язык кода по умолчанию
%    basicstyle=\footnotesize,              % Шрифт для отображения кода
%    basicstyle=\ttfamily\footnotesize,     % Шрифт для отображения кода
    basicstyle=\ttfamily,     % Шрифт для отображения кода
%    basicstyle=\sffamily\footnotesize,    % Шрифт для отображения кода
%    columns = flexible,
    backgroundcolor=\color{bk},             % Цвет фона кода
    frame=none,
    columns=fixed, % make all characters equal width
%    frame=lrb,xleftmargin=\fboxsep,xrightmargin=-\fboxsep,      % Рамка, подогнанная к заголовку
%    rulecolor=\color{frame},                % Цвет рамки
    tabsize=2,                              % Размер табуляции в пробелах
% Настройка отображения номеров строк. Если не нужно, то удалите весь блок
%    numbers=left,                           % Слева отображаются номера строк
%    stepnumber=1,                           % Каждую строку нумеровать
%    numbersep=5pt,                          % Отступ от кода
%    numberstyle=\small\color{black},        % Стиль написания номеров строк
% Для отображения русского языка
    extendedchars=true,
    literate=
    {~}{{\textasciitilde}}1
    {а}{{\selectfont\char224}}1
    {б}{{\selectfont\char225}}1
    {в}{{\selectfont\char226}}1
    {г}{{\selectfont\char227}}1
    {д}{{\selectfont\char228}}1
    {е}{{\selectfont\char229}}1
    {ё}{{\"e}}1
    {ж}{{\selectfont\char230}}1
    {з}{{\selectfont\char231}}1
    {и}{{\selectfont\char232}}1
    {й}{{\selectfont\char233}}1
    {к}{{\selectfont\char234}}1
    {л}{{\selectfont\char235}}1
    {м}{{\selectfont\char236}}1
    {н}{{\selectfont\char237}}1
    {о}{{\selectfont\char238}}1
    {п}{{\selectfont\char239}}1
    {р}{{\selectfont\char240}}1
    {с}{{\selectfont\char241}}1
    {т}{{\selectfont\char242}}1
    {у}{{\selectfont\char243}}1
    {ф}{{\selectfont\char244}}1
    {х}{{\selectfont\char245}}1
    {ц}{{\selectfont\char246}}1
    {ч}{{\selectfont\char247}}1
    {ш}{{\selectfont\char248}}1
    {щ}{{\selectfont\char249}}1
    {ъ}{{\selectfont\char250}}1
    {ы}{{\selectfont\char251}}1
    {ь}{{\selectfont\char252}}1
    {э}{{\selectfont\char253}}1
    {ю}{{\selectfont\char254}}1
    {я}{{\selectfont\char255}}1
    {А}{{\selectfont\char192}}1
    {Б}{{\selectfont\char193}}1
    {В}{{\selectfont\char194}}1
    {Г}{{\selectfont\char195}}1
    {Д}{{\selectfont\char196}}1
    {Е}{{\selectfont\char197}}1
    {Ё}{{\"E}}1
    {Ж}{{\selectfont\char198}}1
    {З}{{\selectfont\char199}}1
    {И}{{\selectfont\char200}}1
    {Й}{{\selectfont\char201}}1
    {К}{{\selectfont\char202}}1
    {Л}{{\selectfont\char203}}1
    {М}{{\selectfont\char204}}1
    {Н}{{\selectfont\char205}}1
    {О}{{\selectfont\char206}}1
    {П}{{\selectfont\char207}}1
    {Р}{{\selectfont\char208}}1
    {С}{{\selectfont\char209}}1
    {Т}{{\selectfont\char210}}1
    {У}{{\selectfont\char211}}1
    {Ф}{{\selectfont\char212}}1
    {Х}{{\selectfont\char213}}1
    {Ц}{{\selectfont\char214}}1
    {Ч}{{\selectfont\char215}}1
    {Ш}{{\selectfont\char216}}1
    {Щ}{{\selectfont\char217}}1
    {Ъ}{{\selectfont\char218}}1
    {Ы}{{\selectfont\char219}}1
    {Ь}{{\selectfont\char220}}1
    {Э}{{\selectfont\char221}}1
    {Ю}{{\selectfont\char222}}1
    {Я}{{\selectfont\char223}}1
    {і}{{\selectfont\char105}}1
    {ї}{{\selectfont\char168}}1
    {є}{{\selectfont\char185}}1
    {ґ}{{\selectfont\char160}}1
    {І}{{\selectfont\char73}}1
    {Ї}{{\selectfont\char136}}1
    {Є}{{\selectfont\char153}}1
    {Ґ}{{\selectfont\char128}}1
    {\{}{{{\color{brackets}\{}}}1 % Цвет скобок {
    {\}}{{{\color{brackets}\}}}}1 % Цвет скобок }
}




\begin{comment}
\lstdefinestyle{basic}{  
  basicstyle=\footnotesize\ttfamily,
  numbers=left,
  numberstyle=\tiny\color{gray}\ttfamily,
  numbersep=5pt,
  backgroundcolor=\color{white},
  showspaces=false,
  showstringspaces=false,
  showtabs=false,
  frame=single,
  rulecolor=\color{black},
  captionpos=b,
  keywordstyle=\color{blue}\bf,
  commentstyle=\color{gray},
  stringstyle=\color{green},
  keywordstyle={[2]\color{red}\bf},
}
\end{comment}

% по умолчанию - сквозная нумерация!!!
\begin{comment}
%Формат: ХХ.YY.ZZ.PP.number
\AtBeginDocument{%
  \renewcommand{\thelstlisting}{%
    \ifnum\value{subsection}=0
      \thesection.\arabic{lstlisting}%
    \else
      \ifnum\value{subsubsection}=0
        \thesubsection.\arabic{lstlisting}%
      \else
        \thesubsubsection.\arabic{lstlisting}%
      \fi
    \fi
  }
}
\end{comment}

% Формат: ХХ.YY.number
\begin{comment}
\AtBeginDocument{%
  \renewcommand{\thelstlisting}{%
%      \arabic{lstlisting}%
      \thesection.\arabic{lstlisting}%
    %  \textbf{\arabic{lstlisting}}%
  }
}
\end{comment}

%%%*** ---------------------------------- ***%%%


%%%*********************************************%%%
%%%*** Настройка формата подписей к листингам***%%%
%%%*********************************************%%%
%\DeclareCaptionFormat{listing}{\rule{\dimexpr\textwidth+17pt\relax}{0.4pt}\par\vskip1pt#1#2#3}
%\captionsetup[lstlisting]{margin=0pt, labelfont = bf}

\DeclareCaptionFormat{listing}{#1#2#3}
\captionsetup[lstlisting]{format=listing,singlelinecheck=false, margin=0pt,labelsep = space,labelfont = bf}


\newcommand{\IncludeListing}[3]{%
%\begin{snugshade}%
\begin{listingExample}
\lstinputlisting[caption=#2,label=#3]{#1}%
\end{listingExample}

%\end{snugshade}%
}

\newcommand{\IncludeLstWithoutCaption}[1]{
\begin{listingExample}
\lstinputlisting{#1}%
\end{listingExample}
}

\newcommand{\IncludeLstWithoutBorder}[3]{%
\lstinputlisting[caption=#2,label=#3]{#1}%
}

\newcommand{\mylstinline}[1]{%
\colorbox{bk}{\lstinline[backgroundcolor=\color{bk}]{#1}}~%
}%

\renewcommand{\lstlistingname}{\textbf{Пример}}         % %\renewcommand{\lstlistingname}{\textbf{Пример кода программы}}         % Переименование Listings в нужное именование структуры

% подписи к листингам
%\DeclareCaptionFont{white}{\color{сaptiontext}}
%\DeclareCaptionFormat{listing}{\parbox{\linewidth}{\colorbox{сaptionbk}{\parbox{\linewidth}{#1#2#3}}\vskip-4pt}}
%\captionsetup[lstlisting]{format=listing,labelfont=white,textfont=white}
%%%*** -------------------------- ***%%%


%%%*** ------------------- ***%%%


%%%**********************************%%%
%%%*** Настройка формата подписей ***%%%
%%%**********************************%%%
% подписи к рисункам и таблицам
\DeclareCaptionLabelFormat{viadot}{#1 #2.}          % после номера Рис№ и Табл№  
\captionsetup{labelsep=space, labelformat=viadot}   % - стоит точка!!!


\newcommand{\keyword}[1]{%
%    \textbf{\textcolor{key_word_color}{#1}}%
    \lstinline{#1}%
}%


\newcommand{\killoverfullbefore}{%
    {\sloppy%     % чтобы был минимум переполнений (чтобы строки не разрастались

    }
}

\ifbiblatex
    \newcommand{\BiblioCite}[1]{%
        \cite{#1}%
    }
\else
    \newcommand{\BiblioCite}[1]{%
        {}%
    }
\fi


%*******************************************
% стиль названий плат расширения
\newcommand{\DSPGATEthree}{%
    \textbf{DSPGATE3} %
}%



%*******************************************
% стиль структур данных
\newcommand{\mystruct}[1]{%
    \textbf{#1}%
%    \textbf{\nameref{sec:#1}}%
}%

% стиль структур данных
\newcommand{\mystructA}[1]{%
    \textbf{#1}%
%    \textbf{\nameref{sec:#1}}%
}%

% стиль структур данных
\newcommand{\mystructall}[1]{%
    \textbf{#1}%
}%



\setrefcountdefault{0}

%\getrefbykeydefault{}{page}{0}

\newcommand{\myreftosec}[1]{%
    \ifnum \getpagerefnumber{sec:#1}=0% 
        \textbf{#1}%                  если ссылка не определена - просто текст
    \else% 
        \textbf{\nameref{sec:#1}}%    если ссылка определена - текст с гиперссылкой
    \fi
%    \IfRefUndefinedBabe{sec:#1}{%
%        \textbf{#1}%                  если ссылка не определена - просто текст
%    }%
%    {% 
%        \textbf{\nameref{sec:#1}}%    если ссылка определена - текст с гиперссылкой
%    }%
}%

\newcommand{\myreftosecwithpage}[1]{%
    \ifnum \getpagerefnumber{sec:#1}=0% 
        \textbf{#1}%                  если ссылка не определена - просто текст
    \else% 
        \textbf{\nameref{sec:#1}}~(стр.~\pageref{sec:#1})%    если ссылка определена - текст с гиперссылкой + указанием номера страницы
    \fi
}%


\begin{comment}


\ifnum \getpagerefnumber{sec:Gate3[i].Chan[j].UVW} = 0 
    <tex-code-1> 
\else 
    <tex-code-2>
\fi

\end{comment}


% стиль значений элементов структур данных
\newcommand{\myval}[1]{%
    \textit{#1}%
}%


%*******************************************
% стиль названий плат расширения
\newcommand{\myacc}[1]{%
    \textit{\textbf{#1}}%
}%


\newcommand{\mytextBFIT}[1]{%
    \textit{\textbf{#1}}%
}%



%*******************************************
% стиль названий типов обработки ДОС
\newcommand{\mytypeECT}[1]{%
    \textit{\textbf{#1}}%
}%



%*******************************************
% стиль записей справа 
\newcommand{\RightHandText}[1]{%
    \textit{#1}%
}%


%*******************************************
% Диапазоны переменных(полей структур)
                            %Floating-point
\newcommand{\myfloatpoint}{%
    Диапазон чисел с плавающей запятой%
}%

\newcommand{\myinteger}{%
    Диапазон целых чисел%
}%

\newcommand{\mynonneginteger}{%
    Диапазон неотрицательных целых чисел%
}%

                            %Positive floating-point (double-precision)
\newcommand{\mypositfloatp}{%
    Диапазон положительных чисел с плавающей запятой%
}%

                            %Positive floating-point
\newcommand{\mypositfloatpoint}{%
    Диапазон положительных чисел с плавающей запятой%
}%

                            %Non-negative floating-point
\newcommand{\mynonnegfloatp}{%
    Диапазон неотрицательных чисел с плавающей запятой%
}%

                            %Non-negative floating-point
\newcommand{\myposfloatp}{%
    Диапазон положительных чисел с плавающей запятой%
}%

\newcommand{\myflpointsixteenbt}{%
    +/-32768 (число с плавающей запятой)%
}%

%Legitimate addresses
\newcommand{\mylegitimadrs}{%
    Все допустимые адреса%
}%
    
\newcommand{\myautoconf}{%
    Автоматическая настройка %
}%


% требуют уточнения
\newcommand{\myReqwestDetail}{%
    \textcolor{red}{ТРЕБУЕТСЯ УТОЧНЕНИЕ}%
}%


%*******************************************


%*******************************************
% Единицы измерения переменных(полей структур)

                    
\newcommand{\myuserset}{%
    Пользовательская настройка%
}%

\newcommand{\axisunit}{%
    Единицы измерения перемещения по оси%
}%

                             %Data structure element addresses
\newcommand{\mydatastreladr}{%
    Адреса элементов структур данных%
}%

                               %Motor units
\newcommand{\motorunits}{%
    Дискреты положения%
}%

                               %Motor units per millisecond
\newcommand{\unitspermsec}{%
    Дискреты положения/мс%
}%

                               %Motor units per servocycle
\newcommand{\unitspermservo}{%
    Дискреты положения/период сервоцикла%
}%
                               %Motor units per millisecond in sq
\newcommand{\unitspermsecinsq}{%
    Дискреты положения/$\text{мс}^2$%
}%

                             %Motor units per units of source data
\newcommand{\unitspersource}{%
    Дискреты положения/единицы измерения исходных данных%
}%

                             %Motor units per LSB of source data
\newcommand{\unitsperlsbsource}{%
    Дискреты положения/МЗР исходных данных%
}%

                %Milliseconds (if >= 0) or milliseconds2 per motor unit (if < 0)
\newcommand{\unitabortta}{%
    мс (если значение >= 0) или мс$^2$/дискреты положения (если значение < 0)%
}%

                %Milliseconds (if >= 0) or milliseconds3 per motor unit (if < 0)
\newcommand{\unitabortts}{%
    мс (если значение >= 0) или мс$^3$/дискреты положения (если значение < 0)%
}%

                               %Motor units per millisecond in sq
\newcommand{\msecinsqperunit}{%
    $\text{мс}^2$/дискреты положения%
}%

                %LSBs of 16-bit output per motor unit of position error
\newcommand{\mylsbperuniterror}{%
    МЗР 16-битного выхода(ЦАП)/дискреты измерения ошибки положения}%

                         %Bit field
\newcommand{\mybitfield}{%
    Битовое поле%
}%

                         %Bits of a signed 16-bit input/output
\newcommand{\myunitsixteenadcdac}{%
    Биты 16-битного значения АЦП/ЦАП со знаком%
}%

                         %Bit field
\newcommand{\myunitbool}{%
    Логический тип данных%
}%

                           %none (unit-less z-transform coefficient)
\newcommand{\myunitlessz}{%
    Нет%
}%




%*******************************************
%     Область действия
\newcommand{\myscope}{%
    Область действия
}%

\newcommand{\ComThreadSpec}{%
    Заданный коммуникационный поток%
}%

\newcommand{\CoordSysSpec}{%
    Заданная координатная система%
}%

\newcommand{\CoordSysAndComThreadSpec}{%
    Заданные коммуникационный поток и координатная система%
}%

\newcommand{\MotorSpec}{%
    Заданный двигатель%
}%


\newcommand{\GlobalScope}{%
    Глобальная%
}%


%*******************************************

%*******************************************
% Значение по умолчанию для переменных (полей структур)
\newcommand{\myautoconfhard}{%
    Автоматически настраивается в зависимости от аппаратной конфигурации%
}%

%*******************************************

%*******************************************
% Тип буферов, для которых предназначена команда 

\newcommand{\TypeBufAllPMC}{%
    Программа движения (\keyword{prog} и~\keyword{rotary})%
}%

\newcommand{\TypeBufProgPMC}{%
    Программа движения (только \keyword{prog})%
}%

\newcommand{\TypeBufAllPMCandPLC}{%
    Программа движения (\keyword{prog} и~\keyword{rotary}), PLC программа%
}%

\newcommand{\TypeBufProgPMCandPLC}{%
    Программа движения (только \keyword{prog}), PLC программа%
}%





%*******************************************
%        Для Смотрите также:
%*******************************************

\newcommand{\SeeAlso}{%
    \vspace{8pt}%
    \textbf{\textit{\underline{Смотрите также:}}}%
    \vspace{2pt}%
}%

\newcommand{\SeeAlsoAcc}{%
    \vspace{8pt}%
    \textbf{\textit{\underline{Для более подробной информации смотрите:}}}%
    \vspace{2pt}%
}%

\newcommand{\DescriptInDev}{%
    \begin{remark}%
    \textbf{Примечание:}\BL 
    \textit{Пункт находится в стадии разработки.}%
    \end{remark}%
%    \vspace{1pt}%
}%

\newcommand{\DescriptInDevTemp}{%
    \BL 
    \begin{remark}%
    \textbf{Примечание:}\BL 
    \textit{Более подробное описание находится в стадии разработки.}%
    \end{remark}%
%    \vspace{1pt}%
}%

\newcommand{\DescriptCMDInDev}{%
    \begin{remark}%
    \textbf{Примечание:}\BL 
    \textit{Описание команды находится в стадии разработки.}%
    \end{remark}%
    \vspace{1pt}%
}%


\newcommand{\SeeParagraphs}{%
    Разделы:%
}%

\newcommand{\SeeVar}{%
    Переменные:%
}%

\newcommand{\SeeProgCmd}{%
    Программные команды:%
}%

\newcommand{\SeeOnlineCmd}{%
    Онлайн команды:%
}%


\newcommand{\tabitem}{~\llap{\textbullet}~~}

\newcommand{\DbgSecSt}[2]{%
% DebugSectionState     DbgSecSt
    % 0 - вообще не готова 
    % 1 - название есть
    % 2 - готова частично
    % 3 - готова полностью
\ifDebugSectState%    
    \ifcase #1%
            \textcolor{red}{#2}%      0    \StZero
        \or \textcolor{violet}{#2}%   1    \StName
        \or \textcolor{blue}{#2}%     2    \StPart
        \or \textcolor{green}{#2}%    3    \StFull
        \or \textcolor{orange}{#2}%   4    \StNote
        \or \textcolor{gray}{#2}%     5    \StDis        
    \else     \textcolor{red}{#2}%    
    \fi%
\else%
    {#2}%        - как есть
\fi
}%

\newcommand{\StZero}{0}
\newcommand{\StName}{1}
\newcommand{\StPart}{2}
\newcommand{\StFull}{3}
\newcommand{\StHome}{4}
\newcommand{\StNote}{4}
\newcommand{\StDis}{5}


% % % % % % % Типы обработки ДОС % % % % % % % %

\newcommand{\myECTtypeZero}{\mytypeECT{Флаг окончания модуля обработки ДОС}}

\newcommand{\myECTtypeOne}{\mytypeECT{Чтение одного 32-битного регистра}}

\newcommand{\myECTtypeTwo}{\mytypeECT{Зарезервировано}}

\newcommand{\myECTtypeThree}{\mytypeECT{Программная <<1/T интерполяция>>}}

\newcommand{\myECTtypeFour}{\mytypeECT{Интерполяция с программным вычислением арктангенса угла}}

\newcommand{\myECTtypeFive}{\mytypeECT{Зарезервировано}}

\newcommand{\myECTtypeSix}{\mytypeECT{Прямое резольверное преобразование арктангенса угла}}

\newcommand{\myECTtypeSeven}{\mytypeECT{Интерполяция с аппаратным вычислением арктангенса угла}}

\newcommand{\myECTtypeEight}{\mytypeECT{Сложение результатов двух записей модуля обработки ДОС}}

\newcommand{\myECTtypeNine}{\mytypeECT{Вычитание результатов двух записей модуля обработки ДОС}}

\newcommand{\myECTtypeTen}{\mytypeECT{Триггерная временная развертка}}

\newcommand{\myECTtypeEleven}{\mytypeECT{Зарезервировано}}

\newcommand{\myECTtypeTwelve}{\mytypeECT{Зарезервировано}}


%в дальнейшем требует корректировки!!!
\newcommand{\ToCorrect}[1]{%
   \textcolor{red}{\textbf{#1}}%
}%


% знак градуса
\newcommand{\degree}{%
$^\circ$ %
}%


% корректный перенос строки - вместо \\ + новый параграф без отступа
\newcommand{\BL}{%
    \par\vspace{\baselineskip}%
    \noindent% 
}%

% новый параграф без отступа
\newcommand{\NL}{%
    \noindent% 
}%


% корректировка ширины таблицы, чтобы не было ошибок
\newcommand{\TBW}{%
    >{\raggedright}%
}%

% корректировка ширины таблицы, чтобы не было ошибок
\newcommand{\TB}{%
    \raggedright%
}%

% если TB в конце, то нужно это:
\newcommand{\TBend}{%
    \tabularnewline%
}%


%\newcommand*{\nom}[2]{#1\nomenclature{#1}{#2}}



\newcounter{word}
\makeatletter
\newcommand*{\LBL}{%
  \@dblarg\@LBL
}
\def\@LBL[#1]#2{%
  \begingroup
    \renewcommand*{\theword}{#2}%
    \refstepcounter{word}%
    \label{#1}%
    #2%
  \endgroup
}
\makeatother






 %       \LBL[BtnKill]{\textit{<<Выключить>>}}


\newcommand{\mylbl}[2]{%
%   \LBL[#2]{\textbf\textit{<<#1>>}}%
   \LBL[#2]{\textcolor{linkcolor}{\textit{\textbf{\textit{#1}}}}}%
}

\newcommand{\myheader}[2]{%
%   \LBL[#2]{\textbf\textit{<<#1>>}}%
   \LBL[#2]{\textcolor{indigo_dye}{\textit{\textbf{\textit{#1}}}}}%
}
%

\newcommand{\myref}[1]{%
 \ref{#1}, стр. \pageref{#1}%
}

% Величина пробелов списка
\renewenvironment{itemize}{
    \begin{list}{\labelitemi}{
    \setlength{\topsep}{0pt}
    \setlength{\partopsep}{0pt}
    \setlength{\parskip}{0pt}
    \setlength{\itemsep}{0pt}
    \setlength{\parsep}{0pt}
    }
}{\end{list}}

% ссылки на слова
%\newcommand{mylbl}[1]{%
%   \textcolor{blue}{\textbf{\LBL{#1}}}%
%}%


%\newcommand{\myreftolbl}[1]{%
%    \ifnum \getpagerefnumber{sec:#1}=0% 
%        \textbf{#1}%                  если ссылка не определена - просто текст
%    \else% 
%        \textbf{\nameref{sec:#1}}%    если ссылка определена - текст с гиперссылкой
%    \fi

\renewcommand{\sfdefault}{cmss}