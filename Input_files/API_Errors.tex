% *******begin section***************
\section{\DbgSecSt{\StPart}{Обработка ошибок}}
%--------------------------------------------------------
\subsection{\DbgSecSt{\StPart}{Типы данных}}

% *******begin subsection***************
\subsubsection{\DbgSecSt{\StPart}{DriveErrors}}
\index{Программный интерфейс ПЛК!Обработка ошибок!Объединение DriveErrors}
\label{sec:DriveErrors}

\begin{fHeader}
    Тип данных:            & \RightHandText{Объединение DriveErrors}\\
    Файл объявления:             & \RightHandText{include/cnc/errors.h} \\
\end{fHeader}

Объединение определяет ошибки и режим работы сервоусилителя. \killoverfullbefore 

\begin{MyTableThreeColAllCntr}{Объединение DriveErrors}{tbl:DriveErrors}{|m{0.33\linewidth}|m{0.22\linewidth}|m{0.45\linewidth}|}{Элемент}{Тип}{Описание}
\hline struct \{ 
\newline
protocol & \newline \centering{unsigned:1} & \newline Ошибка протокола  \\
\hhline{~} ampNotReady & \centering{unsigned:1} & Нет готовности \\
\hhline{~} ampFault & \centering{unsigned:1} & Сервоусилитель в состоянии ошибки \\
\hhline{~} i2tFault & \centering{unsigned:1} &  Ошибка i2t \\
\hhline{~} crc & \centering{unsigned:1} & Ошибка контрольной суммы \\
\hhline{~} igbtFault & \centering{unsigned:1} & Ошибка IGBT модуля \\
\hhline{~} igbtTempFault & \centering{unsigned:1} & Превышение температуры IGBT модуля \\
\hhline{~} highDCFault & \centering{unsigned:1} & Повышенное напряжение в ЗПТ \\
\hhline{~} lowDCFault & \centering{unsigned:1} & Пониженное напряжение в ЗПТ \\
\hhline{~} linkFault & \centering{unsigned:1} & Ошибка связи \\
\hhline{~} brakeOnLowFault & \centering{unsigned:1} & Сигнал на открытие тормозного транзистора в состоянии L (не в слежении) \\
\hhline{~} brakeOnHighFault & \centering{unsigned:1} & Сигнал на открытие тормозного транзистора в состоянии H (не в слежении) \\
\hhline{~} brakeFault & \centering{unsigned:1} & Недостаточная мощность тормозного резистора \\
\hhline{~} currentOutFault & \centering{unsigned:1} & Измеренный ток в фазе в отсечке \\
\hhline{~} adcFault & \centering{unsigned:1} & Ошибка АЦП \\
\hhline{~} pwmShortFault & \centering{unsigned:1} & Период сигнала ШИМ  меньше 50 мкс \\
\hhline{~} pwmLongFault & \centering{unsigned:1} & Период сигнала ШИМ больше 400 мкс \\
\hhline{~} reserved & \centering{unsigned:8} & Резерв \\
\hhline{~} ampState & \centering{unsigned:2} & Состояние сервоусилителя \\
\hhline{~} errorCode \} & \centering{unsigned:4} & Текущий код ошибки \\
\hline errors & \centering{unsigned} & Переменная, содержащая все битовые поля \\
\end{MyTableThreeColAllCntr}
% *******end subsection***************

Поле \texttt{ampState} является 2-битным и содержит коды состояния сервоусилителя:
\begin{itemize}
\item 0 ~-- не подано высокое напряжение; \killoverfullbefore
\item 1 ~-- подано высокое напряжение; \killoverfullbefore
\item 2 ~-- сервоусилитель в слежении. \killoverfullbefore \BL
\end{itemize} 

Поле \texttt{errorCode} является 4-битным и и содержит текущий код ошибки сервоусилителя:
\begin{itemize}
\item 0 ~-- нет ошибок; \killoverfullbefore
\item 1 ~-- ошибка IGBT модуля; \killoverfullbefore
\item 2 ~-- превышение температуры IGBT модуля; \killoverfullbefore 
\item 3 ~-- повышенное напряжение в ЗПТ;
\item 4 ~-- пониженное напряжение в ЗПТ;
\item 5 ~-- ошибка связи;
\item 6 ~-- сигнал на открытие тормозного транзистора в состоянии L (не в слежении);
\item 7 ~-- сигнал на открытие тормозного транзистора в состоянии H (не в слежении);
\item 8 ~-- недостаточная мощность тормозного резистора;
\item 9 ~-- измеренный ток в фазе в отсечке;
\item 10 ~-- ошибка АЦП;
\item 11 ~-- период сигнала ШИМ меньше 50 мкс;
\item 12 ~-- Период сигнала ШИМ больше 400 мкс.\BL
\end{itemize} 

\clearpage
%--------------------------------------------------------
% *******begin subsection***************
\subsubsection{\DbgSecSt{\StPart}{EncoderErrors}}
\index{Программный интерфейс ПЛК!Обработка ошибок!Объединение EncoderErrors}
\label{sec:EncoderErrors}

\begin{fHeader}
    Тип данных:            & \RightHandText{Объединение EncoderErrors}\\
    Файл объявления:             & \RightHandText{include/cnc/errors.h} \\
\end{fHeader}

Объединение определяет ошибки ДОС.

\begin{MyTableThreeColAllCntr}{Объединение EncoderErrors}{tbl:EncoderErrors}{|m{0.33\linewidth}|m{0.22\linewidth}|m{0.45\linewidth}|}{Элемент}{Тип}{Описание}
\hline struct \{ 
\newline
encFault & \newline \centering{unsigned:1} & \newline Комбинированная ошибка датчика  \\
\hhline{~} decode & \centering{unsigned:1} & Ошибка декодирования \\
\hhline{~} sumOfSqr & \centering{unsigned:1} & Неверная сумма квадратов каналов синусно-косинусного датчика \\
\hhline{~} faultN & \centering{unsigned:1} &  Сигнал FAULT\_N \\
\hhline{~} adc & \centering{unsigned:1} & Ошибка АЦП \\
\hhline{~} lineA & \centering{unsigned:1} & Ошибка канала A \\
\hhline{~} lineB & \centering{unsigned:1} & Ошибка канала B \\
\hhline{~} lineC & \centering{unsigned:1} & Ошибка канала C \\
\hhline{~} power & \centering{unsigned:1} & Ошибка питания \\
\hhline{~} serialDataNotReady & \centering{unsigned:1} & Ошибка последовательного ДОС \\
\hhline{~} warningBiSS & \centering{unsigned:1} & Предупреждение ДОС BiSS \\
\hhline{~} faultBiSS & \centering{unsigned:1} & Ошибка ДОС BiSS \\
\hhline{~} statusEnDat \} & \centering{unsigned:1} & Ошибка статуса ДОС с протоколом EnDat \\
\hline errors & \centering{unsigned} & Переменная, содержащая все битовые поля \\
\end{MyTableThreeColAllCntr}
% *******end subsection***************
%--------------------------------------------------------
% *******begin subsection***************
\subsubsection{\DbgSecSt{\StPart}{IOErrors}}
\index{Программный интерфейс ПЛК!Обработка ошибок!Объединение IOErrors}
\label{sec:IOErrors}

\begin{fHeader}
    Тип данных:            & \RightHandText{Объединение IOErrors}\\
    Файл объявления:             & \RightHandText{include/cnc/errors.h} \\
\end{fHeader}

Объединение определяет ошибки последовательного интерфейса плат входов/выходов.

\clearpage

\begin{MyTableThreeColAllCntr}{Объединение IOErrors}{tbl:IOErrors}{|m{0.33\linewidth}|m{0.22\linewidth}|m{0.45\linewidth}|}{Элемент}{Тип}{Описание}
\hline struct \{ 
\newline
parity & \newline \centering{unsigned:1} & \newline Ошибка четности \\
\hhline{~} protocol & \centering{unsigned:1} & Ошибка протокола \\
\hhline{~} crc & \centering{unsigned:1} & Ошибка контрольной суммы \\
\hhline{~} watchdog \} & \centering{unsigned:1} & Срабатывание сторожевого таймера \\
\hline errors & \centering{unsigned} & Переменная, содержащая все битовые поля \\
\end{MyTableThreeColAllCntr}
% *******end subsection***************
%--------------------------------------------------------

% *******begin subsection***************
\subsubsection{\DbgSecSt{\StPart}{MotorErrors}}
\index{Программный интерфейс ПЛК!Обработка ошибок!Объединение MotorErrors}
\label{sec:MotorErrors}

\begin{fHeader}
    Тип данных:            & \RightHandText{Объединение MotorErrors}\\
    Файл объявления:             & \RightHandText{include/cnc/errors.h} \\
\end{fHeader}

Объединение определяет ошибки приводов.

\begin{MyTableThreeColAllCntr}{Объединение MotorErrors}{tbl:MotorErrors}{|m{0.33\linewidth}|m{0.22\linewidth}|m{0.45\linewidth}|}{Элемент}{Тип}{Описание}
\hline struct \{ 
\newline phaseref & \newline \centering{unsigned:1} & \newline Не выполнена фазировка \\
\hhline{~} home & \centering{unsigned:1} & Не выполнен поиск нулевой точки \\
\hhline{~} homeError & \centering{unsigned:1} & Произошла ошибка при поиске нулевой точки \\
\hhline{~} openLoop & \centering{unsigned:1} & Двигатель не в слежении \\

\hhline{~} encoder & \centering{unsigned:1} & Ошибка ДОС \\
\hhline{~} plusLimit & \centering{unsigned:1} & Срабатывание аппаратного ограничителя в положительном направлении \\
\hhline{~} minusLimit & \centering{unsigned:1} & Срабатывание аппаратного ограничителя в отрицательном направлении\\
\hhline{~} swPlusLimit & \centering{unsigned:1} & Срабатывание программного ограничителя в положительном направлении \\
\hhline{~} swMinusLimit & \centering{unsigned:1} & Срабатывание программного ограничителя в отрицательном направлении \\
\hhline{~} folError & \centering{unsigned:1} & Критическая ошибка слежения \\
\hhline{~} folErrorWarning & \centering{unsigned:1} & Предупредительная ошибка слежения \\
\hhline{~} temperature & \centering{unsigned:1} & Перегрев двигателя \\
\hhline{~} tempWarning & \centering{unsigned:1} & Предупреждение о перегреве двигателя \\
\hhline{~} auxFault & \centering{unsigned:1} & Внешняя ошибка \\
\hhline{~} pos2Error & \centering{unsigned:1} & Ошибка рассогласования датчиков положения и скорости \\
\hhline{~} pos2Warning & \centering{unsigned:1} & Предупреждение рассогласования датчиков положения и скорости \\
\hhline{~} phasePosError & \centering{unsigned:1} & Ошибка рассогласования датчиков положения и коммутации \\
\hhline{~} phasePosWarning \} & \centering{unsigned:1} & Предупреждение рассогласования датчиков положения и коммутации \\
\hline errors & \centering{unsigned} & Переменная, содержащая все битовые поля \\
\end{MyTableThreeColAllCntr}
% *******end subsection***************\\

%--------------------------------------------------------
% *******begin subsection***************
\subsubsection{\DbgSecSt{\StPart}{AxisErrors}}
\index{Программный интерфейс ПЛК!Обработка ошибок!Объединение AxisErrors}
\label{sec:AxisErrors}

\begin{fHeader}
    Тип данных:            & \RightHandText{Объединение AxisErrors}\\
    Файл объявления:             & \RightHandText{include/cnc/errors.h} \\
\end{fHeader}

Объединение определяет ошибки оси.

\begin{MyTableThreeColAllCntr}{Объединение AxisErrors}{tbl:AxisErrors}{|m{0.33\linewidth}|m{0.22\linewidth}|m{0.45\linewidth}|}{Элемент}{Тип}{Описание}
\hline struct \{ 
\newline abortTimeout & \newline \centering{unsigned:1} & \newline Истекло время операции аварийного торможения \\
\hhline{~} activateTimeout & \centering{unsigned:1} & Истекло время операции включения в слежение \\
\hhline{~} phaseRefTimeout & \centering{unsigned:1} & Истекло время операции фазировки \\
\hhline{~} deactivateTimeout \} & \centering{unsigned:1} & Истекло время операции выключения \\
\hline errors & \centering{unsigned} & Переменная, содержащая все битовые поля \\
\end{MyTableThreeColAllCntr}
% *******end subsection***************

%--------------------------------------------------------
% *******begin subsection***************
\subsubsection{\DbgSecSt{\StPart}{SpindleErrors}}
\index{Программный интерфейс ПЛК!Обработка ошибок!Объединение SpindleErrors}
\label{sec:SpindleErrors}

\begin{fHeader}
    Тип данных:            & \RightHandText{Объединение SpindleErrors}\\
    Файл объявления:             & \RightHandText{include/cnc/errors.h} \\
\end{fHeader}

Объединение определяет ошибки шпинделя.

\begin{MyTableThreeColAllCntr}{Объединение SpindleErrors}{tbl:SpindleErrors}{|m{0.33\linewidth}|m{0.22\linewidth}|m{0.45\linewidth}|}{Элемент}{Тип}{Описание}
\hline struct \{ 
\newline  abortTimeout & \newline \centering{unsigned:1} &  \newline Истекло время операции аварийного торможения \\
\hhline{~} activateTimeout & \centering{unsigned:1} & Истекло время операции включения в слежение \\
\hhline{~} phaseRefTimeout & \centering{unsigned:1} & Истекло время операции фазировки \\
\hhline{~} deactivateTimeout & \centering{unsigned:1} & Истекло время операции выключения \\
\hhline{~} speedTimeout & \centering{unsigned:1} & Истекло время выхода на заданную скорость \\
\hhline{~} stopTimeout & \centering{unsigned:1} & Истекло время операции останова \\
\hhline{~} homeTimeout & \centering{unsigned:1} & Истекло время операции поиска нулевой точки  \\
\hhline{~} positionTimeout \} & \centering{unsigned:1} & Истекло время выхода в заданное положение \\
\hline errors & \centering{unsigned} & Переменная, содержащая все битовые поля \\
\end{MyTableThreeColAllCntr}
% *******end subsection***************
%--------------------------------------------------------
% *******begin subsection***************
\subsubsection{\DbgSecSt{\StPart}{ChannelErrors}}
\index{Программный интерфейс ПЛК!Обработка ошибок!Объединение ChannelErrors}
\label{sec:ChannelErrors}

\begin{fHeader}
    Тип данных:            & \RightHandText{Объединение ChannelErrors}\\
    Файл объявления:             & \RightHandText{include/cnc/errors.h} \\
\end{fHeader}

Объединение определяет ошибки канала управления.

\begin{MyTableThreeColAllCntr}{Объединение ChannelErrors}{tbl:ChannelErrors}{|m{0.33\linewidth}|m{0.22\linewidth}|m{0.45\linewidth}|}{Элемент}{Тип}{Описание}
\hline struct \{ 
\newline phaseRefTimeout  & \newline \centering{unsigned:1} &  \newline Истекло время операции фазировки \\
\hhline{~} driveOnTimeout  & \centering{unsigned:1} & Истекло время ожидания включения сервоусилителя \\
\hhline{~} driveOffTimeout  & \centering{unsigned:1} & Истекло время ожидания выключения сервоусилителя \\
\hhline{~} abortTimeout & \centering{unsigned:1} & Истекло время операции аварийного торможения \\
\hhline{~} stopTimeout & \centering{unsigned:1} & Истекло время операции останова \\
\hhline{~} homeTimeout & \centering{unsigned:1} & Истекло время операции поиска нулевой точки \\
\hhline{~} homeError & \centering{unsigned:1} & Ошибка поиска нулевой точки \\

\hhline{~} startWithoutHome & \centering{unsigned:1} & Попытка запуска без определения нулевой точки \\
\hhline{~} cannotStart & \centering{unsigned:1} & Ошибка запуска программы \\

\hhline{~} progStopOk & \centering{unsigned:1} & УП выполнена \\
\hhline{~} progStopAbort & \centering{unsigned:1} & УП прервана \\
\hhline{~} progStopSyncError & \centering{unsigned:1} & Ошибка присвоения в буфере синхронных переменных \\
\hhline{~} progStopBufferError & \centering{unsigned:1} & Ошибка в буфере программы движения \\
\hhline{~} progStopCCMove & \centering{unsigned:1} & Неверный кадр в режиме коррекции инструмента \\
\hhline{~} progStopLinToPvt & \centering{unsigned:1} & Ошибка при преобразовании линейного движения в сплайн или pvt-движение \\
\hhline{~} progStopCCLeadOut & \centering{unsigned:1} & Неверный кадр при отмене режима коррекции инструмента \\
\hhline{~} progStopCCLeadIn & \centering{unsigned:1} &  Неверный кадр при активации режима коррекции инструмента \\
\hhline{~} progStopCCBufSize & \centering{unsigned:1} & Недостаточный размер буфера в режиме коррекции инструмента \\
\hhline{~} progStopPvt & \centering{unsigned:1} & Ошибка расчёта pvt-движения\\
\hhline{~} progStopCCFeed & \centering{unsigned:1} & Неверное указание подачи в режиме коррекции инструмента \\
\hhline{~} progStopCCDir & \centering{unsigned:1} & Смена направления движения в режиме коррекции инструмента \\
\hhline{~} progStopNoSolve & \centering{unsigned:1} & Невозможно рассчитать движение в режиме коррекции инструмента \\
\hhline{~} progStopCC3NdotT & \centering{unsigned:1} &  Ошибка расчёта точки резания в режиме трёхмерной коррекции инструмента \\
\hhline{~} progStopCCDist & \centering{unsigned:1} & Невозможно предотвратить «перерез» в режиме коррекции инструмента \\
\hhline{~} progStopCCNoIntersect & \centering{unsigned:1} & Невозможно найти пересечение траекторий в режиме коррекции инструмента \\
\hhline{~} progStopCCNoMoves & \centering{unsigned:1} & Между активацией и отменой  
режима коррекции инструмента кадры без команд движения \\

\hhline{~} progStopRunTime & \centering{unsigned:1} &  Недостаточное время для расчёта движения \\
\hhline{~} progStopInPos & \centering{unsigned:1} & Истекло время ожидания состояния «в позиции» \\
\hhline{~} progStopSoftLimit & \centering{unsigned:1} & Срабатывание программного ограничения \\
\hhline{~} progStopRadiusX & \centering{unsigned:1} & Срабатывание ограничения величины радиальной ошибки в режиме кругового движения \\
\hhline{~} progStopRadiusXX & \centering{unsigned:1} & Срабатывание ограничения величины радиальной ошибки в режиме кругового движения в расширенной системе координат \\
\hhline{~} progPausedM00 & \centering{unsigned:1} & УП временно остановлена по команде М00 или М01 \\
\hhline{~} cycleInvalidArgs & \centering{unsigned:1} & Неверные аргументы функции постоянного цикла \\
\hhline{~} seekingBlock & \centering{unsigned:1} & Поиск кадра \\
\hhline{~} seekBlockFound & \centering{unsigned:1} & Кадр найден \\
\hhline{~} seekBlockNotFound \} & \centering{unsigned:1} & Кадр не найден \\
\hline errors & \centering{unsigned} & Переменная, содержащая все битовые поля \\
\end{MyTableThreeColAllCntr}
% *******end subsection***************
%--------------------------------------------------------
% *******begin subsection***************
\subsubsection{\DbgSecSt{\StPart}{NCErrors}}
\index{Программный интерфейс ПЛК!Обработка ошибок!Объединение NCErrors}
\label{sec:NCErrors}

\begin{fHeader}
    Тип данных:            & \RightHandText{Объединение NCErrors}\\
    Файл объявления:             & \RightHandText{include/cnc/errors.h} \\
\end{fHeader}

Объединение определяет системные ошибки.

\begin{MyTableThreeColAllCntr}{Объединение NCErrors}{tbl:NCErrors}{|m{0.33\linewidth}|m{0.22\linewidth}|m{0.45\linewidth}|}{Элемент}{Тип}{Описание}
\hline struct \{ 
\newline  factory & \newline \centering{unsigned:1} &  \newline Ошибка загрузки системных параметров, используются параметры по умолчанию \\
\hhline{~} userFactory & \centering{unsigned:1} &  Ошибка загрузки параметров пользователя, пользовательские переменные не определены \\
\hhline{~} swClock & \centering{unsigned:1} & Отсутствует аппаратный источник частоты \\
\hhline{~} bgWdt & \centering{unsigned:1} & Срабатывание сторожевого таймера фонового режима \\
\hhline{~} rtWdt & \centering{unsigned:1} &  Срабатывание сторожевого таймера реального времени \\
\hhline{~} sysPlcFault & \centering{unsigned:1} & Ошибка выполнения системных программ ПЛК \\
\hhline{~} hmiWatchdog \} & \centering{unsigned:1} & Срабатывание сторожевого таймера связи с пультом оператора \\
\hline errors & \centering{unsigned} & Переменная, содержащая все битовые поля \\
\end{MyTableThreeColAllCntr}
% *******end subsection***************
\begin{comment}
%--------------------------------------------------------
% *******begin subsection***************
\subsubsection{\DbgSecSt{\StPart}{MachineErrors}}
\index{Программный интерфейс ПЛК!Обработка ошибок!Структура MachineErrors}
\label{sec:MachineErrors}

\begin{fHeader}
    Тип данных:            & \RightHandText{Структура MachineErrors}\\
    Файл объявления:             & \RightHandText{include/cnc/errors.h} \\
\end{fHeader}

Структура определяет ошибки станка.

% *******end subsection***************
\end{comment}
%--------------------------------------------------------
% *******begin subsection***************
\subsubsection{\DbgSecSt{\StPart}{Errors}}
\index{Программный интерфейс ПЛК!Обработка ошибок!Структура Errors}
\label{sec:Errors}

\begin{fHeader}
    Тип данных:            & \RightHandText{Структура Errors}\\
    Файл объявления:             & \RightHandText{include/cnc/errors.h} \\
\end{fHeader}

Структура содержит данные о системных ошибках, об ошибках станка, каналов управления, осей, шпинделей, приводов, ДОС, сервоусилителей и плат входов/выходов.

\begin{MyTableThreeColAllCntr}{Структура Errors}{tbl:Errors}{|m{0.38\linewidth}|m{0.22\linewidth}|m{0.4\linewidth}|}{Элемент}{Тип}{Описание}
\hline machine & \centering{\hyperlink{Machine_Errors}{MachineErrors}} & Ошибки станка \\
\hline nc & \centering{\myreftosec{NCErrors}} & Системные ошибки  \\
\hline channel [ЧИСЛО\_КАНАЛОВ] & \centering{\myreftosec{ChannelErrors}} & Ошибки каналов управления \\
\hline axes [ЧИСЛО\_ОСЕЙ] & \centering{\myreftosec{AxisErrors}} & Ошибки осей \\
\hline spindles [ЧИСЛО\_ШПИНДЕЛЕЙ] & \centering{\myreftosec{SpindleErrors}} & Ошибки шпинделей \\
\hline motors [ЧИСЛО\_ДВИГАТЕЛЕЙ+1] & \centering{\myreftosec{MotorErrors}} & Ошибки приводов \\
\hline encoders [ЧИСЛО\_ДОС] & \centering{\myreftosec{EncoderErrors}} & Ошибки ДОС \\
\hline drive [ЧИСЛО\_ДВИГАТЕЛЕЙ+1] & \centering{\myreftosec{DriveErrors}} & Ошибки сервоусилителей \\
\hline io [ЧИСЛО\_ПЛАТ\_ВХ/ВЫХ] & \centering{\myreftosec{IOErrors}} & Ошибки плат входов/выходов \\
\end{MyTableThreeColAllCntr}
% *******end subsection***************
%--------------------------------------------------------
% *******begin subsection***************
\subsubsection{\DbgSecSt{\StPart}{ErrorReaction}}
\index{Программный интерфейс ПЛК!Обработка ошибок!Перечисление ErrorReaction}
\label{sec:ErrorReaction}

\begin{fHeader}
    Тип данных:            & \RightHandText{Перечисление ErrorReaction}\\
    Файл объявления:             & \RightHandText{include/cnc/errors.h} \\
\end{fHeader}

Перечисление определяет идентификаторы типов реакций на ошибки.

\begin{MyTableTwoColAllCntr}{Перечисление ErrorReaction}{tbl:ErrorReaction}{|m{0.38\linewidth}|m{0.57\linewidth}|}{Идентификатор}{Описание}
\hline reactNone &  Нет реакции \\
% \hline reactLocal &    \\
\hline reactFollowUp  &  Восстановление после ошибки \\
\hline reactStopProgram  & Прервано выполнение программы \\
\hline reactNCNotReady  & Нет готовности системы \\
\hline reactChannelNotReady  & Нет готовности канала \\
\hline reactStartDisable  & Запрет запуска программы в канале \\
\hline reactNeedHome  & Необходим повторный поиск нулевой точки для осей в канале \\
\hline reactShowAlarm  & Показать сообщение об ошибке \\
\hline reactStop &  Останов осей \\
\hline reactStopAtEnd  & Останов осей в конце блока \\
\hline reactAutoOnly  & Фиксация ошибки только в автоматическом режиме \\
\hline reactWarning  & Показать предупреждение \\
\end{MyTableTwoColAllCntr}
% *******end subsection***************
%--------------------------------------------------------
% *******begin subsection***************
\subsubsection{\DbgSecSt{\StPart}{ErrorClear}}
\index{Программный интерфейс ПЛК!Обработка ошибок!Перечисление ErrorClear}
\label{sec:ErrorClear}

\begin{fHeader}
    Тип данных:            & \RightHandText{Перечисление ErrorClear}\\
    Файл объявления:             & \RightHandText{include/cnc/errors.h} \\
\end{fHeader}

Перечисление определяет идентификаторы типов сброса ошибок. 

Самый низкий приоритет  имеет автоматический сброс (\texttt{clearSelf}), самый высокий приоритет ~-- сброс по включению питания (\texttt{clearPowerOn}). \killoverfullbefore

\begin{MyTableTwoColAllCntr}{Перечисление ErrorClear}{tbl:ErrorClear}{|m{0.38\linewidth}|m{0.57\linewidth}|}{Идентификатор}{Описание}
\hline clearSelf &  Автоматический сброс \\
\hline clearCancel &  Сброс из оболочки, отменой текущего режима работы или перезапуском УП \\
\hline clearNCStart &  Сброс отменой текущего режима работы или перезапуском УП \\
\hline clearReset &  Сброс отменой текущего режима работы \\
\hline clearNCReset &  Сброс перезагрузкой системы \\
\hline clearPowerOn &  Сброс по включению питания \\
\end{MyTableTwoColAllCntr}
% *******end subsection***************
%--------------------------------------------------------
% *******begin subsection***************
\subsubsection{\DbgSecSt{\StPart}{DriveErrorReaction}}
\index{Программный интерфейс ПЛК!Обработка ошибок!Перечисление DriveErrorReaction}
\label{sec:DriveErrorReaction}

\begin{fHeader}
    Тип данных:            & \RightHandText{Перечисление DriveErrorReaction}\\
    Файл объявления:             & \RightHandText{include/cnc/errors.h} \\
\end{fHeader}

Перечисление определяет идентификаторы типов реакции на ошибки сервоусилителя.

\begin{MyTableTwoColAllCntr}{Перечисление DriveErrorReaction}{tbl:DriveErrorReaction}{|m{0.38\linewidth}|m{0.57\linewidth}|}{Идентификатор}{Описание}
\hline dreactNone &  Нет реакции \\
\hline dreactOFF1 &  Останов и выключение с задержкой в режиме слежения, иначе выключение \\
\hline dreactOFF1delayed  &  Пауза, останов и выключение в режиме слежения, иначе пауза и выключение \\
\hline dreactOFF2 &  Выключение \\
\hline dreactOFF3 &  Аварийное торможение и выключение с задержкой в режиме слежения, иначе выключение \\
\hline dreactSTOP2 & Аварийное торможение и сохранение режима слежения \\
\hline dreactIASC\_DCBRK  & Для синхронного - закоротить обмотки, для асинхронного - торможение постоянным током \\
\hline dreactENC & Настраивается (по умолчанию dreactOFF2) \\
\end{MyTableTwoColAllCntr}
% *******end subsection***************

\begin{comment}
%--------------------------------------------------------
% *******begin subsection***************
\subsubsection{\DbgSecSt{\StPart}{DriveErrorAcknowledge}}
\index{Программный интерфейс ПЛК!Обработка ошибок!Перечисление DriveErrorAcknowledge}
\label{sec:DriveErrorAcknowledge}

\begin{fHeader}
    Тип данных:            & \RightHandText{Перечисление DriveErrorAcknowledge}\\
    Файл объявления:             & \RightHandText{include/cnc/errors.h} \\
\end{fHeader}

Перечисление определяет идентификаторы подтверждений ошибок сервоусилителя.

\begin{MyTableTwoColAllCntr}{Перечисление DriveErrorAcknowledge}{tbl:DriveErrorAcknowledge}{|m{0.38\linewidth}|m{0.57\linewidth}|}{Идентификатор}{Описание}
\hline dackPowerOn &    \\
\hline dackImmediately  &   \\
\hline dackDisable &    \\
\end{MyTableTwoColAllCntr}
% *******end subsection***************
\end{comment}

%--------------------------------------------------------
% *******begin subsection***************
\subsubsection{\DbgSecSt{\StPart}{ErrorDescription}}
\index{Программный интерфейс ПЛК!Обработка ошибок!Структура ErrorDescription}
\label{sec:ErrorDescription}

\begin{fHeader}
    Тип данных:            & \RightHandText{Структура ErrorDescription}\\
    Файл объявления:             & \RightHandText{include/cnc/errors.h} \\
\end{fHeader}

Структура определяет параметры описания ошибки.

\begin{MyTableThreeColAllCntr}{Структура ErrorDescription}{tbl:ErrorDescription}{|m{0.33\linewidth}|m{0.22\linewidth}|m{0.45\linewidth}|}{Элемент}{Тип}{Описание}
\hline id & \centering{unsigned} & Номер ошибки в категории \\
\hline reaction & \centering{unsigned} & Тип реакции \\
\hline clear & \centering{unsigned} & Тип сброса \\
\end{MyTableThreeColAllCntr}
% *******end subsection***************
%--------------------------------------------------------
% *******begin subsection***************
\subsubsection{\DbgSecSt{\StPart}{ErrorRequests}}
\index{Программный интерфейс ПЛК!Обработка ошибок!Структура ErrorRequests}
\label{sec:ErrorRequests}

\begin{fHeader}
    Тип данных:            & \RightHandText{Структура ErrorRequests}\\
    Файл объявления:             & \RightHandText{include/cnc/errors.cfg} \\
\end{fHeader}

Структура определяет флаги действий системы, которые вызываются согласно реакциям на ошибки в перечислении \myreftosec{ErrorReaction}.

\begin{MyTableThreeColAllCntr}{Структура ErrorDescription}{tbl:ErrorDescription}{|m{0.33\linewidth}|m{0.22\linewidth}|m{0.45\linewidth}|}{Элемент}{Тип}{Описание}
\hline ncNotReady & \centering{int:1} &  Нет готовности системы \\
\hline ncStop & \centering{int:1} & Останов осей \\
\hline ncStopAtEnd & \centering{int:1} & Останов осей в конце блока \\
\hline ncFollowUp & \centering{int:1} &  Восстановление после ошибки \\
\hline channelNotReady & \centering{int:1} & Нет готовности канала \\
\hline startDisable & \centering{int:1} & Запрет запуска программы в канале \\
\hline needHome & \centering{int:1} & Необходим повторный поиск нулевой точки для осей в канале \\
\end{MyTableThreeColAllCntr}
% *******end subsection***************
%--------------------------------------------------------
% *******begin subsection***************
\subsection{\DbgSecSt{\StPart}{Функции и макросы}}

%--------------------------------------------------------
% *******begin subsection***************
\subsubsection{\DbgSecSt{\StPart}{errorSetScan}}
\index{Программный интерфейс ПЛК!Обработка ошибок!Функция errorSetScan}
\label{sec:errorSetScan}

\begin{pHeader}
    Синтаксис:      & \RightHandText{int errorSetScan (unsigned curInput, unsigned input,}\\
    & \RightHandText {const ErrorDescription \&desc, ErrorClear request);} \\
 Аргумент(ы):    & \RightHandText{unsigned curInput ~-- флаг ошибки,} \\ 
 & \RightHandText{unsigned input ~-- состояние соответствующего входа ошибки,} \\   
 & \RightHandText {const \myreftosec{ErrorDescription} \&desc ~-- описание ошибки} \\
    & \RightHandText {\myreftosec{ErrorClear} request ~-- идентификатор запроса на сброс ошибки} \\
%    Возвращаемое значение:       & \RightHandText{Целое знаковое число} \\
    Файл объявления:             & \RightHandText{include/cnc/errors.h} \\      
\end{pHeader}

Функция возвращает 1 (наличие ошибки), если состояние соответствующего входа ошибки отлично от 0. 

Если состояние соответствующего входа равно 0 и уровень сброса ошибки в структуре описания ошибки меньше или равен значению идентификатора запроса на сброс ошибки, то функция возвращает 0 (ошибка сброшена). \killoverfullbefore

Если состояние соответствующего входа равно 0 и уровень сброса ошибки в структуре описания ошибки больше значения идентификатора запроса на сброс ошибки, то функция возвращает текущее значение флага ошибки. \killoverfullbefore

Является системной.
% *******end subsection*****************
%--------------------------------------------------------
% *******begin subsection***************
\subsubsection{\DbgSecSt{\StPart}{errorScanSet}}
\index{Программный интерфейс ПЛК!Обработка ошибок!Макрос errorScanSet}
\label{sec:errorScanSet}

\begin{pHeader}
    Синтаксис:      & \RightHandText{errorScanSet (error, input, desc, request)}\\
 Аргумент(ы):    & \RightHandText{error ~-- флаг ошибки,} \\ 
 & \RightHandText{input ~-- состояние соответствующего входа,} \\   
 & \RightHandText {desc ~-- описание ошибки (переменная типа \myreftosec{ErrorDescription}),} \\
    & \RightHandText {request ~-- идентификатор запроса на сброс ошибки (переменная \newline типа \myreftosec{ErrorClear})} \\
%    Возвращаемое значение:       & \RightHandText{Целое знаковое число} \\
    Файл объявления:             & \RightHandText{include/cnc/errors.h} \\      
\end{pHeader}

Макрос \myreftosec{errorScanSet} вызывает функцию \myreftosec{errorSetScan} и присваивает возвращаемое значение аргументу \texttt{error} (флагу ошибки).

Макрос обновляет флаг выбранной ошибки в зависимости от состояния соответствующего входа и заданного идентификатора запроса на сброс ошибки.

Является системной.
% *******end subsection*****************
%--------------------------------------------------------
% *******begin subsection***************
\subsubsection{\DbgSecSt{\StPart}{errorScanRequest}}
\index{Программный интерфейс ПЛК!Обработка ошибок!Функция errorScanRequest}
\label{sec:errorScanRequest}

\begin{pHeader}
    Синтаксис:      & \RightHandText{void errorScanRequest (ErrorClear request);}\\
    Аргумент(ы):    & \RightHandText{\myreftosec{ErrorClear} request ~-- идентификатор запроса на сброс ошибки} \\    
%    Возвращаемое значение:       & \RightHandText{Нет} \\ 
    Файл объявления:             & \RightHandText{include/cnc/errors.h} \\
\end{pHeader}

Функция выполняет вызовы макроса \myreftosec{errorScanSet} для обновления флагов ошибок станка, УЧПУ, каналов управления, осей, шпинделей, приводов, сервоусилителей, ДОС и последовательного интерфейса плат входов/выходов. \killoverfullbefore

Является системной.
% *******end section*****************
%--------------------------------------------------------
% *******begin subsection***************
\subsubsection{\DbgSecSt{\StPart}{errorScan}}
\index{Программный интерфейс ПЛК!Обработка ошибок!Функция errorScan}
\label{sec:errorScan}

\begin{pHeader}
    Синтаксис:      & \RightHandText{void errorScan();}\\
    Аргумент(ы):    & \RightHandText{Нет} \\   
%    Возвращаемое значение:       & \RightHandText{Нет} \\
    Файл объявления:             & \RightHandText{include/cnc/errors.h} \\      
\end{pHeader}

Функция выполняет вызов \myreftosec{errorScanRequest} с аргументом \texttt{clearSelf} (см. \myreftosec{ErrorClear}) для обновления флагов ошибок с запросом автоматического сброса ошибок. \killoverfullbefore

Является системной.
% *******end subsection*****************
%--------------------------------------------------------
% *******begin subsection***************
\subsubsection{\DbgSecSt{\StPart}{errorReaction}}
\index{Программный интерфейс ПЛК!Обработка ошибок!Функция errorReaction}
\label{sec:errorReaction}

\begin{pHeader}
    Синтаксис:      & \RightHandText{void errorReaction(unsigned input, const ErrorDescription \&desc);}\\
    Аргумент(ы):    & \RightHandText{unsigned input ~-- флаг ошибки,} \\   
     & \RightHandText{const \myreftosec{ErrorDescription} \&desc ~-- описание ошибки} \\
%    Возвращаемое значение:       & \RightHandText{Нет} \\
    Файл объявления:             & \RightHandText{include/cnc/errors.h} \\      
\end{pHeader}

%Устанавливает флаги действий системы (см. \myreftosec{ErrorRequests}), которые вызываются реакциями на возникшую ошибку.

Устанавливает флаги действий системы (см. \myreftosec{ErrorRequests}) согласно реакциям на возникшую ошибку.

Является системной.
% *******end subsection*****************
%--------------------------------------------------------
% *******begin subsection***************
\subsubsection{\DbgSecSt{\StPart}{errorsMachineScan}}
\index{Программный интерфейс ПЛК!Обработка ошибок!Функция errorsMachineScan}
\label{sec:errorsMachineScan}

\begin{pHeader}
    Синтаксис:      & \RightHandText{void errorsMachineScan (int request);}\\
  Аргумент(ы):    & \RightHandText{int request ~-- идентификатор запроса на сброс ошибки} \\   
%    Возвращаемое значение:       & \RightHandText{Нет} \\
    Файл объявления:             & \RightHandText{include/cnc/errors.h} \\
\end{pHeader}

Функция выполняет вызовы макроса \myreftosec{errorScanSet} для обновления флагов ошибок станка. Вызывается из  \myreftosec{errorScanRequest}. \killoverfullbefore

Реализуется пользователем.
% *******end subsection*****************
%--------------------------------------------------------
% *******begin subsection***************
\subsubsection{\DbgSecSt{\StPart}{errorsMachineReaction}}
\index{Программный интерфейс ПЛК!Обработка ошибок!Функция errorsMachineReaction}
\label{sec:errorsMachineReaction}

\begin{pHeader}
    Синтаксис:      & \RightHandText{void errorsMachineReaction();}\\
    Аргумент(ы):    & \RightHandText{Нет} \\   
%    Возвращаемое значение:       & \RightHandText{Нет} \\
    Файл объявления:             & \RightHandText{include/cnc/errors.h} \\
\end{pHeader}

%Функция обрабатывает ошибки станка.
Функция выполняет вызовы функции \myreftosec{errorReaction} для ошибок станка.

%Устанавливает ошибочные состояния системы (см. \myreftosec{ErrorRequests}), которые вызываются реакциями на ошибки станка.

Реализуется пользователем. 
% *******end subsection*****************
%-------------------------------------------------------------------
% *******begin subsection***************
\subsubsection{\DbgSecSt{\StPart}{encoderScanErrors}}
\index{Программный интерфейс ПЛК!Датчики обратной связи!Функция encoderScanErrors}
\label{sec:encoderScanErrors}

\begin{pHeader}
    Синтаксис:      & \RightHandText{void encoderScanErrors(ErrorClear request);}\\
    Аргумент(ы):    & \RightHandText{\myreftosec{ErrorClear} request ~-- идентификатор типа сброса ошибки} \\    
%    Возвращаемое значение:       & \RightHandText{Нет} \\ 
    Файл объявления:             & \RightHandText{include/func/enc.h} \\       
\end{pHeader}

Функция выполняет вызовы макроса \myreftosec{errorScanSet} для обновления флагов ошибок ДОС. Вызывается из  \myreftosec{errorScanRequest}. \killoverfullbefore

Является системной.
% *******end section*****************

%-------------------------------------------------------------------
% *******begin subsection***************
\subsubsection{\DbgSecSt{\StPart}{encoderErrorsReaction}}
\index{Программный интерфейс ПЛК!Датчики обратной связи!Функция encoderErrorsReaction}
\label{sec:encoderErrorsReaction}

\begin{pHeader}
    Синтаксис:      & \RightHandText{void encoderErrorsReaction();}\\
    Аргумент(ы):    & \RightHandText{Нет} \\    
%    Возвращаемое значение:       & \RightHandText{Нет} \\ 
    Файл объявления:             & \RightHandText{include/func/enc.h} \\       
\end{pHeader}

Функция выполняет вызовы функции \myreftosec{errorReaction} для ошибок ДОС.

Является системной.
% *******end section*****************

%--------------------------------------------------------
% *******begin subsection***************
\subsubsection{\DbgSecSt{\StPart}{ampScanErrors}}
\index{Программный интерфейс ПЛК!Обработка ошибок!Функция ampScanErrors}
\label{sec:ampScanErrors}

\begin{pHeader}
    Синтаксис:      & \RightHandText{void ampScanErrors(int motor, int servo, int chan, ErrorClear request,}\\
     & \RightHandText{int isaxis, int id);}\\
  Аргумент(ы):    & \RightHandText{int motor ~-- номер связанного с осью двигателя,} \\   
    & \RightHandText {int servo ~-- номер платы управления,}  \\ 
    & \RightHandText {int chan ~-- номер канала,}  \\ 
    & \RightHandText{\myreftosec{ErrorClear} request ~-- идентификатор типа сброса ошибки,} \\ 
    & \RightHandText {int isaxis ~-- флаг оси (1) или шпинделя (0),}  \\ 
    & \RightHandText {int id ~-- номер оси или шпинделя}  \\ 
%    Возвращаемое значение:       & \RightHandText{Нет} \\
    Файл объявления:             & \RightHandText{include/func/amp\_fault.h} \\      
\end{pHeader}

Функция выполняет вызовы макроса \myreftosec{errorScanSet} для обновления флагов ошибок сервоусилителей. Вызывается из  \myreftosec{errorScanRequest}. \killoverfullbefore

Является системной. 
% *******end subsection*****************
%--------------------------------------------------------
% *******begin subsection***************
\subsubsection{\DbgSecSt{\StPart}{ampErrorsReaction}}
\index{Программный интерфейс ПЛК!Обработка ошибок!Функция ampErrorsReaction}
\label{sec:ampErrorsReaction}

\begin{pHeader}
    Синтаксис:      & \RightHandText{void ampErrorsReaction(int motor);}\\
    Аргумент(ы):    & \RightHandText{int motor ~-- номер связанного с осью двигателя} \\   
%    Возвращаемое значение:       & \RightHandText{Нет} \\
    Файл объявления:             & \RightHandText{include/func/amp\_fault.h} \\      
\end{pHeader}

%Функция обрабатывает ошибки сервоусилителей.
Функция выполняет вызовы функции \myreftosec{errorReaction} для ошибок сервоусилителей.

Является системной.  
% *******end subsection*****************
%--------------------------------------------------------
% *******begin subsection***************
\subsubsection{\DbgSecSt{\StPart}{ioScanErrors}}
\index{Программный интерфейс ПЛК!Обработка ошибок!Функция ioScanErrors}
\label{sec:ioScanErrors}

\begin{pHeader}
    Синтаксис:      & \RightHandText{void ioScanErrors(int ioNum, int servo, int io, ErrorClear request)}\\
    Аргумент(ы):    & \RightHandText{int ioNum ~-- номер платы входов/выходов,} \\ 
    & \RightHandText {int servo ~-- номер платы управления,} \\
    & \RightHandText {int io ~-- номер входа/выхода,} \\           
    & \RightHandText {\myreftosec{ErrorClear} request ~-- идентификатор типа сброса ошибки} \\  
%    Возвращаемое значение:       & \RightHandText{Нет} \\
    Файл объявления:             & \RightHandText{include/func/io.h} \\      
\end{pHeader}

Функция выполняет вызовы макроса \myreftosec{errorScanSet} для обновления флагов ошибок последовательного интерфейса плат входов/выходов. Вызывается из  \myreftosec{errorScanRequest}. \killoverfullbefore

Является системной. 
% *******end subsection*****************
%--------------------------------------------------------
% *******begin subsection***************
\subsubsection{\DbgSecSt{\StPart}{ioErrorsReaction}}
\index{Программный интерфейс ПЛК!Обработка ошибок!Функция ioErrorsReaction}
\label{sec:ioErrorsReaction}

\begin{pHeader}
    Синтаксис:      & \RightHandText{void ioErrorsReaction(int ioNum, int io, int servo);}\\
    Аргумент(ы):    & \RightHandText{int ioNum ~-- номер платы входов/выходов,} \\   
    & \RightHandText {int io ~-- номер входа/выхода,} \\  
    & \RightHandText {int servo ~-- номер платы управления} \\    
%    Возвращаемое значение:       & \RightHandText{Нет} \\
    Файл объявления:             & \RightHandText{include/func/io.h} \\      
\end{pHeader}

%Функция обрабатывает ошибки последовательного интерфейса плат входов/выходов.
Функция выполняет вызовы функции \myreftosec{errorReaction} для ошибок последовательного интерфейса плат входов/выходов.

Является системной.  
% *******end subsection*****************
%-------------------------------------------------------------------

