%--------------------------------------------------------
% *******begin section***************
\section{\DbgSecSt{\StPart}{Очередь команд}}
%--------------------------------------------------------

\subsection{\DbgSecSt{\StPart}{Типы данных}}
%--------------------------------------------------------
% *******begin subsection***************
\subsubsection{\DbgSecSt{\StPart}{CommandRequest}}
\index{Программный интерфейс ПЛК!Очередь команд!Структура CommandRequest}
\label{sec:CommandRequest}

\begin{fHeader}
    Тип данных:            & \RightHandText{Структура CommandRequest}\\
    Файл объявления:             & \RightHandText{include/func/commands.h} \\
\end{fHeader}

Структура определяет параметры команды.

\begin{MyTableThreeColAllCntr}{Структура CommandRequest}{tbl:CommandRequest}{|m{0.3\linewidth}|m{0.25\linewidth}|m{0.45\linewidth}|}{Элемент}{Тип}{Описание}
\hline command & \centering{unsigned} & Идентификатор команды \\
\hline prio & \centering{unsigned} & Приоритет \\
\hline next & \centering{int} & Номер следующей команды в очереди \\
\end{MyTableThreeColAllCntr}
% *******end subsection***************
%--------------------------------------------------------
% *******begin subsection***************
\subsubsection{\DbgSecSt{\StPart}{CommandQueue}}
\index{Программный интерфейс ПЛК!Очередь команд!Структура CommandQueue}
\label{sec:CommandQueue}

\begin{fHeader}
    Тип данных:            & \RightHandText{Структура CommandQueue}\\
    Файл объявления:             & \RightHandText{include/func/commands.h} \\
\end{fHeader}

Структура определяет параметры очереди команд.

\begin{MyTableThreeColAllCntr}{Структура CommandQueue}{tbl:CommandQueue}{|m{0.41\linewidth}|m{0.24\linewidth}|m{0.35\linewidth}|}{Элемент}{Тип}{Описание}
\hline used & \centering{int} & Первый используемый номер команды в очереди \\
\hline free & \centering{int} & Первый неиспользуемый номер команды в очереди  \\
\hline queue[РАЗМЕР\_ОЧЕРЕДИ\_КОМАНД] & \centering{\myreftosec{CommandRequest}} & Массив команд \\
\end{MyTableThreeColAllCntr}
% *******end subsection***************
%--------------------------------------------------------
% *******begin subsection***************
\subsection{\DbgSecSt{\StPart}{Функции}}

% *******begin subsection***************
\subsubsection{\DbgSecSt{\StPart}{commandsInit}}
\index{Программный интерфейс ПЛК!Очередь команд!Функция commandsInit}
\label{sec:commandsInit}

\begin{pHeader}
    Синтаксис:      & \RightHandText{void commandsInit(struct CommandQueue \&queue);}\\
   Аргумент(ы):    & \RightHandText{struct \myreftosec{CommandQueue} \&queue ~-- очередь команд} \\    
%    Возвращаемое значение:       & \RightHandText{Нет} \\ 
    Файл объявления:             & \RightHandText{include/func/commands.h} \\       
\end{pHeader}

Функция инициализирует очередь команд. 

Является системной.
% *******end section*****************

%-------------------------------------------------------------------
% *******begin subsection***************
\subsubsection{\DbgSecSt{\StPart}{commandPush}}
\index{Программный интерфейс ПЛК!Очередь команд!Функция commandPush}
\label{sec:commandPush}

\begin{pHeader}
    Синтаксис:      & \RightHandText{int commandPush(struct CommandQueue \&queue,}\\
      & \RightHandText{unsigned command, unsigned prio);}\\
    Аргумент(ы):    & \RightHandText{struct \myreftosec{CommandQueue} \&queue ~-- очередь команд,} \\
    & \RightHandText{unsigned command ~-- идентификатор команды, } \\   
    & \RightHandText{unsigned prio ~-- приоритет команды} \\  
%    Возвращаемое значение:       & \RightHandText{Целое знаковое число} \\ 
    Файл объявления:             & \RightHandText{include/func/commands.h} \\
\end{pHeader}

Функция помещает команду с учётом заданного приоритета в очередь команд.\killoverfullbefore

Функция возвращает номер команды, если она помещена в очередь, и -1 в случае ошибки.

Является системной.
% *******end section*****************
%-------------------------------------------------------------------
% *******begin subsection***************
\subsubsection{\DbgSecSt{\StPart}{commandPop}}
\index{Программный интерфейс ПЛК!Очередь команд!Функция commandPop}
\label{sec:commandPop}

\begin{pHeader}
    Синтаксис:      & \RightHandText{unsigned commandPop(struct CommandQueue \&queue);}\\
    Аргумент(ы):    & \RightHandText{struct \myreftosec{CommandQueue} \&queue ~-- очередь команд} \\ 
%    Возвращаемое значение:       & \RightHandText{Целое беззнаковое число} \\ 
    Файл объявления:             & \RightHandText{include/func/commands.h} \\
\end{pHeader}

Функция возвращает идентификатор команды, которая должна быть выполнена с учётом приоритета, из очереди команд. 

Является системной.
% *******end section*****************

%-------------------------------------------------------------------
% *******begin subsection***************
\subsubsection{\DbgSecSt{\StPart}{commandFlush}}
\index{Программный интерфейс ПЛК!Очередь команд!Функция commandFlush}
\label{sec:commandFlush}

\begin{pHeader}
    Синтаксис:      & \RightHandText{void commandFlush(struct CommandQueue \&queue);}\\
   Аргумент(ы):    & \RightHandText{struct \myreftosec{CommandQueue} \&queue ~-- очередь команд} \\    
%    Возвращаемое значение:       & \RightHandText{Нет} \\ 
    Файл объявления:             & \RightHandText{include/func/commands.h} \\
\end{pHeader}

Функция очищает очередь команд. 

Является системной.
% *******end section*****************

