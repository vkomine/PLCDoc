\etocsettocdepth.toc {section}

\definecolor{title_color}{HTML}{CDCDB4} %A98040

\lstset{basicstyle=\footnotesize, keywordstyle=\color{indigo_dye}\bfseries, stringstyle=\ttfamily, language=C, numbers=left, numberstyle=\tiny, stepnumber=1, numbersep=5pt}

\begin{comment}
\lstset{% general command to set parameter(s)
basicstyle=\small, % print whole listing small
keywordstyle=\color{black}\bfseries\underbar,
% underlined bold black keywords
identifierstyle=, % nothing happens
commentstyle=\color{white}, % white comments
stringstyle=\ttfamily, % typewriter type for strings
showstringspaces=false} % no special string spaces
\end{comment}


\chapterimage{chapter_head_0} 
\chapter{\DbgSecSt{\StPart}{Реализация программ ПЛК}}
\label{sec:PLCs}
\index{Реализация программ ПЛК|(}

%--------------------------------------------------------
% *******begin section***************

\section{\DbgSecSt{\StPart}{Таймеры}}
\index{Реализация программ ПЛК!Таймеры}

% *******begin subsection***************
\subsection{\DbgSecSt{\StPart}{Таймер однократного запуска}}
\index{Реализация программ ПЛК!Таймеры!Таймер однократного запуска}

Таймер однократного запуска (не периодический) ~-- таймер, выходное значение которого становится равным 1 по истечении заданного интервала и не меняется до повторного запуска (таймер, отмеряющий заданный интервал времени с момента запуска). \killoverfullbefore

Для использования таймера однократного запуска необходимо объявить переменную типа \myreftosec{Timer}. 

Запуск таймера осуществляется вызовом макроса \myreftosec{timerStart}, аргументами которого являются переменная типа \myreftosec{Timer} и величина интервала срабатывания в периодах сервоцикла.

Для проверки срабатывания таймера предназначен макрос \myreftosec{timerTimeout}, который возвращает значение, отличное от 0, если истёк заданный интервал срабатывания.

Перезапуск таймера осуществляется повторным вызовом макроса \myreftosec{timerStart}.

Временная диаграмма таймера представлена на рис. ~\ref{fig:Timer_1}.

\DrawPictEpsFromSvg[0.75\textwidth]{./Pictures/svg/Timer_1}{Таймер однократного запуска}{Timer_1}

Листинг ~\ref{lst:automat_mt} на стр. \pageref{lst:automat_mt} иллюстрирует использование таймера однократного запуска.

% *******end subsection***************

% *******begin subsection***************
\subsection{\DbgSecSt{\StPart}{Периодический таймер}}
\index{Реализация программ ПЛК!Таймеры!Периодический таймер}

Периодический (импульсный) таймер ~-- таймер, выходное значение которого периодически переключается с 0 на 1 и обратно через интервал, равный половине заданного периода таймера.

Функция \myreftosec{timerSc} возвращает 1, если с момента переключения таймера с 1 на 0 истёк интервал, больший или равный половине периода таймера, и 0 в противном случае. 

Временная диаграмма таймера представлена на рис. ~\ref{fig:Timer_1}.

\DrawPictEpsFromSvg[0.75\textwidth]{./Pictures/svg/Timer_2}{Периодический таймер}{Timer_2}

Периодический таймер используется для организации равных промежутков времени, например, для мигающей индикации или разного рода фильтров.

Листинг ~\ref{lst:p_timer} показывает применение периодического таймера для индикации пульта оператора.\killoverfullbefore 

Переменная \texttt{homeLed} (строка 5) определяет состояние индикатора реферирования осей. Если не выполнен выезд в 0 или не выполнено позиционирование всех осей при включении станка, то переменной \texttt{homeLed} будет присваиваться периодически 0 или 1. Значение \texttt{homeLed} записывается в \texttt{mt.PultOut.modeHome} (строка 19) и индикатор реферирования осей пульта оператора будет мигать. \killoverfullbefore \BL

\IncludeListing{./listings/p_timer2.c}{Применение периодического таймера \newline}{lst:p_timer}

% *******end subsection***************

% *******end section*****************
%--------------------------------------------------------

% *******begin section***************

\section{\DbgSecSt{\StPart}{Входы/выходы}}
\index{Реализация программ ПЛК!Входы/выходы}

%\label{MTInputs}
%\label{MTOutputs}
%\label{PultInputs}
%\label{PultOutputs}
%\label{PortablePultInputs}

Обращение ко входам и выходам осуществляется через 4-х байтовые переменные Servo[i].IO[j].DataIn[k] и Servo[i].IO[j].DataOut[k] соответственно, где i ~-- номер платы управления, j ~-- номер порта платы управления, k ~-- номер регистра выходных/выходных данных.

Переменные Servo[i].IO[j].DataIn[k] представляют собой набор 1-но битных полей, каждое из которых содержит состояние отдельного входа.

Переменные Servo[i].IO[j].DataOut[k] представляют собой набор 1-но битных полей, в каждое из которых записывается состояние отдельного выхода.\killoverfullbefore 

Для работы со входами и выходами пользователем объявляются \hypertarget{IO_union}{объединения}:
\begin{itemize}
\item MTInputs ~-- выходные сигналы электрооборудования станка (путь по умолчанию: include\textbackslash platform\textbackslash имя\_проекта\textbackslash stanok.h); \killoverfullbefore
\item MTOutputs ~-- входные сигналы электрооборудования станка (путь по умолчанию: include\textbackslash platform\textbackslash имя\_проекта\textbackslash stanok.h);\killoverfullbefore
\item PultInputs ~-- выходные сигналы пульта оператора (путь по умолчанию: include\textbackslash platform\textbackslash имя\_проекта\textbackslash operator\_pult.h);  \killoverfullbefore
\item PultOutputs ~-- входные сигналы пульта оператора (путь по умолчанию: include\textbackslash platform\textbackslash имя\_проекта\textbackslash operator\_pult.h); \killoverfullbefore
\item PortablePultInputs ~-- выходные сигналы переносного пульта (путь по умолчанию: include\textbackslash platform\textbackslash имя\_проекта\textbackslash portable\_pult.h). \killoverfullbefore \BL
\end{itemize} 

Для указания функционального назначения отдельного входа или выхода следует определить имена идентификаторов соответствующего битового поля в объявлении объединений.

Листинг ~\ref{lst:IO_unions} показывает пример определения имён идентификаторов в объявлении объединений \texttt{MTInputs} и \texttt{MTOutputs}. \killoverfullbefore \BL

\IncludeListing{./listings/IO_unions2.c}{Пример определения имён идентификаторов \newline}{lst:IO_unions}

\begin{comment}
\begin{pExample}
\IncludeLstWithoutBorder{./listings/NullSample.pas}{~}{lst:IO-ex-1}
              
\begin{tabular}{l l}

union MTInputs \{ &  \\
\quad    struct \{ &  \\
\quad \textcolor{exComm}{// Первая плата входов} &   \\
\quad unsigned overloadPumpA:1; & \textcolor{exComm}{//Вход 0 - перегрузка насоса СОЖ А} \\
\quad \vdots &   \\
\quad unsigned onFanSPND:1; & \textcolor{exComm}{//Вход 4 - вентилятор шпинделя включён}\\
\quad \vdots &   \\
\quad unsigned mtOn:1; &  \textcolor{exComm}{//Вход 31 - станок включён}\\
\quad \textcolor{exComm}{// Вторая плата входов} &   \\
\quad lubeLevelLow:1; & \textcolor{exComm}{//Вход 0 - низкий уровень масла} \\
\quad \vdots &   \\
\quad accessWithKey:1; & \textcolor{exComm}{//Вход 10 - доступ с ключом}\\
\quad \vdots &   \\
\quad unsigned unclampingAxisC:1; &  \textcolor{exComm}{//Вход 31 - ось С разжата}\\
\quad \}; & \\
\quad int Inputs[2];  & \\
\}; & \\
  & \\
union MTOutputs \{ &  \\
\quad    struct \{ &  \\
\quad \textcolor{exComm}{// Первая плата реле} &   \\
\quad unsigned clearCoolantOn:1; & \textcolor{exComm}{//Выход 0 - включение очистки СОЖ от масла} \\
\quad \vdots &   \\
\quad unsigned workpieceBlast:1; & \textcolor{exComm}{//Выход 14 - обдув рабочей зоны}\\
\quad \vdots &   \\
\quad operatorDoorOpen:1; &  \textcolor{exComm}{//Выход 23 - открытие двери оператора}\\
\quad \textcolor{exComm}{// Вторая плата реле} &   \\
\quad clampingAxisA:1; & \textcolor{exComm}{//Выход 0 - зажим оси А} \\
\quad \vdots &   \\
\quad pumpB:1; & \textcolor{exComm}{//Выход 10 - включение насоса B подачи СОЖ}\\
\quad \vdots &   \\
\quad unsigned spindleChiller:1; &  \textcolor{exComm}{//Выход 23 - включение охлаждения шпинделя}\\
\quad \}; & \\
\quad int Outputs[2];  & \\
\}; & \\

\end{tabular}

\end{pExample}  

\end{comment}


Плата входов имеет возможность подключения 32 входных сигналов, плата выходов имеет возможность вывода до 24 выходных сигналов.

Перечисленные выше объединения являются полями структуры \myreftosec{MTDesc}.

Пользователь должен объявить переменную \texttt{mt} типа \myreftosec{MTDesc} для возможности работы со входами и выходами.

% *******begin subsection***************
%\subsection{\DbgSecSt{\StPart}{Входы}}

Перед чтением состояний входов необходимо проверить корректность полученных данных: 1-й бит переменной \texttt{Servo[i].IO[j].Status} должен быть установлен (равен 1).

Листинг ~\ref{lst:IO} показывает обращение ко входам и выходам. В данном примере пульт оператора подключён к порту №0, плата входов/выходов для управления электроавтоматикой станка ~-- к порту №1. \killoverfullbefore \BL

\IncludeListing{./listings/IO2.c}{Обращение ко входам и выходам \newline}{lst:IO}

\begin{comment}
\begin{pExample}
\IncludeLstWithoutBorder{./listings/NullSample.pas}{~}{lst:IO-ex-2}
              
\begin{tabular}{l l}
MTDesc mt; &  \\

\vdots &  \\

void readInputs() \{ & \textcolor{exComm}{// Чтение входов} \\
\quad  if (Servo[0].IO[0].Status \& 1) \{ & \textcolor{exComm}{// Проверка корректности} \\
\quad   & \textcolor{exComm}{// данных} \\
\qquad mt.PultIn.PultBtn[0] = Servo[0].IO[0].DataIn[0]; & \textcolor{exComm}{// } \\
\qquad mt.PultIn.PultBtn[1] = Servo[0].IO[0].DataIn[1]; & \textcolor{exComm}{// }\\
\qquad mt.PultIn.PultBtn[2] = Servo[0].IO[0].DataIn[2]; &   \\
\qquad mt.PultIn.PultBtn[3] = Servo[0].IO[0].DataIn[3]; &  \textcolor{exComm}{// }\\
\qquad countErrorLinkOperatorPult = 0; &  \textcolor{exComm}{// Обнуление числа ошибок} \\
\qquad  &  \textcolor{exComm}{//соединения с пультом оператора} \\
\quad \} &   \\
\quad else \{ &   \\
\qquad countErrorLinkOperatorPult++; &   \\
\qquad if (countErrorLinkOperatorPult >= 100) & \textcolor{exComm}{// }\\
\qquad errorSet(systemErrors.machine.linkOperatorPult); &   \\
\quad \} & \\
\quad  if (Servo[0].IO[1].Status \& 1) \{ & \textcolor{exComm}{// Проверка корректности} \\
\quad  & \textcolor{exComm}{// данных} \\
\qquad mt.IN.Inputs[0] = Servo[0].IO[1].DataIn[0]; & \textcolor{exComm}{// } \\
\qquad mt.IN.Inputs[1] = Servo[0].IO[1].DataIn[1]; & \textcolor{exComm}{// }\\
\qquad countErrorLinkIntIO = 0; & \textcolor{exComm}{// Обнуление числа ошибок}  \\
\qquad countErrorLinkIntIO = 0; & \textcolor{exComm}{// соединения с платой входов}  \\
\quad \} &   \\
\quad else \{ &   \\
\qquad countErrorLinkIntIO++; &   \\
\qquad if (countErrorLinkIntIO >= 100) & \textcolor{exComm}{// }\\
\qquad errorSet(systemErrors.machine.linkIntIO); &   \\
\quad \} & \\
\}; & \\
\vdots &  \\

void writeOutputs() \{ & \textcolor{exComm}{// Запись выходов} \\
\quad Servo[0].IO[0].DataOut[0] = mt.PultOut.PultLed[0]; & \textcolor{exComm}{// } \\
\quad Servo[0].IO[0].DataOut[1] = mt.PultOut.PultLed[1]; & \textcolor{exComm}{// } \\
\quad Servo[0].IO[0].DataOut[2] = mt.PultOut.PultLed[2]; & \textcolor{exComm}{// } \\
\quad Servo[0].IO[1].DataOut[0] = mt.OUT.Outputs[0]; & \textcolor{exComm}{// } \\
\} &  \\
\vdots &  \\
\end{tabular}

\end{pExample}  
\end{comment}


% *******end subsection***************

% *******begin subsection***************
%\subsection{\DbgSecSt{\StPart}{Выходы}}



% *******end subsection***************

% *******end section*****************
%--------------------------------------------------------

% *******begin section***************

\section{\DbgSecSt{\StPart}{Программирование алгоритмов управления}}
\index{Реализация программ ПЛК!Программирование алгоритмов управления}

Для программирования алгоритмов управления используются конечные автоматы. 

Конечные автоматы ~-- конструкции, которые описываются ограниченным набором возможных состояний, набором сигналов (событий) и условиями переходов из одного состояния в другое. Последующее состояние автомата определяется текущим состоянием и входными сигналами.

Листинг ~\ref{lst:automat_mt} показывает фрагмент реализации конечного автомата включения/выключения станка с помощью оператора множественного выбора \texttt{switch-case}. Полностью автомат приведён в \textbf{ПРИЛОЖЕНИИ 1} в листинге <<Программа включения/выключения станка>> на стр. \pageref{MtOnOff}.\killoverfullbefore \BL


\IncludeListing{./listings/automat_mt.c}{Фрагмент реализации конечного автомата управления станком \newline}{lst:automat_mt}


\begin{comment}
\begin{pExample}
\IncludeLstWithoutBorder{./listings/NullSample.pas}{~}{lst:automat-ex-1}
              
\begin{tabular}{l l}

\quad switch (mt.State) \{ &  \\

\quad case mtNotReady: \{ & \textcolor{exComm}{// Ожидание включения} \\
\qquad if (CNC.request == mtcncReset) \{ mtReset(); \} & \textcolor{exComm}{// } \\
\qquad if (mtIsOn() \&\& !mt.ncNotReadyReq) mt.State=mtStartOn; & \textcolor{exComm}{// }  \\
\qquad break;&   \\
\quad \} &   \\

\quad case mtStartOn: \{ & \textcolor{exComm}{// Начало включения} \\
\qquad if (CNC.request == mtcncReset) \{ mtReset(); \} & \textcolor{exComm}{// } \\
\qquad if (mt.ncNotReadyReq) \{ mtAbortRequest(); break; \} & \textcolor{exComm}{// } \\
\qquad mt.State = mtDriveOn;
\qquad if (!axesPhaseRefComplete() || !spinsPhaseRefComplete()) mt.State = mtPhaseRef;
\qquad break; &   \\
\quad \} &   \\

\quad case mtPhaseRef: \{ & \textcolor{exComm}{// Фазировка} \\
\qquad if (CNC.request == mtcncReset) \{ mtReset(); \} & \textcolor{exComm}{// } \\
\qquad if (mt.ncNotReadyReq) \{ mtAbortRequest(); break; \} & \textcolor{exComm}{// } \\
\qquad if (axesPhaseRef() || spinsPhaseRef()) \{ &   \\
\qquad \quad mt.State = mtWaitPhaseRef; &   \\
\qquad \quad timerStart(mt.timerState, MT\_TIME\_DRIVE\_PHASE\_REF); &   \\
\qquad \} else \{ &   \\
\qquad \quad mt.State = mtDriveOn;
\qquad \}
\qquad break;&   \\
\quad \} &   \\

%\quad \vdots &   \\

\quad case default: \{ &   \\
\qquad mt.OUT.pumpA = 0; & \textcolor{exComm}{//Насос А выключить}  \\
\qquad mt.OUT.pumpB = 0; & \textcolor{exComm}{//Насос В выключить}  \\
\qquad break;&   \\
\quad \} &   \\

\} & \\
\end{tabular}

\end{pExample} 
\end{comment}

\begin{comment}
\begin{lstlisting}[label=some-code,caption={Это крутой исходный код}]
 // управление сигналом готовности системы
    switch (mt.State) {
    // ждём вход от главного пускателя
    case mtNotReady: {
        if (CNC.request == mtcncReset) { mtReset(); }
        if (mtIsOn() && !mt.ncNotReadyReq) mt.State=mtStartOn;
        // Обработка нажатия кнопки Reset
        //if (MT.PultIn.modeReset) resetCNC();
        break;
    }
        // начало включения
    case mtStartOn: {
        if (CNC.request == mtcncReset) { mtReset(); }
        if (mt.ncNotReadyReq) { mtAbortRequest(); break; }
        mt.State = mtDriveOn;
        if (!axesPhaseRefComplete() || !spinsPhaseRefComplete()) mt.State = mtPhaseRef;
        break;
    }
        // фазировка
    case mtPhaseRef: {
        if (CNC.request == mtcncReset) { mtReset(); }
        if (mt.ncNotReadyReq) { mtAbortRequest(); break; }
        if (axesPhaseRef() || spinsPhaseRef()) {
            mt.State = mtWaitPhaseRef;
            timerStart(mt.timerState,
                       MT_TIME_DRIVE_PHASE_REF);
        } else {
            // уже выполнено
            mt.State = mtDriveOn;
        }
        break;
    }
        // ожидание окончания фазировки
    case mtWaitPhaseRef: {
        if (CNC.request == mtcncReset) { mtReset(); }
        if (mt.ncNotReadyReq) { mtAbortRequest(); break; }
        if (timerTimeout(mt.timerState)) {
            errorSet(systemErrors.channel[0].phaseRefTimeout);
            break;
        }
        if (axesPhaseRefComplete() && spinsPhaseRefComplete()) {
            mt.State=mtDriveOn;
        }
        break;
    }
        // включение приводов
    case mtDriveOn: {
        if (CNC.request == mtcncReset) { mtReset(); }
        if (mt.ncNotReadyReq) { mtAbortRequest(); break; }
        axesActivate();
        //spinsActivate();
        timerStart(mt.timerState,
                   MT_TIME_DRIVE_ON);
        mt.State=mtWaitDriveOn;
        break;
    }
        // ожидание включения приводов
    case mtWaitDriveOn: {
        if (CNC.request == mtcncReset) { mtReset(); }
        if (mt.ncNotReadyReq) { mtAbortRequest(); break; }
        if (timerTimeout(mt.timerState)) {
            errorSet(systemErrors.channel[0].driveOnTimeout);
            break;
        }
        if (axesActive()/* && spinsActive()*/) {
            mt.State=mtOthersMotorOn;
        }
        break;
    }
\end{lstlisting}
\end{comment}

\begin{comment}
\begin{pExample}
\IncludeLstWithoutBorder{./listings/NullSample.pas}{~}{lst:automat-ex-1}
              
\begin{tabular}{l l}

enum CoolantPump \{ & \textcolor{exComm}{//Перечисление ~-- идентификаторы насосов СОЖ}  \\
\quad    coolNone = 0,  &  \\
\quad    coolA = 1,   &  \\
\quad    coolB,  &  \\
\quad    coolAB &  \\
\}; &  \\

\quad \vdots &   \\

void coolantOn (int pump) \{ &  \\
\quad actPump = pump; &  \\
\quad switch (pump) \{ &  \\

\quad case coolA: \{ & \textcolor{exComm}{//actPump = A;} \\
\qquad mt.OUT.pumpA = 1; & \textcolor{exComm}{//Насос А включить} \\
\qquad mt.OUT.pumpB = 0; & \textcolor{exComm}{//Насос В выключить}  \\
\qquad break;&   \\
\quad \} &   \\

\quad case coolB: \{ & \textcolor{exComm}{//actPump = B;} \\
\qquad mt.OUT.pumpA = 0; & \textcolor{exComm}{//Насос А выключить} \\
\qquad mt.OUT.pumpB = 1; & \textcolor{exComm}{//Насос В включить} \\
\qquad break;&   \\
\quad \} &   \\

\quad case coolAB: \{ & \textcolor{exComm}{//actPump = AB;} \\
\qquad mt.OUT.pumpA = 1; & \textcolor{exComm}{//Насос А включить} \\
\qquad mt.OUT.pumpB = 1; & \textcolor{exComm}{//Насос В включить} \\
\qquad break;&   \\
\quad \} &   \\

%\quad \vdots &   \\

\quad case default: \{ &   \\
\qquad mt.OUT.pumpA = 0; & \textcolor{exComm}{//Насос А выключить}  \\
\qquad mt.OUT.pumpB = 0; & \textcolor{exComm}{//Насос В выключить}  \\
\qquad break;&   \\
\quad \} &   \\

\} & \\
\end{tabular}

\end{pExample} 
\end{comment}

% *******end section*****************
%--------------------------------------------------------
% *******begin section***************
\section{\DbgSecSt{\StPart}{Обработка аварийных ситуаций и ошибок электрооборудования станка}}
\index{Реализация программ ПЛК!Обработка аварийных ситуаций и ошибок электрооборудования станка}

Обработка аварийных ситуаций и ошибок электрооборудования станка выполняется с помощью следующих программных средств:\killoverfullbefore
\begin{itemize}
\item объединение \texttt{MachineErrors} ~--  список аварийных ситуаций и ошибок;
\item макрос \texttt{errorSet} ~-- установка ошибки (соответствующему битовому полю объединения \texttt{MachineErrors} присваивается 1); \killoverfullbefore
\item макрос \texttt{DEFINE\_ERROR} ~--  создание и инициализация переменной типа \myreftosec{ErrorDescription}; \killoverfullbefore
\item пользовательская функция \myreftosec{errorsMachineScan} ~-- набор вызовов макроса \texttt{errorScanSet}; \killoverfullbefore
\item пользовательская функция \myreftosec{errorsMachineReaction} ~-- набор вызовов функции \myreftosec{errorReaction}. \killoverfullbefore \BL
\end{itemize}

\hypertarget{Machine_Errors}{В объединении} \texttt{MachineErrors} (путь по умолчанию: include\textbackslash platform\textbackslash имя\_проекта\textbackslash machine\_error.h) пользователем определяются аварийные ситуации и ошибки электрооборудования станка в виде битовых полей. Оно является полем структуры \myreftosec{Errors}. \killoverfullbefore

Листинг ~\ref{lst:machine_errors_union} показывает пример объявления объединения \texttt{MachineErrors}. \killoverfullbefore \BL

\IncludeListing{./listings/machine_errors_union2.c}{Пример объявления объединения \texttt{MachineErrors} \newline}{lst:machine_errors_union}

\BL

Листинг ~\ref{lst:coolant} показывает пример использования макроса \texttt{errorSet} в функции контроля СОЖ. 

Если насос А включен и произошла его перегрузка (проверка соответствующего выхода и входа в строке 7), то выключается подача СОЖ (строка 8) и выставляется соответствующий бит ошибки (строка 9).\killoverfullbefore

Если уровень СОЖ высокий (проверка соответствующего входа в строке 13) и станок включен (строка 14), то выключается подача СОЖ (строка 15) и выставляется соответствующий бит ошибки (строка 17).\killoverfullbefore

Системная переменная \texttt{systemErrors} имеет тип \myreftosec{Errors}. \killoverfullbefore \BL

\IncludeListing{./listings/coolant.c}{Пример использования макроса \texttt{errorSet} \newline}{lst:coolant}

Макрос \texttt{DEFINE\_ERROR(name, code, react, cl)} создаёт переменную типа \myreftosec{ErrorDescription} с именем descError\texttt{\{name\}} и инициализирует её поля аргументами descError\texttt{\{name\}}.id=code, descError\texttt{\{name\}}.reaction=react, descError\texttt{\{name\}}.clear=cl. \killoverfullbefore

Поле descError\texttt{\{name\}}.id ~-- приоритет ошибки. 

Поле descError\texttt{\{name\}}.reaction ~-- идентификаторы перечисления \myreftosec{ErrorReaction}, определяющие реакцию на ошибку. 

Поле descError\texttt{\{name\}}.clear ~-- идентификаторы перечисления \myreftosec{ErrorClear}, определяющие тип сброса ошибки.

Листинг ~\ref{lst:define_error} показывает вызовы макроса \texttt{DEFINE\_ERROR}, которые создают переменные descErrorMachineEmergencyStop и descErrorMachineLubeError. \killoverfullbefore \BL

\IncludeListing{./listings/define_error.c}{Пример вызова макроса \texttt{DEFINE\_ERROR} \newline}{lst:define_error}

Функция \myreftosec{errorsMachineScan} должна быть реализована пользователем. В ней
вызывается макрос \myreftosec{errorScanSet} для ошибок, 
%вызывается макрос \texttt{errorScanSet(error, input, desc, request)} для ошибок,
определенных в объединении \texttt{MachineErrors} и макросом \texttt{DEFINE\_ERROR}. \killoverfullbefore

Агрументы \texttt{errorScanSet}:
\begin{itemize}
\item error ~-- битовое поле объединения \texttt{MachineErrors}; \killoverfullbefore
\item input ~-- значение, которое возвращает функция контроля соответствующего параметра (состояние соответствующего входа); \killoverfullbefore
\item desc ~-- переменная типа \myreftosec{ErrorDescription}; \killoverfullbefore
\item request ~-- переменная типа \myreftosec{ErrorClear}. \killoverfullbefore \BL
\end{itemize}

Функция \myreftosec{errorsMachineReaction} должна быть реализована пользователем. В ней вызывается функция \myreftosec{errorReaction} для ошибок, определенных в объединении \texttt{MachineErrors} и макросом \texttt{DEFINE\_ERROR}. \killoverfullbefore

Листинг ~\ref{lst:machine_errors} показывает пример реализации файла обработки аварийных ситуаций и ошибок электрооборудования станка (путь по умолчанию: include\textbackslash platform\textbackslash имя\_проекта\textbackslash machine\_error.cfg). \killoverfullbefore \BL

\IncludeListing{./listings/machine_errors2.c}{Пример обработки аварийных ситуаций и ошибок электрооборудования станка \newline}{lst:machine_errors}

\BL

Листинг ~\ref{lst:lube} показывает пример реализации функции контроля смазки направляющих \texttt{hasLubeError()}, которая вызывается в строке 21 листинга ~\ref{lst:machine_errors}. \killoverfullbefore \BL

\IncludeListing{./listings/lube.c}{Пример реализации функции контроля смазки направляющих \newline}{lst:lube}

% *******end section*****************
%--------------------------------------------------------
\index{Реализация программ ПЛК|)}

\clearpage