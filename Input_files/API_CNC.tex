\etocsettocdepth.toc {section}

\renewcommand{\arraystretch}{1.0} %% increase table row spacing
\renewcommand{\tabcolsep}{0.1cm}   %% increase table column spacing

\chapterimage{chapter_head_0} 
\chapter{\DbgSecSt{\StPart}{Программный интерфейс ПЛК}}
\label{sec:Functions}
\index{Программный интерфейс ПЛК|(}

%--------------------------------------------------------
% *******begin section***************
\section{\DbgSecSt{\StPart}{Управление УЧПУ}}

\subsection{\DbgSecSt{\StPart}{Типы данных}}

% *******begin subsection***************
\subsubsection{\DbgSecSt{\StPart}{CNCMode}}
\index{Программный интерфейс ПЛК!Управление УЧПУ!Перечисление CNCMode}
\label{sec:CNCMode}

\begin{fHeader}
    Тип данных:            & \RightHandText{Перечисление CNCMode}\\
    Файл объявления:             & \RightHandText{include/cnc/cnc.h} \\
\end{fHeader}

Перечисление определяет идентификаторы режимов работы УЧПУ.

\begin{MyTableTwoColAllCntr}{Перечисление CNCMode}{tbl:CNCMode}{|m{0.38\linewidth}|m{0.57\linewidth}|}{Идентификатор}{Описание}
\hline cncNull &   Режим не определён  \\
\hline cncOff &  УЧПУ не активно \\
\hline cncManual  & Ручной режим \\
\hline cncHome &  Режим выезда в нулевую точку \\
\hline cncHWL &  Режим дискретных перемещений \\
\hline cncAuto &  Автоматический режим \\
\hline cncStep &  Пошаговый режим \\
\hline cncMDI &  Режим преднабора \\
\hline cncVirtual &  Виртуальный режим \\
\hline cncReset &  Режим сброса \\
\hline cncRepos &  Режим возврата на контур \\
\hline cncWaitChangeMode &  Ожидание смены режима \\
\end{MyTableTwoColAllCntr}
% *******end subsection***************
\clearpage
% *******begin subsection***************
\subsubsection{\DbgSecSt{\StPart}{ChannelStatus}}
\index{Программный интерфейс ПЛК!Управление УЧПУ!Перечисление ChannelStatus}
\label{sec:ChannelStatus}

\begin{fHeader}
    Тип данных:            & \RightHandText{Перечисление ChannelStatus}\\
    Файл объявления:             & \RightHandText{include/cnc/cnc.h} \\
\end{fHeader}

Перечисление определяет идентификаторы состояний канала управления.

\begin{MyTableTwoColAllCntr}{Перечисление ChannelStatus}{tbl:ChannelStatus}{|m{0.38\linewidth}|m{0.57\linewidth}|}{Идентификатор}
{Описание}
\hline channelReset &   Готовность  \\
\hline channelInterrupted &   Работа прервана \\
\hline channelActive &   Активен \\
\end{MyTableTwoColAllCntr}
% *******end subsection***************

% *******begin subsection***************
\subsubsection{\DbgSecSt{\StPart}{ModeState}}
\index{Программный интерфейс ПЛК!Управление УЧПУ!Перечисление ModeState}
\label{sec:ModeState}

\begin{fHeader}
    Тип данных:            & \RightHandText{Перечисление ModeState}\\
    Файл объявления:             & \RightHandText{include/cnc/cnc.h} \\
\end{fHeader}

Перечисление определяет идентификаторы состояний текущего режима УЧПУ.

\begin{MyTableTwoColAllCntr}{Перечисление ModeState}{tbl:ModeState}{|m{0.38\linewidth}|m{0.57\linewidth}|}{Идентификатор}{Описание}
\hline modeReset &   Готовность  \\
\hline modeRunning  &  Выполнение \\
\hline modeStopped  &  Останов \\
\end{MyTableTwoColAllCntr}
% *******end subsection***************

% *******begin subsection***************
\subsubsection{\DbgSecSt{\StPart}{ProgramSeekMode}}
\index{Программный интерфейс ПЛК!Управление УЧПУ!Перечисление ProgramSeekMode}
\label{sec:ProgramSeekMode}

\begin{fHeader}
    Тип данных:            & \RightHandText{Перечисление ProgramSeekMode}\\
    Файл объявления:             & \RightHandText{include/cnc/cnc.h} \\
\end{fHeader}

Перечисление определяет идентификаторы режимов выполнения УП с произвольного кадра.

\begin{MyTableTwoColAllCntr}{Перечисление ProgramSeekMode}{tbl:ProgramSeekMode}{|m{0.38\linewidth}|m{0.57\linewidth}|}{Идентификатор}{Описание}
\hline seekNone &  Режим не активен  \\
\hline seekApproach &  Выполнение УП с начала выбранного кадра  \\
\hline seekWithoutApproach  & Выполнение УП с конца выбранного кадра \\
\hline seekWithoutCalc &  Выполнение УП без расчёта фрагмента программы до выбранного кадра \\
\end{MyTableTwoColAllCntr}
% *******end subsection***************

% *******begin subsection***************
\subsubsection{\DbgSecSt{\StPart}{ProgramStatus}}
\index{Программный интерфейс ПЛК!Управление УЧПУ!Перечисление ProgramStatus}
\label{sec:ProgramStatus}

\begin{fHeader}
    Тип данных:            & \RightHandText{Перечисление ProgramStatus}\\
    Файл объявления:             & \RightHandText{include/cnc/cnc.h} \\
\end{fHeader}

Перечисление определяет идентификаторы состояний УП.

\begin{MyTableTwoColAllCntr}{Перечисление ProgramStatus}{tbl:ProgramStatus}{|m{0.38\linewidth}|m{0.57\linewidth}|}{Идентификатор}{Описание}
\hline programAborted &  Выполнение УП прервано и завершено  \\
\hline programInterrupted  & Выполнение УП временно прервано для  какой-либо операции\\
\hline programStopped  & Выполнение УП остановлено \\
\hline programRunning  &  УП выполняется \\
\end{MyTableTwoColAllCntr}
% *******end subsection***************

% *******begin subsection***************
\subsubsection{\DbgSecSt{\StPart}{ShutdownState}}
\index{Программный интерфейс ПЛК!Управление УЧПУ!Перечисление ShutdownState}
\label{sec:ShutdownState}

\begin{fHeader}
    Тип данных:            & \RightHandText{Перечисление ShutdownState}\\
    Файл объявления:             & \RightHandText{include/cnc/cnc.h} \\
\end{fHeader}

Перечисление определяет идентификаторы состояний автомата выключения УЧПУ и станка.

\begin{MyTableTwoColAllCntr}{Перечисление ShutdownState}{tbl:ShutdownState}{|m{0.38\linewidth}|m{0.57\linewidth}|}{Идентификатор}{Описание}
\hline shutdownWaitCommand &  Ожидание команды выключения \\
\hline shutdownWaitAck &  Ожидание подтверждения команды выключения \\
\end{MyTableTwoColAllCntr}
% *******end subsection***************

% *******begin subsection***************
\subsubsection{\DbgSecSt{\StPart}{ChannelInfo}}
\index{Программный интерфейс ПЛК!Управление УЧПУ!Структура ChannelInfo}
\label{sec:ChannelInfo}

\begin{fHeader}
    Тип данных:            & \RightHandText{Структура ChannelInfo}\\
    Файл объявления:             & \RightHandText{include/cnc/cnc.h} \\
\end{fHeader}

Структура определяет данные канала управления.

\begin{MyTableThreeColAllCntr}{Структура ChannelInfo}{tbl:ChannelInfo}{|m{0.33\linewidth}|m{0.22\linewidth}|m{0.45\linewidth}|}{Элемент}{Тип}{Описание}
\hline canLoad & \centering{Битовое поле:1} &  Разрешение загрузки УП  \\
\hline starting & \centering{Битовое поле:1} &  Подготовка к выполнению УП \\
\hline running & \centering{Битовое поле:1} & Выполнение УП \\
\hline holding & \centering{Битовое поле:1} & УП в процессе останова или возобновления  \\
\hline stopped & \centering{Битовое поле:1} & УП не выполняется \\
\hline waitingBlock & \centering{Битовое поле:1} & Запрос поиска кадра  \\
\hline seekingBlock & \centering{Битовое поле:1} & Выполнение поиска кадра \\
\hline virtualStart & \centering{Битовое поле:1} & Подготовка к выполнению УП в виртуальном режиме \\
\hline virtualRun & \centering{Битовое поле:1} & Выполнение УП в виртуальном режиме \\
\hline canLoadMDI & \centering{Битовое поле:1} & Разрешение загрузки УП в режиме преднабора \\
\hline startingMDI & \centering{Битовое поле:1} & Подготовка к выполнению УП в режиме преднабора \\
\hline runningMDI & \centering{Битовое поле:1} &  Выполнение УП в режиме преднабора \\
\hline holdingMDI & \centering{Битовое поле:1} &  УП в режиме преднабора в процессе останова или возобновления \\
\hline waitingMDI & \centering{Битовое поле:2} &  0 ~-- УП загружена для выполнения в режиме преднабора \newline 1 ~-- запрос загрузки УП для выполнения в режиме преднабора \newline 2 ~-- ошибка загрузки УП для выполнения в режиме преднабора \\
\hline mdiReady & \centering{Битовое поле:1} &  УП в режиме преднабора готова к выполнению \\
\hline switchToRepos & \centering{Битовое поле:1} &  Разрешение перехода в режим возврата на контур \\
\hline setActual & \centering{Битовое поле:8} & Младшие 4 бита ~-- команда: \newline 1 ~-- текущая позиция = 0; \newline 2 ~-- текущая позиция = машинная позиция; \newline  3 ~-- текущая позиция = программная позиция \newline   
Старшие 4 бита ~-- область применения: 0 ~-- все оси; другие значения определяются конфигурацией станка \\
\hline res & \centering{Битовое поле:7} &  Резерв \\
\hline Pos[ЧИСЛО\_ОСЕЙ] & \centering{double} &  Программная позиция \\
\hline WorkPos[ЧИСЛО\_ОСЕЙ] & \centering{double} & Программная позиция относительно базового смещения \\
\hline MachPos[ЧИСЛО\_ОСЕЙ] & \centering{double} &  Машинная позиция \\
\hline TargetPos[ЧИСЛО\_ОСЕЙ] & \centering{double} &  Конечная позиция текущего кадра \\
\hline DistToGo[ЧИСЛО\_ОСЕЙ] & \centering{double} &  Остаток пути \\
\hline ActualPos[ЧИСЛО\_ОСЕЙ] & \centering{double} &  Текущая позиция \\
\hline ActualBase[ЧИСЛО\_ОСЕЙ] & \centering{double} &  Базовое смещение текущей позиции \\
\hline state & \centering{\myreftosec{ChannelStatus}} & Состояние канала управления \\
\hline modeState & \centering{\myreftosec{ModeState}} &  Состояние текущего режима УЧПУ \\
\hline runtime & \centering{\myreftosec{ProgramRuntime}} & Данные УП \\
\hline startBlock & \centering{unsigned} & Начальный блок поиска кадра при выполнении УП с произвольного кадра\\
\hline blockMode & \centering{unsigned} &  Режим выполнения УП с произвольного кадра \\
\hline seekCount & \centering{unsigned} &  Номер итерации поиска кадра \\
\end{MyTableThreeColAllCntr}
% *******end subsection***************

% *******begin subsection***************
\subsubsection{\DbgSecSt{\StPart}{CNCDesc}}
\index{Программный интерфейс ПЛК!Управление УЧПУ!Структура CNCDesc}
\label{sec:CNCDesc}

\begin{fHeader}
    Тип данных:            & \RightHandText{Структура CNCDesc}\\
    Файл объявления:             & \RightHandText{include/cnc/cnc.h} \\
\end{fHeader}

Структура определяет данные УЧПУ.

\begin{MyTableThreeColAllCntr}{Структура CNCDesc}{tbl:CNCDesc}{|m{0.33\linewidth}|m{0.22\linewidth}|m{0.45\linewidth}|}{Элемент}{Тип}{Описание}
\hline mode & \centering{\myreftosec{CNCMode}} &  Текущий режим работы УЧПУ  \\
\hline prevMode & \centering{\myreftosec{CNCMode}} & Предыдущий режим работы УЧПУ \\
\hline nextMode & \centering{\myreftosec{CNCMode}} & Следующий режим работы УЧПУ \\
\hline Watchdog & \centering{int} & Счётчик сторожевого таймера \\
\hline HMIFeedback & \centering{int} &  Флаг обратной связи пульта оператора \\
\hline HMIFirstStart & \centering{int} &  Флаг включения пульта оператора (до включения пульта оператора равен 1) \\
\hline hmiTripped & \centering{int} & Флаг срабатывания сторожевого таймера \\
\hline HMIWatchdog & \centering{\myreftosec{Timer}} &  Таймер сторожевого таймера связи с пультом оператора\\
\hline shutdown & \centering{\myreftosec{Timer}} &  Таймер выключения УЧПУ и станка \\
\hline modeAutoStep & \centering{unsigned} & Флаг покадровой отработки УП \\
\hline modeAutoVirtual & \centering{unsigned} & Флаг отработки УП в виртуальном режиме \\
\hline modeAutoSkip & \centering{unsigned} &  Флаг программного пропуска кадров при отработке УП \\
\hline modeAutoOptStop & \centering{unsigned} & Флаг опционального останова при отработке УП  \\
\hline modeAutoRepos & \centering{unsigned} &  Флаг возврата на контур при возобновлении выполнения УП \\
\hline alarmCancel & \centering{unsigned} &  Запрос сброса ошибок \\
\hline modeDryRun & \centering{unsigned} &  Флаг пробной подачи при отработке УП  \\
\hline modeReducedG0 & \centering{unsigned} &  Флаг уменьшенной подачи быстрого хода при отработке УП  \\
\hline nodeNoMovement & \centering{unsigned} &  Флаг отработки УП с блокировкой движения \\
\hline request & \centering{\myreftosec{MTCNCRequests}} & Текущая исполняемая команда УЧПУ  \\
\hline channel[ЧИСЛО\_КАНАЛОВ] & \centering{\myreftosec{ChannelInfo}} &  Данные канала управления \\
\hline notReadyReq & \centering{Битовое поле:1} &  УЧПУ не готово \\
\hline startDisableReq & \centering{Битовое поле:1} &  Запрет запуска УП \\
\hline enablePortablePult & \centering{Битовое поле:1} &  Разрешение работы переносного пульта \\
\hline ShutdownHMI & \centering{int} &  Переменная выключения УЧПУ и станка принимает значения: \newline 0x5A при включении УЧПУ, \newline 0xA5 ~-- при получении команды, выключения, \newline 0x55 ~-- при подтверждении команды выключения\\
\hline ShutdownState & \centering{int} & Состояние автомата выключения УЧПУ и станка \\
\hline commands & \centering{\myreftosec{CommandQueue}} &  Очередь команд \\
\end{MyTableThreeColAllCntr}
% *******end subsection***************

% *******begin subsection***************
\subsubsection{\DbgSecSt{\StPart}{CNCSettings}}
\index{Программный интерфейс ПЛК!Управление УЧПУ!Структура CNCSettings}
\label{sec:CNCSettings}

\begin{fHeader}
    Тип данных:            & \RightHandText{Структура CNCSettings}\\
    Файл объявления:             & \RightHandText{include/cnc/cnc.h} \\
\end{fHeader}

Структура определяет значения подачи для различных режимов.

\begin{MyTableThreeColAllCntr}{Структура CNCSettings}{tbl:CNCSettings}{|m{0.33\linewidth}|m{0.22\linewidth}|m{0.45\linewidth}|}{Элемент}{Тип}{Описание}
\hline Frapid & \centering{double} &  Значение подачи быстрого хода \\
\hline Fdry & \centering{double} & Значение пробной подачи \\
\hline FrapidReduced & \centering{double} & Значение уменьшенной подачи быстрого хода \\
\end{MyTableThreeColAllCntr}
% *******end subsection***************

%-------------------------------------------------------------------
% *******begin subsection***************
\subsection{\DbgSecSt{\StPart}{Функции}}
\begin{comment}
% *******begin subsection***************
\subsubsection{\DbgSecSt{\StPart}{void InitCnc()}}
\index{Программный интерфейс ПЛК!Управление УЧПУ!void InitCnc()}
\label{sec:InitCnc}

\begin{pHeader}
%    Синтаксис:      & \RightHandText{void InitCnc();}\\
    Аргумент(ы):    & \RightHandText{Нет} \\    
%    Возвращаемое значение:       & \RightHandText{Нет} \\ 
    Файл объявления:             & \RightHandText{include/cnc/cnc.h} \\
\end{pHeader}

Функция инициализации УЧПУ. 

Является системной.
% *******end section*****************
\end{comment}
% *******begin subsection***************
\subsubsection{\DbgSecSt{\StPart}{InitCnc}}
\index{Программный интерфейс ПЛК!Управление УЧПУ!Функция InitCnc}
\label{sec:InitCnc}

\begin{pHeader}
    Синтаксис:      & \RightHandText{void InitCnc();}\\
    Аргумент(ы):    & \RightHandText{Нет} \\    
%    Возвращаемое значение:       & \RightHandText{Нет} \\ 
    Файл объявления:             & \RightHandText{include/cnc/cnc.h} \\
\end{pHeader}

Функция инициализации УЧПУ. 

Является системной.
% *******end section*****************
%--------------------------------------------------------
% *******begin subsection***************
\subsubsection{\DbgSecSt{\StPart}{mtIsReady}}
\index{Программный интерфейс ПЛК!Управление УЧПУ!Функция mtIsReady}
\label{sec:mtIsReady}

\begin{pHeader}
    Синтаксис:      & \RightHandText{int mtIsReady();}\\
    Аргумент(ы):    & \RightHandText{Нет} \\   
%    Возвращаемое значение:       & \RightHandText{Целое знаковое число} \\
    Файл объявления:             & \RightHandText{include/cnc/cnc.h} \\      
\end{pHeader}

Функция проверки готовности станка к работе. \killoverfullbefore

Функция возвращает 1, если станок готов, и 0 в противном случае.  

Реализуется пользователем. 
% *******end subsection*****************
%--------------------------------------------------------
% *******begin subsection***************
\subsubsection{\DbgSecSt{\StPart}{cncSetMode}}
\index{Программный интерфейс ПЛК!Управление УЧПУ!Функция cncSetMode}
\label{sec:cncSetMode}

\begin{pHeader}
    Синтаксис:      & \RightHandText{void cncSetMode(CNCMode mode);}\\
    Аргумент(ы):    & \RightHandText{\myreftosec{CNCMode} mode ~-- идентификатор режима работы УЧПУ} \\   
%    Возвращаемое значение:       & \RightHandText{Нет} \\    
    Файл объявления:             & \RightHandText{include/cnc/cnc.h} \\
\end{pHeader}

Функция устанавливает режим работы УЧПУ, принимая в качестве аргумента значение одного из идентификаторов перечисления \myreftosec{CNCMode}. 

Является системной.
% *******end subsection*****************
%--------------------------------------------------------
% *******begin subsection***************
\subsubsection{\DbgSecSt{\StPart}{cncRequest}}
\index{Программный интерфейс ПЛК!Управление УЧПУ!Функция cncRequest}
\label{sec:cncRequest}

\begin{pHeader}
    Синтаксис:      & \RightHandText{void cncRequest (MTCNCRequests request);}\\
    Аргумент(ы):    & \RightHandText{\myreftosec{MTCNCRequests} request ~-- идентификатор команды управления станком} \\
%    Возвращаемое значение:       & \RightHandText{Нет} \\    
    Файл объявления:             & \RightHandText{include/cnc/cnc.h} \\
\end{pHeader}

Функция посылает команду УЧПУ, принимая в качестве аргумента значение одного из идентификаторов перечисления \myreftosec{MTCNCRequests}. 

Является системной.
% *******end subsection*****************

\begin{comment}
%--------------------------------------------------------
% *******begin subsection***************
\subsubsection{\DbgSecSt{\StPart}{void cncCustomRequest (MTCNCRequests request)}}
\index{Программный интерфейс ПЛК!Управление УЧПУ!void cncCustomRequest (MTCNCRequests request)}
\label{sec:cncCustomRequest}

\begin{pHeader}
%    Синтаксис:      & \RightHandText{void cncSetMode(CNCMode mode);}\\
    Аргумент(ы):    & \RightHandText{Идентификатор перечисления \myreftosec{MTCNCRequests}} \\
%    Возвращаемое значение:       & \RightHandText{Нет} \\    
    Файл объявления:             & \RightHandText{include/cnc/cnc.h} \\
\end{pHeader}

Функция посылает команду УЧПУ, принимая в качестве аргумента значение одного из идентификаторов перечисления \myreftosec{MTCNCRequests}. 
% *******end subsection*****************
\end{comment}
%--------------------------------------------------------
% *******begin subsection***************
\subsubsection{\DbgSecSt{\StPart}{cncChangeMode}}
\index{Программный интерфейс ПЛК!Управление УЧПУ!Функция cncChangeMode}
\label{sec:cncChangeMode}

\begin{pHeader}
    Синтаксис:      & \RightHandText{void cncChangeMode (int newMode);}\\
    Аргумент(ы):    & \RightHandText{int newMode ~-- идентификатор режима работы УЧПУ} \\
%    Возвращаемое значение:       & \RightHandText{Нет} \\    
    Файл объявления:             & \RightHandText{include/cnc/cnc.h} \\
\end{pHeader}

Функция выполняет запрос изменения режима работы УЧПУ. 

Является системной.
% *******end subsection*****************
%--------------------------------------------------------
% *******begin subsection***************
\subsubsection{\DbgSecSt{\StPart}{channelUpdate}}
\index{Программный интерфейс ПЛК!Управление УЧПУ!Функция channelUpdate}
\label{sec:channelUpdate}

\begin{pHeader}
    Синтаксис:      & \RightHandText{void channelUpdate (int channel);}\\
    Аргумент(ы):    & \RightHandText{int channel ~-- номер канала} \\
%    Возвращаемое значение:       & \RightHandText{Нет} \\    
    Файл объявления:             & \RightHandText{include/cnc/cnc.h} \\
\end{pHeader}

Функция обновляет данные канала, номер которого задаётся в качестве аргумента. 

Является системной.
% *******end subsection*****************
%--------------------------------------------------------
% *******begin subsection***************
\subsubsection{\DbgSecSt{\StPart}{cncModeManual}}
\index{Программный интерфейс ПЛК!Управление УЧПУ!Функция cncModeManual}
\label{sec:cncModeManual}

\begin{pHeader}
    Синтаксис:      & \RightHandText{void cncModeManual();}\\
    Аргумент(ы):    & \RightHandText{Нет} \\
%    Возвращаемое значение:       & \RightHandText{Нет} \\    
    Файл объявления:             & \RightHandText{include/cnc/cnc.h} \\
\end{pHeader}

Функция обработки команд в ручном режиме работы УЧПУ. 

Является системной.
% *******end subsection*****************
%--------------------------------------------------------
% *******begin subsection***************
\subsubsection{\DbgSecSt{\StPart}{cncModeHome}}
\index{Программный интерфейс ПЛК!Управление УЧПУ!Функция cncModeHome}
\label{sec:cncModeHome}

\begin{pHeader}
    Синтаксис:      & \RightHandText{void cncModeHome();}\\
    Аргумент(ы):    & \RightHandText{Нет} \\
%    Возвращаемое значение:       & \RightHandText{Нет} \\    
    Файл объявления:             & \RightHandText{include/cnc/cnc.h} \\
\end{pHeader}

Функция обработки команд в режиме выезда в нулевую точку УЧПУ. 

Является системной.
% *******end subsection*****************

%--------------------------------------------------------
% *******begin subsection***************
\subsubsection{\DbgSecSt{\StPart}{cncModeHandwheel}}
\index{Программный интерфейс ПЛК!Управление УЧПУ!Функция cncModeHandwheel}
\label{sec:cncModeHandwheel}

\begin{pHeader}
    Синтаксис:      & \RightHandText{void cncModeHandwheel();}\\
    Аргумент(ы):    & \RightHandText{Нет} \\
%    Возвращаемое значение:       & \RightHandText{Нет} \\    
    Файл объявления:             & \RightHandText{include/cnc/cnc.h} \\
\end{pHeader}

Функция обработки команд в режиме дискретных перемещений УЧПУ. 

Является системной.
% *******end subsection*****************
%--------------------------------------------------------
% *******begin subsection***************
\subsubsection{\DbgSecSt{\StPart}{cncModeAuto}}
\index{Программный интерфейс ПЛК!Управление УЧПУ!Функция cncModeAuto}
\label{sec:cncModeAuto}

\begin{pHeader}
    Синтаксис:      & \RightHandText{void cncModeAuto();}\\
    Аргумент(ы):    & \RightHandText{Нет} \\
%    Возвращаемое значение:       & \RightHandText{Нет} \\    
    Файл объявления:             & \RightHandText{include/cnc/cnc.h} \\
\end{pHeader}

Функция обработки команд в автоматическом режиме работы УЧПУ. 

Является системной.
% *******end subsection*****************
%--------------------------------------------------------
% *******begin subsection***************
\subsubsection{\DbgSecSt{\StPart}{cncModeMDI}}
\index{Программный интерфейс ПЛК!Управление УЧПУ!Функция cncModeMDI}
\label{sec:cncModeMDI}

\begin{pHeader}
    Синтаксис:      & \RightHandText{void cncModeMDI();}\\
    Аргумент(ы):    & \RightHandText{Нет} \\
%    Возвращаемое значение:       & \RightHandText{Нет} \\    
    Файл объявления:             & \RightHandText{include/cnc/cnc.h} \\
\end{pHeader}

Функция обработки команд в режиме преднабора УЧПУ. 

Является системной.
% *******end subsection*****************
\clearpage
%--------------------------------------------------------
% *******begin subsection***************
\subsubsection{\DbgSecSt{\StPart}{cncModeRepos}}
\index{Программный интерфейс ПЛК!Управление УЧПУ!Функция cncModeRepos}
\label{sec:cncModeRepos}

\begin{pHeader}
    Синтаксис:      & \RightHandText{void cncModeRepos();}\\
    Аргумент(ы):    & \RightHandText{Нет} \\
%    Возвращаемое значение:       & \RightHandText{Нет} \\    
    Файл объявления:             & \RightHandText{include/cnc/cnc.h} \\
\end{pHeader}

Функция обработки команд в режиме возврата на контур УЧПУ. 

Является системной.
% *******end subsection*****************
%--------------------------------------------------------
% *******begin subsection***************
\subsubsection{\DbgSecSt{\StPart}{cncManualEnter}}
\index{Программный интерфейс ПЛК!Управление УЧПУ!Функция cncManualEnter}
\label{sec:cncManualEnter}

\begin{pHeader}
    Синтаксис:      & \RightHandText{void cncManualEnter());}\\
    Аргумент(ы):    & \RightHandText{Нет} \\
%    Возвращаемое значение:       & \RightHandText{Нет} \\    
    Файл объявления:             & \RightHandText{include/cnc/cnc.h} \\
\end{pHeader}

Функция вызывается при установке ручного режима работы УЧПУ. В ней должны определяться действия, выполняемые при входе в данный режим. \killoverfullbefore

Реализуется пользователем. 
% *******end subsection*****************
%--------------------------------------------------------
% *******begin subsection***************
\subsubsection{\DbgSecSt{\StPart}{cncManualLeave}}
\index{Программный интерфейс ПЛК!Управление УЧПУ!Функция cncManualLeave}
\label{sec:cncManualLeave}

\begin{pHeader}
    Синтаксис:      & \RightHandText{int cncManualLeave (CNCMode newMode);}\\
    Аргумент(ы):    & \RightHandText{\myreftosec{CNCMode} newMode ~-- идентификатор режима работы УЧПУ} \\ 
    %    Возвращаемое значение:       & \RightHandText{Целое знаковое число} \\    
    Файл объявления:             & \RightHandText{include/cnc/cnc.h} \\
\end{pHeader}

Функция вызывается при выходе из ручного режима работы УЧПУ. В ней должны определяться действия, выполняемые при выходе из данного режима, а также осуществляться проверка возможности установки нового режима работы УЧПУ, который задаётся аргументом ~-- значением одного из идентификаторов перечисления \myreftosec{CNCMode}.\killoverfullbefore

 Возвращаемое значение должно быть отлично от 0 для разрешения нового режима работы. \killoverfullbefore

Реализуется пользователем.  
% *******end subsection*****************
%--------------------------------------------------------
% *******begin subsection***************
\subsubsection{\DbgSecSt{\StPart}{cncHwlEnter}}
\index{Программный интерфейс ПЛК!Управление УЧПУ!Функция cncHwlEnter}
\label{sec:cncHwlEnter}

\begin{pHeader}
    Синтаксис:      & \RightHandText{void cncHwlEnter();}\\
    Аргумент(ы):    & \RightHandText{Нет} \\
%    Возвращаемое значение:       & \RightHandText{Нет} \\    
    Файл объявления:             & \RightHandText{include/cnc/cnc.h} \\
\end{pHeader}

Функция вызывается при установке режима дискретных перемещений УЧПУ. В ней должны определяться действия, выполняемые при входе в данный режим. \killoverfullbefore

Реализуется пользователем. 
% *******end subsection*****************
%--------------------------------------------------------
% *******begin subsection***************
\subsubsection{\DbgSecSt{\StPart}{cncHwlLeave}}
\index{Программный интерфейс ПЛК!Управление УЧПУ!Функция cncHwlLeave}
\label{sec:cncHwlLeave}

\begin{pHeader}
    Синтаксис:      & \RightHandText{int cncHwlLeave (CNCMode newMode);}\\
    Аргумент(ы):    & \RightHandText{\myreftosec{CNCMode} newMode ~-- идентификатор режима работы УЧПУ} \\ 
%    Возвращаемое значение:       & \RightHandText{Целое знаковое число} \\    
    Файл объявления:             & \RightHandText{include/cnc/cnc.h} \\
\end{pHeader}

Функция вызывается при выходе из режима дискретных перемещений УЧПУ. В ней должны определяться действия, выполняемые при выходе из данного режима, а также осуществляться проверка возможности установки нового режима работы УЧПУ, который задаётся аргументом ~-- значением одного из идентификаторов перечисления \myreftosec{CNCMode}.\killoverfullbefore

 Возвращаемое значение должно быть отлично от 0 для разрешения нового режима работы. \killoverfullbefore

Реализуется пользователем. 
% *******end subsection*****************
%--------------------------------------------------------
% *******begin subsection***************
\subsubsection{\DbgSecSt{\StPart}{cncHomeEnter}}
\index{Программный интерфейс ПЛК!Управление УЧПУ!Функция cncHomeEnter}
\label{sec:cncHomeEnter}

\begin{pHeader}
    Синтаксис:      & \RightHandText{void cncHomeEnter();}\\
    Аргумент(ы):    & \RightHandText{Нет} \\
%    Возвращаемое значение:       & \RightHandText{Нет} \\    
    Файл объявления:             & \RightHandText{include/cnc/cnc.h} \\
\end{pHeader}

Функция вызывается при установке режима выезда в нулевую точку УЧПУ. В ней должны определяться действия, выполняемые при входе в данный режим. \killoverfullbefore

Реализуется пользователем. 
% *******end subsection*****************
%--------------------------------------------------------
% *******begin subsection***************
\subsubsection{\DbgSecSt{\StPart}{cncHomeLeave}}
\index{Программный интерфейс ПЛК!Управление УЧПУ!Функция cncHomeLeave}
\label{sec:cncHomeLeave}

\begin{pHeader}
    Синтаксис:      & \RightHandText{int cncHomeLeave (CNCMode newMode);}\\
    Аргумент(ы):    & \RightHandText{\myreftosec{CNCMode} newMode ~-- идентификатор режима работы УЧПУ} \\ 
%    Возвращаемое значение:       & \RightHandText{Целое знаковое число} \\    
    Файл объявления:             & \RightHandText{include/cnc/cnc.h} \\
\end{pHeader}

Функция вызывается при выходе из режима выезда в нулевую точку УЧПУ. В ней должны определяться действия, выполняемые при выходе из данного режима, а также осуществляться проверка возможности установки нового режима работы УЧПУ, который задаётся аргументом ~-- значением одного из идентификаторов перечисления \myreftosec{CNCMode}.\killoverfullbefore

 Возвращаемое значение должно быть отлично от 0 для разрешения нового режима работы. \killoverfullbefore

Реализуется пользователем.
% *******end subsection*****************
%--------------------------------------------------------
% *******begin subsection***************
\subsubsection{\DbgSecSt{\StPart}{cncAutoEnter}}
\index{Программный интерфейс ПЛК!Управление УЧПУ!Функция cncAutoEnter}
\label{sec:cncAutoEnter}

\begin{pHeader}
    Синтаксис:      & \RightHandText{void cncAutoEnter();}\\
    Аргумент(ы):    & \RightHandText{Нет} \\
%    Возвращаемое значение:       & \RightHandText{Нет} \\    
    Файл объявления:             & \RightHandText{include/cnc/cnc.h} \\
\end{pHeader}

Функция вызывается при установке автоматического режима УЧПУ. В ней должны определяться действия, выполняемые при входе в данный режим. \killoverfullbefore

Реализуется пользователем. 
% *******end subsection*****************
%--------------------------------------------------------
% *******begin subsection***************
\subsubsection{\DbgSecSt{\StPart}{cncAutoLeave}}
\index{Программный интерфейс ПЛК!Управление УЧПУ!Функция cncAutoLeave}
\label{sec:cncAutoLeave}

\begin{pHeader}
    Синтаксис:      & \RightHandText{int cncAutoLeave (CNCMode newMode);}\\
    Аргумент(ы):    & \RightHandText{\myreftosec{CNCMode} newMode ~-- идентификатор режима работы УЧПУ} \\ 
%    Возвращаемое значение:       & \RightHandText{Целое знаковое число} \\    
    Файл объявления:             & \RightHandText{include/cnc/cnc.h} \\
\end{pHeader}

Функция вызывается при выходе из автоматического режима УЧПУ. В ней должны определяться действия, выполняемые при выходе из данного режима, а также осуществляться проверка возможности установки нового режима работы УЧПУ, который задаётся аргументом ~-- значением одного из идентификаторов перечисления \myreftosec{CNCMode}. \killoverfullbefore

Возвращаемое значение должно быть отлично от 0 для разрешения нового режима работы. \killoverfullbefore

Реализуется пользователем.
% *******end subsection*****************
%--------------------------------------------------------
% *******begin subsection***************
\subsubsection{\DbgSecSt{\StPart}{cncMDIEnter}}
\index{Программный интерфейс ПЛК!Управление УЧПУ!Функция cncMDIEnter}
\label{sec:cncMDIEnter}

\begin{pHeader}
    Синтаксис:      & \RightHandText{void cncMDIEnter();}\\
    Аргумент(ы):    & \RightHandText{Нет} \\
%    Возвращаемое значение:       & \RightHandText{Нет} \\    
    Файл объявления:             & \RightHandText{include/cnc/cnc.h} \\
\end{pHeader}

Функция вызывается при установке режима преднабора УЧПУ. В ней должны определяться действия, выполняемые при входе в данный режим. \killoverfullbefore

Реализуется пользователем. 
% *******end subsection*****************

%--------------------------------------------------------
% *******begin subsection***************
\subsubsection{\DbgSecSt{\StPart}{cncMDILeave}}
\index{Программный интерфейс ПЛК!Управление УЧПУ!Функция cncMDILeave}
\label{sec:cncMDILeave}

\begin{pHeader}
    Синтаксис:      & \RightHandText{int cncMDILeave (CNCMode newMode);}\\
    Аргумент(ы):    & \RightHandText{\myreftosec{CNCMode} newMode ~-- идентификатор режима работы УЧПУ} \\ 
%    Возвращаемое значение:       & \RightHandText{Целое знаковое число} \\    
    Файл объявления:             & \RightHandText{include/cnc/cnc.h} \\
\end{pHeader}

Функция вызывается при выходе из режима преднабора УЧПУ. В ней должны определяться действия, выполняемые при выходе из данного режима, а также осуществляться проверка возможности установки нового режима работы УЧПУ, который задаётся аргументом ~-- значением одного из идентификаторов перечисления \myreftosec{CNCMode}.\killoverfullbefore

 Возвращаемое значение должно быть отлично от 0 для разрешения нового режима работы. \killoverfullbefore

Реализуется пользователем.
% *******end subsection*****************
%--------------------------------------------------------
% *******begin subsection***************
\subsubsection{\DbgSecSt{\StPart}{cncReposEnter}}
\index{Программный интерфейс ПЛК!Управление УЧПУ!Функция cncReposEnter}
\label{sec:cncReposEnter}

\begin{pHeader}
    Синтаксис:      & \RightHandText{void cncReposEnter();}\\
    Аргумент(ы):    & \RightHandText{Нет} \\
%    Возвращаемое значение:       & \RightHandText{Нет} \\    
    Файл объявления:             & \RightHandText{include/cnc/cnc.h} \\
\end{pHeader}

Функция вызывается при установке режима возврата на контур УЧПУ. В ней должны определяться действия, выполняемые при входе в данный режим. \killoverfullbefore

Реализуется пользователем. 
% *******end subsection*****************
%--------------------------------------------------------
% *******begin subsection***************
\subsubsection{\DbgSecSt{\StPart}{cncReposLeave}}
\index{Программный интерфейс ПЛК!Управление УЧПУ!Функция cncReposLeave}
\label{sec:cncReposLeave}

\begin{pHeader}
    Синтаксис:      & \RightHandText{int cncReposLeave (CNCMode newMode);}\\
    Аргумент(ы):    & \RightHandText{\myreftosec{CNCMode} newMode ~-- идентификатор режима работы УЧПУ} \\ 
%    Возвращаемое значение:       & \RightHandText{Целое знаковое число} \\    
    Файл объявления:             & \RightHandText{include/cnc/cnc.h} \\
\end{pHeader}

Функция вызывается при выходе из режима возврата на контур УЧПУ. В ней должны определяться действия, выполняемые при выходе из данного режима, а также осуществляться проверка возможности установки нового режима работы УЧПУ, который задаётся аргументом ~-- значением одного из идентификаторов перечисления \myreftosec{CNCMode}. \killoverfullbefore

Возвращаемое значение должно быть отлично от 0 для разрешения нового режима работы. \killoverfullbefore

Реализуется пользователем.
% *******end subsection*****************
%--------------------------------------------------------
% *******begin subsection***************
\subsubsection{\DbgSecSt{\StPart}{controlPowerCNC}}
\index{Программный интерфейс ПЛК!Управление УЧПУ!Функция controlPowerCNC}
\label{sec: controlPowerCNC}

\begin{pHeader}
    Синтаксис:      & \RightHandText{void controlPowerCNC (int request);}\\
    Аргумент(ы):    & \RightHandText{int request ~-- идентификатор команды управления станком} \\
%    Возвращаемое значение:       & \RightHandText{Нет} \\    
    Файл объявления:             & \RightHandText{include/cnc/cnc.h} \\
\end{pHeader}

Функция обработки запроса выключения УЧПУ и станка.  Аргументом функции является  значение одного из идентификаторов перечисления \myreftosec{MTCNCRequests}.

Является системной.
% *******end subsection*****************
%--------------------------------------------------------
% *******begin subsection***************
\subsubsection{\DbgSecSt{\StPart}{cncAutoOnProgramExit}}
\index{Программный интерфейс ПЛК!Управление УЧПУ!Функция cncAutoOnProgramExit}
\label{sec: cncAutoOnProgramExit}

\begin{pHeader}
    Синтаксис:      & \RightHandText{void cncAutoOnProgramExit (int channel);}\\
    Аргумент(ы):    & \RightHandText{int channel ~-- номер канала} \\
%    Возвращаемое значение:       & \RightHandText{Нет} \\    
    Файл объявления:             & \RightHandText{include/cnc/cnc.h} \\
\end{pHeader}

Функция вызывается при выходе из автоматического режима УЧПУ. В ней должны определяться действия, выполняемые при выходе из данного режима для канала, номер которого является аргументом функции. \killoverfullbefore

Реализуется пользователем.
% *******end subsection*****************
%--------------------------------------------------------
% *******begin subsection***************
\subsubsection{\DbgSecSt{\StPart}{cncCustomRequestManual}}
\index{Программный интерфейс ПЛК!Управление УЧПУ!Функция cncCustomRequestManual}
\label{sec: cncCustomRequestManual}

\begin{pHeader}
    Синтаксис:      & \RightHandText{void cncCustomRequestManual (int request);} \\
   Аргумент(ы):    & \RightHandText{int request ~-- идентификатор команды пользователя} \\
%    Возвращаемое значение:       & \RightHandText{Нет} \\    
    Файл объявления:             & \RightHandText{include/cnc/cnc.h} \\
\end{pHeader}

Функция обработки пользовательских команд в ручном режиме УЧПУ. Аргументом функции является команда пользователя.

Реализуется пользователем.
% *******end subsection*****************
%--------------------------------------------------------
% *******begin subsection***************
\subsubsection{\DbgSecSt{\StPart}{cncCustomRequestHome}}
\index{Программный интерфейс ПЛК!Управление УЧПУ!Функция cncCustomRequestHome}
\label{sec: cncCustomRequestHome}

\begin{pHeader}
    Синтаксис:      & \RightHandText{void cncCustomRequestHome (int request);}\\
   Аргумент(ы):    & \RightHandText{int request ~-- идентификатор команды пользователя} \\
%    Возвращаемое значение:       & \RightHandText{Нет} \\    
    Файл объявления:             & \RightHandText{include/cnc/cnc.h} \\
\end{pHeader}

Функция обработки пользовательских команд в режиме выезда в нулевую точку УЧПУ.  Аргументом функции является команда пользователя.

Реализуется пользователем.
% *******end subsection*****************
%--------------------------------------------------------
% *******begin subsection***************
\subsubsection{\DbgSecSt{\StPart}{cncCustomRequestAuto}}
\index{Программный интерфейс ПЛК!Управление УЧПУ!Функция cncCustomRequestAuto}
\label{sec: cncCustomRequestAuto}

\begin{pHeader}
    Синтаксис:      & \RightHandText{void cncCustomRequestAuto (int request);}\\
   Аргумент(ы):    & \RightHandText{int request ~-- идентификатор команды пользователя} \\
%    Возвращаемое значение:       & \RightHandText{Нет} \\    
    Файл объявления:             & \RightHandText{include/cnc/cnc.h} \\
\end{pHeader}

Функция обработки пользовательских команд в автоматическом режиме УЧПУ.  Аргументом функции является команда пользователя.

Реализуется пользователем.
% *******end subsection*****************
%--------------------------------------------------------
% *******begin subsection***************
\subsubsection{\DbgSecSt{\StPart}{cncCustomRequestMDI}}
\index{Программный интерфейс ПЛК!Управление УЧПУ!Функция cncCustomRequestMDI}
\label{sec: cncCustomRequestMDI}

\begin{pHeader}
    Синтаксис:      & \RightHandText{void cncCustomRequestMDI (int request);}\\
   Аргумент(ы):    & \RightHandText{int request ~-- идентификатор команды пользователя} \\
%    Возвращаемое значение:       & \RightHandText{Нет} \\    
    Файл объявления:             & \RightHandText{include/cnc/cnc.h} \\
\end{pHeader}

Функция обработки пользовательских команд в режиме преднабора УЧПУ.  Аргументом функции является команда пользователя.

Реализуется пользователем.
% *******end subsection*****************
%--------------------------------------------------------
% *******begin subsection***************
\subsubsection{\DbgSecSt{\StPart}{cncCustomRequestHwl}}
\index{Программный интерфейс ПЛК!Управление УЧПУ!Функции cncCustomRequestHwl}
\label{sec: cncCustomRequestHwl}

\begin{pHeader}
   Синтаксис:      & \RightHandText{void cncCustomRequestHwl (int request);}\\
    Аргумент(ы):    & \RightHandText{Целое знаковое число} \\   Аргумент(ы):    & \RightHandText{int request ~-- идентификатор команды пользователя} \\%    Возвращаемое значение:       & \RightHandText{Нет} \\    
    Файл объявления:             & \RightHandText{include/cnc/cnc.h} \\
\end{pHeader}

Функция обработки пользовательских команд в режиме дискретных перемещений УЧПУ.  Аргументом функции является команда пользователя.

Реализуется пользователем.
% *******end subsection*****************
%--------------------------------------------------------
% *******begin subsection***************
\subsubsection{\DbgSecSt{\StPart}{cncCustomRequestRepos}}
\index{Программный интерфейс ПЛК!Управление УЧПУ!Функция cncCustomRequestRepos}
\label{sec: cncCustomRequestRepos}

\begin{pHeader}
    Синтаксис:      & \RightHandText{void cncCustomRequestRepos (int request);}\\
   Аргумент(ы):    & \RightHandText{int request ~-- идентификатор команды пользователя} \\
%    Возвращаемое значение:       & \RightHandText{Нет} \\    
    Файл объявления:             & \RightHandText{include/cnc/cnc.h} \\
\end{pHeader}

Функция обработки пользовательских команд в режиме возврата на контур УЧПУ.  Аргументом функции является команда пользователя. 

Реализуется пользователем.
% *******end subsection*****************
%--------------------------------------------------------
% *******begin subsection***************
\subsubsection{\DbgSecSt{\StPart}{cncManualCanChangeOverride}}
\index{Программный интерфейс ПЛК!Управление УЧПУ!Функция cncManualCanChangeOverride}
\label{sec: cncManualCanChangeOverride}

\begin{pHeader}
    Синтаксис:      & \RightHandText{int cncManualCanChangeOverride();}\\
    Аргумент(ы):    & \RightHandText{Нет} \\
%    Возвращаемое значение:       & \RightHandText{Целое знаковое число} \\
    Файл объявления:             & \RightHandText{include/cnc/cnc.h} \\
\end{pHeader}

Функция выполняет запрос на разрешение применения коррекции подачи. \killoverfullbefore

Возвращает 1, если коррекция разрешена, и 0 в противном случае.

Реализуется пользователем.
% *******end subsection*****************
% *******end section*****************

%--------------------------------------------------------
