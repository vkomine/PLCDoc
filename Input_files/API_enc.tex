%--------------------------------------------------------
% *******begin section***************
\section{\DbgSecSt{\StPart}{Датчики обратной связи}}
%--------------------------------------------------------
\subsection{\DbgSecSt{\StPart}{Типы данных}}

% *******begin subsection***************
\subsubsection{\DbgSecSt{\StPart}{EncType}}
\index{Программный интерфейс ПЛК!Датчики обратной связи!Перечисление EncType}
\label{sec:EncType}

\begin{fHeader}
    Тип данных:            & \RightHandText{Перечисление EncType}\\
    Файл объявления:             & \RightHandText{include/func/enc.h} \\
\end{fHeader}

Перечисление определяет идентификаторы типов датчиков обратной связи.

\begin{MyTableTwoColAllCntr}{Перечисление EncType}{tbl:EncType}{|m{0.38\linewidth}|m{0.57\linewidth}|}{Идентификатор}{Описание}
\hline encNone & Нет \\
\hline encIncrement & Инкрементальный ДОС \\
\hline encSinCos & Синусно-косинусный ДОС \\
\hline encEnDat & ДОС с интерфейсом EnDat 2.2\\
\hline encBiSS & ДОС с интерфейсом BiSS \\
\end{MyTableTwoColAllCntr}
% *******end subsection***************
%--------------------------------------------------------
% *******begin subsection***************
\subsubsection{\DbgSecSt{\StPart}{EncConfig}}
\index{Программный интерфейс ПЛК!Датчики обратной связи!Структура EncConfig}
\label{sec:EncConfig}

\begin{fHeader}
    Тип данных:            & \RightHandText{Структура EncConfig} \\
    Файл объявления:             & \RightHandText{include/func/enc.h} \\
\end{fHeader}

Структура определяет параметры датчика.

\begin{MyTableThreeColAllCntr}{Структура EncConfig}{tbl:EncConfig}{|m{0.33\linewidth}|m{0.22\linewidth}|m{0.45\linewidth}|}{Элемент}{Тип}{Описание}
\hline servo & \centering{unsigned} & Номер платы (0$\div$3) \\
\hline chan & \centering{unsigned} & Номер канала (0$\div$7) \\
\hline type & \centering{unsigned} & Тип датчика (см. \myreftosec{EncType})\\
\end{MyTableThreeColAllCntr}
% *******end subsection***************
%--------------------------------------------------------
\begin{comment}
% *******begin subsection***************
\subsection{\DbgSecSt{\StPart}{Функции}}

% *******begin subsection***************
\subsubsection{\DbgSecSt{\StPart}{encoderScanErrors}}
\index{Программный интерфейс ПЛК!Датчики обратной связи!Функция encoderScanErrors}
\label{sec:encoderScanErrors}

\begin{pHeader}
    Синтаксис:      & \RightHandText{void encoderScanErrors(ErrorClear request);}\\
    Аргумент(ы):    & \RightHandText{\myreftosec{ErrorClear} request ~-- идентификатор типа сброса ошибки} \\    
%    Возвращаемое значение:       & \RightHandText{Нет} \\ 
    Файл объявления:             & \RightHandText{include/func/enc.h} \\       
\end{pHeader}

 

Является системной.
% *******end section*****************

%-------------------------------------------------------------------
% *******begin subsection***************
\subsubsection{\DbgSecSt{\StPart}{encoderErrorsReaction}}
\index{Программный интерфейс ПЛК!Датчики обратной связи!Функция encoderErrorsReaction}
\label{sec:encoderErrorsReaction}

\begin{pHeader}
    Синтаксис:      & \RightHandText{void encoderErrorsReaction();}\\
    Аргумент(ы):    & \RightHandText{Нет} \\    
%    Возвращаемое значение:       & \RightHandText{Нет} \\ 
    Файл объявления:             & \RightHandText{include/func/enc.h} \\       
\end{pHeader}



Является системной.
% *******end section*****************
\end{comment}

%--------------------------------------------------------
