%--------------------------------------------------------
% *******begin section***************
\section{\DbgSecSt{\StPart}{Управление шпинделями}}
%--------------------------------------------------------
\subsection{\DbgSecSt{\StPart}{Типы данных}}

% *******begin subsection***************
\subsubsection{\DbgSecSt{\StPart}{SpindleStates}}
\index{Программный интерфейс ПЛК!Управление шпинделями!Перечисление SpindleStates}
\label{sec:SpindleStates}

\begin{fHeader}
    Тип данных:            & \RightHandText{Перечисление SpindleStates}\\
    Файл объявления:             & \RightHandText{include/func/spnd.h} \\
\end{fHeader}

Перечисление определяет идентификаторы состояний шпинделя.

\begin{MyTableTwoColAllCntr}{Перечисление SpindleStates}{tbl:SpindleStates}{|m{0.38\linewidth}|m{0.57\linewidth}|}{Идентификатор}{Описание}
\hline spndInactive & Шпиндель выключен  \\
\hline spndActive &  Шпиндель находится в слежении \\
\hline spndCW & Вращение по часовой стрелке\\
\hline spndCCW & Вращение против часовой стрелки \\
\hline spndStopping & Останов шпинделя \\
\hline spndHomeWaitHW & Ожидание применения аппаратных настроек выезда в нулевую точку \\
\hline spndHoming & Выезд в нулевую точку \\
\hline spndIndexWaitHW & Ожидание применения аппаратных настроек поиска индексной метки \\
\hline spndIndexing & Поиск индексной метки \\
\hline spndAborting & Аварийное торможение \\
\hline spndWaitActivate & Ожидание включения \\
\hline spndWaitDeactivate & Ожидание выключения \\
\hline spndWaitPhaseRef & Ожидание фазировки \\
\end{MyTableTwoColAllCntr}
% *******end subsection***************
%--------------------------------------------------------
% *******begin subsection***************
\subsubsection{\DbgSecSt{\StPart}{SpindleCommands}}
\index{Программный интерфейс ПЛК!Управление шпинделями!Перечисление SpindleCommands}
\label{sec:SpindleCommands}

\begin{fHeader}
    Тип данных:            & \RightHandText{Перечисление SpindleCommands}\\
    Файл объявления:             & \RightHandText{include/func/spnd.h} \\
\end{fHeader}

Перечисление определяет идентификаторы команд управления шпинделями. 

\begin{MyTableTwoColAllCntr}{Перечисление SpindleCommands}{tbl:SpindleCommands}{|m{0.38\linewidth}|m{0.57\linewidth}|}{Идентификатор}{Описание}
\hline spndCmdIdle &  Нет команды  \\
\hline spndCmdKill  &  Выключить шпиндель \\
\hline spndCmdActivate  &  Включить шпиндель в слежение \\
\hline spndCmdDeactivate  &  Выключить шпиндель  \\
\hline spndCmdCW &  Выполнить вращение по часовой стрелке \\
\hline spndCmdCCW & Выполнить вращение против часовой стрелки \\
\hline spndCmdStop & Выполнить останов \\
\hline spndCmdInc  & Выполнить поворот на заданный угол \\
\hline spndCmdAbs & Выполнить поворот в заданное положение \\
\hline spndCmdHome & Выполнить движение в нулевую точку \\
\hline spndCmdIndex & Выполнить движение до индексной метки \\
\hline spndCmdPhaseRef & Выполнить фазировку \\
\hline spndCmdAbort & Выполнить аварийное выключение \\
\end{MyTableTwoColAllCntr}
% *******end subsection***************
%--------------------------------------------------------
% *******begin subsection***************
\subsubsection{\DbgSecSt{\StPart}{SpindleStage}}
\index{Программный интерфейс ПЛК!Управление шпинделями!Структура SpindleStage}
\label{sec:SpindleStage}

\begin{fHeader}
    Тип данных:            & \RightHandText{Структура SpindleStage} \\
    Файл объявления:             & \RightHandText{include/func/spnd.h} \\
\end{fHeader}

Структура определяет настройки ступенчатого разгона шпинделя.

\begin{MyTableThreeColAllCntr}{Структура SpindleStage}{tbl:SpindleStage}{|m{0.33\linewidth}|m{0.22\linewidth}|m{0.45\linewidth}|}{Элемент}{Тип}{Описание}
\hline Vmin & \centering{double} & Минимальная скорость \\
\hline Vmax & \centering{doubled} & Максимальная скорость \\
\hline Amax & \centering{double} & Максимальное ускорение \\
\hline Jmax & \centering{double} & Максимальный рывок \\
\end{MyTableThreeColAllCntr}
% *******end subsection***************
%--------------------------------------------------------
%--------------------------------------------------------
% *******begin subsection***************
\subsubsection{\DbgSecSt{\StPart}{SpindleConfig}}
\index{Программный интерфейс ПЛК!Управление шпинделями!Структура SpindleConfig}
\label{sec:SpindleConfig}

\begin{fHeader}
    Тип данных:            & \RightHandText{Структура SpindleConfig} \\
    Файл объявления:             & \RightHandText{include/func/spnd.h} \\
\end{fHeader}

Структура определяет настройки шпинделя.

\begin{MyTableThreeColAllCntr}{Структура SpindleConfig}{tbl:SpindleConfig}{|m{0.35\linewidth}|m{0.2\linewidth}|m{0.45\linewidth}|}{Элемент}{Тип}{Описание}
\hline servo & \centering{unsigned} & Номер платы управления ($0\div3$) \\
\hline chan & \centering{unsigned} & Номер канала ($0\div7$) \\
\hline motor & \centering{unsigned} & Номер связанного с осью двигателя ($0\div31$)\\
\hline homeOrder & \centering{unsigned} & Порядок выезда в нулевую точку \\
\hline needDKill & \centering{Битовое поле:1} & Требуется задержка перед отключением \\
\hline needPhaseRef & \centering{Битовое поле:1} & Требуется фазировка \\

\hline needHome & \centering{Битовое поле:1} & Требуется выезд в нулевую точку \\
\hline needIndex & \centering{Битовое поле:1} & Требуется позиционирование по индексной метке \\
\hline hasAbsPos & \centering{Битовое поле:1} & Установлен абсолютный датчик \\

\hline abortMode & \centering{Битовое поле:2} & Реакции на команду аварийного выключения (см. \myreftosec{AxisAbortMode})\\
\hline homeCaptCtrl & \centering{Битовое поле:4} & Настройка CaptCtrl для выезда в нулевую точку по входу (флагу) \\
\hline indexCaptCtrl & \centering{Битовое поле:4} & Настройка CaptCtrl для выезда в нулевую точку по индексной метке \\
\hline killAfterStop & \centering{Битовое поле:1} &  Выключение после останова \\
\hline reserved & \centering{Битовое поле:16} &  Резерв \\
\hline homeVel & \centering{double} & Скорость выезда в нулевую точку \\
\hline indexVel & \centering{double} & Скорость поиска индексной метки \\
\hline homeOffset & \centering{double} & Смещение нулевой точки относительно позиции ДОС\\
\hline indexOffset & \centering{double} & Смещение индексной метки относительно позиции ДОС\\
\hline homeOfsVel & \centering{double} & Скорость движения в позицию смещения нулевой точки (не используется)\\
\hline indexOfsVel & \centering{double} & Скорость движения в позицию смещения индексной метки (не используется)\\
\hline minPos & \centering{double} & Программное ограничение в отрицательном направлении (для абсолютного ДОС настраивается в дискретах датчика, определённых в энкодерной таблице) \\
\hline maxPos & \centering{double} &  Программное ограничение в положительном направлении (для абсолютного ДОС настраивается в дискретах датчика, определённых в энкодерной таблице) \\
\hline spinEncRes & \centering{double} & Число дискрет датчика на оборот \\
\hline atSpeedBand & \centering{double} & Амплитуда зоны ошибки заданной скорости \\
\hline stages\newline[ЧИСЛО\_СТУПЕНЕЙ\_РАЗГОНА] & \centering{\myreftosec{SpindleStage}} & Параметры ступенчатого разгона \\
\end{MyTableThreeColAllCntr}
% *******end subsection***************

Поля \texttt{homeCaptCtrl} и \texttt{indexCaptCtrl} являются 4-битными и содержат настройки захвата положения для выезда в нулевую точку:
\begin{itemize}
\item биты 0 и 1 определяют тип захвата положения (0 ~-- непосредственный захват, 1 ~--  по индексному сигналу датчика, 2 ~-- захват по флагу, 3 ~-- по флагу и индексному сигналу); \killoverfullbefore
\item бит 2 управляет инверсией индексного сигнала ДОС (0 ~-- не инвертировать, 1 ~-- инвертировать); \killoverfullbefore
\item бит 3 управляет инверсией флага при захвате положения (0 ~-- не инвертировать, 1 ~-- инвертировать). \killoverfullbefore \BL
\end{itemize} 
%--------------------------------------------------------
% *******begin subsection***************
\subsubsection{\DbgSecSt{\StPart}{Spindle}}
\index{Программный интерфейс ПЛК!Управление шпинделями!Структура Spindle}
\label{sec:Spindle}

\begin{fHeader}
    Тип данных:            & \RightHandText{Структура Spindle}\\
    Файл объявления:             & \RightHandText{include/func/spnd.h} \\
\end{fHeader}

Структура определяет состояние, параметры и данные шпинделя.

\begin{MyTableThreeColAllCntr}{Структура Spindle}{tbl:Spindle}{|m{0.31\linewidth}|m{0.29\linewidth}|m{0.4\linewidth}|}{Элемент}{Тип}{Описание}
\hline state & \centering{\myreftosec{SpindleStates}} & Текущее состояние \\
\hline command & \centering{\myreftosec{SpindleCommands}} & Текущая команда \\
\hline commandAfterActivate & \centering{\myreftosec{SpindleCommands}} & Команда после включения \\
\hline followup & \centering{Битовое поле:1} & Восстановление после ошибки \\
\hline phaseRefComplete & \centering{Битовое поле:1} & Фазировка выполнена \\
\hline phaseRefError & \centering{Битовое поле:1} & Ошибка фазировки \\
\hline homeComplete & \centering{Битовое поле:1} & Выполнен выезд в нулевую точку \\
\hline homeErrorFlag & \centering{Битовое поле:1} & Ошибка выезда в нулевую точку \\
\hline atSpeed & \centering{Битовое поле:1} & Заданная скорость достигнута \\
\hline timer & \centering{\myreftosec{Timer}} & Таймер \\
\hline SpeedValue & \centering{double} &  Значение скорости \\
\hline SpeedOverride & \centering{double} & Значение корректора скорости \\
\hline spinStage & \centering{unsigned} & Номер ступени разгона \\
\hline platform & \centering{\hyperlink{Spindle_Platform_Control}{SpindlePlatformControl}} & Пользовательские параметры и переменные шпинделя \\
\end{MyTableThreeColAllCntr}

\hypertarget{Spindle_Platform_Control}{Структура} \texttt{SpindlePlatformControl} является пользовательской и служит для введения дополнительных параметров и переменных шпинделя.\killoverfullbefore

Если структура \texttt{SpindlePlatformControl} задана пользователем, то должен быть определён идентификатор \texttt{SPINDLE\_PLATFORM\_CONTROL\_DEFINED}: \texttt{\#define SPINDLE\_PLATFORM\_CONTROL\_DEFINED}. \killoverfullbefore
% *******end subsection***************
%--------------------------------------------------------
% *******begin subsection***************
\subsubsection{\DbgSecSt{\StPart}{SpindleControl}}
\index{Программный интерфейс ПЛК!Управление шпинделями!Структура SpindleControl}
\label{sec:SpindleControl}

\begin{fHeader}
    Тип данных:            & \RightHandText{Структура SpindleControl}\\
    Файл объявления:             & \RightHandText{include/func/spnd.h} \\
\end{fHeader}

Структура определяет состояние, параметры и данные шпинделей. 

\begin{MyTableThreeColAllCntr}{Структура SpindleControl}{tbl:SpindleControl}{|m{0.31\linewidth}|m{0.29\linewidth}|m{0.4\linewidth}|}{Элемент}{Тип}{Описание}
\hline homeState & \centering{\myreftosec{HomeStates}} & Состояние выезда в нулевую точку\\
\hline homeComplete & \centering{Битовое поле:1} & Выполнен выезд в нулевую точку \\
\hline homeErrorFlag & \centering{Битовое поле:1} & Ошибка выезда в нулевую точку \\
\hline homeStage & \centering{int} & Этап выезда в нулевую точку \\
\hline spin[ЧИСЛО\_ШПИНДЕЛЕЙ] & \centering{\myreftosec{Spindle}} & Данные шпинделя(-ей) \\
\hline timerHome & \centering{\myreftosec{Timer}} & Таймер для задержек переключений состояний в режиме выезда в нулевую точку\\
\hline platform & \centering{\hyperlink{Spindles_Platform_Control}{SpindlesPlatformControl}} & Пользовательские параметры и переменные шпинделей \\
\end{MyTableThreeColAllCntr}

\hypertarget{Spindles_Platform_Control}{Структура} \texttt{SpindlesPlatformControl} является пользовательской и служит для введения дополнительных параметров и переменных шпинделей.\killoverfullbefore

Если структура \texttt{SpindlesPlatformControl} задана пользователем, то должен быть определён идентификатор \texttt{SPINDLE\_PLATFORM\_CONTROL\_DEFINED}: \texttt{\#define SPINDLE\_PLATFORM\_CONTROL\_DEFINED}. \killoverfullbefore
% *******end subsection***************

%-------------------------------------------------------------------
% *******begin subsection***************
\subsection{\DbgSecSt{\StPart}{Функции}}

% *******begin subsection***************
\subsubsection{\DbgSecSt{\StPart}{spinsForceKill}}
\index{Программный интерфейс ПЛК!Управление шпинделями!Функция spinsForceKill}
\label{sec:spinsForceKill}

\begin{pHeader}
    Синтаксис:      & \RightHandText{void spinsForceKill();}\\
    Аргумент(ы):    & \RightHandText{Нет} \\    
%    Возвращаемое значение:       & \RightHandText{Нет} \\ 
    Файл объявления:             & \RightHandText{include/func/spnd.h} \\       
\end{pHeader}

Функция вызывает принудительное выключение всех шпинделей.

Является системной.
% *******end section*****************

%-------------------------------------------------------------------
% *******begin subsection***************
\subsubsection{\DbgSecSt{\StPart}{spinForceKill}}
\index{Программный интерфейс ПЛК!Управление шпинделями!Функция spinForceKill}
\label{sec:spinForceKill}

\begin{pHeader}
    Синтаксис:      & \RightHandText{void spinForceKill(unsigned spin);}\\
    Аргумент(ы):    & \RightHandText{unsigned spin ~-- номер шпинделя} \\ 
%    Возвращаемое значение:       & \RightHandText{Нет} \\ 
    Файл объявления:             & \RightHandText{include/func/spnd.h} \\       
\end{pHeader}

Функция вызывает принудительное выключение шпинделя, номер которого является аргументом функции.

Является системной.
% *******end section*****************
%-------------------------------------------------------------------
% *******begin subsection***************
\subsubsection{\DbgSecSt{\StPart}{spinsDeactivate}}
\index{Программный интерфейс ПЛК!Управление шпинделями!Функция spinsDeactivate}
\label{sec:spinsDeactivate}

\begin{pHeader}
    Синтаксис:      & \RightHandText{void spinsDeactivate();}\\
    Аргумент(ы):    & \RightHandText{Нет} \\    
%    Возвращаемое значение:       & \RightHandText{Нет} \\ 
    Файл объявления:             & \RightHandText{include/func/spnd.h} \\       
\end{pHeader}

Функция вызывает выключение всех шпинделей.

Является системной.
% *******end section*****************
%-------------------------------------------------------------------
% *******begin subsection***************
\subsubsection{\DbgSecSt{\StPart}{spinsActivate}}
\index{Программный интерфейс ПЛК!Управление шпинделями!Функция spinsActivate}
\label{sec:spinsActivate}

\begin{pHeader}
    Синтаксис:      & \RightHandText{void spinsActivate();}\\
    Аргумент(ы):    & \RightHandText{Нет} \\    
%    Возвращаемое значение:       & \RightHandText{Нет} \\ 
    Файл объявления:             & \RightHandText{include/func/spnd.h} \\
\end{pHeader}

Функция вызывает включение в слежение всех шпинделей.

Является системной.
% *******end section*****************
\clearpage
%-------------------------------------------------------------------
% *******begin subsection***************
\subsubsection{\DbgSecSt{\StPart}{spinsInactive}}
\index{Программный интерфейс ПЛК!Управление шпинделями!Функция spinsInactive}
\label{sec:spinsInactive}

\begin{pHeader}
    Синтаксис:      & \RightHandText{int spinsInactive();}\\
    Аргумент(ы):    & \RightHandText{Нет} \\    
%    Возвращаемое значение:       & \RightHandText{Целое знаковое число} \\ 
    Файл объявления:             & \RightHandText{include/func/spnd.h} \\       
\end{pHeader}

Функция возвращает 1, если хотя бы один шпиндель не находится в слежении, и 0 в противном случае.

Является системной.
% *******end section*****************
%-------------------------------------------------------------------
% *******begin subsection***************
\subsubsection{\DbgSecSt{\StPart}{spinsActive}}
\index{Программный интерфейс ПЛК!Управление шпинделями!Функция spinsActive}
\label{sec:spinsActive}

\begin{pHeader}
    Синтаксис:      & \RightHandText{int spinsActive();}\\
    Аргумент(ы):    & \RightHandText{Нет} \\    
%    Возвращаемое значение:       & \RightHandText{Целое знаковое число} \\ 
    Файл объявления:             & \RightHandText{include/func/spnd.h} \\       
\end{pHeader}

Функция возвращает 1, если все шпиндели находятся в слежении, и 0 в противном случае.

Является системной.
% *******end section*****************
%-------------------------------------------------------------------
% *******begin subsection***************
\subsubsection{\DbgSecSt{\StPart}{spinsPhaseRefComplete}}
\index{Программный интерфейс ПЛК!Управление шпинделями!Функция spinsPhaseRefComplete}
\label{sec:spinsPhaseRefComplete}

\begin{pHeader}
    Синтаксис:      & \RightHandText{int spinsPhaseRefComplete();}\\
    Аргумент(ы):    & \RightHandText{Нет} \\    
%    Возвращаемое значение:       & \RightHandText{Целое знаковое число} \\ 
    Файл объявления:             & \RightHandText{include/func/spnd.h} \\       
\end{pHeader}

Функция возвращает 1, если фазировка выполнена для всех шпинделей, и 0 в противном случае.

Является системной.
% *******end section*****************
%-------------------------------------------------------------------
% *******begin subsection***************
\subsubsection{\DbgSecSt{\StPart}{spinsPhaseRef}}
\index{Программный интерфейс ПЛК!Управление шпинделями!Функция spinsPhaseRef}
\label{sec:spinsPhaseRef}

\begin{pHeader}
    Синтаксис:      & \RightHandText{int spinsPhaseRef();}\\
    Аргумент(ы):    & \RightHandText{Нет} \\    
%    Возвращаемое значение:       & \RightHandText{Целое знаковое число} \\ 
    Файл объявления:             & \RightHandText{include/func/spnd.h} \\       
\end{pHeader}

Функция возвращает 1, если хотя бы один шпиндель требует фазировки, и 0 в противном случае. Для шпинделя, фазировка которого не выполнена, даётся команда фазировки.

Является системной.
% *******end section*****************

%--------------------------------------------------------
% *******begin subsection***************
\subsubsection{\DbgSecSt{\StPart}{spinAborted}}
\index{Программный интерфейс ПЛК!Управление шпинделями!Функция spinAborted}
\label{sec:spinAborted}

\begin{pHeader}
    Синтаксис:      & \RightHandText{int spinAborted(unsigned spin);}\\
    Аргумент(ы):    & \RightHandText{unsigned spin ~-- номер шпинделя} \\ 
%    Возвращаемое значение:       & \RightHandText{Целое знаковое число} \\ 
    Файл объявления:             & \RightHandText{include/func/spnd.h} \\
\end{pHeader}

Функция возвращает 1, если шпиндель, номер которого является аргументом функции, аварийно остановлен, и 0 в противном случае.  

Является системной.
% *******end section*****************
%-------------------------------------------------------------------
% *******begin subsection***************
\subsubsection{\DbgSecSt{\StPart}{spinsAborted}}
\index{Программный интерфейс ПЛК!Управление шпинделями!Функция spinsAborted}
\label{sec:spinsAborted}

\begin{pHeader}
    Синтаксис:      & \RightHandText{int spinsAborted();}\\
    Аргумент(ы):    & \RightHandText{Нет} \\    
%    Возвращаемое значение:       & \RightHandText{Целое знаковое число} \\ 
    Файл объявления:             & \RightHandText{include/func/spnd.h} \\       
\end{pHeader}

Функция возвращает 1, если все шпиндели аварийно остановлены, и 0 в противном случае.

Является системной.
% *******end section*****************
%--------------------------------------------------------
% *******begin subsection***************
\subsubsection{\DbgSecSt{\StPart}{spinIsStopped}}
\index{Программный интерфейс ПЛК!Управление шпинделями!Функция spinIsStopped}
\label{sec:spinIsStopped}

\begin{pHeader}
    Синтаксис:      & \RightHandText{int spinIsStopped(unsigned spin);}\\
    Аргумент(ы):    & \RightHandText{unsigned spin ~-- номер шпинделя} \\ 
%    Возвращаемое значение:       & \RightHandText{Целое знаковое число} \\ 
    Файл объявления:             & \RightHandText{include/func/spnd.h} \\
\end{pHeader}

Функция возвращает 1, если шпиндель, номер которого является аргументом функции, остановлен (в слежении имеет равную нулю заданную скорость), и 0 в противном случае.  

Является системной.
% *******end section*****************
%-------------------------------------------------------------------
% *******begin subsection***************
\subsubsection{\DbgSecSt{\StPart}{spinsStopped}}
\index{Программный интерфейс ПЛК!Управление шпинделями!Функция spinsStopped}
\label{sec:spinsStopped}

\begin{pHeader}
    Синтаксис:      & \RightHandText{int spinsStopped();}\\
    Аргумент(ы):    & \RightHandText{Нет} \\    
%    Возвращаемое значение:       & \RightHandText{Целое знаковое число} \\ 
    Файл объявления:             & \RightHandText{include/func/spnd.h} \\
\end{pHeader}

Функция возвращает 1, если все шпиндели остановлены (в слежении имеют равную нулю заданную скорость), и 0 в противном случае.

Является системной.
% *******end section*****************

%-------------------------------------------------------------------
% *******begin subsection***************
\subsubsection{\DbgSecSt{\StPart}{spinsAbortAll}}
\index{Программный интерфейс ПЛК!Управление шпинделями!Функция spinsAbortAll}
\label{sec:spinsAbortAll}

\begin{pHeader}
    Синтаксис:      & \RightHandText{void spinsAbortAll();}\\
    Аргумент(ы):    & \RightHandText{Нет} \\    
%    Возвращаемое значение:       & \RightHandText{Нет} \\ 
    Файл объявления:             & \RightHandText{include/func/spnd.h} \\
\end{pHeader}

Функция вызывает аварийное выключение всех шпинделей.

Является системной.
% *******end section*****************
%-------------------------------------------------------------------
% *******begin subsection***************
\subsubsection{\DbgSecSt{\StPart}{spinsStopAll}}
\index{Программный интерфейс ПЛК!Управление шпинделями!Функция spinsStopAll}
\label{sec:spinsStopAll}

\begin{pHeader}
    Синтаксис:      & \RightHandText{void spinsStopAll();}\\
    Аргумент(ы):    & \RightHandText{Нет} \\    
%    Возвращаемое значение:       & \RightHandText{Нет} \\ 
    Файл объявления:             & \RightHandText{include/func/spnd.h} \\
\end{pHeader}

Функция вызывает останов всех шпинделей при толчковых перемещениях.

Является системной.
% *******end section*****************
%--------------------------------------------------------
% *******begin subsection***************
\subsubsection{\DbgSecSt{\StPart}{spinAtSpeed}}
\index{Программный интерфейс ПЛК!Управление шпинделями!Функция spinAtSpeed}
\label{sec:spinAtSpeed}

\begin{pHeader}
    Синтаксис:      & \RightHandText{int spinAtSpeed(unsigned spin);}\\
    Аргумент(ы):    & \RightHandText{unsigned spin ~-- номер шпинделя} \\ 
%    Возвращаемое значение:       & \RightHandText{Целое знаковое число} \\ 
    Файл объявления:             & \RightHandText{include/func/spnd.h} \\
\end{pHeader}

Функция возвращает 1, если шпиндель, номер которого является аргументом функции, имеет скорость равную заданной, и 0 в противном случае. 

Является системной.
% *******end section*****************
%--------------------------------------------------------
% *******begin subsection***************
\subsubsection{\DbgSecSt{\StPart}{spinPosition}}
\index{Программный интерфейс ПЛК!Управление шпинделями!Функция spinPosition}
\label{sec:spinPosition}

\begin{pHeader}
    Синтаксис:      & \RightHandText{double spinPosition(unsigned spin);}\\
    Аргумент(ы):    & \RightHandText{unsigned spin ~-- номер шпинделя} \\ 
%    Возвращаемое значение:       & \RightHandText{Число с плавающей запятой двойной точности} \\ 
    Файл объявления:             & \RightHandText{include/func/spnd.h} \\
\end{pHeader}

Функция возвращает заданную позицию шпинделя, номер которого является аргументом функции.

Возвращаемое значение измеряется в единицах \texttt{spinEncRes} (см. структуру \myreftosec{SpindleConfig}). \killoverfullbefore

Является системной.
% *******end section*****************
%--------------------------------------------------------
% *******begin subsection***************
\subsubsection{\DbgSecSt{\StPart}{spinSpeedCommand}}
\index{Программный интерфейс ПЛК!Управление шпинделями!Функция spinSpeedCommand}
\label{sec:spinSpeedCommand}

\begin{pHeader}
    Синтаксис:      & \RightHandText{void spinSpeedCommand(unsigned spin, double speed,}\\
    & \RightHandText{int direction);}\\
    Аргумент(ы):    & \RightHandText{unsigned spin ~-- номер шпинделя, } \\ 
    & \RightHandText{double speed ~--  скорость,} \\   
    & \RightHandText{int direction ~-- направление вращения} \\   
%    Возвращаемое значение:       & \RightHandText{Нет} \\ 
    Файл объявления:             & \RightHandText{include/func/spnd.h} \\
\end{pHeader}

Функция задаёт скорость и направление вращения шпинделя, номер которого является аргументом функции. 

Является системной.
% *******end section*****************
%--------------------------------------------------------
% *******begin subsection***************
\subsubsection{\DbgSecSt{\StPart}{spinCurStage}}
\index{Программный интерфейс ПЛК!Управление шпинделями!Функция spinCurStage}
\label{sec:spinCurStage}

\begin{pHeader}
    Синтаксис:      & \RightHandText{unsigned spinCurStage(unsigned spin);}\\
    Аргумент(ы):    & \RightHandText{unsigned spin ~-- номер шпинделя} \\ 
%    Возвращаемое значение:       & \RightHandText{Целое беззнаковое число} \\ 
    Файл объявления:             & \RightHandText{include/func/spnd.h} \\
\end{pHeader}

Функция возвращает номер ступени разгона шпинделя, номер которого является аргументом функции. 

Является системной.
% *******end section*****************
\clearpage

%--------------------------------------------------------
% *******begin subsection***************
\subsubsection{\DbgSecSt{\StPart}{spinNeedChangeStage}}
\index{Программный интерфейс ПЛК!Управление шпинделями!Функция spinNeedChangeStage}
\label{sec:spinNeedChangeStage}

\begin{pHeader}
    Синтаксис:      & \RightHandText{int spinNeedChangeStage(unsigned spin);}\\
    Аргумент(ы):    & \RightHandText{unsigned spin ~-- номер шпинделя} \\ 
%    Возвращаемое значение:       & \RightHandText{Целое знаковое число} \\ 
    Файл объявления:             & \RightHandText{include/func/spnd.h} \\
\end{pHeader}

Функция возвращает 1, если требуется смена ступени разгона шпинделя, номер которого является аргументом функции, и 0 в противном случае.

Является системной.
% *******end section*****************
%--------------------------------------------------------
% *******begin subsection***************
\subsubsection{\DbgSecSt{\StPart}{spinsFollowup}}
\index{Программный интерфейс ПЛК!Управление шпинделями!Функция spinsFollowup}
\label{sec:spinsFollowup}

\begin{pHeader}
    Синтаксис:      & \RightHandText{void spinsFollowup();}\\
    Аргумент(ы):    & \RightHandText{Нет} \\    
%    Возвращаемое значение:       & \RightHandText{Нет} \\ 
    Файл объявления:             & \RightHandText{include/func/axis.h} \\
\end{pHeader}

Функция устанавливает для всех шпинделей флаг <<Восстановление после ошибки>> (см. структуру \myreftosec{Spindle}).

Является системной.
% *******end section*****************

%--------------------------------------------------------
% *******begin subsection***************
\subsubsection{\DbgSecSt{\StPart}{initSpindle}}
\index{Программный интерфейс ПЛК!Управление шпинделями!Функция initSpindle}
\label{sec:initSpindle}

\begin{pHeader}
    Синтаксис:      & \RightHandText{void initSpindle(int spin);}\\
    Аргумент(ы):    & \RightHandText{int spin ~-- номер шпинделя} \\ 
%    Возвращаемое значение:       & \RightHandText{Нет} \\ 
    Файл объявления:             & \RightHandText{include/func/axis.h} \\
\end{pHeader}

Функция выполняет инициализацию шпинделя, номер которого является аргументом функции, параметрами по умолчанию. 

Является системной.
% *******end section*****************

%--------------------------------------------------------
% *******begin subsection***************
\subsubsection{\DbgSecSt{\StPart}{initSpindles}}
\index{Программный интерфейс ПЛК!Управление шпинделями!Функция initSpindles}
\label{sec:initSpindles}

\begin{pHeader}
    Синтаксис:      & \RightHandText{void initSpindles();}\\
    Аргумент(ы):    & \RightHandText{Нет} \\    
%    Возвращаемое значение:       & \RightHandText{Нет} \\ 
    Файл объявления:             & \RightHandText{include/func/axis.h} \\
\end{pHeader}

Функция выполняет инициализацию шпинделей параметрами по умолчанию. 

Является системной.
% *******end section*****************
%--------------------------------------------------------
% *******begin subsection***************
\subsubsection{\DbgSecSt{\StPart}{spinInitPlatform}}
\index{Программный интерфейс ПЛК!Управление шпинделями!Функция spinInitPlatform}
\label{sec:spinInitPlatform}

\begin{pHeader}
    Синтаксис:      & \RightHandText{void spinInitPlatform(int spin);}\\
    Аргумент(ы):    & \RightHandText{int spin ~-- номер шпинделя} \\    
%    Возвращаемое значение:       & \RightHandText{Нет} \\ 
    Файл объявления:             & \RightHandText{include/func/axis.h} \\
\end{pHeader}

Функция выполняет инициализацию параметров шпинделя, номер которого является аргументом функции, пользовательскими значениями, в том числе структуры \hyperlink{Spindle_Platform_Control}{SpindlePlatformControl}. \killoverfullbefore 

Реализуется пользователем.
% *******end section*****************
%--------------------------------------------------------
% *******begin subsection***************
\subsubsection{\DbgSecSt{\StPart}{spinsInitPlatform}}
\index{Программный интерфейс ПЛК!Управление шпинделями!Функция spinsInitPlatform}
\label{sec:spinsInitPlatform}

\begin{pHeader}
    Синтаксис:      & \RightHandText{void spinsInitPlatform();}\\
    Аргумент(ы):    & \RightHandText{Нет} \\    
%%    Возвращаемое значение:       & \RightHandText{Нет} \\ 
    Файл объявления:             & \RightHandText{include/func/axis.h} \\
\end{pHeader}

Функция выполняет инициализацию параметров шпинделей пользовательскими значениями, в том числе структуры \hyperlink{Spindles_Platform_Control}{SpindlesPlatformControl}. \killoverfullbefore 

Реализуется пользователем.
% *******end section*****************
