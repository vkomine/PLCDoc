%--------------------------------------------------------
% *******begin section***************
\section{\DbgSecSt{\StPart}{Реферирование осей}}
%--------------------------------------------------------
\subsection{\DbgSecSt{\StPart}{Типы данных}}

% *******begin subsection***************
\subsubsection{\DbgSecSt{\StPart}{HomeStates}}
\index{Программный интерфейс ПЛК!Реферирование осей!Перечисление HomeStates}
\label{sec:HomeStates}

\begin{fHeader}
    Тип данных:            & \RightHandText{Перечисление HomeStates}\\
    Файл объявления:             & \RightHandText{include/func/home.h} \\
\end{fHeader}

Перечисление определяет идентификаторы состояний выезда в нулевую точку. \killoverfullbefore

\begin{MyTableTwoColAllCntr}{Перечисление HomeStates}{tbl:HomeStates}{|m{0.38\linewidth}|m{0.57\linewidth}|}{Идентификатор}{Описание}
\hline homeReady & Выезд в ноль не выполнен \\
\hline homeStart &  Начало выезда в нулевую точку \\
\hline homeWaitStage & Ожидание этапа выезда в нулевую точку \\
\hline homeComplete & Выезд в нулевую точку выполнен \\
\hline homeError & Ошибка выезда в нулевую точку \\
\end{MyTableTwoColAllCntr}
% *******end subsection***************
%--------------------------------------------------------
% *******begin subsection***************
\subsection{\DbgSecSt{\StPart}{Функции}}

% *******begin subsection***************
\subsubsection{\DbgSecSt{\StPart}{isHomeComplete}}
\index{Программный интерфейс ПЛК!Реферирование осей!Функция isHomeComplete}
\label{sec:isHomeComplete}

\begin{pHeader}
    Синтаксис:      & \RightHandText{int isHomeComplete();}\\
    Аргумент(ы):    & \RightHandText{Нет} \\    
%    Возвращаемое значение:       & \RightHandText{Целое знаковое число} \\ 
    Файл объявления:             & \RightHandText{include/func/home.h} \\       
\end{pHeader}

Функция возвращает 1, если завершено реферирование осей и шпинделей, и 0 в противном случае. 

Реализуется пользователем.
% *******end section*****************
%-------------------------------------------------------------------
% *******begin subsection***************
\subsubsection{\DbgSecSt{\StPart}{isHoming}}
\index{Программный интерфейс ПЛК!Реферирование осей!Функция isHoming}
\label{sec:isHoming}

\begin{pHeader}
    Синтаксис:      & \RightHandText{int isHoming();}\\
    Аргумент(ы):    & \RightHandText{Нет} \\    
%    Возвращаемое значение:       & \RightHandText{Целое знаковое число} \\ 
    Файл объявления:             & \RightHandText{include/func/home.h} \\       
\end{pHeader}

Функция возвращает 1, если выполняется реферирование осей и шпинделей, и 0 в противном случае.

Реализуется пользователем.
% *******end section*****************
%-------------------------------------------------------------------
% *******begin subsection***************
\subsubsection{\DbgSecSt{\StPart}{startHoming}}
\index{Программный интерфейс ПЛК!Реферирование осей!Функция startHoming}
\label{sec:startHoming}

\begin{pHeader}
    Синтаксис:      & \RightHandText{void startHoming();}\\
    Аргумент(ы):    & \RightHandText{Нет} \\    
%    Возвращаемое значение:       & \RightHandText{Нет} \\ 
    Файл объявления:             & \RightHandText{include/func/home.h} \\       
\end{pHeader}

Функция инициирует начало реферирование осей и шпинделей.

Реализуется пользователем.
% *******end section*****************
%-------------------------------------------------------------------
% *******begin subsection***************
\subsubsection{\DbgSecSt{\StPart}{isHomingError}}
\index{Программный интерфейс ПЛК!Реферирование осей!Функция isHomingError}
\label{sec:isHomingError}

\begin{pHeader}
    Синтаксис:      & \RightHandText{int isHomingError();}\\
    Аргумент(ы):    & \RightHandText{Нет} \\    
%    Возвращаемое значение:       & \RightHandText{Целое знаковое число} \\ 
    Файл объявления:             & \RightHandText{include/func/home.h} \\       
\end{pHeader}

Функция возвращает 1, если произошла ошибка при реферировании осей или шпинделей, и 0 в противном случае.

Реализуется пользователем.
% *******end section*****************
%-------------------------------------------------------------------
% *******begin subsection***************
\subsubsection{\DbgSecSt{\StPart}{homeCancel}}
\index{Программный интерфейс ПЛК!Реферирование осей!Функция homeCancel}
\label{sec:homeCancel}

\begin{pHeader}
    Синтаксис:      & \RightHandText{void homeCancel();}\\
    Аргумент(ы):    & \RightHandText{Нет} \\    
%    Возвращаемое значение:       & \RightHandText{Нет} \\ 
    Файл объявления:             & \RightHandText{include/func/home.h} \\       
\end{pHeader}

Функция выполняет останов реферирования осей и шпинделей.

Реализуется пользователем.
% *******end section*****************
%--------------------------------------------------------
