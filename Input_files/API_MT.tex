%--------------------------------------------------------
% *******begin section***************
\section{\DbgSecSt{\StPart}{Управление станком}}
%--------------------------------------------------------
\subsection{\DbgSecSt{\StPart}{Типы данных}}

% *******begin subsection***************
\subsubsection{\DbgSecSt{\StPart}{MTState}}
\index{Программный интерфейс ПЛК!Управление станком!Перечисление MTState}
\label{sec:MTState}

\begin{fHeader}
    Тип данных:            & \RightHandText{Перечисление MTState}\\
    Файл объявления:             & \RightHandText{include/cnc/mt.h} \\
\end{fHeader}

Перечисление определяет идентификаторы состояний станка.

\begin{MyTableTwoColAllCntr}{Перечисление MTState}{tbl:MTState}{|m{0.38\linewidth}|m{0.57\linewidth}|}{Идентификатор}{Описание}
\hline mtNotReady &  Станок выключен  \\
\hline mtStartOn &  Начало включения \\
\hline mtDriveOn &  Включение приводов \\
\hline mtWaitDriveOn &  Ожидание включения приводов \\
\hline mtOthersMotorOn & Включение вспомогательных моторов \\
\hline mtReady & Станок включен \\
\hline mtStartOff & Начало выключения \\
\hline mtOthersMotorOff & Выключение вспомогательных моторов \\
\hline mtAxisStop & Останов осей и шпинделя \\
\hline mtAxisWaitStop & Ожидание останова осей и шпинделя \\
\hline mtDriveOff & Выключение приводов \\
\hline mtAbort & Аварийное торможение \\
\hline mtPhaseRef & Фазировка \\
\hline mtWaitPhaseRef  & Ожидание фазировки \\
\hline mtWaitOff & Ожидание выключения питания станка \\
\hline mtWaitAbsPos & Ожидание данных от абсолютного ДОС \\
\end{MyTableTwoColAllCntr}
% *******end subsection***************
%--------------------------------------------------------
% *******begin subsection***************
\subsubsection{\DbgSecSt{\StPart}{MTCNCRequests}}
\index{Программный интерфейс ПЛК!Управление станком!Перечисление MTCNCRequests}
\label{sec:MTCNCRequests}

\begin{fHeader}
    Тип данных:            & \RightHandText{Перечисление MTCNCRequests}\\
    Файл объявления:             & \RightHandText{include/cnc/mt.h} \\
\end{fHeader}

Перечисление определяет идентификаторы команд управления станком. 

Начальный номер блока пользовательских команд  (mtcncCommandStart) равен 1000, конечный (mtcncCommandEnd) ~-- 1999. \killoverfullbefore

Начальный номер блока пользовательских команд движения (mtcncMoveCommandStart) равен 2000, конечный (mtcncMoveCommandEnd) ~-- 2999. \killoverfullbefore

\begin{MyTableTwoColAllCntr}{Перечисление MTCNCRequests}{tbl:MTCNCRequests}{|m{0.38\linewidth}|m{0.57\linewidth}|}{Идентификатор}{Описание}
\hline mtcncNone &  Нет команды  \\
\hline mtcncPowerOn  &  Включение станка \\
\hline mtcncPowerOff  &  Выключение станка \\
\hline mtcncEmergencyStop  &  Аварийный останов  \\
\hline mtcncReset  &  Сброс в начальное состояние \\
\hline mtcncStart  & Запуск операции в текущем режиме \\
\hline mtcncStop &  Останов операции в текущем режиме  \\
\hline mtcncCncOff &  Выключение УЧПУ \\

\hline mtcncActivateManual  & Включение ручного режима  \\
\hline mtcncActivateHandwheel  & Включение режима дискретных перемещений \\
\hline mtcncActivateRef &  Включение режима выезда в нулевую точку \\
\hline mtcncActivateMDI &  Включение режима преднабора \\
\hline mtcncActivateAuto & Включение  автоматического режима \\
\hline mtcncActivateRepos &  Включение режима возврата на контур \\

\hline mtcncToggleStep &  Покадровая отработка УП \\
\hline mtcncToggleRepos &  Возврат на контур \\
\hline mtcncToggleVirtual &  Отработка УП в виртуальном режиме \\
\hline mtcncToggleOptionalSkip & Отработка УП с программным пропуском кадров \\
\hline mtcncToggleOptionalStop & Отработка УП с опциональным остановом \\
\hline mtcncSelectSpeed1 & Выбор первой скорости/дискреты \newline безразмерных/дискретных
перемещений  \\
\hline mtcncSelectSpeed2 & Выбор второй скорости/дискреты \newline безразмерных/дискретных
перемещений \\
\hline mtcncSelectSpeed3 & Выбор третьей скорости/дискреты \newline безразмерных/дискретных перемещений  \\
\hline mtcncSelectSpeed4 & Выбор четвёртой скорости/дискреты \newline безразмерных/дискретных перемещений \\
\hline mtcncSelectRapid & Перемещение на скорости быстрого хода \\
\hline mtcncDryRun & Пробная подача   \\
\hline mtcncReducedRapid &  Уменьшенная подача быстрого хода \\
\hline mtcncMoveLock & Отработка УП с блокировкой движения \\
\hline mtcncAlarmCancel & Сброс ошибок \\

\hline mtcncCommandStart & Начальный номер блока пользовательских команд \\
\hline mtcncCommandEnd &  Конечный номер блока пользовательских команд \\

\hline mtcncMoveCommandStart & Начальный номер блока пользовательских команд движения \\
\hline mtcncMoveCommandEnd & Конечный номер блока пользовательских команд движения  \\

\end{MyTableTwoColAllCntr}
% *******end subsection***************
%--------------------------------------------------------
\begin{comment}
% *******begin subsection***************
\subsubsection{\DbgSecSt{\StPart}{UsedPult}}
\index{Программный интерфейс ПЛК!Управление станком!Перечисление UsedPult}
\label{sec:UsedPult}

\begin{fHeader}
    Тип данных:            & \RightHandText{Перечисление UsedPult}\\
    Файл объявления:             & \RightHandText{include/cnc/mt.h} \\
\end{fHeader}

Перечисление определяет идентификаторы терминальных устройств.

\begin{MyTableThreeColAllCntr}{Перечисление UsedPult}{tbl:UsedPult}{|m{0.33\linewidth}|m{0.22\linewidth}|m{0.45\linewidth}|}{Идентификатор}{Значение}{Описание}
\hline opertor & \centering{0} & Пульт оператора  \\
\hline portable & \centering{1} & Переносной пульт \\
\end{MyTableThreeColAllCntr}
% *******end subsection***************
\end{comment}
%--------------------------------------------------------
% *******begin subsection***************
\subsubsection{\DbgSecSt{\StPart}{MTDesc}}
\index{Программный интерфейс ПЛК!Управление станком!Структура MTDesc}
\label{sec:MTDesc}

\begin{fHeader}
    Тип данных:            & \RightHandText{Структура MTDesc}\\
    Файл объявления:             & \RightHandText{include/cnc/mt.h} \\
\end{fHeader}

Структура определяет данные станка.

\begin{MyTableThreeColAllCntr}{Структура MTDesc}{tbl:MTDesc}{|m{0.33\linewidth}|m{0.222\linewidth}|m{0.45\linewidth}|}{Элемент}{Тип}{Описание}
\hline State & \centering{int} &  Состояние автомата включения/выключения станка  \\
\hline IN & \centering{\hyperlink{IO_union}{MTInputs}} & Входы плат входов\\
\hline OUT & \centering{\hyperlink{IO_union}{MTOutputs}} & Выходы плат реле\\
\hline PultIn & \centering{\hyperlink{IO_union}{PultInputs}} & Входы пульта оператора \\
\hline PultOut & \centering{\hyperlink{IO_union}{PultOutputs}} & Выходы пульта оператора \\
\hline PortablePultIn & \centering{\hyperlink{IO_union}{PortablePultInputs}} &  Входы переносного пульта \\

\hline timerState & \centering{\myreftosec{Timer}} & Таймер состояния \\
\hline timerReset & \centering{\myreftosec{Timer}} & Таймер сброса \\
\hline timerScan & \centering{\myreftosec{Timer}} & Таймер выполнения операции \\

\hline ncNotReadyReq & \centering{Битовое поле:1} & Запрос готовности системы \\
\hline ncFollowUpReq & \centering{Битовое поле:1} & Запрос восстановления после ошибки  \\
\hline ncStopReq & \centering{Битовое поле:1} & Запрос немедленного останова УП или движения  \\
\hline ncStopAtEndReq & \centering{Битовое поле:1} & Запрос останова в конце текущего кадра\\
%\hline usedPult & \centering{int} &   \\
%\hline corrFTemp & \centering{double} &   \\
\end{MyTableThreeColAllCntr}
% *******end subsection***************
%-------------------------------------------------------------------
% *******begin subsection***************
\subsection{\DbgSecSt{\StPart}{Функции}}

% *******begin subsection***************
\subsubsection{\DbgSecSt{\StPart}{systemPlcActive}}
\index{Программный интерфейс ПЛК!Управление станком!Функция systemPlcActive}
\label{sec:systemPlcActive}

\begin{pHeader}
    Синтаксис:      & \RightHandText{int systemPlcActive();}\\
    Аргумент(ы):    & \RightHandText{Нет} \\    
%    Возвращаемое значение:       & \RightHandText{Целое знаковое число} \\ 
    Файл объявления:             & \RightHandText{include/cnc/mt.h} \\       
\end{pHeader}

Функция возвращает 1, если нет ошибок программ ПЛК, и 0 в противном случае.

Реализуется пользователем.

% *******end section*****************
%-------------------------------------------------------------------
% *******begin subsection***************
\subsubsection{\DbgSecSt{\StPart}{hasEmergencyStopRequest}}
\index{Программный интерфейс ПЛК!Управление станком!Функция hasEmergencyStopRequest}
\label{sec:hasEmergencyStopRequest}

\begin{pHeader}
    Синтаксис:      & \RightHandText{int hasEmergencyStopRequest();}\\
    Аргумент(ы):    & \RightHandText{Нет} \\    
%    Возвращаемое значение:       & \RightHandText{Целое знаковое число} \\ 
    Файл объявления:             & \RightHandText{include/cnc/mt.h} \\       
\end{pHeader}

Функция возвращает 1, если есть запрос аварийного останова, и 0 в противном случае.

Реализуется пользователем.
% *******end section*****************
\clearpage
\begin{comment}
%-------------------------------------------------------------------
% *******begin subsection***************
\subsubsection{\DbgSecSt{\StPart}{int hasEmergencyStopMt()}}
\index{Программный интерфейс ПЛК!Управление станком!Функция int hasEmergencyStopMt()}
\label{sec:hasEmergencyStopMt}

\begin{pHeader}
%    Синтаксис:      & \RightHandText{void InitCnc();}\\
    Аргумент(ы):    & \RightHandText{Нет} \\    
%    Возвращаемое значение:       & \RightHandText{Целое знаковое число} \\ 
    Файл объявления:             & \RightHandText{include/cnc/mt.h} \\       
\end{pHeader}


% *******end section*****************
\end{comment}
%-------------------------------------------------------------------
% *******begin subsection***************
\subsubsection{\DbgSecSt{\StPart}{mtControlRequest}}
\index{Программный интерфейс ПЛК!Управление станком!Функция mtControlRequest}
\label{sec:mtControlRequest}

\begin{pHeader}
    Синтаксис:      & \RightHandText{void mtControlRequest();}\\
    Аргумент(ы):    & \RightHandText{Нет} \\    
%    Возвращаемое значение:       & \RightHandText{Нет} \\ 
    Файл объявления:             & \RightHandText{include/cnc/mt.h} \\
\end{pHeader}

Функция добавляет команды в очередь.

Реализуется пользователем.
% *******end section*****************
%-------------------------------------------------------------------
% *******begin subsection***************
\subsubsection{\DbgSecSt{\StPart}{mtUpdateCNCIndication}}
\index{Программный интерфейс ПЛК!Управление станком!Функция mtUpdateCNCIndication}
\label{sec:mtUpdateCNCIndication}

\begin{pHeader}
    Синтаксис:      & \RightHandText{void mtUpdateCNCIndication();}\\
    Аргумент(ы):    & \RightHandText{Нет} \\    
%    Возвращаемое значение:       & \RightHandText{Нет} \\ 
    Файл объявления:             & \RightHandText{include/cnc/mt.h} \\       
\end{pHeader}

Функция обновляет индикацию пульта оператора.

Реализуется пользователем.
% *******end section*****************
%-------------------------------------------------------------------
