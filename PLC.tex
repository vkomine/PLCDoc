% !TeX document-id = {2a4736f1-9131-4047-9e92-6f2668d80547}
%%%%%**********************************Настройки*************************************** %%%%% 
% Inkscape, Ghostscript, GSView должны быть установлены 
% Установить biber.exe, если не установлен или устарел
% Также нужно установить SED и ICONV (из пакета GNUWIN32), брать тут: 
% http://sourceforge.net/projects/gnuwin32/files/sed/
% http://sourceforge.net/projects/gnuwin32/files/libiconv/
% В переменую среды PATH должен быть прописан путь к папке,
% в которой находится inkscape.exe, biber.exe, 
% и sed.exe с iconv.exe (..GnuWin32\bin)   
% Также нужно добавить в PATH ..miktex\bin\  
%_______________________________________________________________________________________________________
% Команды:                       latex.exe -shell-escape -src -synctex=1 -interaction=nonstopmode %.tex
% Построение и просмотр:        DVI->PS->PDF постобработка
% Библиография по умолчанию:    biber
% Makeindex:                    good_index.bat %
% 
%%%%%********************************************************************************** %%%%% 


% Спец-комментарии - для жесткой настройки
% !TeX spellcheck = ru_RU
% !TeX encoding = utf8
% !BIB program = biber
    
                % draft - для коррекции
                % openany     - чтобы новые главы начинались с любой страницы
                % openright - чтобы новые главы начинались только с нечетной страницы (по умолчанию)
%\documentclass[dvips, 11pt,fleqn, draft, openany]{book}    
%\documentclass[dvips, 11pt,fleqn, openany]{book}    

%\documentclass[dvips, 12pt, twocolumn, openany]{book}    
%\documentclass[dvips, 12pt, a4paper, pagesize, BCOR=5mm, openany]{book}    
%\documentclass[dvips, 12pt, draft, openany]{book}    
\documentclass[dvips, 12pt, openany]{book}    
                                                % Тип документа - книга 
                                                % шрифт по умолчанию; выравнивание слева
        
%%%****************************************%%%
%%%*** Настройка шрифтов и кодировок    ***%%%
%%%****************************************%%%
\usepackage{cmap}    % Улучшенный поиск русских слов в полученном pdf-файле
% % % % % % % % % должна быть после documentclass!!!

%\usepackage[auto]{chappg} %НОМЕРГЛАВЫ-Страница 7-11
%%%*****************************************%%%
%%%*** Индексация (предметный указатель) ***%%%
%%%***      обязательно в начале         ***%%%
%%%*****************************************%%%
\makeindex % Tells LaTeX to create the files required for indexing
\usepackage{calc} % For simpler calculation - used for spacing the index letter headings correctly
\usepackage{makeidx} % Required to make an index
%%%*** -------------------------------- ***%%%


%%%*** -------------------------------- ***%%%
%\usepackage{nomencl}		% номенклатура - список терминов и определений
%\makenomenclature    		% Нужно закомментить, если список не нужен
%%%*** -------------------------------- ***%%%

\usepackage{datetime}
\usepackage[T2A]{fontenc}                % Поддержка русских букв
\usepackage[utf8]{inputenc}                % Кодировка utf8
\usepackage[english, russian]{babel}    % Языки: русский, английский

\usepackage[default,scale=0.95]{opensans}		% гладкий, но без засечек
%\usepackage{DejaVuSerif}	% гладкий, с засечками . тоже можно использовать!

%\usepackage{ptserif}		% гладкий, с засечками	!!!	оптимальнее всего

\frenchspacing


%\usepackage{parskip}
%\setlength\parskip{\baselineskip}
%\setlength\parindent{0pt}


%\usepackage{avant} % Use the Avantgarde font for headings
\usepackage{microtype} % Slightly tweak font spacing for aesthetics

\usepackage{setspace} 

\linespread{1.1} %  интервал между строк


%%%*** -------------------------------- ***%%%

%%%***************************%%%
%%%*** Блочные комментарии ***%%%
%%%***************************%%%
\usepackage{comment}
% uncomment to include stuff in standard comment-environment
%\includecomment{comment}
% define a mysection env which content is excluded
%\excludecomment{mysections}
%%%*** ------------------- ***%%%


%%%***************************%%%
%%%***    Библиография     ***%%%
%%%***************************%%%
\usepackage{csquotes}
\usepackage[style=alphabetic,sorting=nyt,sortcites=true,autopunct=true,babel=hyphen,hyperref=true,abbreviate=false,backref=true,backend=biber]{biblatex}

%\usepackage[style=apa, apabackref = true, backend = biber, hyperref = true,    backref=true, maxbibnames = 99, sorting = debug] {Biblatex} 
         
%\usepackage[natbib=true,citestyle=verbose-ibid,isbn=false,maxnames=3,bibstyle=authoryear,useprefix=true,citereset=chapter]{biblatex}

\addbibresource{bibliography.bib} % BibTeX bibliography file
\defbibheading{bibempty}{}

%\usepackage{cite} % Красивые ссылки на литературу

%%%*** ------------------- ***%%%

%%%***************************%%%
%%%***      Графика        ***%%%
%%%***************************%%%
%\usepackage[pdftex]{graphicx}
\usepackage{graphicx} % Required for including pictures
%%%*** ------------------- ***%%%

%%%***************************%%%
%%%*** Оформление документа***%%%
%%%***************************%%%
\usepackage[top=1.7cm,bottom=2.3cm,left=2cm,right=2cm,headsep=10pt,a4paper]{geometry}
                                % Page margins                                  
\usepackage{indentfirst}        % Задает отступ в первых абзацах
\usepackage{multicol}           % возможность представления текста в несколько колонок 


\usepackage{titlesec}           % Allows customization of titles

\usepackage[section,above,below]{placeins}

\usepackage{framed}

\usepackage{lipsum}             % Inserts dummy text
\usepackage{tikz}               % Required for drawing custom shapes

\usepackage{longtable}  % для работы с longtable
\usepackage{multirow}  % 

\usepackage{verbatim}
   
\usepackage{afterpage}

\usepackage{array}

\usepackage{enumitem} % Customize lists
\setlist{nolistsep} % Reduce spacing between bullet points and numbered lists

\usepackage{booktabs} % Required for nicer horizontal rules in tables

\usepackage{eso-pic} % Required for specifying an image background in the title page
%%%*** ------------------- ***%%%
   

%%%***************************%%%
%%%***    PAGE HEADERS     ***%%%
%%%***************************%%%
\usepackage{fancyhdr} % Required for header and footer configuration
%%%*** ------------------- ***%%%

%%%***************************%%%
%%%***    THEOREM STYLES   ***%%%
%%%***************************%%%
\usepackage{amsmath,amsfonts,amssymb,amsthm} % For including math equations, theorems, symbols, etc
%%%*** ------------------- ***%%%


%%%**************************************%%%
%%%***      Цветовые настройки        ***%%%
%%%**************************************%%%
\usepackage{color}
\usepackage{xcolor}         % Required for specifying colors by name

% Общие цвета
\definecolor{ocre}{RGB}{10,175,15} 

% Цвета ссылок
\definecolor{linkcolor}{RGB}{25,125,225}     % цвет ссылок
\definecolor{urlcolor}{HTML}{0000FF}         % цвет гиперссылок
\definecolor{my_color}{RGB}{25,125,225}        % оформления TOC

% Цвета ремарок
\definecolor{remark_color}{HTML}{15AF15}    %цвет ремарок!!!
\definecolor{warning_color}{HTML}{E01515}    %цвет предупреждений!!!
\definecolor{darkelectricblue}{rgb}{0.33, 0.41, 0.47}
\definecolor{darkgray}{rgb}{0.66, 0.66, 0.66}
\definecolor{battleshipgrey}{rgb}{0.52, 0.52, 0.51}
\definecolor{ashgrey}{rgb}{0.7, 0.75, 0.71}
\definecolor{lightgray}{rgb}{0.83, 0.83, 0.83}  
\definecolor{indigo_dye}{rgb}{0.0, 0.25, 0.42}   
\definecolor{jasper}{rgb}{0.84, 0.23, 0.24}
\definecolor{exComm}{rgb}{0.18,0.55,0.34}
\definecolor{title_color}{HTML}{CDCDB4} %A98040
%%%***************************%%%
%%%***    Листинги         ***%%%
%%%***************************%%%
\usepackage{listings}
%%%*** ------------------- ***%%%


%%%***********************************%%%
%%%***    Оформление оглавления    ***%%%
%%%***********************************%%%
\usepackage{etoolbox}

\usepackage{titletoc} % Required for manipulating the table of contents

\usepackage{etoc}

\usepackage[titles]{tocloft}    % необходим для форматирования ОГЛАВЛЕНИЯ и пр.
                                % нужно загружать с этой опцией - иначе ломает  
                                % все настройки по  хеадеру1
%%%*** -------------------------- ***%%%
        
%%%**********************************%%%
%%%*** Настройка формата подписей ***%%%
%%%**********************************%%%
\usepackage{caption}
%%%*** -------------------------- ***%%%

\usepackage{nameref}

\usepackage{refcount}

\usepackage{ragged2e}

\usepackage{setspace}

%	!!!!!!!!!!!!!!!!!!!!!!!!!	%
%\pdfcompresslevel=9 % сжимать PDF


%\usepackage[hyphenbreaks]{breakurl}
%\usepackage[toc,style=treenoname,order=word,subentrycounter]{glossaries}

%\usepackage[refpages]{gloss} 
\usepackage{gloss} 

\usepackage{hhline}

\makegloss                    % Используемые пакеты
\newif\ifShowAllVarI        % Старые переменные (I, Q и пр)

\newif\ifbiblatex            % Библиография

\newif\ifKeyWordUpCase        % Keyword - Upcase/LOcase

\newif\ifTempDescript        % временное комментирование

\newif\ifDebugSectState        % показать состояние готовности пункта


%%%******************************************%%%
%%%***    Настройка условий компиляции    ***%%%
%%%******************************************%%%

% Старые переменные (I, Q и пр)
    %\ShowAllVarItrue          % показать все недоделанное
    \ShowAllVarIfalse        % скрыть все недоделанное

% Библиография
    %\biblatextrue 
    \biblatexfalse

% Keyword - Upcase/LOcase
    %\KeyWordUpCasetrue 
    \KeyWordUpCasefalse

% временное комментирование
    \TempDescripttrue
%    \TempDescriptfalse

% показать состояние готовности пунктов
    \DebugSectStatetrue
%\DebugSectStatefalse
    


%%%*** ---------------------------------- ***%%%

% для борьбы с висячими строками
    \clubpenalty = 10000
    \widowpenalty = 10000


 \setcounter{tocdepth}{3}            % чтобы добавить подпараграфы...
 \setcounter{secnumdepth}{4}
 \setcounter{totalnumber}{10}
 \setcounter{topnumber}{10}
 \setcounter{bottomnumber}{10}
 \renewcommand{\topfraction}{1}
 \renewcommand{\bottomfraction}{1}
 \renewcommand{\textfraction}{0}
 
 
% **********************************************************


\graphicspath{{./}{./Pictures/eps/}{./Pictures/svg/}}


% **********************************************************
%                 ЦВЕТА
% **********************************************************
% Цвета для кода
\definecolor{string}{HTML}{B40000}          % цвет строк в коде
\definecolor{comment}{HTML}{228B22}         % цвет комментариев в коде
\definecolor{keyword}{HTML}{1A00FF}         % цвет ключевых слов в коде

\definecolor{key_word_color}{HTML}{0000FF}

\definecolor{key_word_1_color}{HTML}{3A2FCD}

%\definecolor{key_word_1_color}{HTML}{0000FF}

\definecolor{key_word_2_color}{HTML}{BF0000}

\definecolor{key_word_3_color}{HTML}{000070}

\definecolor{key_word_4_color}{HTML}{700070}

\definecolor{morecomment}{HTML}{8000FF}     % цвет include и других элементов в коде

%\definecolor{сaptiontext}{HTML}{FFFFFF}     % цвет текста заголовка в коде
\definecolor{сaptiontext}{HTML}{000000}     % цвет текста заголовка в коде


%\definecolor{сaptionbk}{HTML}{999999}       % цвет фона заголовка в коде
\definecolor{сaptionbk}{rgb}{.9, .9, .9}       % цвет фона заголовка в коде

\definecolor{bk}{HTML}{F8F8FF}              % цвет фона в коде

%\definecolor{frame}{HTML}{999999}           % цвет рамки в коде

\definecolor{frame}{rgb}{.9, .9, .9}           % цвет рамки в коде
%\definecolor{frame}{RGB}{10,175,15}           % цвет рамки в коде
 

\definecolor{brackets}{HTML}{B40000}        % цвет скобок в коде

\definecolor{digits}{HTML}{FF0000}        % цвет скобок в коде
\definecolor{digits_color}{HTML}{FF0000}        % цвет скобок в коде


%\definecolor{shadecolor}{HTML}{999999}
\definecolor{shadecolor}{rgb}{.9, .9, .9}
%\definecolor{shadecolor}{RGB}{10,175,15}
% **********************************************************



%\begin{comment}
\makeatletter
\setlength{\@fptop}{2pt}
\setlength{\@fpbot}{0pt plus 1fil}
\makeatother
%\end{comment}

% **********************************************************
\begin{comment}
\makeatletter
   \def\relativepath{\import@path}
\makeatother
\end{comment}

\makeatletter
\long\def\@makecaption#1#2{%
  \vskip\abovecaptionskip
  \hbox to\textwidth{\hfill\parbox{0.9\textwidth}{\begin{center}#1 #2\end{center}}\hfill}
  \vskip\belowcaptionskip}
\makeatother

%**************************************************

\IfFileExists{/dev/null}{%
  \newcommand{\Inkscape}{inkscape}%
  }{%
  \newcommand{\Inkscape}{"C:/Program Files (x86)/Inkscape/inkscape.exe"}%
}

\newcommand{\executeiffilenewer}[3]{%
 \ifnum\pdfstrcmp{\pdffilemoddate{#1}}%
 {\pdffilemoddate{#2}}>0%
 {\immediate\write18{#3}}\fi%
}

\newcommand{\DrawPictEpsFromSvg}[4][0.9\textwidth]{%
    \executeiffilenewer{#2.svg}{#2.eps}%
    {inkscape -z --file=#2.svg --export-eps=#2.eps --export-text-to-path}%
    \begin{figure}[htb]%
    \noindent\centering\includegraphics[keepaspectratio=true, width=#1]{#2.eps}%
    \caption{#3}%
    \label{fig:#4}%
    \end{figure}%
    \FloatBarrier%
    \afterpage{\FloatBarrier}%
}

\newcommand{\DrawOnlyEpsFromSvg}[2][0.9\textwidth]{%
    \executeiffilenewer{#2.svg}{#2.eps}%
    {inkscape -z --file=#2.svg --export-eps=#2.eps --export-text-to-path}%
    \begin{figure}[htb]%
    \noindent\centering\includegraphics[width=#1]{#2.eps}%
    \end{figure}%
    \FloatBarrier%
    \afterpage{\FloatBarrier}%
}

\newcommand{\IncludeEpsFromSvg}[2][\textwidth]{%
    \executeiffilenewer{#2.svg}{#2.eps}%
    {inkscape -z --file=#2.svg --export-eps=#2.eps --export-text-to-path}%
    \includegraphics[width=#1]{#2.eps}%
}
    
\begin{comment}
\newcommand{\includesvg}[1]{%
    \executeiffilenewer{#1.svg}{#1.eps}%
    {inkscape -z --file=#1.svg --export-eps=#1.eps --export-text-to-path}%
    \includegraphics{#1.eps}
}
\end{comment}




\newenvironment{MyItemize}[1]%
{%
    %\begin{center}%
        \begin{longtable}[h]{#1}
}%
{%
        \end{longtable}    \addtocounter{table}{-1}%
    %\end{center}%
%    \FloatBarrier%
%    \afterpage{\FloatBarrier}%
}    


\newcommand*{\TblLeftHeaderCenter}[1]%
{%
\multicolumn{1}{c|}{\textbf{#1}}%
}%


\newcommand{\TblLeftHeaderCenterNew}[1]%
{%
\multicolumn{1}{c|}{\textbf{#1}}%
}%

\newcommand{\TBHdr}[1]%
{%
\textbf{#1}%
}%

\newcommand{\TBHdrCntr}[1]%
{%
\centering\textbf{#1}%
}%

\newcommand{\TBRightHdrCntr}[1]%
{%
\multicolumn{1}{c|}{\textbf{#1}}%
}%

\newenvironment{MyTableTwoCol}[5]%
{%
    \begin{center}%
        \begin{longtable}[h]{#3}%
            \caption{#1}\label{#2}\\
            \hline                  &                \\
                    \TBHdrCntr{#4}      & \TBHdr{#5}   \\
                                     &               \\
            \hline
            \endfirsthead
            \caption*{Продолжение таблицы \ref{#2}.}\\
            \hline                     &                \\
                    \TBHdrCntr{#4}      & \TBHdr{#5}       \\
                                     &                \\
            \hline
            \endhead
            \hline 
            \endfoot
            \hline
            \endlastfoot
}%
{%
        \end{longtable}%
    \end{center}%
    \FloatBarrier%
    \afterpage{\FloatBarrier}%
}    

\newenvironment{MyTableTwoColCntr}[5]%
{%
    \begin{center}%
        \begin{longtable}[h]{#3}%
            \caption{#1}\label{#2}\\
            \hline                  &                \\
                    \TBHdrCntr{#4}      & \TBRightHdrCntr{#5}   \\
                                     &               \\
            \hline
            \endfirsthead
            \caption*{Продолжение таблицы \ref{#2}.}\\
            \hline                     &                \\
                    \TBHdrCntr{#4}      & \TBHdr{#5}       \\
                                     &                \\
            \hline
            \endhead
            \hline 
            \endfoot
            \hline
            \endlastfoot
}%
{%
        \end{longtable}%
    \end{center}%
    \FloatBarrier%
    \afterpage{\FloatBarrier}%
} 

\newenvironment{MyTableTwoColAllCntr}[5]%
{%
    \begin{center}%
        \begin{longtable}[h]{#3}%
            \caption{#1}\label{#2}\\
            \hline                  &                \\
                    \TBHdrCntr{#4}      & \TBRightHdrCntr{#5}   \\
                                     &               \\
            \hline
            \endfirsthead
            \caption*{Продолжение таблицы \ref{#2}.}\\
            \hline                     &                \\
                    \TBHdrCntr{#4}      & \TBRightHdrCntr{#5}       \\
                                     &                \\
            \hline
            \endhead
            \hline 
            \endfoot
            \hline
            \endlastfoot
}%
{%
        \end{longtable}%
    \end{center}%
    \FloatBarrier%
    \afterpage{\FloatBarrier}%
}    



%\centering\bfseries
\newenvironment{MyTableThreeCol}[6]%
{%
    \begin{center}%
        \begin{longtable}[h]{#3}%
            \caption{#1}\label{#2}\\
            \hline                  &                & \\
                    \TBHdrCntr{#4}      &  \TBHdrCntr{#5}     & \TBHdr{#6}     \\
                                     &                & \\
            \hline
            \endfirsthead
            \caption*{Продолжение таблицы \ref{#2}.}\\
            \hline                     &                & \\
                    \TBHdrCntr{#4}      & \TBHdrCntr{#5}     & \TBHdr{#6}     \\
                   &                & \\
            \hline
            \endhead
            \hline 
            \endfoot
            \hline
            \endlastfoot
}%
{%
        \end{longtable}%
    \end{center}%
    \FloatBarrier%
    \afterpage{\FloatBarrier}%
}    

\newenvironment{MyTableThreeColCntr}[6]%
{%
    \begin{center}%
        \begin{longtable}{#3}%
            \caption{#1}\label{#2}\\
            \hline                  &                & \\
                    \TBHdrCntr{#4}      &  \TBHdrCntr{#5}     & \TBRightHdrCntr{#6}     \\
                                     &                & \\
            \hline
            \endfirsthead
            \caption*{Продолжение таблицы \ref{#2}.}\\
            \hline                     &                & \\
                    \TBHdrCntr{#4}      & \TBHdr{#5}     & \TBHdr{#6}     \\
                   &                & \\
            \hline
            \endhead
            \hline 
            \endfoot
            \hline
            \endlastfoot
}%
{%
        \end{longtable}%
    \end{center}%
    \FloatBarrier%
    \afterpage{\FloatBarrier}%
} 

%\centering\bfseries
\newenvironment{MyTableThreeColAllCntr}[6]%
{%
    \begin{center}%
        \begin{longtable}{#3}%
            \caption{#1}\label{#2}\\
            \hline                  &                & \\
                    \TBHdrCntr{#4}      &  \TBHdrCntr{#5}     & \TBRightHdrCntr{#6}     \\
                                     &                & \\
            \hline
            \endfirsthead
            \caption*{Продолжение таблицы \ref{#2}.}\\
            \hline                     &                & \\
                    \TBHdrCntr{#4}      & \TBHdrCntr{#5}     & \TBRightHdrCntr{#6}     \\
                   &                & \\
            \hline
            \endhead
            \hline 
            \endfoot
            \hline
            \endlastfoot
}%
{%
        \end{longtable}%
    \end{center}%
    \FloatBarrier%
    \afterpage{\FloatBarrier}%
}   

\newenvironment{MyTableFourCol}[7]%
{%
    \begin{center}%
        \begin{longtable}{#3}
            \caption{#1}\label{#2}\\
            \hline                  &                &              & \\
                    \TBHdrCntr{#4}      & \TBHdrCntr{#5}     & \TBHdrCntr{#6} & \TBHdr{#7}     \\
                                    &                &              & \\
            \hline
            \endfirsthead
            \caption*{Продолжение таблицы \ref{#2}.}\\
            \hline                  &                &              & \\
                    \TBHdrCntr{#4}      & \TBHdrCntr{#5}     & \TBHdrCntr{#6} & \TBHdr{#7}     \\
                                    &                &              & \\
            \hline
            \endhead
            \hline 
            \endfoot
            \hline
            \endlastfoot
}%
{%
        \end{longtable}%
    \end{center}%
    \FloatBarrier%
    \afterpage{\FloatBarrier}%
}


\newenvironment{MyTableFourColAllCntr}[7]%
{%
    \begin{center}%
        \begin{longtable}{#3}
            \caption{#1}\label{#2}\\
            \hline                  &                &              & \\
                    \TBHdrCntr{#4}      & \TBHdrCntr{#5}     & \TBHdrCntr{#6} & \TBRightHdrCntr{#7}     \\
                                    &                &              & \\
            \hline
            \endfirsthead
            \caption*{Продолжение таблицы \ref{#2}.}\\
            \hline                  &                &              & \\
                    \TBHdrCntr{#4}      & \TBHdrCntr{#5}     & \TBHdrCntr{#6} & \TBRightHdrCntr{#7}     \\
                                    &                &              & \\
            \hline
            \endhead
            \hline 
            \endfoot
            \hline
            \endlastfoot
}%
{%
        \end{longtable}%
    \end{center}%
    \FloatBarrier%
    \afterpage{\FloatBarrier}%
}

\newenvironment{MyTableFiveCol}[8]%
{%
    \begin{center}%
        \begin{longtable}{#3}
            \caption{#1}\label{#2}\\
            \hline                  &                &              &             &\\
                    \TBHdrCntr{#4}      & \TBHdrCntr{#5}     & \TBHdrCntr{#6} & \TBHdrCntr{#7} & \TBHdr{#8}     \\
                                    &                &              &             &\\
            \hline
            \endfirsthead
            \caption*{Продолжение таблицы \ref{#2}.}\\
            \hline  &                &              &             &\\
                    \TBHdrCntr{#4}      & \TBHdrCntr{#5}     & \TBHdrCntr{#6} & \TBHdrCntr{#7} & \TBHdr{#8}     \\
                    &                &              &             &\\
            \hline
            \endhead
            \hline 
            \endfoot
            \hline
            \endlastfoot
}%
{%
        \end{longtable}%
    \end{center}%
    \FloatBarrier%
    \afterpage{\FloatBarrier}%
}    

\newenvironment{MyTableFiveColAllCntr}[8]%
{%
    \begin{center}%
        \begin{longtable}{#3}
            \caption{#1}\label{#2}\\
            \hline                  &                &              &             &\\
                    \TBHdrCntr{#4}      & \TBHdrCntr{#5}     & \TBHdrCntr{#6} & \TBHdrCntr{#7} & \TBRightHdrCntr{#8}     \\
                                    &                &              &             &\\
            \hline
            \endfirsthead
            \caption*{Продолжение таблицы \ref{#2}.}\\
            \hline  &                &              &             &\\
                    \TBHdrCntr{#4}      & \TBHdrCntr{#5}     & \TBHdrCntr{#6} & \TBHdrCntr{#7} & \TBRightHdrCntr{#8}     \\
                    &                &              &             &\\
            \hline
            \endhead
            \hline 
            \endfoot
            \hline
            \endlastfoot
}%
{%
        \end{longtable}%
    \end{center}%
    \FloatBarrier%
    \afterpage{\FloatBarrier}%
}    

\newenvironment{MyTableSixCol}[9]%
{%
    \begin{center}%
        \begin{longtable}{#3}
            \caption{#1}\label{#2}\\
            \hline  &                &              &             &                &\\
                    \TBHdrCntr{#4}      & \TBHdrCntr{#5}     & \TBHdrCntr{#6} & \TBHdrCntr{#7} & \TBHdrCntr{#8} & \TBRightHdrCntr{#9}     \\
                    &                &              &             &                &\\
            \hline
            \endfirsthead
            \caption*{Продолжение таблицы \ref{#2}.}\\
            \hline  &                &              &             &                &\\
                    \TBHdrCntr{#4}      & \TBHdrCntr{#5}     & \TBHdrCntr{#6} & \TBHdrCntr{#7} & \TBHdrCntr{#8} & \TBRightHdrCntr{#9}    \\
                    &                &              &             &                &\\
            \hline
            \endhead
            \hline 
            \endfoot
            \hline
            \endlastfoot
}%
{%
        \end{longtable}%
    \end{center}%
    \FloatBarrier%
    \afterpage{\FloatBarrier}%
}    

\newenvironment{MyTableSixColAllCntr}[9]%
{%
    \begin{center}%
        \begin{longtable}{#3}
            \caption{#1}\label{#2}\\
            \hline  &                &              &             &                &\\
                    \TBHdrCntr{#4}      & \TBHdrCntr{#5}     & \TBHdrCntr{#6} & \TBHdrCntr{#7} & \TBHdrCntr{#8} & \TBHdr{#9}     \\
                    &                &              &             &                &\\
            \hline
            \endfirsthead
            \caption*{Продолжение таблицы \ref{#2}.}\\
            \hline  &                &              &             &                &\\
                    \TBHdrCntr{#4}      & \TBHdrCntr{#5}     & \TBHdrCntr{#6} & \TBHdrCntr{#7} & \TBHdrCntr{#8} & \TBHdr{#9}    \\
                    &                &              &             &                &\\
            \hline
            \endhead
            \hline 
            \endfoot
            \hline
            \endlastfoot
}%
{%
        \end{longtable}%
    \end{center}%
    \FloatBarrier%
    \afterpage{\FloatBarrier}%
}    






\begin{comment}

\newenvironment{MyTableSevenCol}[10]%
{%
    \begin{center}%
        \begin{longtable}{#3}
            \caption{#1}\label{#2}\\
            \hline  &        &        &         &        &        &\\
                    \textbf{#4}      & \textbf{#5}     & \textbf{#6} & \textbf{#7} & \textbf{#8} & \textbf{#9} & \textbf{#10}    \\
                    &        &        &        &         &        &\\
            \hline
            \endfirsthead
            \caption*{Продолжение таблицы \ref{#2}.}\\
            \hline  &                &              &     &        &                &\\
                    \textbf{#4}      & \textbf{#5}     & \textbf{#6} & \textbf{#7} & \textbf{#8} & \textbf{#9}    & \textbf{#10} \\
                    &                &              &     &        &                &\\
            \hline
            \endhead
            \hline 
            \endfoot
            \hline
            \endlastfoot
}%
{%
        \end{longtable}%
    \end{center}%
    \FloatBarrier%
    \afterpage{\FloatBarrier}%
}    

\end{comment}

%----------------------------------------------------------------------------------------
%    MINI TABLE OF CONTENTS IN CHAPTER HEADS
%----------------------------------------------------------------------------------------

% Section text styling
\titlecontents{lsection}[0em] % Indendating
%{\footnotesize\sffamily} % Font settings
{\footnotesize} % Font settings
{}
{}
{}

% Subsection text styling
\titlecontents{lsubsection}[.5em] % Indentation
%{\normalfont\footnotesize\sffamily} % Font settings
{\normalfont\footnotesize} % Font settings
{}
{}
{}
 
%----------------------------------------------------------------------------------------
%    PAGE HEADERS
%----------------------------------------------------------------------------------------
%\newdateformat{specialdate}{\THEYEAR-\twodigit{\THEMONTH}-\twodigit{\THEDAY}}

\newdateformat{monthyeardate}{\monthname[\THEMONTH], \THEYEAR}

% Для печати на одной странице
\pagestyle{fancy}
\fancyhead{}
\fancyfoot{}
\renewcommand{\chaptermark}[1]{\markboth{\normalsize\bfseries #1}{}} % Chapter text font settings
\renewcommand{\sectionmark}[1]{\markright{\normalsize\thesection\hspace{5pt}#1}{}} % Section text font settings
\fancyhf{} \fancyfoot[CE,CO]{\normalsize\thepage} % Font setting for the page number in the header
\rhead{\rightmark}
%\fancyhead[RE,RO]{\rightmark} % Print the nearest section name on the left side of odd pages
%\fancyhead[LE,LO]{\leftmark} % Print the current chapter name on the right side of even pages
\renewcommand{\headrulewidth}{0.4pt} % Width of the rule under the header
\addtolength{\headheight}{12pt} % Increase the spacing around the header slightly
\setlength{\headsep}{13pt} %
\renewcommand{\footrulewidth}{0.0pt} % Removes the rule in the footer

\fancypagestyle{plain}{ 
    \fancyhf{}
    \fancyfoot[C]{\thepage}}

\begin{comment}
\pagestyle{fancy}
\renewcommand{\chaptermark}[1]{\markboth{\normalsize\bfseries #1}{}} % Chapter text font settings
\renewcommand{\sectionmark}[1]{\markright{\normalsize\thesection\hspace{5pt}#1}{}} % Section text font settings
\fancyhf{} \fancyhead[LE,RO]{\normalsize\thepage} % Font setting for the page number in the header
\fancyhead[LO]{\rightmark} % Print the nearest section name on the left side of odd pages
\fancyhead[RE]{\leftmark} % Print the current chapter name on the right side of even pages
\renewcommand{\headrulewidth}{0.5pt} % Width of the rule under the header
\addtolength{\headheight}{12pt} % Increase the spacing around the header slightly
\setlength{\headsep}{13pt} %
\renewcommand{\footrulewidth}{0pt} % Removes the rule in the footer
\fancypagestyle{plain}{\fancyhead{}\renewcommand{\headrulewidth}{0pt}} % Style for when a plain pagestyle is specified
\fancyfoot[LE]{\copyright INELSY 2017}
\fancyfoot[RO]{Ревизия 1.0 от \today}
\fancyfoot[C]{}
 \end{comment}
% Removes the header from odd empty pages at the end of chapters
\makeatletter
\renewcommand{\cleardoublepage}{
\afterpage{\clearpage}\ifodd\c@page\else
\hbox{}
\vspace*{\fill}
\thispagestyle{empty}
\newpage
\fi}



%----------------------------------------------------------------------------------------
%    THEOREM STYLES
%----------------------------------------------------------------------------------------

\newcommand{\intoo}[2]{\mathopen{]}#1\,;#2\mathclose{[}}
\newcommand{\ud}{\mathop{\mathrm{{}d}}\mathopen{}}
\newcommand{\intff}[2]{\mathopen{[}#1\,;#2\mathclose{]}}
\newtheorem{notation}{Notation}[chapter]

%%%%%%%%%%%%%%%%%%%%%%%%%%%%%%%%%%%%%%%%%%%%%%%%%%%%%%%%%%%%%%%%%%%%%%%%%%%
%%%%%%%%%%%%%%%%%%%% dedicated to boxed/framed environements %%%%%%%%%%%%%%
%%%%%%%%%%%%%%%%%%%%%%%%%%%%%%%%%%%%%%%%%%%%%%%%%%%%%%%%%%%%%%%%%%%%%%%%%%%
\newtheoremstyle{ocrenumbox}% % Theorem style name
{0pt}% Space above
{0pt}% Space below
{\normalfont}% % Body font
{}% Indent amount
{\small\bf\sffamily\color{ocre}}% % Theorem head font
{\;}% Punctuation after theorem head
{0.25em}% Space after theorem head
{\small\sffamily\color{ocre}\thmname{#1}\nobreakspace\thmnumber{\@ifnotempty{#1}{}\@upn{#2}}% Theorem text (e.g. Theorem 2.1)
%\thmnote{\nobreakspace\the\thm@notefont\sffamily\bfseries\color{black}---\nobreakspace#3.}} % Optional theorem note
\thmnote{\nobreakspace\the\thm@notefont\bfseries\color{black}---\nobreakspace#3.}} % Optional theorem note
\renewcommand{\qedsymbol}{$\blacksquare$}% Optional qed square

\newtheoremstyle{blacknumex}% Theorem style name
{5pt}% Space above
{5pt}% Space below
{\normalfont}% Body font
{} % Indent amount
%{\small\bf\sffamily}% Theorem head font
{\small\bf}% Theorem head font
{\;}% Punctuation after theorem head
{0.25em}% Space after theorem head
%{\small\sffamily{\tiny\ensuremath{\blacksquare}}\nobreakspace\thmname{#1}\nobreakspace\thmnumber{\@ifnotempty{#1}{}\@upn{#2}}% Theorem text (e.g. Theorem 2.1)
{\small{\tiny\ensuremath{\blacksquare}}\nobreakspace\thmname{#1}\nobreakspace\thmnumber{\@ifnotempty{#1}{}\@upn{#2}}% Theorem text (e.g. Theorem 2.1)
%\thmnote{\nobreakspace\the\thm@notefont\sffamily\bfseries---\nobreakspace#3.}}% Optional theorem note
\thmnote{\nobreakspace\the\thm@notefont\bfseries---\nobreakspace#3.}}% Optional theorem note

\newtheoremstyle{blacknumbox} % Theorem style name
{0pt}% Space above
{0pt}% Space below
{\normalfont}% Body font
{}% Indent amount
{\small\bf}% Theorem head font
%{\small\bf\sffamily}% Theorem head font
{\;}% Punctuation after theorem head
{0.25em}% Space after theorem head
%{\small\sffamily\thmname{#1}\nobreakspace\thmnumber{\@ifnotempty{#1}{}\@upn{#2}}% Theorem text (e.g. Theorem 2.1)
{\small\thmname{#1}\nobreakspace\thmnumber{\@ifnotempty{#1}{}\@upn{#2}}% Theorem text (e.g. Theorem 2.1)
%\thmnote{\nobreakspace\the\thm@notefont\sffamily\bfseries---\nobreakspace#3.}}% Optional theorem note
\thmnote{\nobreakspace\the\thm@notefont\bfseries---\nobreakspace#3.}}% Optional theorem note

%%%%%%%%%%%%%%%%%%%%%%%%%%%%%%%%%%%%%%%%%%%%%%%%%%%%%%%%%%%%%%%%%%%%%%%%%%%
%%%%%%%%%%%%% dedicated to non-boxed/non-framed environements %%%%%%%%%%%%%
%%%%%%%%%%%%%%%%%%%%%%%%%%%%%%%%%%%%%%%%%%%%%%%%%%%%%%%%%%%%%%%%%%%%%%%%%%%
\newtheoremstyle{ocrenum}% % Theorem style name
{5pt}% Space above
{5pt}% Space below
{\normalfont}% % Body font
{}% Indent amount
%{\small\bf\sffamily\color{ocre}}% % Theorem head font
{\small\bf\color{ocre}}% % Theorem head font
{\;}% Punctuation after theorem head
{0.25em}% Space after theorem head
{\small\sffamily\color{ocre}\thmname{#1}\nobreakspace\thmnumber{\@ifnotempty{#1}{}\@upn{#2}}% Theorem text (e.g. Theorem 2.1)
%\thmnote{\nobreakspace\the\thm@notefont\sffamily\bfseries\color{black}---\nobreakspace#3.}} % Optional theorem note
\thmnote{\nobreakspace\the\thm@notefont\bfseries\color{black}---\nobreakspace#3.}} % Optional theorem note
\renewcommand{\qedsymbol}{$\blacksquare$}% Optional qed square
\makeatother

% Defines the theorem text style for each type of theorem to one of the three styles above
\newcounter{dummy} 
\numberwithin{dummy}{section}
\theoremstyle{ocrenumbox}
\newtheorem{theoremeT}[dummy]{Theorem}
\newtheorem{problem}{Problem}[chapter]
\newtheorem{exerciseT}{Exercise}[chapter]
\theoremstyle{blacknumex}
\newtheorem{exampleT}{Example}[chapter]
\theoremstyle{blacknumbox}
\newtheorem{vocabulary}{Vocabulary}[chapter]
\newtheorem{definitionT}{Definition}[section]
\newtheorem{corollaryT}[dummy]{Corollary}
\theoremstyle{ocrenum}
\newtheorem{proposition}[dummy]{Proposition}


%******************************************************
%******************************************************

%----------------------------------------------------------------------------------------
%    DEFINITION OF COLORED BOXES
%----------------------------------------------------------------------------------------

\RequirePackage[framemethod=default]{mdframed} % Required for creating the theorem, definition, exercise and corollary boxes

% Theorem box
\newmdenv[skipabove=7pt,
skipbelow=7pt,
backgroundcolor=black!5,
linecolor=ocre,
innerleftmargin=5pt,
innerrightmargin=5pt,
innertopmargin=5pt,
leftmargin=0cm,
rightmargin=0cm,
innerbottommargin=5pt]{tBox}

% Exercise box      
\newmdenv[skipabove=7pt,
skipbelow=7pt,
rightline=false,
leftline=true,
topline=false,
bottomline=false,
backgroundcolor=ocre!10,
linecolor=ocre,
innerleftmargin=5pt,
innerrightmargin=5pt,
innertopmargin=5pt,
innerbottommargin=5pt,
leftmargin=0cm,
rightmargin=0cm,
linewidth=4pt]{eBox}    

% Definition box
\newmdenv[skipabove=7pt,
skipbelow=7pt,
rightline=false,
leftline=true,
topline=false,
bottomline=false,
linecolor=ocre,
backgroundcolor=bk,%
innerleftmargin=5pt,
innerrightmargin=5pt,
innertopmargin=0pt,
leftmargin=0cm,
rightmargin=0cm,
linewidth=4pt,
innerbottommargin=0pt]{dBox}    

% Corollary box
\newmdenv[skipabove=7pt,
skipbelow=7pt,
rightline=false,
leftline=true,
topline=false,
bottomline=false,
linecolor=gray,
backgroundcolor=black!5,
innerleftmargin=5pt,
innerrightmargin=5pt,
innertopmargin=5pt,
leftmargin=0cm,
rightmargin=0cm,
linewidth=4pt,
innerbottommargin=5pt]{cBox}                  
          

% Corollary box
\newmdenv[skipabove=7pt,%
skipbelow=7pt,%
rightline=false,%
leftline=true,%
topline=false,%
bottomline=false,%
linecolor= orange,% %warning_color,%
backgroundcolor=bk,%
innerleftmargin=5pt,%
innerrightmargin=5pt,%
innertopmargin=0pt,%
leftmargin=0cm,%
rightmargin=0cm,%
linewidth=4pt,%
innerbottommargin=0pt]{zBox}                  
          
% G box
\newmdenv[skipabove=7pt,%
skipbelow=7pt,%
rightline=false,%
leftline=true,%
topline=false,%
bottomline=false,%
linecolor=ashgrey,% %warning_color,%
backgroundcolor=lightgray,%
innerleftmargin=5pt,%
innerrightmargin=5pt,%
innertopmargin=0pt,%
leftmargin=0cm,%
rightmargin=0cm,%
linewidth=8pt,%
innerbottommargin=0pt]{gBox} 

\newmdenv[skipabove=7pt,%
skipbelow=7pt,%
rightline=false,%
leftline=true,%
topline=false,%
bottomline=false,%
linecolor=title_color,
backgroundcolor=bk,%
innerleftmargin=5pt,%
innerrightmargin=4pt,%
innertopmargin=0pt,%
leftmargin=0cm,%
rightmargin=0cm,%
linewidth=12pt,%
innerbottommargin=0pt]{listingBox}
          
% Definition box
\newmdenv[skipabove=7pt,
skipbelow=7pt,
rightline=false,
leftline=true,
topline=false,
bottomline=false,
linecolor=linkcolor,
innerleftmargin=5pt,
innerrightmargin=5pt,
innertopmargin=0pt,
leftmargin=0cm,
rightmargin=0cm,
linewidth=4pt,
innerbottommargin=0pt]{SeeAlsoBox}    

% Creates an environment for each type of theorem and assigns it a theorem text style from the "Theorem Styles" section above and a colored box from above
\newenvironment{theorem}{\begin{tBox}\begin{theoremeT}}{\end{theoremeT}\end{tBox}}
\newenvironment{exercise}{\begin{eBox}\begin{exerciseT}}{\hfill{\color{ocre}\tiny\ensuremath{\blacksquare}}\end{exerciseT}\end{eBox}}        
        
\newenvironment{definition}{\begin{dBox}\begin{definitionT}}{\end{definitionT}\end{dBox}}    
\newenvironment{pHeader}{\begin{dBox}\begin{tabular}[h]{lp{34em}}}{\end{tabular}\end{dBox}}    

\newenvironment{fHeader}{\begin{zBox}\begin{tabular}[h]{lp{28em}}}{\end{tabular}\end{zBox}}  

\newenvironment{cHeader}{\begin{gBox}\begin{tabular}[h]{lp{48em}}}{\end{tabular}\end{gBox}} 

\newenvironment{smallTblBits}{\begin{tabular}[h]{llp{31em}}}{\end{tabular}}  
%\newenvironment{pHeader}{\begin{dBox}\begin{longtable}[h]{lp{26em}}}{\addtocounter{table}{-1}\end{longtable}\end{dBox}}    

%\newenvironment{pHeader}{\begin{dBox}}{\end{dBox}}    
\newenvironment{pExample}{\begin{zBox}}{\end{zBox}}
\newenvironment{listingExample}{\begin{listingBox}}{\end{listingBox}}
\newenvironment{gExample}{\begin{cBox}}{\end{cBox}}
\newenvironment{example}{\begin{exampleT}}{\hfill{\tiny\ensuremath{\blacksquare}}\end{exampleT}}        
\newenvironment{corollary}{\begin{cBox}\begin{corollaryT}}{\end{corollaryT}\end{cBox}}    

%\newenvironment{MyDescription}{\begin{tabular}[h]{p{18em}p{14em}}}{\end{tabular}}    

%\newenvironment{SeeAlsoList}{\begin{SeeAlsoBox}\begin{tabular}[t]{lp{23em}}}{\end{tabular}\end{SeeAlsoBox}}    
\newenvironment{SeeAlsoList}{\begin{SeeAlsoBox}\begin{longtable}[h]{lp{23em}}}{\addtocounter{table}{-1}\end{longtable}\end{SeeAlsoBox}}    

\newenvironment{SeeAlsoLstAcc24}{\begin{SeeAlsoBox}\begin{longtable}[h]{m{7em}m{28em}}}{ \addtocounter{table}{-1}\end{longtable}\end{SeeAlsoBox}
%\newenvironment{SeeAlsoLstAcc24}{\begin{SeeAlsoBox}\begin{tabular}[h]{p{10em}p{35em}}}{\end{tabular}\end{SeeAlsoBox}
}    


% форматирование нумерованных списков
\renewcommand{\labelenumii}{\arabic{enumi}.\arabic{enumii}.}


%----------------------------------------------------------------------------------------
%    REMARK ENVIRONMENT
%----------------------------------------------------------------------------------------

\newenvironment{remark}{%
    \par\vskip10pt\small % Vertical white space above the remark and smaller font size
    \begin{list}{}{
        \leftmargin=35pt % Indentation on the left
        \rightmargin=25pt}\item\ignorespaces % Indentation on the right
        \makebox[-2.5pt]{%
            \begin{tikzpicture}[overlay]%
                \node[draw=remark_color,line width=1pt,circle,%
                fill=remark_color,font=\bfseries,inner sep=2pt,%
%                fill=remark_color,font=\sffamily\bfseries,inner sep=2pt,%
                outer sep=0pt] at (-15pt,0pt){\normalfont\bfseries\large\textcolor{white}{i}};%
            \end{tikzpicture}} % Orange R in a circle
        \advance\baselineskip -1pt}
    {\end{list}\vskip5pt%
} % Tighter line spacing and white space after remark

%----------------------------------------------------------------------------------------
%    WARNING ENVIRONMENT
%----------------------------------------------------------------------------------------

\newenvironment{warning}{%
    \par\vskip10pt\small % Vertical white space above the warning and smaller font size
    \begin{list}{}{%
        \leftmargin=35pt % Indentation on the left
        \rightmargin=25pt}\item\ignorespaces % Indentation on the right
        \makebox[-2.5pt]{%
            \begin{tikzpicture}[overlay]%
                \node[draw=warning_color,line width=1pt,circle,%
%                fill=white,font=\sffamily\bfseries,inner sep=2pt,%
                fill=white,font=\bfseries,inner sep=2pt,%
                outer sep=0pt] at (-15pt,0pt){\normalfont\bfseries\large\textcolor{warning_color}{!}};%
            \end{tikzpicture}} % Orange R in a circle
        \advance\baselineskip -1pt}% 
    {\end{list}\vskip5pt%
} % Tighter line spacing and white space after warning

%----------------------------------------------------------------------------------------
%    SECTION NUMBERING IN THE MARGIN
%----------------------------------------------------------------------------------------

\makeatletter

\begin{comment}

    \renewcommand{\@seccntformat}[1]{%
        \llap{%
            \textcolor{ocre}%
            {\csname the#1\endcsname}%
            \hspace{1em}%
        }%
    }%                    
\end{comment}

\begin{comment}

    \renewcommand{\@seccntformat}[1]{%
     \csname the#1\endcsname.\quad
    }


% настройка нумерации

    \renewcommand{\@seccntformat}[1]{%
        \llap{%
            %\textcolor{ocre}%
            {\csname the#1\endcsname}%            %.\quad%
            \hspace{5mm}%
        }%    
    }%
\end{comment}

% 
\begin{comment}

%    Стиль оформления section
    \renewcommand{\section}{\@startsection
        {section}%                                    name
        {1}%                                        level    
        {\z@}%                                        indent
        {-4ex \@plus -1ex \@minus -.4ex}%            space above header
        {1ex \@plus.2ex }%                            space under header
        {\normalfont\LARGE\bfseries}%                style
    }%
\end{comment}

%    Стиль оформления section
    \renewcommand{\section}{\@startsection%
        {section}%                                     name
           {1}%                                           level
%        {\z@}%                                        indent
           {0mm}%                                         indent
%        {-4ex \@plus -1ex \@minus -.4ex}%            space above header
        {-5.5ex \@plus -1.5ex \@minus -.3ex}%                space above header
%           {-3\baselineskip}%                           space above header
%        {1ex \@plus.2ex }%                            space under header
        {3ex \@plus1.5ex }%                            space under header
%          {1.5\baselineskip}%                            space under header
           {\normalfont\Large\bfseries}%                 style
    }%

%\begin{comment}
    \renewcommand{\subsection}{\@startsection%
        {subsection}%                                name
        {2}%                                        level    
%        {\z@}%                                        indent
           {0mm}%                                         indent
%        {-3ex \@plus -0.1ex \@minus -.4ex}%            space above header
        {-4ex \@plus -1.5ex \@minus -0.2ex}%            space above header
%       {-4ex \@plus -1.5ex \@minus -0.2ex}%       ace above header  last settings      
%        {-3.5ex \@plus -1ex \@minus -.2ex}%            space above header
%        {-5ex \@plus -0.1ex \@minus -.8ex}%            space above header
%        {-1.5\baselineskip}%                           space above header
%        {1.5ex}%                        space under header
         {2ex \@plus1.5ex }%                        space under header
%        {1ex \@plus 0.5ex \@minus -1ex}%                        space under header
%        {1ex \@plus.2ex }%                            space under header
%        {1ex \@plus -0.8ex \@minus-.8ex }%            space under header
%          {0.5\baselineskip}%                            space under header
        {\normalfont\large\bfseries}%                        style
    }%
%\end{comment}

\begin{comment}
    \renewcommand{\subsection}{\@startsection%
        {subsection}%                                name
        {2}%                                        level    
%        {\z@}%                                        indent
           {0mm}%                                         indent
        {-5ex \@plus -20ex \@minus -2.5ex}%            space above header
        {-1.5ex \@minus -1ex}%                        space under header
        {\normalfont\bfseries}%                        style
    }%
\end{comment}



    \renewcommand{\subsubsection}{\@startsection%    
        {subsubsection}%                              name
        {3}%                                          level
        {0mm}%                                        indent
        {-2ex \@plus -1ex \@minus -.2ex}%           space above header
        {2ex \@plus.2ex }%                          space under header
        {\normalfont\bfseries}%                       style
    }                    
    
%\begin{comment}
    
    \titleformat{\subsubsection} [runin]
      {}
      {\thesubsubsection}
      {1ex}{}[.]
    \titlespacing*{\subsubsection}{\parindent}{*4}{1ex}
    
%\end{comment}
    
    
    \renewcommand\paragraph{%
        \@startsection{paragraph}{4}{\z@}%
        {-2ex \@plus-.2ex \@minus .2ex}%
        {0.1ex}%
%        {\normalfont\small\sffamily\bfseries}%
        {\normalfont\small\bfseries}%
    }%

\makeatother
    
    

    
    
    %----------------------------------------------------------------------------------------
    %    CHAPTER HEADINGS
    %----------------------------------------------------------------------------------------



\begin{comment}
\makeatletter
    
    \newcommand{\thechapterimage}{}
    \newcommand{\chapterimage}[1]{\renewcommand{\thechapterimage}{#1}}
    \def\thechapter{\arabic{chapter}}
    \def\@makechapterhead#1{
    \thispagestyle{empty}
%    {\centering \normalfont\sffamily
    {\centering \normalfont
    \ifnum \c@secnumdepth >\m@ne
    \if@mainmatter
    \startcontents
    \begin{tikzpicture}[remember picture,overlay]
    \node at (current page.north west)
    {\begin{tikzpicture}[remember picture,overlay]
    
    \node[anchor=north west,inner sep=0pt] at (0,0) {\includegraphics[width=\paperwidth]{\thechapterimage}};
    %\node[anchor=north west,inner sep=0pt] at (0,0) {\IncludeEpsFromSvg[width=\paperwidth]{\thechapterimage}};
    %Commenting the 3 lines below removes the small contents box in the chapter heading
    %\draw[fill=white,opacity=.6] (1cm,0) rectangle (8cm,-7cm);
    %\node[anchor=north west] at (1cm,.25cm) {\parbox[t][8cm][t]{6.5cm}{\huge\bfseries\flushleft \printcontents{l}{1}{\setcounter{tocdepth}{2}}}};
%\normalfont\Large\bfseries    
    \draw[anchor=west] (1.5cm,-8.4cm) node [rounded corners=25pt,fill=white,fill opacity=.6,text opacity=1,draw=ocre,draw opacity=1,line width=2pt,inner sep=15pt]{\normalfont\Large\bfseries\textcolor{black}{\thechapter\ ---\ #1\vphantom{plPQq}\makebox[22cm]{}}};    
    \end{tikzpicture}};
    \end{tikzpicture}}\par\vspace*{230\p@}
    \fi
    \fi
    }
    \def\@makeschapterhead#1{
    \thispagestyle{empty}
%    {\centering \normalfont\sffamily
    {\centering \normalfont
    \ifnum \c@secnumdepth >\m@ne
    \if@mainmatter
    \startcontents
    \begin{tikzpicture}[remember picture,overlay]
    \node at (current page.north west)
    {\begin{tikzpicture}[remember picture,overlay]
    \node[anchor=north west] at (-4pt,4pt) {\includegraphics[width=\paperwidth]{\thechapterimage}};
    %\node[anchor=north west] at (-4pt,4pt) {\IncludeEpsFromSvg[width=\paperwidth]{\thechapterimage}};
%    \draw[anchor=west] (5cm,-9cm) node [rounded corners=25pt,fill=white,opacity=.7,inner sep=15.5pt]{\huge\sffamily\bfseries\textcolor{black}{\vphantom{plPQq}\makebox[22cm]{}}};
    \draw[anchor=west] (5cm,-9cm) node [rounded corners=25pt,fill=white,opacity=.7,inner sep=15.5pt]{\huge\bfseries\textcolor{black}{\vphantom{plPQq}\makebox[22cm]{}}};    
%    \draw[anchor=west] (5cm,-9cm) node [rounded corners=25pt,draw=ocre,line width=2pt,inner sep=15pt]{\huge\sffamily\bfseries\textcolor{black}{#1\vphantom{plPQq}\makebox[22cm]{}}};
    \draw[anchor=west] (5cm,-9cm) node [rounded corners=25pt,draw=ocre,line width=2pt,inner sep=15pt]{\huge\bfseries\textcolor{black}{#1\vphantom{plPQq}\makebox[22cm]{}}};    
    \end{tikzpicture}};
    \end{tikzpicture}}\par\vspace*{230\p@}
    \fi
    \fi
    }
\makeatother



\definecolor{gray75}{gray}{0.75} % определяем цвет
\newcommand{\hsp}{\hspace{10pt}} % длина линии в 20pt
% titleformat определяет стиль
\titleformat{\chapter}[hang]{\Huge\bfseries}{\thechapter\hsp\textcolor{gray75}{|}\hsp}{0pt}{\Huge\bfseries}
\end{comment}

\makeatletter
    
    \newcommand{\thechapterimage}{}
    \newcommand{\chapterimage}[1]{\renewcommand{\thechapterimage}{#1}}
    \def\thechapter{\arabic{chapter}}
    \def\@makechapterhead#1{
    \thispagestyle{empty}
%    {\centering \normalfont\sffamily
    {\centering \normalfont
    \ifnum \c@secnumdepth >\m@ne
    \if@mainmatter
    \startcontents
    \begin{tikzpicture}[remember picture,overlay]
    \node at (current page.north west)
    {\begin{tikzpicture}[remember picture,overlay]
    
    \node[anchor=north west,inner sep=0pt] at (0,0) {\includegraphics[width=\paperwidth]{\thechapterimage}};
    %\node[anchor=north west,inner sep=0pt] at (0,0) {\IncludeEpsFromSvg[width=\paperwidth]{\thechapterimage}};
    %Commenting the 3 lines below removes the small contents box in the chapter heading
    %\draw[fill=white,opacity=.6] (1cm,0) rectangle (8cm,-7cm);
    %\node[anchor=north west] at (1cm,.25cm) {\parbox[t][8cm][t]{6.5cm}{\huge\bfseries\flushleft \printcontents{l}{1}{\setcounter{tocdepth}{2}}}};
%\normalfont\Large\bfseries    
    \draw[anchor=west] (1.5cm,-6.3cm) node [rounded corners=25pt,fill=white,fill opacity=.8,text opacity=1,draw=ocre, draw opacity=1,line width=2pt,inner sep=15pt]{\normalfont\LARGE\bfseries\textcolor{black}{\thechapter\ .\ #1\vphantom{plPQq}\makebox[22cm]{}}};    
    \end{tikzpicture}};
    \end{tikzpicture}}\par\vspace*{150\p@}
    \fi
    \fi
    }
    \def\@makeschapterhead#1{
    \thispagestyle{empty}
%    {\centering \normalfont\sffamily
    {\centering \normalfont
    \ifnum \c@secnumdepth >\m@ne
    \if@mainmatter
    \startcontents
    \begin{tikzpicture}[remember picture,overlay]
    \node at (current page.north west)
    {\begin{tikzpicture}[remember picture,overlay]
    \node[anchor=north west] at (-4pt,4pt) {\includegraphics[width=\paperwidth]{\thechapterimage}};
    %\node[anchor=north west] at (-4pt,4pt) {\IncludeEpsFromSvg[width=\paperwidth]{\thechapterimage}};
%    \draw[anchor=west] (5cm,-9cm) node [rounded corners=25pt,fill=white,opacity=.7,inner sep=15.5pt]{\huge\sffamily\bfseries\textcolor{black}{\vphantom{plPQq}\makebox[22cm]{}}};
    \draw[anchor=west] (5cm,-6.3cm) node [rounded corners=25pt,fill=white,opacity=.8,inner sep=15.5pt]{\normalfont\LARGE\bfseries\textcolor{black}{\vphantom{plPQq}\makebox[22cm]{}}};    
%    \draw[anchor=west] (5cm,-9cm) node [rounded corners=25pt,draw=ocre,line width=2pt,inner sep=15pt]{\huge\sffamily\bfseries\textcolor{black}{#1\vphantom{plPQq}\makebox[22cm]{}}};
    \draw[anchor=west] (5cm,-6.3cm) node [rounded corners=25pt,draw=ocre,line width=2pt,inner sep=15pt]{\normalfont\LARGE\bfseries\textcolor{black}{#1\vphantom{plPQq}\makebox[22cm]{}}};    
    \end{tikzpicture}};
    \end{tikzpicture}}\par\vspace*{150\p@}
    \fi
    \fi
    }
\makeatother

%----------------------------------------------------------------------------------------
%    Оформление ОГЛАВЛЕНИя
%----------------------------------------------------------------------------------------

% точки после названий частей !!!
\renewcommand\cftchapaftersnum{.}

%все номера частей, секций и т.п. и точки - одним цветом !
\renewcommand\cftchappagefont{\color{my_color}}

\renewcommand\cftchappagefont{\color{my_color}}
\renewcommand\cftchapleader{\color{my_color}\cftdotfill{\cftdotsep}}

\renewcommand\cftsecpagefont{\color{my_color}}
\renewcommand\cftsecleader{\color{my_color}\cftdotfill{\cftsecdotsep}}

\renewcommand\cftsubsecpagefont{\color{my_color}}
\renewcommand\cftsubsecleader{\color{my_color}\cftdotfill{\cftsubsecdotsep}}


% отступы между частями!!!
%\preto\section{\ifnum\value{section}=0\addtocontents{toc}{\vskip10pt}\fi}
\makeatletter
\pretocmd{\chapter}{\addtocontents{toc}{\protect\addvspace{7\p@}}}{}{}
\pretocmd{\section}{\addtocontents{toc}{\protect\addvspace{7\p@}}}{}{}
\pretocmd{\subsection}{\addtocontents{toc}{\protect\addvspace{5\p@}}}{}{}
\pretocmd{\subsubsection}{\addtocontents{toc}{\protect\addvspace{5\p@}}}{}{}
\makeatother


%\preto\subsection{\ifnum\value{subsection}=0\addtocontents{toc}{\vskip7pt}\fi}

%\preto\subsection{\addtocontents{toc}{\vskip7pt}}

% отступы!!! (для subsection - обязательно оставить!!!)

    \setlength{\cftsecindent}{7pt}  % Indent of section No.
    \setlength{\cftsecnumwidth}{34pt}  % Width of section No.

    \setlength{\cftsubsecindent}{21pt}  % Indent of subsection No.
    \setlength{\cftsubsecnumwidth}{52pt}  % Width of subsection No.

    \setlength{\cftsubsubsecindent}{35pt}  % Indent of subsubsection No.
    \setlength{\cftsubsubsecnumwidth}{62pt}  % Width of subsubsection No.

\begin{comment}
\contentsmargin{0cm} % Removes the default margin
% Chapter text styling
\titlecontents{chapter}[1.25cm] % Indentation
{\addvspace{15pt}\large\sffamily\bfseries} % Spacing and font options for chapters
{\color{ocre!60}\contentslabel[\Large\thecontentslabel]{1.25cm}\color{ocre}} % Chapter number
{}  
{\color{ocre!60}\normalsize\sffamily\bfseries\;\titlerule*[.5pc]{.}\;\thecontentspage} % Page number
% Section text styling
\titlecontents{section}[1.25cm] % Indentation
{\addvspace{5pt}\sffamily\bfseries} % Spacing and font options for sections
{\contentslabel[\thecontentslabel]{1.25cm}} % Section number
{}
{\sffamily\hfill\color{black}\thecontentspage} % Page number
[]
% Subsection text styling
\titlecontents{subsection}[1.25cm] % Indentation
{\addvspace{1pt}\sffamily\small} % Spacing and font options for subsections
{\contentslabel[\thecontentslabel]{1.25cm}} % Subsection number
{}
{\sffamily\;\titlerule*[.5pc]{.}\;\thecontentspage} % Page number
[] 

\end{comment}


% % % % % % % % % % % Настройка параметров и переопределение предметного указателя % % % % %
\makeatletter
% 2-й параметр - отступ от левого края основного текста
% 1-й параметр - сдвиг (от левого края) при переносе текста на новую строку
\begin{comment}
\renewcommand{\@idxitem}{\par\hangindent=7pt\hspace*{0pt}}

%\renewcommand{\subitem}{\par\hangindent=70pt\hspace*{12pt}}
\renewcommand{\subitem}{\par\hangindent=21pt\hspace*{14pt}}

%\renewcommand{\subsubitem}{\par\hangindent=35pt\hspace*{24pt}}
\renewcommand{\subsubitem}{\par\hangindent=35pt\hspace*{28pt}}
\end{comment}

\renewcommand{\@idxitem}{\par\hangindent=5pt\hspace*{0pt}}

%\renewcommand{\subitem}{\par\hangindent=70pt\hspace*{12pt}}
\renewcommand{\subitem}{\par\hangindent=17pt\hspace*{10pt}}

%\renewcommand{\subsubitem}{\par\hangindent=35pt\hspace*{24pt}}
\renewcommand{\subsubitem}{\par\hangindent=28pt\hspace*{21pt}}


\renewenvironment*{theindex}{\columnseprule=0pt\columnsep=15pt
\@makeschapterhead{\indexname}%
\@mkboth{\uppercase{\indexname}}{\uppercase{\indexname}}%
\thispagestyle{plain}\parindent=0pt
\setlength{\parskip}{0pt plus .3pt}%
\let\item=\@idxitem
\begin{multicols}{2}}%
{\end{multicols}}

\makeatother



%%%***************************%%%
%%%***    Листинги         ***%%%
%%%***************************%%%


%\color{frame}

\begin{comment}
\makeatletter

\renewenvironment{snugshade}{%
 \def\FrameCommand##1{\hskip\@totalleftmargin \hskip-\fboxsep
 \colorbox{shadecolor}{##1}\hskip-\fboxsep
     % There is no \@totalrightmargin, so:
     \hskip-\linewidth \hskip-\@totalleftmargin \hskip\columnwidth}%
 \MakeFramed {\advance\hsize-\width
   \@totalleftmargin\z@ \linewidth\hsize
   \@setminipage}}%
 {\par\unskip\endMakeFramed}

\makeatother
\end{comment}

%!!! определение стиля для служебных слов

\newcommand{\KeyWordUpcaseBold}{\normalfont\bfseries}%  нужен знак *
\newcommand{\KeyWordLocaseBold}{\normalfont\bfseries}%    без знака *
\newcommand{\KeyWordUpcase}{\bfseries}%                   нужен знак *
\newcommand{\KeyWordLocase}{\bfseries}%                      без знака *

%\newcommand{\KeyWordSt}{\bfseries}%

%    *\normalfont   - для UPcase
%     \normalfont\bfseries   - для LOcase

\newcommand{\KeyWordSt}{\KeyWordLocaseBold}%

%   \bfseries - lo case
%  *\bfseries - up case
% просто \bfseries не дает нужного эффекта, нужен \bf, но он только на аргумент
% хорошая альтернатива для нормального болда: \normalfont\bfseries
% 

\ifKeyWordUpCase

    \newcommand{\MyKeyWordSt}{%
        *\normalfont}%
        
    }%

\else
    \newcommand{\MyKeyWordSt}{%
        \normalfont\bfseries%
    }%
\fi


\lstdefinelanguage{cnc_lang}
{%
    classoffset=0,
    morekeywords={abort, abs, adisable, begin, bstart, bstop,% break,
    cclr, ccmode0, ccmode1, ccmode2, ccmode3, ccr,  cexec,% 
    cmd, continue, cout, cset, cskip, D, ddisable,  delay,%default,
    disable, bgcplc, rticplc, dkill, do, dread, dtogread, dwell,%
    enable, F, frax, fread, G, hold, home, homez, inc, jog, jogret,%
%    kill, lh, lhpurge, M, N, nofrax, normal, nxyz, pause, pclear, pload,% 
    kill, lh, lhpurge, M, nofrax, normal, nxyz, pause, pclear, pload,% 
    pmatch, pread, pset, pstore, read, resume, return, run, s, S, send,% 
    sendall, sendallcmds, sendallsystemsmds, start, step, system, %stop,
    struct,union, void, __attribute__%  
    T, td, tm, tread, tsel, tsel1, tsel2, tsel3, txyz,%ts,
%    rot, lookahead, plc, plcc, undefine, r, R, A, b, B, c, C, S,% 
    rot, lookahead, plc, plcc, undefine, r, R, b, B, c, C, S,% 
    p, q, v, h, u, U, % 
%    a, j, p, q, v, k, h, u, U, I, i, % 
    open, inverse, forward, prog, prog1000, close, clear, delete, gather,%
    define, all, rotary, ecat, echo, buffer, buffers, hmz, hm, free,%
    program, list, kinematic, backup, bpclear, bpclearall, bpset, %
    brickacver, bricklvver, cpu, cpx, cx, alias, assign, config, slaves,%
%    apc, pc, subprog, out, reboot, rotfree, save, size, string, type, date,%  
    apc, pc, subprog, out, reboot, rotfree, save, size, string, date,%  
    G00, G01, G02, G03, G04, G05, G06, G07, G08, G09,%
    G10, G11, G12, G13, G14, G15, G16, G17, G18, G19,%
    G20, G21, G22, G23, G24, G25, G26, G27, G28, G29,%
    G30, G31, G32, G33, G34, G35, G36, G37, G38, G39,%
    G40, G41, G42, G43, G44, G45, G46, G47, G48, G49,%
    G50, G51, G52, G53, G54, G55, G56, G57, G58, G59,%
    G60, G61, G62, G63, G64, G65, G66, G67, G68, G69,%
    G70, G71, G72, G73, G74, G75, G76, G77, G78, G79,%
    G80, G81, G82, G83, G84, G85, G86, G87, G88, G89,%
    G90, G91, G92, G93, G94, G95, G96, G97, G98, G99,%
    G100,Gnn, G115, F, tm, Global, csglobal,%ts, ta,
    },%
%    keywordstyle=*\ttfamily\color{key_word_1_color},%
    keywordstyle=\ttfamily\normalfont\bfseries\color{key_word_1_color},%
    classoffset=1,%
    morekeywords={call, callsub, case, if, else, endif, and, or, endwhile, while,%
    , wait, goto, gosub, switch, break, default,},%
%    keywordstyle=*\ttfamily\color{key_word_2_color},%
    keywordstyle=\ttfamily\normalfont\bfseries\color{key_word_2_color},%
    classoffset=2,%
    morekeywords={circle, circle1, circle2, circle3, circle4, linear,pvt,%
    rapid,spline, spline1, spline2, circlen, splinen,},%
%    keywordstyle=*\ttfamily\color{key_word_3_color},%
    keywordstyle=\ttfamily\normalfont\bfseries\color{key_word_3_color},%
    classoffset=3,
    morekeywords={cos, sin, tan, asin, acos, atan2,%
%    morekeywords={cos, sin, tan, asin, acos, atan, atan2,%
    int, double, int32_t, uint32_t, exp, ln, sqrt},%
%    keywordstyle=*\ttfamily\color{key_word_4_color},%
    keywordstyle=\ttfamily\normalfont\bfseries\color{key_word_4_color},%
    classoffset=0,
    %otherkeywords={$, &, $},
    %otherkeywords={&,\#},
%    otherkeywords={\#,\$,&,\%},
    otherkeywords={\#,\%},    %&,
     morecomment=[l]{//},% l is for line comment
    morecomment=[l]{;},%
    morecomment=[s]{/*}{*/}, % s is for start and end delimiter
    %morestring=[b]" % defines that strings are enclosed in double quotes
% Цвета
    stringstyle=\color{string}%
    commentstyle=\ttfamily\normalfont\footnotesize\bfseries\itshape\color{comment},%
        sensitive=false,%
%    commentstyle=\sffamily\footnotesize\itshape\bfseries\color{comment},%
%    commentstyle=\ttfamily\normalfont\bfseries\itshape\color{comment},%
    commentstyle=\ttfamily\normalfont\footnotesize\bfseries\itshape\color{comment},%
    %numberstyle=\color{kew_word_2_color},
    postbreak=\space, 
    breakindent=5pt, 
    breaklines,                            % Перенос длинных строк
    breakatwhitespace,
%    xleftmargin = 12pt,                % отступ слева
    lineskip = 5pt,                        % расстояние между строками
    emptylines =*1,                        % максимальное количество "пустых" линий между строками - остальные просто режутся
}

%breaklines=true,                        % Automatic line breaking?
%   breakatwhitespace=true,                % Automatic breaks only at whitespace
   

% Настройки отображения кода
\lstset{%
    language=cnc_lang,                      % Язык кода по умолчанию
%    basicstyle=\footnotesize,              % Шрифт для отображения кода
%    basicstyle=\ttfamily\footnotesize,     % Шрифт для отображения кода
    basicstyle=\ttfamily,     % Шрифт для отображения кода
%    basicstyle=\sffamily\footnotesize,    % Шрифт для отображения кода
%    columns = flexible,
    backgroundcolor=\color{bk},             % Цвет фона кода
    frame=none,
    columns=fixed, % make all characters equal width
%    frame=lrb,xleftmargin=\fboxsep,xrightmargin=-\fboxsep,      % Рамка, подогнанная к заголовку
%    rulecolor=\color{frame},                % Цвет рамки
    tabsize=2,                              % Размер табуляции в пробелах
% Настройка отображения номеров строк. Если не нужно, то удалите весь блок
%    numbers=left,                           % Слева отображаются номера строк
%    stepnumber=1,                           % Каждую строку нумеровать
%    numbersep=5pt,                          % Отступ от кода
%    numberstyle=\small\color{black},        % Стиль написания номеров строк
% Для отображения русского языка
    extendedchars=true,
    literate=
    {~}{{\textasciitilde}}1
    {а}{{\selectfont\char224}}1
    {б}{{\selectfont\char225}}1
    {в}{{\selectfont\char226}}1
    {г}{{\selectfont\char227}}1
    {д}{{\selectfont\char228}}1
    {е}{{\selectfont\char229}}1
    {ё}{{\"e}}1
    {ж}{{\selectfont\char230}}1
    {з}{{\selectfont\char231}}1
    {и}{{\selectfont\char232}}1
    {й}{{\selectfont\char233}}1
    {к}{{\selectfont\char234}}1
    {л}{{\selectfont\char235}}1
    {м}{{\selectfont\char236}}1
    {н}{{\selectfont\char237}}1
    {о}{{\selectfont\char238}}1
    {п}{{\selectfont\char239}}1
    {р}{{\selectfont\char240}}1
    {с}{{\selectfont\char241}}1
    {т}{{\selectfont\char242}}1
    {у}{{\selectfont\char243}}1
    {ф}{{\selectfont\char244}}1
    {х}{{\selectfont\char245}}1
    {ц}{{\selectfont\char246}}1
    {ч}{{\selectfont\char247}}1
    {ш}{{\selectfont\char248}}1
    {щ}{{\selectfont\char249}}1
    {ъ}{{\selectfont\char250}}1
    {ы}{{\selectfont\char251}}1
    {ь}{{\selectfont\char252}}1
    {э}{{\selectfont\char253}}1
    {ю}{{\selectfont\char254}}1
    {я}{{\selectfont\char255}}1
    {А}{{\selectfont\char192}}1
    {Б}{{\selectfont\char193}}1
    {В}{{\selectfont\char194}}1
    {Г}{{\selectfont\char195}}1
    {Д}{{\selectfont\char196}}1
    {Е}{{\selectfont\char197}}1
    {Ё}{{\"E}}1
    {Ж}{{\selectfont\char198}}1
    {З}{{\selectfont\char199}}1
    {И}{{\selectfont\char200}}1
    {Й}{{\selectfont\char201}}1
    {К}{{\selectfont\char202}}1
    {Л}{{\selectfont\char203}}1
    {М}{{\selectfont\char204}}1
    {Н}{{\selectfont\char205}}1
    {О}{{\selectfont\char206}}1
    {П}{{\selectfont\char207}}1
    {Р}{{\selectfont\char208}}1
    {С}{{\selectfont\char209}}1
    {Т}{{\selectfont\char210}}1
    {У}{{\selectfont\char211}}1
    {Ф}{{\selectfont\char212}}1
    {Х}{{\selectfont\char213}}1
    {Ц}{{\selectfont\char214}}1
    {Ч}{{\selectfont\char215}}1
    {Ш}{{\selectfont\char216}}1
    {Щ}{{\selectfont\char217}}1
    {Ъ}{{\selectfont\char218}}1
    {Ы}{{\selectfont\char219}}1
    {Ь}{{\selectfont\char220}}1
    {Э}{{\selectfont\char221}}1
    {Ю}{{\selectfont\char222}}1
    {Я}{{\selectfont\char223}}1
    {і}{{\selectfont\char105}}1
    {ї}{{\selectfont\char168}}1
    {є}{{\selectfont\char185}}1
    {ґ}{{\selectfont\char160}}1
    {І}{{\selectfont\char73}}1
    {Ї}{{\selectfont\char136}}1
    {Є}{{\selectfont\char153}}1
    {Ґ}{{\selectfont\char128}}1
    {\{}{{{\color{brackets}\{}}}1 % Цвет скобок {
    {\}}{{{\color{brackets}\}}}}1 % Цвет скобок }
}




\begin{comment}
\lstdefinestyle{basic}{  
  basicstyle=\footnotesize\ttfamily,
  numbers=left,
  numberstyle=\tiny\color{gray}\ttfamily,
  numbersep=5pt,
  backgroundcolor=\color{white},
  showspaces=false,
  showstringspaces=false,
  showtabs=false,
  frame=single,
  rulecolor=\color{black},
  captionpos=b,
  keywordstyle=\color{blue}\bf,
  commentstyle=\color{gray},
  stringstyle=\color{green},
  keywordstyle={[2]\color{red}\bf},
}
\end{comment}

% по умолчанию - сквозная нумерация!!!
\begin{comment}
%Формат: ХХ.YY.ZZ.PP.number
\AtBeginDocument{%
  \renewcommand{\thelstlisting}{%
    \ifnum\value{subsection}=0
      \thesection.\arabic{lstlisting}%
    \else
      \ifnum\value{subsubsection}=0
        \thesubsection.\arabic{lstlisting}%
      \else
        \thesubsubsection.\arabic{lstlisting}%
      \fi
    \fi
  }
}
\end{comment}

% Формат: ХХ.YY.number
\begin{comment}
\AtBeginDocument{%
  \renewcommand{\thelstlisting}{%
%      \arabic{lstlisting}%
      \thesection.\arabic{lstlisting}%
    %  \textbf{\arabic{lstlisting}}%
  }
}
\end{comment}

%%%*** ---------------------------------- ***%%%


%%%*********************************************%%%
%%%*** Настройка формата подписей к листингам***%%%
%%%*********************************************%%%
%\DeclareCaptionFormat{listing}{\rule{\dimexpr\textwidth+17pt\relax}{0.4pt}\par\vskip1pt#1#2#3}
%\captionsetup[lstlisting]{margin=0pt, labelfont = bf}

\DeclareCaptionFormat{listing}{#1#2#3}
\captionsetup[lstlisting]{format=listing,singlelinecheck=false, margin=0pt,labelsep = space,labelfont = bf}


\newcommand{\IncludeListing}[3]{%
%\begin{snugshade}%
\begin{listingExample}
\lstinputlisting[caption=#2,label=#3]{#1}%
\end{listingExample}

%\end{snugshade}%
}

\newcommand{\IncludeLstWithoutCaption}[1]{
\begin{listingExample}
\lstinputlisting{#1}%
\end{listingExample}
}

\newcommand{\IncludeLstWithoutBorder}[3]{%
\lstinputlisting[caption=#2,label=#3]{#1}%
}

\newcommand{\mylstinline}[1]{%
\colorbox{bk}{\lstinline[backgroundcolor=\color{bk}]{#1}}~%
}%

\renewcommand{\lstlistingname}{\textbf{Пример}}         % %\renewcommand{\lstlistingname}{\textbf{Пример кода программы}}         % Переименование Listings в нужное именование структуры

% подписи к листингам
%\DeclareCaptionFont{white}{\color{сaptiontext}}
%\DeclareCaptionFormat{listing}{\parbox{\linewidth}{\colorbox{сaptionbk}{\parbox{\linewidth}{#1#2#3}}\vskip-4pt}}
%\captionsetup[lstlisting]{format=listing,labelfont=white,textfont=white}
%%%*** -------------------------- ***%%%


%%%*** ------------------- ***%%%


%%%**********************************%%%
%%%*** Настройка формата подписей ***%%%
%%%**********************************%%%
% подписи к рисункам и таблицам
\DeclareCaptionLabelFormat{viadot}{#1 #2.}          % после номера Рис№ и Табл№  
\captionsetup{labelsep=space, labelformat=viadot}   % - стоит точка!!!


\newcommand{\keyword}[1]{%
%    \textbf{\textcolor{key_word_color}{#1}}%
    \lstinline{#1}%
}%


\newcommand{\killoverfullbefore}{%
    {\sloppy%     % чтобы был минимум переполнений (чтобы строки не разрастались

    }
}

\ifbiblatex
    \newcommand{\BiblioCite}[1]{%
        \cite{#1}%
    }
\else
    \newcommand{\BiblioCite}[1]{%
        {}%
    }
\fi


%*******************************************
% стиль названий плат расширения
\newcommand{\DSPGATEthree}{%
    \textbf{DSPGATE3} %
}%



%*******************************************
% стиль структур данных
\newcommand{\mystruct}[1]{%
    \textbf{#1}%
%    \textbf{\nameref{sec:#1}}%
}%

% стиль структур данных
\newcommand{\mystructA}[1]{%
    \textbf{#1}%
%    \textbf{\nameref{sec:#1}}%
}%

% стиль структур данных
\newcommand{\mystructall}[1]{%
    \textbf{#1}%
}%



\setrefcountdefault{0}

%\getrefbykeydefault{}{page}{0}

\newcommand{\myreftosec}[1]{%
    \ifnum \getpagerefnumber{sec:#1}=0% 
        \textbf{#1}%                  если ссылка не определена - просто текст
    \else% 
        \textbf{\nameref{sec:#1}}%    если ссылка определена - текст с гиперссылкой
    \fi
%    \IfRefUndefinedBabe{sec:#1}{%
%        \textbf{#1}%                  если ссылка не определена - просто текст
%    }%
%    {% 
%        \textbf{\nameref{sec:#1}}%    если ссылка определена - текст с гиперссылкой
%    }%
}%

\newcommand{\myreftosecwithpage}[1]{%
    \ifnum \getpagerefnumber{sec:#1}=0% 
        \textbf{#1}%                  если ссылка не определена - просто текст
    \else% 
        \textbf{\nameref{sec:#1}}~(стр.~\pageref{sec:#1})%    если ссылка определена - текст с гиперссылкой + указанием номера страницы
    \fi
}%


\begin{comment}


\ifnum \getpagerefnumber{sec:Gate3[i].Chan[j].UVW} = 0 
    <tex-code-1> 
\else 
    <tex-code-2>
\fi

\end{comment}


% стиль значений элементов структур данных
\newcommand{\myval}[1]{%
    \textit{#1}%
}%


%*******************************************
% стиль названий плат расширения
\newcommand{\myacc}[1]{%
    \textit{\textbf{#1}}%
}%


\newcommand{\mytextBFIT}[1]{%
    \textit{\textbf{#1}}%
}%



%*******************************************
% стиль названий типов обработки ДОС
\newcommand{\mytypeECT}[1]{%
    \textit{\textbf{#1}}%
}%



%*******************************************
% стиль записей справа 
\newcommand{\RightHandText}[1]{%
    \textit{#1}%
}%


%*******************************************
% Диапазоны переменных(полей структур)
                            %Floating-point
\newcommand{\myfloatpoint}{%
    Диапазон чисел с плавающей запятой%
}%

\newcommand{\myinteger}{%
    Диапазон целых чисел%
}%

\newcommand{\mynonneginteger}{%
    Диапазон неотрицательных целых чисел%
}%

                            %Positive floating-point (double-precision)
\newcommand{\mypositfloatp}{%
    Диапазон положительных чисел с плавающей запятой%
}%

                            %Positive floating-point
\newcommand{\mypositfloatpoint}{%
    Диапазон положительных чисел с плавающей запятой%
}%

                            %Non-negative floating-point
\newcommand{\mynonnegfloatp}{%
    Диапазон неотрицательных чисел с плавающей запятой%
}%

                            %Non-negative floating-point
\newcommand{\myposfloatp}{%
    Диапазон положительных чисел с плавающей запятой%
}%

\newcommand{\myflpointsixteenbt}{%
    +/-32768 (число с плавающей запятой)%
}%

%Legitimate addresses
\newcommand{\mylegitimadrs}{%
    Все допустимые адреса%
}%
    
\newcommand{\myautoconf}{%
    Автоматическая настройка %
}%


% требуют уточнения
\newcommand{\myReqwestDetail}{%
    \textcolor{red}{ТРЕБУЕТСЯ УТОЧНЕНИЕ}%
}%


%*******************************************


%*******************************************
% Единицы измерения переменных(полей структур)

                    
\newcommand{\myuserset}{%
    Пользовательская настройка%
}%

\newcommand{\axisunit}{%
    Единицы измерения перемещения по оси%
}%

                             %Data structure element addresses
\newcommand{\mydatastreladr}{%
    Адреса элементов структур данных%
}%

                               %Motor units
\newcommand{\motorunits}{%
    Дискреты положения%
}%

                               %Motor units per millisecond
\newcommand{\unitspermsec}{%
    Дискреты положения/мс%
}%

                               %Motor units per servocycle
\newcommand{\unitspermservo}{%
    Дискреты положения/период сервоцикла%
}%
                               %Motor units per millisecond in sq
\newcommand{\unitspermsecinsq}{%
    Дискреты положения/$\text{мс}^2$%
}%

                             %Motor units per units of source data
\newcommand{\unitspersource}{%
    Дискреты положения/единицы измерения исходных данных%
}%

                             %Motor units per LSB of source data
\newcommand{\unitsperlsbsource}{%
    Дискреты положения/МЗР исходных данных%
}%

                %Milliseconds (if >= 0) or milliseconds2 per motor unit (if < 0)
\newcommand{\unitabortta}{%
    мс (если значение >= 0) или мс$^2$/дискреты положения (если значение < 0)%
}%

                %Milliseconds (if >= 0) or milliseconds3 per motor unit (if < 0)
\newcommand{\unitabortts}{%
    мс (если значение >= 0) или мс$^3$/дискреты положения (если значение < 0)%
}%

                               %Motor units per millisecond in sq
\newcommand{\msecinsqperunit}{%
    $\text{мс}^2$/дискреты положения%
}%

                %LSBs of 16-bit output per motor unit of position error
\newcommand{\mylsbperuniterror}{%
    МЗР 16-битного выхода(ЦАП)/дискреты измерения ошибки положения}%

                         %Bit field
\newcommand{\mybitfield}{%
    Битовое поле%
}%

                         %Bits of a signed 16-bit input/output
\newcommand{\myunitsixteenadcdac}{%
    Биты 16-битного значения АЦП/ЦАП со знаком%
}%

                         %Bit field
\newcommand{\myunitbool}{%
    Логический тип данных%
}%

                           %none (unit-less z-transform coefficient)
\newcommand{\myunitlessz}{%
    Нет%
}%




%*******************************************
%     Область действия
\newcommand{\myscope}{%
    Область действия
}%

\newcommand{\ComThreadSpec}{%
    Заданный коммуникационный поток%
}%

\newcommand{\CoordSysSpec}{%
    Заданная координатная система%
}%

\newcommand{\CoordSysAndComThreadSpec}{%
    Заданные коммуникационный поток и координатная система%
}%

\newcommand{\MotorSpec}{%
    Заданный двигатель%
}%


\newcommand{\GlobalScope}{%
    Глобальная%
}%


%*******************************************

%*******************************************
% Значение по умолчанию для переменных (полей структур)
\newcommand{\myautoconfhard}{%
    Автоматически настраивается в зависимости от аппаратной конфигурации%
}%

%*******************************************

%*******************************************
% Тип буферов, для которых предназначена команда 

\newcommand{\TypeBufAllPMC}{%
    Программа движения (\keyword{prog} и~\keyword{rotary})%
}%

\newcommand{\TypeBufProgPMC}{%
    Программа движения (только \keyword{prog})%
}%

\newcommand{\TypeBufAllPMCandPLC}{%
    Программа движения (\keyword{prog} и~\keyword{rotary}), PLC программа%
}%

\newcommand{\TypeBufProgPMCandPLC}{%
    Программа движения (только \keyword{prog}), PLC программа%
}%





%*******************************************
%        Для Смотрите также:
%*******************************************

\newcommand{\SeeAlso}{%
    \vspace{8pt}%
    \textbf{\textit{\underline{Смотрите также:}}}%
    \vspace{2pt}%
}%

\newcommand{\SeeAlsoAcc}{%
    \vspace{8pt}%
    \textbf{\textit{\underline{Для более подробной информации смотрите:}}}%
    \vspace{2pt}%
}%

\newcommand{\DescriptInDev}{%
    \begin{remark}%
    \textbf{Примечание:}\BL 
    \textit{Пункт находится в стадии разработки.}%
    \end{remark}%
%    \vspace{1pt}%
}%

\newcommand{\DescriptInDevTemp}{%
    \BL 
    \begin{remark}%
    \textbf{Примечание:}\BL 
    \textit{Более подробное описание находится в стадии разработки.}%
    \end{remark}%
%    \vspace{1pt}%
}%

\newcommand{\DescriptCMDInDev}{%
    \begin{remark}%
    \textbf{Примечание:}\BL 
    \textit{Описание команды находится в стадии разработки.}%
    \end{remark}%
    \vspace{1pt}%
}%


\newcommand{\SeeParagraphs}{%
    Разделы:%
}%

\newcommand{\SeeVar}{%
    Переменные:%
}%

\newcommand{\SeeProgCmd}{%
    Программные команды:%
}%

\newcommand{\SeeOnlineCmd}{%
    Онлайн команды:%
}%


\newcommand{\tabitem}{~\llap{\textbullet}~~}

\newcommand{\DbgSecSt}[2]{%
% DebugSectionState     DbgSecSt
    % 0 - вообще не готова 
    % 1 - название есть
    % 2 - готова частично
    % 3 - готова полностью
\ifDebugSectState%    
    \ifcase #1%
            \textcolor{red}{#2}%      0    \StZero
        \or \textcolor{violet}{#2}%   1    \StName
        \or \textcolor{blue}{#2}%     2    \StPart
        \or \textcolor{green}{#2}%    3    \StFull
        \or \textcolor{orange}{#2}%   4    \StNote
        \or \textcolor{gray}{#2}%     5    \StDis        
    \else     \textcolor{red}{#2}%    
    \fi%
\else%
    {#2}%        - как есть
\fi
}%

\newcommand{\StZero}{0}
\newcommand{\StName}{1}
\newcommand{\StPart}{2}
\newcommand{\StFull}{3}
\newcommand{\StHome}{4}
\newcommand{\StNote}{4}
\newcommand{\StDis}{5}


% % % % % % % Типы обработки ДОС % % % % % % % %

\newcommand{\myECTtypeZero}{\mytypeECT{Флаг окончания модуля обработки ДОС}}

\newcommand{\myECTtypeOne}{\mytypeECT{Чтение одного 32-битного регистра}}

\newcommand{\myECTtypeTwo}{\mytypeECT{Зарезервировано}}

\newcommand{\myECTtypeThree}{\mytypeECT{Программная <<1/T интерполяция>>}}

\newcommand{\myECTtypeFour}{\mytypeECT{Интерполяция с программным вычислением арктангенса угла}}

\newcommand{\myECTtypeFive}{\mytypeECT{Зарезервировано}}

\newcommand{\myECTtypeSix}{\mytypeECT{Прямое резольверное преобразование арктангенса угла}}

\newcommand{\myECTtypeSeven}{\mytypeECT{Интерполяция с аппаратным вычислением арктангенса угла}}

\newcommand{\myECTtypeEight}{\mytypeECT{Сложение результатов двух записей модуля обработки ДОС}}

\newcommand{\myECTtypeNine}{\mytypeECT{Вычитание результатов двух записей модуля обработки ДОС}}

\newcommand{\myECTtypeTen}{\mytypeECT{Триггерная временная развертка}}

\newcommand{\myECTtypeEleven}{\mytypeECT{Зарезервировано}}

\newcommand{\myECTtypeTwelve}{\mytypeECT{Зарезервировано}}


%в дальнейшем требует корректировки!!!
\newcommand{\ToCorrect}[1]{%
   \textcolor{red}{\textbf{#1}}%
}%


% знак градуса
\newcommand{\degree}{%
$^\circ$ %
}%


% корректный перенос строки - вместо \\ + новый параграф без отступа
\newcommand{\BL}{%
    \par\vspace{\baselineskip}%
    \noindent% 
}%

% новый параграф без отступа
\newcommand{\NL}{%
    \noindent% 
}%


% корректировка ширины таблицы, чтобы не было ошибок
\newcommand{\TBW}{%
    >{\raggedright}%
}%

% корректировка ширины таблицы, чтобы не было ошибок
\newcommand{\TB}{%
    \raggedright%
}%

% если TB в конце, то нужно это:
\newcommand{\TBend}{%
    \tabularnewline%
}%


%\newcommand*{\nom}[2]{#1\nomenclature{#1}{#2}}



\newcounter{word}
\makeatletter
\newcommand*{\LBL}{%
  \@dblarg\@LBL
}
\def\@LBL[#1]#2{%
  \begingroup
    \renewcommand*{\theword}{#2}%
    \refstepcounter{word}%
    \label{#1}%
    #2%
  \endgroup
}
\makeatother






 %       \LBL[BtnKill]{\textit{<<Выключить>>}}


\newcommand{\mylbl}[2]{%
%   \LBL[#2]{\textbf\textit{<<#1>>}}%
   \LBL[#2]{\textcolor{linkcolor}{\textit{\textbf{\textit{#1}}}}}%
}

\newcommand{\myheader}[2]{%
%   \LBL[#2]{\textbf\textit{<<#1>>}}%
   \LBL[#2]{\textcolor{indigo_dye}{\textit{\textbf{\textit{#1}}}}}%
}
%

\newcommand{\myref}[1]{%
 \ref{#1}, стр. \pageref{#1}%
}

% Величина пробелов списка
\renewenvironment{itemize}{
    \begin{list}{\labelitemi}{
    \setlength{\topsep}{0pt}
    \setlength{\partopsep}{0pt}
    \setlength{\parskip}{0pt}
    \setlength{\itemsep}{0pt}
    \setlength{\parsep}{0pt}
    }
}{\end{list}}

% ссылки на слова
%\newcommand{mylbl}[1]{%
%   \textcolor{blue}{\textbf{\LBL{#1}}}%
%}%


%\newcommand{\myreftolbl}[1]{%
%    \ifnum \getpagerefnumber{sec:#1}=0% 
%        \textbf{#1}%                  если ссылка не определена - просто текст
%    \else% 
%        \textbf{\nameref{sec:#1}}%    если ссылка определена - текст с гиперссылкой
%    \fi

\renewcommand{\sfdefault}{cmss}                      % Применяемые стили


%%%*****************************************%%%
%%%***           Гиперссылки                  ***%%%
%%%***  обязательно должен быть           ***%%%    
%%%***  последним подключаемым пакетом     ***%%%
%%%*****************************************%%%
% В списке загружаемых пакетов hyperref, должен стоять последним, так как он модифицирует работу других пакетов. 


\usepackage[    %dvips, 
                %linktocpage=true,  - ссылки только номера страниц
            linktoc=all,        %ссылки и текст и номера страниц!!!
            colorlinks=true, linkcolor=my_color, citecolor = my_color, urlcolor=urlcolor,
               breaklinks=true,
               plainpages=false,pdfpagelabels=false,
               pdfauthor={inelsy},
            pdftitle={user manual},
            pdfsubject={IntAMP},
            pdfkeywords={inelsy IntAMP},
            pdfproducer={inelsy},
            pdfcreator={inelsy}]{hyperref}
%%%*** -------------------------------- ***%%%

\begin{document}
    \begingroup

\thispagestyle{empty}

%\definecolor{title_color}{HTML}{CDCDB4} %A98040
\pagecolor{title_color}
 
\begin{figure}[h!]{
        \noindent \centering{%
            \IncludeEpsFromSvg[1\textwidth]{./Pictures/svg/inelsy_logo}
        }
}

\end{figure}

    \centering
    \vspace*{3cm}

    {\LARGE CЧПУ серии IntNC PRO}\par 

    \vspace*{0.5cm}
    
    \vspace*{1cm}

    {\LARGE РУКОВОДСТВО ПО ПРОГРАММИРОВАНИЮ}\par 
    
    \vspace*{1cm}
        
    {\LARGE ПРОГРАММЫ ПЛК}\par 

\vspace*{1cm}
    
     {\large ВЕРСИЯ 1.0}\par 


\begin{figure}[b]{
        \noindent \centering{%
            \IncludeEpsFromSvg[0.2\textwidth]{./Pictures/svg/inelsy}\BL %
            %ИНЭЛСИ \copyright\ 2014 - 2017\BL % Copyright notice
            \url{www.inelsy.com}\BL \BL \BL% URL
            \copyright\ ИНЭЛСИ 2018\BL % License information
        }
}
\end{figure}


\endgroup

\newpage

\pagecolor{white}              % Титульный лист

    \chapterimage{chapter_head_0} % Table of contents heading image

%\pagestyle{empty} % No headers

\setstretch{0.7}

\tableofcontents % Print the table of contents itself

\cleardoublepage % Forces the first chapter to start on an odd page so it's on the right

%\pagestyle{fancy} % Print headers again

\setstretch{1.1}       % Оглавление
    \phantomsection

\etocsettocdepth.toc {chapter}

\chapterimage{chapter_head_0}
\chapter*{Список сокращений}
\addcontentsline{toc}{chapter}{Список сокращений}
\BL

\begin{doublespace}
\begin{smallTblBits}
$\bullet$ ДОС & \hspace{6pt} ~-- & датчик обратной связи;\\
$\bullet$ ПЛК & \hspace{6pt} ~-- & программируемый логический контроллер;\\
$\bullet$ ПО & \hspace{6pt} ~-- & программное обеспечение;\\
$\bullet$ ПЭС & \hspace{6pt} ~-- & программа электроавтоматики станка;\\
$\bullet$ СЧПУ & \hspace{6pt} ~-- & система числового программного управления;\\
$\bullet$ УП & \hspace{6pt} ~-- & управляющая программа;\\
$\bullet$ УЧПУ & \hspace{6pt} ~-- & устройство числового программного управления;\\
$\bullet$ GNU & \hspace{6pt} ~-- & General Public License ~-- лицензия на свободное программное обеспечение;\\
$\bullet$ IDE & \hspace{6pt} ~-- & Integrated Development Environment ~-- интегрированная среда разработки.\\
\end{smallTblBits}

\end{doublespace}


\newpage           % Список сокращений
    \phantomsection

\etocsettocdepth.toc {chapter}

\chapterimage{chapter_head_0}
\chapter*{Введение}
\addcontentsline{toc}{chapter}{Введение}

Настоящее руководство по программированию (далее РП) предназначено для изучения  принципов создания программ ПЛК для системы \mbox{ЧПУ} серии \textbf{IntNC PRO}. 

Системы \mbox{ЧПУ} серии \textbf{IntNC PRO} имеют программно реализованный встроенный логический контроллер, программы для которого разрабатываются на основе языка программирования высокого уровня IntLang.

Описание языка программирования \textbf{IntLang} приведено в Руководстве по программированию <<ЯЗЫК ПРОГРАММИРОВАНИЯ IntLang CЧПУ серии IntNC PRO>>.

%Для выполнения таких задач, как работа со строковыми параметрами, математические операции, управление запуском программ, предусмотрены наборы встроенных функций и макросов.

Настоящее РП распространяется на все модификации CЧПУ серии \textbf{IntNC PRO}. \killoverfullbefore \BL

\textbf{Сохраняется право внесения изменений!}
%Напечатано в РФ
%Все права защищены. Воспроизведение любой части данного издания в любой форме (фотокопия, микрофильм или иной метод) или редактирование, размножение или распространение с помощью электронных систем без письменного разрешения запрещаются.

\BL\BL\BL

\copyright\ Inelsy 11/09/2018

\BL

WINDOWS является зарегистрированной торговой маркой Microsoft Corporation

\newpage                   % Введение
    \etocsettocdepth.toc {section}

\chapterimage{chapter_head_0} 
\chapter{\DbgSecSt{\StPart}{Особенности реализации программ ПЛК}}
\label{sec:Elements}
\index{Особенности реализации программ ПЛК|(}

%--------------------------------------------------------
% *******begin section***************
\section{\DbgSecSt{\StPart}{Встроенный логический контроллер}}
\index{Особенности реализации программ ПЛК!Встроенный логический контроллер}

Системы \mbox{ЧПУ} серии \textbf{IntNC PRO} имеют встроенный механизм выполнения логических программ управления ~-- программно реализованный встроенный логический контроллер.\killoverfullbefore

Интегрированный в системное программное обеспечение (рис. ~\ref{fig:PLC}) логический контроллер гарантирует:
\begin{itemize}
\item одно адресное пространство для выполнения системных задач и программ логического управления;
\item синхронизацию между различными задачами УЧПУ;
\item выполнение до 4-х программ ПЛК в режиме реального времени;
\item выполнение до 32-х программ ПЛК в фоновом режиме. \killoverfullbefore 
\end{itemize}

%\centering{\includegraphics[scale=0.7]{./Pictures/eps/buttons/1.eps}}

\DrawPictEpsFromSvg[0.7\textwidth]{./Pictures/svg/PLC}{Функциональная схема системного программного обеспечения}{PLC}
% *******end section*****************
%--------------------------------------------------------
\begin{comment}
% *******begin section***************
\section{\DbgSecSt{\StPart}{Программы ПЛК реального времени и фонового режима}}
\index{Программы ПЛК реального времени и фонового режима}

Константа ~-- число, символ или строка символов. Константы используются в программе для задания постоянных величин. Различают четыре типа констант: целые, с плавающей точкой, символьные константы и cтроковые литералы.\BL

\index{Элементы языка!Константы!Целые константы}
\mylbl{Целые константы}{IntegerConstant} \BL

Строковые литералы имеют тип массива char, то есть строка ~-- массив элементов типа char. Число элементов массива равно числу символов в строке плюс один для заканчивающего пустого символа. \killoverfullbefore
% *******end section*****************
\end{comment}
%--------------------------------------------------------

% *******begin section***************
\section{\DbgSecSt{\StPart}{Язык программ ПЛК}}
\index{Особенности реализации программ ПЛК!Язык программ ПЛК}

Для создания программ ПЛК используется процедурный язык программирования IntLang, разработанный на основе стандарта ANSI C.

Язык программирования IntLang имеет следующие особенности:
\begin{itemize}
\item простую языковую базу;
\item минимальное число ключевых слов;
\item систему типов;
\item области действия имён;
\item определяемые пользователем собирательные типы данных ~-- структуры и объединения;
\item передачу параметров в функцию по значению;
\item препроцессор для определения макросов и включения файлов с исходным кодом;
\item математические функции и функции работы с массивами. \killoverfullbefore 
\end{itemize}
% *******end section*****************
%--------------------------------------------------------

% *******begin section***************
\section{\DbgSecSt{\StPart}{Организация программ ПЛК}}
\index{Особенности реализации программ ПЛК!Организация программ ПЛК}

Программы ПЛК реализуются в виде текстовых файлов с расширением \texttt{cfg} и входят 
в состав конфигурационных файлов УЧПУ для станка. \killoverfullbefore 

Программы ПЛК размещаются в директории пользовательских файлов \texttt{<<source/platform/имя\_проекта>>} и их имена включаются в файл \texttt{<<source/platform/имя\_проекта/target.cfg>>}. \killoverfullbefore 

\DrawPictEpsFromSvg[0.6\textwidth]{./Pictures/svg/Struct_1}{Организация конфигурационных файлов проекта <<stanok>>}{Struct_1}
% *******end section*****************
%--------------------------------------------------------

\index{Особенности реализации программ ПЛК|)}

\clearpage                     % Обосбенности программ ПЛК
    \etocsettocdepth.toc {subsection}

\chapterimage{chapter_head_0} 
\chapter{\DbgSecSt{\StPart}{Создание программ ПЛК}}
\label{sec:DataTypes}
\index{Создание программ ПЛК|(}

\renewcommand{\arraystretch}{1.2} %% increase table row spacing
\renewcommand{\tabcolsep}{0.2cm}   %% increase table column spacing
%--------------------------------------------------------
% *******begin section***************
\section{\DbgSecSt{\StPart}{Краткое описание языка программ ПЛК}}
\index{Создание программ ПЛК!Краткое описание языка программ ПЛК}

Краткое описание языка программ ПЛК содержит информацию об основных элементах языка IntLang.

Полное описание языка IntLang приведено в Руководстве по программированию <<ЯЗЫК ПРОГРАММИРОВАНИЯ IntLang CЧПУ серии IntNC PRO>>.

\subsection{\DbgSecSt{\StPart}{Набор символов}}
\index{Создание программ ПЛК!Описание языка программ ПЛК!Набор символов}

Множество символов языка содержит буквы, цифры и знаки пунктуации.\killoverfullbefore

Набор символов содержит прописные и строчные буквы латинского алфавита, 10 десятичных цифр арабской системы исчисления и символ подчеркивания ( \_ ). Они используются для формирования констант, идентификаторов и ключевых слов. Прописные и строчные буквы обрабатываются как разные символы. 

\renewcommand{\arraystretch}{1.4} %% increase table row spacing
\renewcommand{\tabcolsep}{0.5cm}   %% increase table column spacing
\begin{center}
\begin{tabular}{ l l }
Прописные английские буквы: & A B C D E F G H I J K L M N O P Q R S T U V W X Y Z \\
Строчные английские буквы: & a b c d e f g h i j k l m n o p q r s t u v w x y z  \\
Десятичные цифры: & 0 1 2 3 4 5 6 7 8 9  \\
Символ подчеркивания:  & \_ \\
\end{tabular}
\end{center}

Знаки пунктуации и специальные символы из набора символов имеют самое разное предназначение, от организации текста программы до определения задач, которые будут выполнены программой.

\renewcommand{\arraystretch}{1.2} %% increase table row spacing
\renewcommand{\tabcolsep}{0.2cm}   %% increase table column spacing
\begin{center}
\begin{tabular}{ l l|l l }
, &  запятая  &  >  &  правая угловая скобка   \\
. &  точка &   ! &   восклицательный знак   \\
; &  точка с запятой  &   $\vert$   &  вертикальная черта      \\
: &  двоеточие  &   /   &  знак деления   \\
? &  знак вопроса     &  \textbackslash   &  знак обратного деления    \\
' &  одинарная цитатная скобка     &  \textasciitilde  &  тильда    \\
\end{tabular}
\end{center}

\begin{center}
\begin{tabular}{ l l|l l }
" &  двойная цитатная скобка     &  +   &  плюс    \\
( & левая круглая скобка     &   \#   &  номер   \\
) &  правая круглая скобка    &   \%    &  процент \\
$ [ $ &  левая прямоугольная скобка     &  \&   &  амперсанд\\
$ ] $ &  правая прямоугольная скобка     &   \textasciicircum   &  крышечка\\
\{ &  левая фигурная скобка     &   *    &  звездочка\\
\} &  правая фигурная скобка     &   -   &  минус\\
< &  левая угловая скобка    &   =   &  равно \textcolor{white}{отступ отступ отступ} \\
\end{tabular}
\end{center}

% *******end subsection*****************
%--------------------------------------------------------
\subsection{\DbgSecSt{\StPart}{Ключевые слова}}
\index{Создание программ ПЛК!Описание языка программ ПЛК!Ключевые слова}

Ключевые слова ~-- заранее определенные идентификаторы, которые имеют специальное значение. Их использование строго регламентировано. Имя элемента программы не может совпадать по написанию с ключевым словом. 

\begin{center}
\begin{tabular}{ c | c | c | c }
break    & double   & int      & switch \\
case     & else     & long     & typedef \\
char     & enum     & return   & union \\
const    & extern   & short    & unsigned \\
continue & float    & signed   & void \\
default  & for      & static   &  while \\
do       & if       & struct   &  \\
\end{tabular}
\end{center}

\subsection{\DbgSecSt{\StPart}{Базовые типы данных}}
\index{Создание программ ПЛК!Описание языка программ ПЛК!Базовые типы данных}

В языке реализован набор типов данных, называемых базовыми типами. Спецификации этих типов перечислены ниже.
%в табл.~\ref{tbl:DataTypes}.

\begin{center}
\begin{tabular}{ l l }
Целые типы: & char, short, int, long, enum   \\
Типы с плавающей точкой: & float, double  \\
Пpочие:  & void, const \\
\end{tabular}
\end{center}

\begin{comment}
\begin{center}%
        \begin{longtable}{|m{0.45\linewidth}|m{0.30\linewidth}|}
        \caption{Базовые типы данных}\label{tbl:DataTypes}\\ 
    \hline  
            &       \\
            \TB{\textbf{Тип данных}} &  
            \textbf{Спецификация типа}  \TBend 
            &       \\
\hline

\hline  Целые типы         & char \\ \cline{2-2}
                    & short  \\ \cline{2-2}
                        & int  \\ \cline{2-2}
                        & long \\ \cline{2-2}
                        & enum  \\ 

\hline  Типы с плавающей точкой     & float \\ \cline{2-2}
                   & double \\ 

\hline  Пpочие          & void \\ \cline{2-2}
                        & const \\ \cline{2-2}

\hline
\end{longtable}
\end{center}%
\end{comment}


\begin{comment}
Типы данных, определяемые пользователем, указаны в табл.~\ref{tbl:UserDataTypes}.

\begin{MyTableTwoColCntr}{Типы данных пользователя}{tbl:UserDataTypes}{|m{0.17\linewidth}|m{0.73\linewidth}|}{Тип данных}{Описание}
\hline Массивы  & Одномерный:\textcolor{white}{отс}  имя\_массива[размер] \newline Двумерный:\textcolor{white}{отсту}   имя\_массива[размер][размер]\\
\hline Структуры & struct имя\_структуры \{описание элемента структуры, ... \};  \newline
Поля бит в структурах: \textcolor{white}{отс} struct имя\_структуры \{ \newline
описание элемента структуры : кол-во бит, ... \};\\
\hline  Объединения  & union имя\_объединения \{описание элемента объединения, ... \}; \\
\hline  Перечисления & enum имя\_перечисления \{список значений \}; \\
\end{MyTableTwoColCntr}
\end{comment}

% *******end subsection*****************
%--------------------------------------------------------

% *******begin section***************
\subsection{\DbgSecSt{\StPart}{Области значений}}
\index{Создание программ ПЛК!Описание языка программ ПЛК!Области значений}

Область значений ~-- интервал от минимального до максимального значения, которое может быть представлено в переменной данного типа. В табл.~\ref{tbl:Range} приведен размер занимаемой памяти и области значений переменных каждого типа. \killoverfullbefore
 
\begin{MyTableThreeCol}{Область значений типов}{tbl:Range}{|>{\TB}m{0.20\linewidth}|>{\TB}m{0.10\linewidth}|m{0.6\linewidth}|}{Тип}{Размер, байт}{Область значений}
\hline  
unsigned char & \centering{1} & $  0  \div  255  $\\ 
\hline 
signed char (char) & \centering{1} &  $  -128  \div  127  $\\
\hline 
unsigned short & \centering{2} &  $  0  \div  65535  $\\
\hline 
signed short (short) & \centering{2} &  $  -32768  \div  32767  $\\
\hline 
unsigned int (unsigned) & \centering{4} &  $  0  \div  4294967295  $\\
\hline 
signed int (int) & \centering{4} &  $  -2147483648  \div  2147483647  $\\
\hline 
unsigned long  & \centering{8} &  $  0  \div  18446744073709551615  $\\
\hline 
signed long (long) & \centering{8} &  $ -9223372036854775808 \div 9223372036854775807 $ \\
\hline  
float & \centering{4} & $1.175494351\cdot10^{-38} \div 3.402823466\cdot10^{+38} $ \\ 
\hline 
double & \centering{8} & $2.2250738585072014\cdot10^{-308} \div 1.7976931348623158\cdot10^{+308} $ \\
\end{MyTableThreeCol}
% *******end section*****************
%--------------------------------------------------------
% *******begin section***************
\subsection{\DbgSecSt{\StPart}{Объявления переменных}}
\index{Создание программ ПЛК!Описание языка программ ПЛК!Объявления переменных}

Переменные используются для хранения значений. Переменная характеризуется типом и именем. Типы переменных приведены в табл.~\ref{tbl:VT}. Имя переменной может начинаться с подчеркивания или буквы, но не с числа. Имя переменной может включать в себя символы английского алфавита, цифры и знак подчёркивания, но не должно совпадать с ключевыми словами.

\begin{MyTableTwoColCntr}{Типы переменных}{tbl:VT}{|m{0.18\linewidth}|m{0.73\linewidth}|}{Тип переменной}{Описание}
\hline \centering{Простая переменная}  & Отдельная переменная с одним значением целого типа или с плавающей точкой \\
\hline \centering {Перечисляемая переменная}   & Простая переменная целого типа, принимающая одно значение из предопределенного набора именованных констант \newline enum имя\_перечисления \{список значений \};\\
\hline  \centering {Структура}  & Переменная, содержащая совокупность элементов, которые могут иметь различные типы \newline struct имя\_структуры \{описание элемента структуры, ... \};  \newline
Поля бит в структурах: \textcolor{white}{отс} struct имя\_структуры \{ \newline
описание элемента структуры : кол-во бит, ... \};\\
\hline  \centering {Объединение} & Переменная, содержащая совокупность элементов, которые могут иметь различные типы, но занимают одну и ту же область памяти \newline union имя\_объединения \{описание элемента объединения, ... \};\\
\hline \centering {Массив} & Переменная, содержащая совокупность элементов одинакового типа \newline Одномерный:\textcolor{white}{отс}  имя\_массива[размер] \newline Двумерный:\textcolor{white}{отсту}   имя\_массива[размер][размер]\\
\end{MyTableTwoColCntr}

% *******end section*****************
%--------------------------------------------------------
% *******begin section***************
\subsection{\DbgSecSt{\StPart}{Операции}}
\index{Создание программ ПЛК!Описание языка программ ПЛК!Операции}

Любое выражение состоит из операндов, соединенных знаками операций. Знак операции - это символ или группа символов, которые сообщают о необходимости выполнения определенных арифметических, логических или других действий. 

Операции имеют либо один операнд (унарные операции), либо два операнда (бинарные операции), либо три (тернарная операция). Операция присваивания может быть как унарной, так и бинарной.\killoverfullbefore

Унарные операции приведены в табл.~\ref{tbl:UO}. Унарные операции выполняются справа налево. \killoverfullbefore

Операции увеличения и уменьшения увеличивают или уменьшают значение операнда на единицу и могут быть записаны как справа так и слева от операнда. \killoverfullbefore

Если знак операции записан перед операндом (префиксная форма), то изменение операнда происходит до его использования в выражении. Если знак операции записан после операнда (постфиксная форма), то операнд вначале используется в выражении, а затем происходит его изменение. \killoverfullbefore \BL

\begin{MyTableTwoColCntr}{Унарные операции}{tbl:UO}{|m{0.25\linewidth}|m{0.65\linewidth}|}{Знак операции}{Операция}
\hline \centering{{--}}    & арифметическое отрицание (отрицание и дополнение) \\
\hline \centering {\textasciitilde}   & побитовое логическое отрицание (дополнение) \\
\hline  \centering {!}  & логическое отрицание \\
\hline  \centering {*} & разадресация (косвенная адресация) \\
\hline \centering {\&} & вычисление адреса \\
\hline \centering {+} & унарный плюс \\
\hline \centering {++} & увеличение (инкремент) \\
\hline \centering {{--}{--}} & уменьшение (декремент) \\
\end{MyTableTwoColCntr}

Бинарные операции приведены в табл.~\ref{tbl:BO}. В отличие от унарных, бинарные операции выполняются слева направо. 

\begin{center}%
        \begin{longtable}{|m{0.25\linewidth}|>{\centering}m{0.15\linewidth}|m{0.55\linewidth}|}
        \caption{Бинарные операции}\label{tbl:BO}\\ 
    \hline  
            &       &         \\
            \TB{\textbf{Группа операций}} &  
            \textbf{Знак операции} & \textbf{Операция} \TBend 
            &       &        \\
\hline

\hline  Мультипликативные       & * & умножение \\ \cline{2-3}
                        & / & деление \\ \cline{2-3}
                        & \% & остаток от деления \\                         
                        
\hline  Аддитивные      & + & сложение \\ \cline{2-3}
                        & {--} & вычитание \\ 

\hline  Операции сдвига & $\ll$ & сдвиг влево \\ \cline{2-3}
                        & $\gg$ & сдвиг вправо \\ 
   
\hline Операции         & < & меньше \\ \cline{2-3}
отношения               & <= & меньше или равно \\ \cline{2-3}
                        & > & больше \\ \cline{2-3} 
                        & >= & больше или равно \\ \cline{2-3}                        
                        & == & равно \\ \cline{2-3}                           
                        & != & не равно \\   
                        
\hline Поразрядные      & \& & поразрядное И \\ \cline{2-3}
операции                & $\vert$ & поразрядное ИЛИ \\ \cline{2-3}
                        & \textasciicircum & поразрядное исключающее ИЛИ \\  
 
\hline   Операция последовательного вычисления     & , & последовательное вычисление \\

\hline  Операции        & = & присваивание \\ \cline{2-3}
присваивания            & *= & умножение с присваиванием \\ \cline{2-3}
                        & /= & деление с присваиванием \\ \cline{2-3} 
                        & \%= & остаток от деления с присваиванием \\ \cline{2-3}                        
                        & {--}= & вычитание с присваиванием \\ \cline{2-3}                           
                        & += & сложение с присваиванием \\   \cline{2-3} 
                        & $\ll$= & сдвиг влево с присваиванием \\   \cline{2-3} 
                        & $\gg$= & сдвиг вправо с присваиванием \\  \cline{2-3}   
                        & \&= & поразрядное И с присваиванием \\ \cline{2-3} 
                        & $\vert$= & поразрядное ИЛИ с присваиванием \\ \cline{2-3} 
                        & \textasciicircum= & поразрядное исключающее ИЛИ с присваиванием \\
\hline
\end{longtable}
\end{center}%

Тернарное выражение состоит из трех операндов, разделенных знаками тернарной операции (?) и (:), и имеет вид: <операнд1>?<операнд2>:<операнд3>.
% *******end section*****************
%--------------------------------------------------------
% *******begin subsection***************
\subsection{\DbgSecSt{\StPart}{Операторы}}
\index{Создание программ ПЛК!Описание языка программ ПЛК!Операторы}

Оператор ~-- законченная конструкция языка, реализующая определенные действия
в программе. Операторы языка приведены в табл.~\ref{tbl:Operators}.

\begin{MyTableThreeColCntr}{Операторы}{tbl:Operators}{|m{0.12\linewidth}|m{0.4\linewidth}|m{0.38\linewidth}|}{Оператор}{Описание}{Синтаксис}
\hline Простой оператор  & Любое выражение, которое заканчивается точкой с запятой. & выражение; \\
\hline Составной оператор & Последовательность операторов, заключенная в фигурные
скобки.  & \{ \newline [объявление] \newline ... \newline оператор; \newline
[оператор]; \newline ... \newline \} \\
\hline Оператор if  &  Условный оператор.  & if (<выражение>) \newline <оператор1> \newline [else <оператор2>] \\
\hline Оператор for   &  Оператор цикла, позволяющий повторять некоторый набор операторов в программе определенное количество раз.  & for([<начальное-выражение>];\newline [<условное-выражение>];\newline
[<выражение-цикла>]) \newline тело-оператора\\
\hline  Оператор while  & Оператор цикла, применяемый, когда заранее неизвестно количество повторений.  & while (<выражение>) \newline тело оператора \\
\hline  Оператор do while  &  Оператор цикла с постусловием, в котором сначала выполняется оператор, затем анализируется условие.  & do \newline тело-оператора \newline while (<выражение>) \\
\hline Оператор switch & Выбор одного оператора(-ов) из нескольких.  & switch (<выражение>)  \{\newline
[объявление]\newline
...\newline
[case константное-выражение:]\newline
[список операторов]\newline
...\newline
[case константное-выражение:]\newline
[список операторов]\newline
...\newline
[default:\newline
[список операторов]]\newline
\}\\
\hline Оператор break & Прерывает выполнение операторов do, for, switch или while, в которых он появляется. & break; \\
\hline Оператор continue & Передает управление на следующую итерацию в
операторах цикла do, for, while. & continue; \\
\hline Оператор return & Оператор return завершает выполнение функции, в которой он задан, и возвращает управление в вызывающую функцию. & return [выражение]; \\
\end{MyTableThreeColCntr}
% *******end section*****************
%--------------------------------------------------------
% *******begin subsection***************
\subsection{\DbgSecSt{\StPart}{Функции}}
\index{Создание программ ПЛК!Описание языка программ ПЛК!Функции}

Функция ~-- совокупность объявлений и операторов, предназначенная для выполнения некоторой отдельной задачи. Количество функций в программе не ограничивается.

С использованием функций связаны три понятия - определение функции (описание действий, выполняемых функцией), объявление или прототип функции (задание формы обращения к функции) и вызов функции:
\begin{itemize}
\item Определение функции специфицирует имя функции, тип и число её формальных
параметров, а также тело функции, содержащее объявления и операторы («тело функции»);  в нем также может устанавливаться тип возвращаемого значения и класс памяти.
\item Объявление или прототип функции определяет её имя, тип возвращаемого значения и класс памяти; в нем также могут быть установлены типы и идентификаторы
для некоторых или всех аргументов функции.
\item Вызов функции передает управление и фактические аргументы заданной функции. \killoverfullbefore 
\end{itemize}
% *******end section*****************
%--------------------------------------------------------
% *******begin subsection***************
\subsection{\DbgSecSt{\StPart}{Директивы препроцессора}}
\index{Создание программ ПЛК!Описание языка программ ПЛК!Директивы препроцессора}

Директивы препроцессора ~-- инструкции препроцессору, то есть текстовому процессору, который обрабатывает текст исходного файла. Директивы препроцессора приведены в табл.~\ref{tbl:Direct}.\killoverfullbefore 

Знак решётки (\#) должен быть первым неразделительным символом в строке, содержащей директиву. Некоторые директивы содержат аргументы или значения. Любой текст, который следует за директивой (кроме аргумента или значения, который является частью директивы) должен быть заключен в скобки комментария (/*  */). \killoverfullbefore

\begin{MyTableThreeColCntr}{Директивы препроцессора}{tbl:Direct}{|m{0.14\linewidth}|m{0.36\linewidth}|m{0.40\linewidth}|}{Директива}{Описание}{Синтаксис}
\hline \#define  & Используется для замены часто используемых в программе констант, ключевых слов, операторов и выражений содержательными идентификаторами. & \#define <идентификатор> <текст> \\
\hline \#include & Включает содержимое исходного файла, имя пути которого задано, в текущий исходный файл.  & \#include "имя пути" \newline \#include <имя пути> \\
\hline \#if, \newline \#elif, \newline \#else, \newline \#endif  &  Управляют условной компиляцией, то есть позволяют подавить компиляцию части исходного файла, проверяя постоянное выражение или идентификатор.  & \#if <ограниченное-константное-выражение> <текст> \newline \#elif <ограниченное-константное-выражение> <текст> \newline \#else <текст> \newline \#endif\\
\end{MyTableThreeColCntr}

Директивы препроцессора могут появляться в произвольном месте исходного файла, но они будут воздействовать только на оставшуюся часть исходного файла, в котором они появились. \killoverfullbefore
% *******end section*****************
%--------------------------------------------------------
% *******begin subsection***************
\subsection{\DbgSecSt{\StPart}{Математические функции}}
\index{Создание программ ПЛК!Описание языка программ ПЛК!Математические функции}

\begin{MyTableTwoColCntr}{Математические функции}{tbl:MFunc}{|m{0.37\linewidth}|m{0.53\linewidth}|}{Функция}{Описание}
\hline int isnan(double x); \newline int isnanf(float x); & Функция используется для проверки, является ли аргумент x не числом NaN. \\
\hline double cos(double x); \newline float cosf(float x); & Функция возвращает значение косинуса аргумента х. \\
\hline double sin(double x); \newline float sinf(float x);  &  Функция возвращает значение синуса аргумента х. \\
\hline double tan(double x); \newline float tanf(float x);  &  Функция возвращает значение тангенса аргумента х. \\
\hline double acos(double x); \newline float acosf(float x);  &  Функция возвращает главное значение арккосинуса аргумента х. \\
\hline double asin(double x); \newline float asinf(float x);  &  Функция возвращает главное значение арксинуса аргумента х. \\
\hline double atan(double x); \newline float atanf(float x);  &  Функция возвращает главное значение арктангенса аргумента х. \\
\hline double atan2(double y, double x); \newline float atan2f(float y, float x);  &  Функция возвращает главное значение арктангенса аргумента y/x. \\
\hline double sqrt(double x); \newline float sqrtf(float x);  &  Функция возвращает значение квадратного корня аргумента x. \\
\hline double fabs(double x); \newline float fabsf(float x);  &  Функция возвращает абсолютное значение (модуль) аргумента x. \\
\hline double pow(double x, double p); \newline float powf(float x, float p);  &  Функция возвращает значение аргумента х, возведенного в степень р ($\text{x}^p$). \\
\hline double exp(double x); \newline float expf(float x);  &  Функция возвращает значение экспоненты аргумента х ($\text{e}^x$). \\
\hline double exp2(double x); \newline float exp2f(float x);  &  Функция возвращает значение числа 2 в степени x ($\text{2}^x$). \\
\hline double log(double x); \newline float logf(float x);  &  Функция возвращает значение натурального логарифма аргумента х. \\
\hline double log10(double x); \newline float log10f(float x);  &  Функция возвращает значение логарифма по основанию 10 аргумента x. \\
\hline double log2(double x); \newline float log2f(float x);  &  Функция возвращает значение логарифма по основанию 2 аргумента x. \\
\hline double min(double x, double y); \newline float minf(float x, float y); &  Функция возвращает наименьшее из двух значений аргументов x и y. \\
\hline double max(double x, double y); \newline float maxf(float x, float y); &  Функция возвращает наибольшее из двух значений аргументов x и y. \\
\hline double floor(double x); \newline float floorf(float x);  &  Функция округляет аргумент x до наибольшего целого числа, которое меньше или равно аргументу. \\
\hline double ceil(double x); \newline float ceilf(float x);  &  Функция округляет аргумент x до наименьшего целого числа, которое больше или равно аргументу. \\
\hline double trunc(double x); \newline float truncf(float x);  &  Функция округляет аргумент x путем отброса дробной части, то есть возвращает целую часть аргумента. \\
\hline double round(double x); \newline float roundf(float x);  &  Функция округляет аргумент x до ближайшего целого числа. \\
\hline double fmod(double x, double y); \newline fmodf(double x, double y);  &  Функция возвращает остаток от деления х на у. \\
\end{MyTableTwoColCntr}
% *******end section*****************
%--------------------------------------------------------
% *******begin subsection***************
\subsection{\DbgSecSt{\StPart}{Функции работы с памятью}}
\index{Создание программ ПЛК!Описание языка программ ПЛК!Функции работы с памятью}

\begin{MyTableTwoColCntr}{Функции работы с памятью}{tbl:ArrayFunc}{|m{0.37\linewidth}|m{0.53\linewidth}|}{Функция}{Описание}
\hline void memcpy (void *dst, void *src, int size); & Функция копирует size байт из области памяти, адресуемой аргументом src, в область памяти, адресуемую аргументом dst. \\
\hline void memmove (void *dst, void *src, int size); & Функция копирует size байт из области памяти, адресуемой аргументом src, в область памяти, адресуемую аргументом dst. \\
\hline void memset (void *ptr, int n, int size); & Функция заполняет первые size байт области памяти, адресуемой аргументом ptr, символом n после его преобразования в unsigned char. \\
\end{MyTableTwoColCntr}
% *******end section*****************
%--------------------------------------------------------

\clearpage

% *******begin section***************
\section{\DbgSecSt{\StPart}{Среда проектирования и разработки}}
\index{Создание программ ПЛК!Среда проектирования и разработки}

Для создания программ ПЛК используется кросс-платформенная свободно распространяемая
интегрированная среда разработки IDE Qt Creator, которая представляет собой комплекс настраиваемых программных средств для разработки программного обеспечения. \killoverfullbefore 

Данное решение предлагает:
\begin{itemize}
\item редактор кода с подсветкой синтаксиса, определяемой пользователем;  
\item удобную навигацию внутри проекта;
\item дополнительные элементы, помогающие визуализировать проект;
\item поддержку для сборки приложений;
\item использование различных компиляторов;
\item возможность вывода сообщений об ошибках и предупреждений. \killoverfullbefore 
\end{itemize}

\begin{comment}
 не просто редактор кода, но и ряд дополнительных элементов, помогающих визуализировать проект, а также протестировать его на работоспособность и взаимодействие с конечным пользователем. 

Qt Creator (ранее известная под кодовым названием Greenhouse) — кроссплатформенная свободная IDE для разработки на С, С++ и QML. Разработана Trolltech (Digia) для работы с фреймворком Qt. Включает в себя графический интерфейс отладчика и визуальные средства разработки интерфейса как с использованием QtWidgets, так и QML.

, ИСP, также единая среда разработки, ЕСР — комплекс программных средств, используемый программистами для разработки программного обеспечения (ПО).

IDE используется для программирования на популярных языках, среди которых C, C++ и QML. Подобно другим интегрированным средам разработки, 

Юзеру доступно использование редактора кода, компилятора, (способного компилировать не только готовые проекты, но и их отдельные блоки), отладчика приложений и дополнительного редактора интерфейса. Работать с кодом в Qt Creator довольно удобно благодаря встроенной подсветке синтаксиса – ее, к слову, можно настроить под свои нужды, изменив стандартные параметры на свои собственные. 

Ключевые особенности
подсветка синтаксиса популярных языков программирования; 
возможность использовать редактор кода и интерфейса, отладчик и компилятор;
автоматическое завершение строк;
обеспечение удобной навигации внутри проекта;
компиляция отдельных блоков кода;
разработка под Windows, а также другие популярные десткопные и мобильные операционные системы;
расширение базового функционала программы с помощью плагинов.

Напомню, что Qt Creator является кросс-платформенной свободной IDE для работы с фреймворком Qt, разработанной Trolltech (Nokia). Что не мешает сделать из него простой текстовый редактор с подсветкой синтаксиса, простым отключением всех расширений. Внимание, сотни картинок!
\end{comment}

\subsection{\DbgSecSt{\StPart}{Открытие и редактирование проекта}}
\index{Создание программ ПЛК!Среда проектирования и разработки!Открытие проекта}

После запуска Qt Creator открывается режим <<Начало>> (рис. ~\ref{fig:Qt_1}), в котором пользователь может:
\begin{itemize}
\item открыть проект;  
\item открыть недавние сессии и проекты;  
\item создать новый проект;
\item открыть справочную информацию. \killoverfullbefore 
\end{itemize}

\DrawPictEpsFromSvg[0.95\textwidth]{./Pictures/svg/Qt_1}{Начальное окно Qt Creator}{Qt_1}

Для переключения режимов предназначена левая боковая панель ~-- переключатель режимов:
\begin{itemize}
\item режим <<Редактор>> используется для редактирования проекта и файлов исходных кодов;
\item режим <<Отладка>> используется для просмотра состояние вашей программы во время отладки;
\item режим <<Проекты>> используется для настройки сборки и запуска проекта (режим доступен, если открыт проект);
\item режим <<Справка>> используется для просмотра документации.\killoverfullbefore \BL
\end{itemize}

Для открытия проекта следует нажать на кнопку <<Открыть проект>> (сочетание клавиш Ctrl+Shift+O), перейти в каталог, в котором находятся конфигурационные файлы, и выбрать файл <<project.creator>>. Если имя проекта присутствует в списке последних проектов, выбрать его из данного списка.

%Для открытия проекта следует выбрать его из списка последних проектов в начальном окне или использовать сочетание клавиш Ctrl+Shift+O. 

После открытия проекта Qt Creator переходит в режим <<Редактор>> (рис. ~\ref{fig:Qt_2}).

\DrawPictEpsFromSvg[0.95\textwidth]{./Pictures/svg/Qt_2}{Содержимое проекта}{Qt_2}

В меню <<Проекты>> боковой панели выбирается её содержимое:
\begin{itemize}
\item пункт <<Проекты>> показывает список открытых проектов в текущей сессии;
\item пункт <<Открытые документы>> показывает открытые в настоящий момент документы;
\item пункт <<Закладки>> показывает установленные закладки для текущей сессии;
\item пункт <<Файловая система>> показывает содержимое проекта в каталоге;
\item пункт <<Обзор классов>> показывает функции и пользовательские типы;
\item пункт <<Иерархия включений>> показывает зависимости между файлами проекта. \killoverfullbefore \BL
\end{itemize}

Дерево файлов проекта на боковой панели позволяет перемещаться между директориями проекта, открывать файлы в редакторе. С помощью контекстного меню возможно добавлять существующие файлы и каталоги, переименовывать, удалять файлы и т.д.\killoverfullbefore

Нижняя панель имеет несколько вкладок: <<Проблемы>>,  <<Результаты поиска>>, <<Вывод приложения>>, <<Консоль сборки>> и др, число которых настраивается пользователем.\killoverfullbefore 

Вкладка <<Проблемы>> (рис. ~\ref{fig:Qt_5}) предоставляет список произошедших во время сборки ошибок и предупреждений. Нажатие правой кнопкой мыши на строке вызовет контекстное меню, с помощью которого можно копировать содержимое, показать в редакторе, в консоли сборки и т.д.\killoverfullbefore

\DrawPictEpsFromSvg[0.95\textwidth]{./Pictures/svg/Qt_5}{Вывод ошибок и предупреждений}{Qt_5}

Вкладка <<Результаты поиска>>, вызываемая также сочетанием клавиш Ctrl+Shift+F, отображает результаты глобальных поисков, таких как поиск внутри текущего документа, проекта, во всех проектах или на диске. Рис. ~\ref{fig:Qt_3} показывает пример результатов поиска всех упоминаний \texttt{<<PLC>>} в текущем проекте.\killoverfullbefore

Вкладка <<Вывод приложения>> отображает статус программы при её выполнении и отладочную информацию. 

Вкладка <<Консоль сборки>> предоставляет список произошедших во время сборки ошибок и предупреждений, который является более расширенным по сравнению с вкладкой <<Проблемы>>.\killoverfullbefore

\DrawPictEpsFromSvg[0.95\textwidth]{./Pictures/svg/Qt_3}{Результаты поиска}{Qt_3}

\subsection{\DbgSecSt{\StPart}{Сборка проекта}}
\index{Создание программ ПЛК!Среда проектирования и разработки!Сборка проекта}

Режим <<Проекты>> используется для настройки сборки проекта (рис. ~\ref{fig:Qt_4}). 

\DrawPictEpsFromSvg[0.95\textwidth]{./Pictures/svg/Qt_4}{Настройка сборки}{Qt_4}

В окне <<Настройка сборки>> указывается каталог сборки ~-- каталог, в котором находятся конфигурационные файлы, и этап сборки <<Особый: servovmc имя\_проекта.cfg>>\killoverfullbefore

Сборка проекта выполняется из верхнего меню <<Сборка>> выбором пункта <<Собрать проект>> (сочетание клавиш Ctrl+В) или нажатием нижней кнопки левой боковой панели (рис. ~\ref{fig:Qt_6}).\killoverfullbefore

\DrawPictEpsFromSvg[0.95\textwidth]{./Pictures/svg/Qt_6}{Сборка проекта}{Qt_6}

После успешной сборки проекта в каталоге сборки будет создан файл конфигурации <<config.mcc>>, который записывается в УЧПУ.\killoverfullbefore
% *******end section*****************
%--------------------------------------------------------

% *******begin section***************
\section{\DbgSecSt{\StPart}{Объявление и реализация программ ПЛК}}
\index{Создание программ ПЛК!Объявление и реализация программ ПЛК}

Для создания программы ПЛК необходимо создать новый файл с расширением \texttt{cfg} в каталоге \texttt{<<source/platform/имя\_проекта>>}. \killoverfullbefore

В рассматриваемом примере: \texttt{<<source/platform/stanok>>}. \killoverfullbefore 

\begin{comment}
\begin{itemize}
\item в окне дерева файлов проекта на боковой панели правой кнопкой мыши на папке с именем проекта вызвать контекстное меню, в котором выбрать пункт <<Добавить новый>>;
\item в появившемся окне <<Новый файл>> выбрать шаблон <<С++>> и <<Файл исходных текстов С++>>;
\item в следующем окне <<Файл исходных текстов С++>> задать имя файла с расширением;
\item добавить файл в текущий проект. \killoverfullbefore \BL
\end{itemize} 
\end{comment}

После открытия проекта в окне дерева файлов проекта на боковой панели правой кнопкой мыши на папке с именем проекта вызвать контекстное меню, в котором выбрать пункт <<Добавить новый>> (рис. ~\ref{fig:CreatePLC_1}). \killoverfullbefore

\DrawPictEpsFromSvg[0.7\textwidth]{./Pictures/svg/CreatePLC_1}{Создание нового файла в проекте}{CreatePLC_1}

В появившемся окне <<Новый файл>> выбрать шаблон <<С++>> и <<Файл исходных текстов С++>> (рис. ~\ref{fig:CreatePLC_2}).

\DrawPictEpsFromSvg[0.7\textwidth]{./Pictures/svg/CreatePLC_2}{Выбор типа файла}{CreatePLC_2}

В следующем окне <<Файл исходных текстов С++>> задать имя файла с расширением \texttt{cfg}, в котором будет реализована программа ПЛК (рис. ~\ref{fig:CreatePLC_3}).

В рассматриваемом примере: <<user\_plc.cfg>>.

\DrawPictEpsFromSvg[0.7\textwidth]{./Pictures/svg/CreatePLC_3}{Задание имени файла программы ПЛК с расширением}{CreatePLC_3}

Добавить файл в текущий проект, нажав кнопку <<Завершить>>. После добавления нового файла его имя должно появиться в окне дерева файлов проекта.\killoverfullbefore

\DrawPictEpsFromSvg[0.7\textwidth]{./Pictures/svg/CreatePLC_4}{Добавление нового файла в проект}{CreatePLC_4}

Открыв созданный файл <<user\_plc.cfg>> в редакторе, следует объявить в нём ПЛК программу строкой \texttt{PLC (номер\_программы, имя\_функции)} и реализовать определение функции.\killoverfullbefore

В рассматриваемом примере: \texttt{PLC (9, user\_conrol)}.

\DrawPictEpsFromSvg[0.7\textwidth]{./Pictures/svg/CreatePLC_5}{Объявление программы ПЛК }{CreatePLC_5}

В файл \texttt{<<source/platform/имя\_проекта/target.cfg>>} добавить строку с именем созданного файла.\killoverfullbefore

В рассматриваемом примере: \texttt{\#include "user\_plc.cfg"}.

\DrawPictEpsFromSvg[0.7\textwidth]{./Pictures/svg/CreatePLC_6}{Включение файла программы ПЛК в конфигурационный файл \texttt{<<target.cfg>>}}{CreatePLC_6}

В файле \texttt{<<include/platform/имя\_проекта/stanok\_desc.h>>} определить идентификатор, соответствующий номеру программы ПЛК.\killoverfullbefore 

В рассматриваемом примере: \texttt{\#define PLC\_USER\_CTRL \textcolor{white}{от} 9}.

\DrawPictEpsFromSvg[0.7\textwidth]{./Pictures/svg/CreatePLC_7}{Определение идентификатора программы ПЛК}{CreatePLC_7}

Для разрешения выполнения программы ПЛК необходимо добавить вызов функции \texttt{enablePLC(идентификатор)} в файле \texttt{<<source/platform/имя\_проекта/stanok.cfg>>} в функцию \texttt{initMachine()}.\killoverfullbefore

В рассматриваемом примере: \texttt{enablePLC(PLC\_USER\_CTRL)}.

\DrawPictEpsFromSvg[0.7\textwidth]{./Pictures/svg/CreatePLC_8}{Разрешение выполнения программы ПЛК}{CreatePLC_8}
% *******end section*****************
%--------------------------------------------------------
% *******begin section***************
\section{\DbgSecSt{\StPart}{Предопределённые функции}}
\index{Создание программ ПЛК!Предопределённые функции}

Предопределёнными функциями в системе управления являются:
\begin{itemize}
\item \texttt{void setup()};  
\item \texttt{void motion\_nc()};
\item \texttt{void motion\#\#()}, где \#\# ~-- целое число (1, 2, 3, \dots). \killoverfullbefore \BL
\end{itemize}

Функция \texttt{void setup()} ~-- точка старта программы пользователя,  определяемая в файле \texttt{setup.cfg}. Данной функции передаётся управление после запуска системы. 

\DrawPictEpsFromSvg[0.7\textwidth]{./Pictures/svg/CreatePLC_9}{Содержимое файла \texttt{setup.cfg}}{CreatePLC_9}

Функция \texttt{void motion\_nc()} является пользовательской программой движения №0, функции \texttt{void motion\#\#()} ~-- программы движения с номерами \#\#.

Пример программы движение приведён в \textbf{ПРИЛОЖЕНИИ 1} в листинге <<Задание программы движения>> на стр. \pageref{MotionNC}.

% *******end section*****************
%--------------------------------------------------------
% *******begin section***************
\section{\DbgSecSt{\StPart}{Загрузка конфигурации в УЧПУ}}
\index{Создание программ ПЛК!Загрузка конфигурации в УЧПУ}

Загрузка файла конфигурации <<config.mcc>> в УЧПУ осуществляется по протоколу SCP, предназначенного для защищённого копирования файлов.\killoverfullbefore \BL

Для загрузки файла конфигурации из ОС Linux используется команда \texttt{<<scp>>}, которая имеет следующий синтаксис: \texttt{<<scp source\_file\_name username@destination\_host:destination\_folder>>}. Основная команда SCP без параметров копирует файлы в фоновом режиме. Параметр <<-v>> команды \texttt{<<scp>>} служит для вывода отладочной информации на экран, что может помочь настроить соединение, аутентификацию и устранить проблемы конфигурации. \killoverfullbefore 

Пример использования команды \texttt{<<scp>>} загрузки файла <<config.mcc>> в УЧПУ: \newline \texttt{<<scp config.mcc root@192.168.1.90:/root/>>}. \killoverfullbefore \BL

Загрузка файла конфигурации из ОС Windows выполняется посредством свободно распространяемой (лицензия GNU GPL) программы ~-- графической оболочки-клиента WinSCP. \killoverfullbefore 

%SCP (от англ. secure copy) — протокол особого удалённого копирования файлов, использующий в качестве транспорта не RSH, а SSH.

После запуска программы WinSCP необходимо ввести параметры нового подключения в окне <<Вход>> (рис. ~\ref{fig:WinSCP_1}):
\begin{itemize}
\item протокол передачи ~-- SCP;
\item имя хоста ~-- IP-адрес УЧПУ, номер порта оставить по умолчанию;
\item имя пользователя и пароль (по умолчанию root и 123456). \killoverfullbefore \BL
\end{itemize}

\DrawPictEpsFromSvg[0.6\textwidth]{./Pictures/svg/WinSCP_1}{Ввод параметров подключения}{WinSCP_1}

Записать введённые параметры, нажав кнопку <<Сохранить>>. 

В окне диалоговом окне <<Сохранить как новое подключение>> оставить настройки сохранения без изменений и нажать кнопку <<ОК>> (рис. ~\ref{fig:WinSCP_2}).

\DrawPictEpsFromSvg[0.6\textwidth]{./Pictures/svg/WinSCP_2}{Сохранение параметров подключения}{WinSCP_2}

Подключиться к УЧПУ, нажав кнопку <<Войти>> (рис. ~\ref{fig:WinSCP_3}).

\DrawPictEpsFromSvg[0.6\textwidth]{./Pictures/svg/WinSCP_3}{Подключение к УЧПУ}{WinSCP_3}

После подключения к УЧПУ в правой панели программы отобразится удалённый каталог УЧПУ для загрузки файлов конфигурации. В левой панели следует перейти в каталог сборки проекта и переписать файл <<config.mcc>> в каталог УЧПУ \texttt{var/lib/motioncore/config/} в правой панели.\killoverfullbefore

\DrawPictEpsFromSvg[0.85\textwidth]{./Pictures/svg/WinSCP_4}{Загрузка файла конфигурации в УЧПУ}{WinSCP_4}

Для того, чтобы изменения вступили в силу необходимо перезагрузить УЧПУ командой \texttt{\$\$\$} через программную оболочку ServoIDE или отключением и включением питания.\killoverfullbefore

% *******end section*****************
\index{Создание программ ПЛК|)}

\clearpage   			    % Язык и среда 
    \etocsettocdepth.toc {section}

\renewcommand{\arraystretch}{1.0} %% increase table row spacing
\renewcommand{\tabcolsep}{0.1cm}   %% increase table column spacing

\chapterimage{chapter_head_0} 
\chapter{\DbgSecSt{\StPart}{Программный интерфейс ПЛК}}
\label{sec:Functions}
\index{Программный интерфейс ПЛК|(}

Программный интерфейс ПЛК ~-- набор типов данных, констант, макросов и функций, предоставляемых системой пользователю для создания программ ПЛК. 
%--------------------------------------------------------
% *******begin section***************
\section{\DbgSecSt{\StPart}{Управление УЧПУ}}

\subsection{\DbgSecSt{\StPart}{Типы данных}}

% *******begin subsection***************
\subsubsection{\DbgSecSt{\StPart}{CNCMode}}
\index{Программный интерфейс ПЛК!Управление УЧПУ!Перечисление CNCMode}
\label{sec:CNCMode}

\begin{fHeader}
    Тип данных:            & \RightHandText{Перечисление CNCMode}\\
    Файл объявления:             & \RightHandText{include/cnc/cnc.h} \\
\end{fHeader}

Перечисление определяет идентификаторы режимов работы УЧПУ.

\begin{MyTableTwoColAllCntr}{Перечисление CNCMode}{tbl:CNCMode}{|m{0.38\linewidth}|m{0.57\linewidth}|}{Идентификатор}{Описание}
\hline cncNull &   Режим не определён  \\
\hline cncOff &  УЧПУ не активно \\
\hline cncManual  & Ручной режим \\
\hline cncHome &  Режим выезда в нулевую точку \\
\hline cncHWL &  Режим дискретных перемещений \\
\hline cncAuto &  Автоматический режим \\
\hline cncStep &  Пошаговый режим \\
\hline cncMDI &  Режим преднабора \\
\hline cncVirtual &  Виртуальный режим \\
\hline cncReset &  Режим сброса \\
\hline cncRepos &  Режим возврата на контур \\
\hline cncWaitChangeMode &  Ожидание смены режима \\
\end{MyTableTwoColAllCntr}
% *******end subsection***************
\clearpage
% *******begin subsection***************
\subsubsection{\DbgSecSt{\StPart}{ChannelStatus}}
\index{Программный интерфейс ПЛК!Управление УЧПУ!Перечисление ChannelStatus}
\label{sec:ChannelStatus}

\begin{fHeader}
    Тип данных:            & \RightHandText{Перечисление ChannelStatus}\\
    Файл объявления:             & \RightHandText{include/cnc/cnc.h} \\
\end{fHeader}

Перечисление определяет идентификаторы состояний канала управления.

\begin{MyTableTwoColAllCntr}{Перечисление ChannelStatus}{tbl:ChannelStatus}{|m{0.38\linewidth}|m{0.57\linewidth}|}{Идентификатор}
{Описание}
\hline channelReset &   Готовность  \\
\hline channelInterrupted &   Работа прервана \\
\hline channelActive &   Активен \\
\end{MyTableTwoColAllCntr}
% *******end subsection***************

% *******begin subsection***************
\subsubsection{\DbgSecSt{\StPart}{ModeState}}
\index{Программный интерфейс ПЛК!Управление УЧПУ!Перечисление ModeState}
\label{sec:ModeState}

\begin{fHeader}
    Тип данных:            & \RightHandText{Перечисление ModeState}\\
    Файл объявления:             & \RightHandText{include/cnc/cnc.h} \\
\end{fHeader}

Перечисление определяет идентификаторы состояний текущего режима УЧПУ.

\begin{MyTableTwoColAllCntr}{Перечисление ModeState}{tbl:ModeState}{|m{0.38\linewidth}|m{0.57\linewidth}|}{Идентификатор}{Описание}
\hline modeReset &   Готовность  \\
\hline modeRunning  &  Выполнение \\
\hline modeStopped  &  Останов \\
\end{MyTableTwoColAllCntr}
% *******end subsection***************

% *******begin subsection***************
\subsubsection{\DbgSecSt{\StPart}{ProgramSeekMode}}
\index{Программный интерфейс ПЛК!Управление УЧПУ!Перечисление ProgramSeekMode}
\label{sec:ProgramSeekMode}

\begin{fHeader}
    Тип данных:            & \RightHandText{Перечисление ProgramSeekMode}\\
    Файл объявления:             & \RightHandText{include/cnc/cnc.h} \\
\end{fHeader}

Перечисление определяет идентификаторы режимов выполнения УП с произвольного кадра.

\begin{MyTableTwoColAllCntr}{Перечисление ProgramSeekMode}{tbl:ProgramSeekMode}{|m{0.38\linewidth}|m{0.57\linewidth}|}{Идентификатор}{Описание}
\hline seekNone &  Режим не активен  \\
\hline seekApproach &  Выполнение УП с начала выбранного кадра  \\
\hline seekWithoutApproach  & Выполнение УП с конца выбранного кадра \\
\hline seekWithoutCalc &  Выполнение УП без расчёта фрагмента программы до выбранного кадра \\
\end{MyTableTwoColAllCntr}
% *******end subsection***************

% *******begin subsection***************
\subsubsection{\DbgSecSt{\StPart}{ProgramStatus}}
\index{Программный интерфейс ПЛК!Управление УЧПУ!Перечисление ProgramStatus}
\label{sec:ProgramStatus}

\begin{fHeader}
    Тип данных:            & \RightHandText{Перечисление ProgramStatus}\\
    Файл объявления:             & \RightHandText{include/cnc/cnc.h} \\
\end{fHeader}

Перечисление определяет идентификаторы состояний УП.

\begin{MyTableTwoColAllCntr}{Перечисление ProgramStatus}{tbl:ProgramStatus}{|m{0.38\linewidth}|m{0.57\linewidth}|}{Идентификатор}{Описание}
\hline programAborted &  Выполнение УП прервано и завершено  \\
\hline programInterrupted  & Выполнение УП временно прервано для  какой-либо операции\\
\hline programStopped  & Выполнение УП остановлено \\
\hline programRunning  &  УП выполняется \\
\end{MyTableTwoColAllCntr}
% *******end subsection***************

% *******begin subsection***************
\subsubsection{\DbgSecSt{\StPart}{ShutdownState}}
\index{Программный интерфейс ПЛК!Управление УЧПУ!Перечисление ShutdownState}
\label{sec:ShutdownState}

\begin{fHeader}
    Тип данных:            & \RightHandText{Перечисление ShutdownState}\\
    Файл объявления:             & \RightHandText{include/cnc/cnc.h} \\
\end{fHeader}

Перечисление определяет идентификаторы состояний автомата выключения УЧПУ и станка.

\begin{MyTableTwoColAllCntr}{Перечисление ShutdownState}{tbl:ShutdownState}{|m{0.38\linewidth}|m{0.57\linewidth}|}{Идентификатор}{Описание}
\hline shutdownWaitCommand &  Ожидание команды выключения \\
\hline shutdownWaitAck &  Ожидание подтверждения команды выключения \\
\end{MyTableTwoColAllCntr}
% *******end subsection***************

% *******begin subsection***************
\subsubsection{\DbgSecSt{\StPart}{ChannelInfo}}
\index{Программный интерфейс ПЛК!Управление УЧПУ!Структура ChannelInfo}
\label{sec:ChannelInfo}

\begin{fHeader}
    Тип данных:            & \RightHandText{Структура ChannelInfo}\\
    Файл объявления:             & \RightHandText{include/cnc/cnc.h} \\
\end{fHeader}

Структура определяет данные канала управления.

\begin{MyTableThreeColAllCntr}{Структура ChannelInfo}{tbl:ChannelInfo}{|m{0.33\linewidth}|m{0.22\linewidth}|m{0.45\linewidth}|}{Элемент}{Тип}{Описание}
\hline canLoad & \centering{Битовое поле:1} &  Разрешение загрузки УП  \\
\hline starting & \centering{Битовое поле:1} &  Подготовка к выполнению УП \\
\hline running & \centering{Битовое поле:1} & Выполнение УП \\
\hline holding & \centering{Битовое поле:1} & УП в процессе останова или возобновления  \\
\hline stopped & \centering{Битовое поле:1} & УП не выполняется \\
\hline waitingBlock & \centering{Битовое поле:1} & Запрос поиска кадра  \\
\hline seekingBlock & \centering{Битовое поле:1} & Выполнение поиска кадра \\
\hline virtualStart & \centering{Битовое поле:1} & Подготовка к выполнению УП в виртуальном режиме \\
\hline virtualRun & \centering{Битовое поле:1} & Выполнение УП в виртуальном режиме \\
\hline canLoadMDI & \centering{Битовое поле:1} & Разрешение загрузки УП в режиме преднабора \\
\hline startingMDI & \centering{Битовое поле:1} & Подготовка к выполнению УП в режиме преднабора \\
\hline runningMDI & \centering{Битовое поле:1} &  Выполнение УП в режиме преднабора \\
\hline holdingMDI & \centering{Битовое поле:1} &  УП в режиме преднабора в процессе останова или возобновления \\
\hline waitingMDI & \centering{Битовое поле:2} &  0 ~-- УП загружена для выполнения в режиме преднабора \newline 1 ~-- запрос загрузки УП для выполнения в режиме преднабора \newline 2 ~-- ошибка загрузки УП для выполнения в режиме преднабора \\
\hline mdiReady & \centering{Битовое поле:1} &  УП в режиме преднабора готова к выполнению \\
\hline switchToRepos & \centering{Битовое поле:1} &  Разрешение перехода в режим возврата на контур \\
\hline setActual & \centering{Битовое поле:8} & Младшие 4 бита ~-- команда: \newline 1 ~-- текущая позиция = 0; \newline 2 ~-- текущая позиция = машинная позиция; \newline  3 ~-- текущая позиция = программная позиция \newline   
Старшие 4 бита ~-- область применения: 0 ~-- все оси; другие значения определяются конфигурацией станка \\
\hline res & \centering{Битовое поле:7} &  Резерв \\
\hline Pos[ЧИСЛО\_ОСЕЙ] & \centering{double} &  Программная позиция \\
\hline WorkPos[ЧИСЛО\_ОСЕЙ] & \centering{double} & Программная позиция относительно базового смещения \\
\hline MachPos[ЧИСЛО\_ОСЕЙ] & \centering{double} &  Машинная позиция \\
\hline TargetPos[ЧИСЛО\_ОСЕЙ] & \centering{double} &  Конечная позиция текущего кадра \\
\hline DistToGo[ЧИСЛО\_ОСЕЙ] & \centering{double} &  Остаток пути \\
\hline ActualPos[ЧИСЛО\_ОСЕЙ] & \centering{double} &  Текущая позиция \\
\hline ActualBase[ЧИСЛО\_ОСЕЙ] & \centering{double} &  Базовое смещение текущей позиции \\
\hline state & \centering{\myreftosec{ChannelStatus}} & Состояние канала управления \\
\hline modeState & \centering{\myreftosec{ModeState}} &  Состояние текущего режима УЧПУ \\
\hline runtime & \centering{\myreftosec{ProgramRuntime}} & Данные УП \\
\hline startBlock & \centering{unsigned} & Начальный блок поиска кадра при выполнении УП с произвольного кадра\\
\hline blockMode & \centering{unsigned} &  Режим выполнения УП с произвольного кадра \\
\hline seekCount & \centering{unsigned} &  Номер итерации поиска кадра \\
\end{MyTableThreeColAllCntr}
% *******end subsection***************

% *******begin subsection***************
\subsubsection{\DbgSecSt{\StPart}{CNCDesc}}
\index{Программный интерфейс ПЛК!Управление УЧПУ!Структура CNCDesc}
\label{sec:CNCDesc}

\begin{fHeader}
    Тип данных:            & \RightHandText{Структура CNCDesc}\\
    Файл объявления:             & \RightHandText{include/cnc/cnc.h} \\
\end{fHeader}

Структура определяет данные УЧПУ.

\begin{MyTableThreeColAllCntr}{Структура CNCDesc}{tbl:CNCDesc}{|m{0.33\linewidth}|m{0.22\linewidth}|m{0.45\linewidth}|}{Элемент}{Тип}{Описание}
\hline mode & \centering{\myreftosec{CNCMode}} &  Текущий режим работы УЧПУ  \\
\hline prevMode & \centering{\myreftosec{CNCMode}} & Предыдущий режим работы УЧПУ \\
\hline nextMode & \centering{\myreftosec{CNCMode}} & Следующий режим работы УЧПУ \\
\hline Watchdog & \centering{int} & Счётчик сторожевого таймера \\
\hline HMIFeedback & \centering{int} &  Флаг обратной связи пульта оператора \\
\hline HMIFirstStart & \centering{int} &  Флаг включения пульта оператора (до включения пульта оператора равен 1) \\
\hline hmiTripped & \centering{int} & Флаг срабатывания сторожевого таймера \\
\hline HMIWatchdog & \centering{\myreftosec{Timer}} &  Таймер сторожевого таймера связи с пультом оператора\\
\hline shutdown & \centering{\myreftosec{Timer}} &  Таймер выключения УЧПУ и станка \\
\hline modeAutoStep & \centering{unsigned} & Флаг покадровой отработки УП \\
\hline modeAutoVirtual & \centering{unsigned} & Флаг отработки УП в виртуальном режиме \\
\hline modeAutoSkip & \centering{unsigned} &  Флаг программного пропуска кадров при отработке УП \\
\hline modeAutoOptStop & \centering{unsigned} & Флаг опционального останова при отработке УП  \\
\hline modeAutoRepos & \centering{unsigned} &  Флаг возврата на контур при возобновлении выполнения УП \\
\hline alarmCancel & \centering{unsigned} &  Запрос сброса ошибок \\
\hline modeDryRun & \centering{unsigned} &  Флаг пробной подачи при отработке УП  \\
\hline modeReducedG0 & \centering{unsigned} &  Флаг уменьшенной подачи быстрого хода при отработке УП  \\
\hline nodeNoMovement & \centering{unsigned} &  Флаг отработки УП с блокировкой движения \\
\hline request & \centering{\myreftosec{MTCNCRequests}} & Текущая исполняемая команда УЧПУ  \\
\hline channel[ЧИСЛО\_КАНАЛОВ] & \centering{\myreftosec{ChannelInfo}} &  Данные канала управления \\
\hline notReadyReq & \centering{Битовое поле:1} &  УЧПУ не готово \\
\hline startDisableReq & \centering{Битовое поле:1} &  Запрет запуска УП \\
\hline enablePortablePult & \centering{Битовое поле:1} &  Разрешение работы переносного пульта \\
\hline ShutdownHMI & \centering{int} &  Переменная выключения УЧПУ и станка принимает значения: \newline 0x5A при включении УЧПУ, \newline 0xA5 ~-- при получении команды, выключения, \newline 0x55 ~-- при подтверждении команды выключения\\
\hline ShutdownState & \centering{int} & Состояние автомата выключения УЧПУ и станка \\
\hline commands & \centering{\myreftosec{CommandQueue}} &  Очередь команд \\
\end{MyTableThreeColAllCntr}
% *******end subsection***************

% *******begin subsection***************
\subsubsection{\DbgSecSt{\StPart}{CNCSettings}}
\index{Программный интерфейс ПЛК!Управление УЧПУ!Структура CNCSettings}
\label{sec:CNCSettings}

\begin{fHeader}
    Тип данных:            & \RightHandText{Структура CNCSettings}\\
    Файл объявления:             & \RightHandText{include/cnc/cnc.h} \\
\end{fHeader}

Структура определяет значения подачи для различных режимов.

\begin{MyTableThreeColAllCntr}{Структура CNCSettings}{tbl:CNCSettings}{|m{0.33\linewidth}|m{0.22\linewidth}|m{0.45\linewidth}|}{Элемент}{Тип}{Описание}
\hline Frapid & \centering{double} &  Значение подачи быстрого хода \\
\hline Fdry & \centering{double} & Значение пробной подачи \\
\hline FrapidReduced & \centering{double} & Значение уменьшенной подачи быстрого хода \\
\end{MyTableThreeColAllCntr}
% *******end subsection***************

%-------------------------------------------------------------------
% *******begin subsection***************
\subsection{\DbgSecSt{\StPart}{Функции}}
\begin{comment}
% *******begin subsection***************
\subsubsection{\DbgSecSt{\StPart}{void InitCnc()}}
\index{Программный интерфейс ПЛК!Управление УЧПУ!void InitCnc()}
\label{sec:InitCnc}

\begin{pHeader}
%    Синтаксис:      & \RightHandText{void InitCnc();}\\
    Аргумент(ы):    & \RightHandText{Нет} \\    
%    Возвращаемое значение:       & \RightHandText{Нет} \\ 
    Файл объявления:             & \RightHandText{include/cnc/cnc.h} \\
\end{pHeader}

Функция инициализации УЧПУ. 

Является системной.
% *******end section*****************
\end{comment}
% *******begin subsection***************
\subsubsection{\DbgSecSt{\StPart}{InitCnc}}
\index{Программный интерфейс ПЛК!Управление УЧПУ!Функция InitCnc}
\label{sec:InitCnc}

\begin{pHeader}
    Синтаксис:      & \RightHandText{void InitCnc();}\\
    Аргумент(ы):    & \RightHandText{Нет} \\    
%    Возвращаемое значение:       & \RightHandText{Нет} \\ 
    Файл объявления:             & \RightHandText{include/cnc/cnc.h} \\
\end{pHeader}

Функция инициализации УЧПУ. 

Является системной.
% *******end section*****************
%--------------------------------------------------------
% *******begin subsection***************
\subsubsection{\DbgSecSt{\StPart}{mtIsReady}}
\index{Программный интерфейс ПЛК!Управление УЧПУ!Функция mtIsReady}
\label{sec:mtIsReady}

\begin{pHeader}
    Синтаксис:      & \RightHandText{int mtIsReady();}\\
    Аргумент(ы):    & \RightHandText{Нет} \\   
%    Возвращаемое значение:       & \RightHandText{Целое знаковое число} \\
    Файл объявления:             & \RightHandText{include/cnc/cnc.h} \\      
\end{pHeader}

Функция проверки готовности станка к работе. \killoverfullbefore

Функция возвращает 1, если станок готов, и 0 в противном случае.  

Реализуется пользователем. 
% *******end subsection*****************
%--------------------------------------------------------
% *******begin subsection***************
\subsubsection{\DbgSecSt{\StPart}{cncSetMode}}
\index{Программный интерфейс ПЛК!Управление УЧПУ!Функция cncSetMode}
\label{sec:cncSetMode}

\begin{pHeader}
    Синтаксис:      & \RightHandText{void cncSetMode(CNCMode mode);}\\
    Аргумент(ы):    & \RightHandText{\myreftosec{CNCMode} mode ~-- идентификатор режима работы УЧПУ} \\   
%    Возвращаемое значение:       & \RightHandText{Нет} \\    
    Файл объявления:             & \RightHandText{include/cnc/cnc.h} \\
\end{pHeader}

Функция устанавливает режим работы УЧПУ, принимая в качестве аргумента значение одного из идентификаторов перечисления \myreftosec{CNCMode}. 

Является системной.
% *******end subsection*****************
%--------------------------------------------------------
% *******begin subsection***************
\subsubsection{\DbgSecSt{\StPart}{cncRequest}}
\index{Программный интерфейс ПЛК!Управление УЧПУ!Функция cncRequest}
\label{sec:cncRequest}

\begin{pHeader}
    Синтаксис:      & \RightHandText{void cncRequest (MTCNCRequests request);}\\
    Аргумент(ы):    & \RightHandText{\myreftosec{MTCNCRequests} request ~-- идентификатор команды управления станком} \\
%    Возвращаемое значение:       & \RightHandText{Нет} \\    
    Файл объявления:             & \RightHandText{include/cnc/cnc.h} \\
\end{pHeader}

Функция посылает команду УЧПУ, принимая в качестве аргумента значение одного из идентификаторов перечисления \myreftosec{MTCNCRequests}. 

Является системной.
% *******end subsection*****************

\begin{comment}
%--------------------------------------------------------
% *******begin subsection***************
\subsubsection{\DbgSecSt{\StPart}{void cncCustomRequest (MTCNCRequests request)}}
\index{Программный интерфейс ПЛК!Управление УЧПУ!void cncCustomRequest (MTCNCRequests request)}
\label{sec:cncCustomRequest}

\begin{pHeader}
%    Синтаксис:      & \RightHandText{void cncSetMode(CNCMode mode);}\\
    Аргумент(ы):    & \RightHandText{Идентификатор перечисления \myreftosec{MTCNCRequests}} \\
%    Возвращаемое значение:       & \RightHandText{Нет} \\    
    Файл объявления:             & \RightHandText{include/cnc/cnc.h} \\
\end{pHeader}

Функция посылает команду УЧПУ, принимая в качестве аргумента значение одного из идентификаторов перечисления \myreftosec{MTCNCRequests}. 
% *******end subsection*****************
\end{comment}
%--------------------------------------------------------
% *******begin subsection***************
\subsubsection{\DbgSecSt{\StPart}{cncChangeMode}}
\index{Программный интерфейс ПЛК!Управление УЧПУ!Функция cncChangeMode}
\label{sec:cncChangeMode}

\begin{pHeader}
    Синтаксис:      & \RightHandText{void cncChangeMode (int newMode);}\\
    Аргумент(ы):    & \RightHandText{int newMode ~-- идентификатор режима работы УЧПУ} \\
%    Возвращаемое значение:       & \RightHandText{Нет} \\    
    Файл объявления:             & \RightHandText{include/cnc/cnc.h} \\
\end{pHeader}

Функция выполняет запрос изменения режима работы УЧПУ. 

Является системной.
% *******end subsection*****************
%--------------------------------------------------------
% *******begin subsection***************
\subsubsection{\DbgSecSt{\StPart}{channelUpdate}}
\index{Программный интерфейс ПЛК!Управление УЧПУ!Функция channelUpdate}
\label{sec:channelUpdate}

\begin{pHeader}
    Синтаксис:      & \RightHandText{void channelUpdate (int channel);}\\
    Аргумент(ы):    & \RightHandText{int channel ~-- номер канала} \\
%    Возвращаемое значение:       & \RightHandText{Нет} \\    
    Файл объявления:             & \RightHandText{include/cnc/cnc.h} \\
\end{pHeader}

Функция обновляет данные канала, номер которого задаётся в качестве аргумента. 

Является системной.
% *******end subsection*****************
%--------------------------------------------------------
% *******begin subsection***************
\subsubsection{\DbgSecSt{\StPart}{cncModeManual}}
\index{Программный интерфейс ПЛК!Управление УЧПУ!Функция cncModeManual}
\label{sec:cncModeManual}

\begin{pHeader}
    Синтаксис:      & \RightHandText{void cncModeManual();}\\
    Аргумент(ы):    & \RightHandText{Нет} \\
%    Возвращаемое значение:       & \RightHandText{Нет} \\    
    Файл объявления:             & \RightHandText{include/cnc/cnc.h} \\
\end{pHeader}

Функция обработки команд в ручном режиме работы УЧПУ. 

Является системной.
% *******end subsection*****************
%--------------------------------------------------------
% *******begin subsection***************
\subsubsection{\DbgSecSt{\StPart}{cncModeHome}}
\index{Программный интерфейс ПЛК!Управление УЧПУ!Функция cncModeHome}
\label{sec:cncModeHome}

\begin{pHeader}
    Синтаксис:      & \RightHandText{void cncModeHome();}\\
    Аргумент(ы):    & \RightHandText{Нет} \\
%    Возвращаемое значение:       & \RightHandText{Нет} \\    
    Файл объявления:             & \RightHandText{include/cnc/cnc.h} \\
\end{pHeader}

Функция обработки команд в режиме выезда в нулевую точку УЧПУ. 

Является системной.
% *******end subsection*****************

%--------------------------------------------------------
% *******begin subsection***************
\subsubsection{\DbgSecSt{\StPart}{cncModeHandwheel}}
\index{Программный интерфейс ПЛК!Управление УЧПУ!Функция cncModeHandwheel}
\label{sec:cncModeHandwheel}

\begin{pHeader}
    Синтаксис:      & \RightHandText{void cncModeHandwheel();}\\
    Аргумент(ы):    & \RightHandText{Нет} \\
%    Возвращаемое значение:       & \RightHandText{Нет} \\    
    Файл объявления:             & \RightHandText{include/cnc/cnc.h} \\
\end{pHeader}

Функция обработки команд в режиме дискретных перемещений УЧПУ. 

Является системной.
% *******end subsection*****************
%--------------------------------------------------------
% *******begin subsection***************
\subsubsection{\DbgSecSt{\StPart}{cncModeAuto}}
\index{Программный интерфейс ПЛК!Управление УЧПУ!Функция cncModeAuto}
\label{sec:cncModeAuto}

\begin{pHeader}
    Синтаксис:      & \RightHandText{void cncModeAuto();}\\
    Аргумент(ы):    & \RightHandText{Нет} \\
%    Возвращаемое значение:       & \RightHandText{Нет} \\    
    Файл объявления:             & \RightHandText{include/cnc/cnc.h} \\
\end{pHeader}

Функция обработки команд в автоматическом режиме работы УЧПУ. 

Является системной.
% *******end subsection*****************
%--------------------------------------------------------
% *******begin subsection***************
\subsubsection{\DbgSecSt{\StPart}{cncModeMDI}}
\index{Программный интерфейс ПЛК!Управление УЧПУ!Функция cncModeMDI}
\label{sec:cncModeMDI}

\begin{pHeader}
    Синтаксис:      & \RightHandText{void cncModeMDI();}\\
    Аргумент(ы):    & \RightHandText{Нет} \\
%    Возвращаемое значение:       & \RightHandText{Нет} \\    
    Файл объявления:             & \RightHandText{include/cnc/cnc.h} \\
\end{pHeader}

Функция обработки команд в режиме преднабора УЧПУ. 

Является системной.
% *******end subsection*****************
\clearpage
%--------------------------------------------------------
% *******begin subsection***************
\subsubsection{\DbgSecSt{\StPart}{cncModeRepos}}
\index{Программный интерфейс ПЛК!Управление УЧПУ!Функция cncModeRepos}
\label{sec:cncModeRepos}

\begin{pHeader}
    Синтаксис:      & \RightHandText{void cncModeRepos();}\\
    Аргумент(ы):    & \RightHandText{Нет} \\
%    Возвращаемое значение:       & \RightHandText{Нет} \\    
    Файл объявления:             & \RightHandText{include/cnc/cnc.h} \\
\end{pHeader}

Функция обработки команд в режиме возврата на контур УЧПУ. 

Является системной.
% *******end subsection*****************
%--------------------------------------------------------
% *******begin subsection***************
\subsubsection{\DbgSecSt{\StPart}{cncManualEnter}}
\index{Программный интерфейс ПЛК!Управление УЧПУ!Функция cncManualEnter}
\label{sec:cncManualEnter}

\begin{pHeader}
    Синтаксис:      & \RightHandText{void cncManualEnter());}\\
    Аргумент(ы):    & \RightHandText{Нет} \\
%    Возвращаемое значение:       & \RightHandText{Нет} \\    
    Файл объявления:             & \RightHandText{include/cnc/cnc.h} \\
\end{pHeader}

Функция вызывается при установке ручного режима работы УЧПУ. В ней должны определяться действия, выполняемые при входе в данный режим. \killoverfullbefore

Реализуется пользователем. 
% *******end subsection*****************
%--------------------------------------------------------
% *******begin subsection***************
\subsubsection{\DbgSecSt{\StPart}{cncManualLeave}}
\index{Программный интерфейс ПЛК!Управление УЧПУ!Функция cncManualLeave}
\label{sec:cncManualLeave}

\begin{pHeader}
    Синтаксис:      & \RightHandText{int cncManualLeave (CNCMode newMode);}\\
    Аргумент(ы):    & \RightHandText{\myreftosec{CNCMode} newMode ~-- идентификатор режима работы УЧПУ} \\ 
    %    Возвращаемое значение:       & \RightHandText{Целое знаковое число} \\    
    Файл объявления:             & \RightHandText{include/cnc/cnc.h} \\
\end{pHeader}

Функция вызывается при выходе из ручного режима работы УЧПУ. В ней должны определяться действия, выполняемые при выходе из данного режима, а также осуществляться проверка возможности установки нового режима работы УЧПУ, который задаётся аргументом ~-- значением одного из идентификаторов перечисления \myreftosec{CNCMode}.\killoverfullbefore

 Возвращаемое значение должно быть отлично от 0 для разрешения нового режима работы. \killoverfullbefore

Реализуется пользователем.  
% *******end subsection*****************
%--------------------------------------------------------
% *******begin subsection***************
\subsubsection{\DbgSecSt{\StPart}{cncHwlEnter}}
\index{Программный интерфейс ПЛК!Управление УЧПУ!Функция cncHwlEnter}
\label{sec:cncHwlEnter}

\begin{pHeader}
    Синтаксис:      & \RightHandText{void cncHwlEnter();}\\
    Аргумент(ы):    & \RightHandText{Нет} \\
%    Возвращаемое значение:       & \RightHandText{Нет} \\    
    Файл объявления:             & \RightHandText{include/cnc/cnc.h} \\
\end{pHeader}

Функция вызывается при установке режима дискретных перемещений УЧПУ. В ней должны определяться действия, выполняемые при входе в данный режим. \killoverfullbefore

Реализуется пользователем. 
% *******end subsection*****************
%--------------------------------------------------------
% *******begin subsection***************
\subsubsection{\DbgSecSt{\StPart}{cncHwlLeave}}
\index{Программный интерфейс ПЛК!Управление УЧПУ!Функция cncHwlLeave}
\label{sec:cncHwlLeave}

\begin{pHeader}
    Синтаксис:      & \RightHandText{int cncHwlLeave (CNCMode newMode);}\\
    Аргумент(ы):    & \RightHandText{\myreftosec{CNCMode} newMode ~-- идентификатор режима работы УЧПУ} \\ 
%    Возвращаемое значение:       & \RightHandText{Целое знаковое число} \\    
    Файл объявления:             & \RightHandText{include/cnc/cnc.h} \\
\end{pHeader}

Функция вызывается при выходе из режима дискретных перемещений УЧПУ. В ней должны определяться действия, выполняемые при выходе из данного режима, а также осуществляться проверка возможности установки нового режима работы УЧПУ, который задаётся аргументом ~-- значением одного из идентификаторов перечисления \myreftosec{CNCMode}.\killoverfullbefore

 Возвращаемое значение должно быть отлично от 0 для разрешения нового режима работы. \killoverfullbefore

Реализуется пользователем. 
% *******end subsection*****************
%--------------------------------------------------------
% *******begin subsection***************
\subsubsection{\DbgSecSt{\StPart}{cncHomeEnter}}
\index{Программный интерфейс ПЛК!Управление УЧПУ!Функция cncHomeEnter}
\label{sec:cncHomeEnter}

\begin{pHeader}
    Синтаксис:      & \RightHandText{void cncHomeEnter();}\\
    Аргумент(ы):    & \RightHandText{Нет} \\
%    Возвращаемое значение:       & \RightHandText{Нет} \\    
    Файл объявления:             & \RightHandText{include/cnc/cnc.h} \\
\end{pHeader}

Функция вызывается при установке режима выезда в нулевую точку УЧПУ. В ней должны определяться действия, выполняемые при входе в данный режим. \killoverfullbefore

Реализуется пользователем. 
% *******end subsection*****************
%--------------------------------------------------------
% *******begin subsection***************
\subsubsection{\DbgSecSt{\StPart}{cncHomeLeave}}
\index{Программный интерфейс ПЛК!Управление УЧПУ!Функция cncHomeLeave}
\label{sec:cncHomeLeave}

\begin{pHeader}
    Синтаксис:      & \RightHandText{int cncHomeLeave (CNCMode newMode);}\\
    Аргумент(ы):    & \RightHandText{\myreftosec{CNCMode} newMode ~-- идентификатор режима работы УЧПУ} \\ 
%    Возвращаемое значение:       & \RightHandText{Целое знаковое число} \\    
    Файл объявления:             & \RightHandText{include/cnc/cnc.h} \\
\end{pHeader}

Функция вызывается при выходе из режима выезда в нулевую точку УЧПУ. В ней должны определяться действия, выполняемые при выходе из данного режима, а также осуществляться проверка возможности установки нового режима работы УЧПУ, который задаётся аргументом ~-- значением одного из идентификаторов перечисления \myreftosec{CNCMode}.\killoverfullbefore

 Возвращаемое значение должно быть отлично от 0 для разрешения нового режима работы. \killoverfullbefore

Реализуется пользователем.
% *******end subsection*****************
%--------------------------------------------------------
% *******begin subsection***************
\subsubsection{\DbgSecSt{\StPart}{cncAutoEnter}}
\index{Программный интерфейс ПЛК!Управление УЧПУ!Функция cncAutoEnter}
\label{sec:cncAutoEnter}

\begin{pHeader}
    Синтаксис:      & \RightHandText{void cncAutoEnter();}\\
    Аргумент(ы):    & \RightHandText{Нет} \\
%    Возвращаемое значение:       & \RightHandText{Нет} \\    
    Файл объявления:             & \RightHandText{include/cnc/cnc.h} \\
\end{pHeader}

Функция вызывается при установке автоматического режима УЧПУ. В ней должны определяться действия, выполняемые при входе в данный режим. \killoverfullbefore

Реализуется пользователем. 
% *******end subsection*****************
%--------------------------------------------------------
% *******begin subsection***************
\subsubsection{\DbgSecSt{\StPart}{cncAutoLeave}}
\index{Программный интерфейс ПЛК!Управление УЧПУ!Функция cncAutoLeave}
\label{sec:cncAutoLeave}

\begin{pHeader}
    Синтаксис:      & \RightHandText{int cncAutoLeave (CNCMode newMode);}\\
    Аргумент(ы):    & \RightHandText{\myreftosec{CNCMode} newMode ~-- идентификатор режима работы УЧПУ} \\ 
%    Возвращаемое значение:       & \RightHandText{Целое знаковое число} \\    
    Файл объявления:             & \RightHandText{include/cnc/cnc.h} \\
\end{pHeader}

Функция вызывается при выходе из автоматического режима УЧПУ. В ней должны определяться действия, выполняемые при выходе из данного режима, а также осуществляться проверка возможности установки нового режима работы УЧПУ, который задаётся аргументом ~-- значением одного из идентификаторов перечисления \myreftosec{CNCMode}. \killoverfullbefore

Возвращаемое значение должно быть отлично от 0 для разрешения нового режима работы. \killoverfullbefore

Реализуется пользователем.
% *******end subsection*****************
%--------------------------------------------------------
% *******begin subsection***************
\subsubsection{\DbgSecSt{\StPart}{cncMDIEnter}}
\index{Программный интерфейс ПЛК!Управление УЧПУ!Функция cncMDIEnter}
\label{sec:cncMDIEnter}

\begin{pHeader}
    Синтаксис:      & \RightHandText{void cncMDIEnter();}\\
    Аргумент(ы):    & \RightHandText{Нет} \\
%    Возвращаемое значение:       & \RightHandText{Нет} \\    
    Файл объявления:             & \RightHandText{include/cnc/cnc.h} \\
\end{pHeader}

Функция вызывается при установке режима преднабора УЧПУ. В ней должны определяться действия, выполняемые при входе в данный режим. \killoverfullbefore

Реализуется пользователем. 
% *******end subsection*****************

%--------------------------------------------------------
% *******begin subsection***************
\subsubsection{\DbgSecSt{\StPart}{cncMDILeave}}
\index{Программный интерфейс ПЛК!Управление УЧПУ!Функция cncMDILeave}
\label{sec:cncMDILeave}

\begin{pHeader}
    Синтаксис:      & \RightHandText{int cncMDILeave (CNCMode newMode);}\\
    Аргумент(ы):    & \RightHandText{\myreftosec{CNCMode} newMode ~-- идентификатор режима работы УЧПУ} \\ 
%    Возвращаемое значение:       & \RightHandText{Целое знаковое число} \\    
    Файл объявления:             & \RightHandText{include/cnc/cnc.h} \\
\end{pHeader}

Функция вызывается при выходе из режима преднабора УЧПУ. В ней должны определяться действия, выполняемые при выходе из данного режима, а также осуществляться проверка возможности установки нового режима работы УЧПУ, который задаётся аргументом ~-- значением одного из идентификаторов перечисления \myreftosec{CNCMode}.\killoverfullbefore

 Возвращаемое значение должно быть отлично от 0 для разрешения нового режима работы. \killoverfullbefore

Реализуется пользователем.
% *******end subsection*****************
%--------------------------------------------------------
% *******begin subsection***************
\subsubsection{\DbgSecSt{\StPart}{cncReposEnter}}
\index{Программный интерфейс ПЛК!Управление УЧПУ!Функция cncReposEnter}
\label{sec:cncReposEnter}

\begin{pHeader}
    Синтаксис:      & \RightHandText{void cncReposEnter();}\\
    Аргумент(ы):    & \RightHandText{Нет} \\
%    Возвращаемое значение:       & \RightHandText{Нет} \\    
    Файл объявления:             & \RightHandText{include/cnc/cnc.h} \\
\end{pHeader}

Функция вызывается при установке режима возврата на контур УЧПУ. В ней должны определяться действия, выполняемые при входе в данный режим. \killoverfullbefore

Реализуется пользователем. 
% *******end subsection*****************
%--------------------------------------------------------
% *******begin subsection***************
\subsubsection{\DbgSecSt{\StPart}{cncReposLeave}}
\index{Программный интерфейс ПЛК!Управление УЧПУ!Функция cncReposLeave}
\label{sec:cncReposLeave}

\begin{pHeader}
    Синтаксис:      & \RightHandText{int cncReposLeave (CNCMode newMode);}\\
    Аргумент(ы):    & \RightHandText{\myreftosec{CNCMode} newMode ~-- идентификатор режима работы УЧПУ} \\ 
%    Возвращаемое значение:       & \RightHandText{Целое знаковое число} \\    
    Файл объявления:             & \RightHandText{include/cnc/cnc.h} \\
\end{pHeader}

Функция вызывается при выходе из режима возврата на контур УЧПУ. В ней должны определяться действия, выполняемые при выходе из данного режима, а также осуществляться проверка возможности установки нового режима работы УЧПУ, который задаётся аргументом ~-- значением одного из идентификаторов перечисления \myreftosec{CNCMode}. \killoverfullbefore

Возвращаемое значение должно быть отлично от 0 для разрешения нового режима работы. \killoverfullbefore

Реализуется пользователем.
% *******end subsection*****************
%--------------------------------------------------------
% *******begin subsection***************
\subsubsection{\DbgSecSt{\StPart}{controlPowerCNC}}
\index{Программный интерфейс ПЛК!Управление УЧПУ!Функция controlPowerCNC}
\label{sec: controlPowerCNC}

\begin{pHeader}
    Синтаксис:      & \RightHandText{void controlPowerCNC (int request);}\\
    Аргумент(ы):    & \RightHandText{int request ~-- идентификатор команды управления станком} \\
%    Возвращаемое значение:       & \RightHandText{Нет} \\    
    Файл объявления:             & \RightHandText{include/cnc/cnc.h} \\
\end{pHeader}

Функция обработки запроса выключения УЧПУ и станка.  Аргументом функции является  значение одного из идентификаторов перечисления \myreftosec{MTCNCRequests}.

Является системной.
% *******end subsection*****************
%--------------------------------------------------------
% *******begin subsection***************
\subsubsection{\DbgSecSt{\StPart}{cncAutoOnProgramExit}}
\index{Программный интерфейс ПЛК!Управление УЧПУ!Функция cncAutoOnProgramExit}
\label{sec: cncAutoOnProgramExit}

\begin{pHeader}
    Синтаксис:      & \RightHandText{void cncAutoOnProgramExit (int channel);}\\
    Аргумент(ы):    & \RightHandText{int channel ~-- номер канала} \\
%    Возвращаемое значение:       & \RightHandText{Нет} \\    
    Файл объявления:             & \RightHandText{include/cnc/cnc.h} \\
\end{pHeader}

Функция вызывается при выходе из автоматического режима УЧПУ. В ней должны определяться действия, выполняемые при выходе из данного режима для канала, номер которого является аргументом функции. \killoverfullbefore

Реализуется пользователем.
% *******end subsection*****************
%--------------------------------------------------------
% *******begin subsection***************
\subsubsection{\DbgSecSt{\StPart}{cncCustomRequestManual}}
\index{Программный интерфейс ПЛК!Управление УЧПУ!Функция cncCustomRequestManual}
\label{sec: cncCustomRequestManual}

\begin{pHeader}
    Синтаксис:      & \RightHandText{void cncCustomRequestManual (int request);} \\
   Аргумент(ы):    & \RightHandText{int request ~-- идентификатор команды пользователя} \\
%    Возвращаемое значение:       & \RightHandText{Нет} \\    
    Файл объявления:             & \RightHandText{include/cnc/cnc.h} \\
\end{pHeader}

Функция обработки пользовательских команд в ручном режиме УЧПУ. Аргументом функции является команда пользователя.

Реализуется пользователем.
% *******end subsection*****************
%--------------------------------------------------------
% *******begin subsection***************
\subsubsection{\DbgSecSt{\StPart}{cncCustomRequestHome}}
\index{Программный интерфейс ПЛК!Управление УЧПУ!Функция cncCustomRequestHome}
\label{sec: cncCustomRequestHome}

\begin{pHeader}
    Синтаксис:      & \RightHandText{void cncCustomRequestHome (int request);}\\
   Аргумент(ы):    & \RightHandText{int request ~-- идентификатор команды пользователя} \\
%    Возвращаемое значение:       & \RightHandText{Нет} \\    
    Файл объявления:             & \RightHandText{include/cnc/cnc.h} \\
\end{pHeader}

Функция обработки пользовательских команд в режиме выезда в нулевую точку УЧПУ.  Аргументом функции является команда пользователя.

Реализуется пользователем.
% *******end subsection*****************
%--------------------------------------------------------
% *******begin subsection***************
\subsubsection{\DbgSecSt{\StPart}{cncCustomRequestAuto}}
\index{Программный интерфейс ПЛК!Управление УЧПУ!Функция cncCustomRequestAuto}
\label{sec: cncCustomRequestAuto}

\begin{pHeader}
    Синтаксис:      & \RightHandText{void cncCustomRequestAuto (int request);}\\
   Аргумент(ы):    & \RightHandText{int request ~-- идентификатор команды пользователя} \\
%    Возвращаемое значение:       & \RightHandText{Нет} \\    
    Файл объявления:             & \RightHandText{include/cnc/cnc.h} \\
\end{pHeader}

Функция обработки пользовательских команд в автоматическом режиме УЧПУ.  Аргументом функции является команда пользователя.

Реализуется пользователем.
% *******end subsection*****************
%--------------------------------------------------------
% *******begin subsection***************
\subsubsection{\DbgSecSt{\StPart}{cncCustomRequestMDI}}
\index{Программный интерфейс ПЛК!Управление УЧПУ!Функция cncCustomRequestMDI}
\label{sec: cncCustomRequestMDI}

\begin{pHeader}
    Синтаксис:      & \RightHandText{void cncCustomRequestMDI (int request);}\\
   Аргумент(ы):    & \RightHandText{int request ~-- идентификатор команды пользователя} \\
%    Возвращаемое значение:       & \RightHandText{Нет} \\    
    Файл объявления:             & \RightHandText{include/cnc/cnc.h} \\
\end{pHeader}

Функция обработки пользовательских команд в режиме преднабора УЧПУ.  Аргументом функции является команда пользователя.

Реализуется пользователем.
% *******end subsection*****************
%--------------------------------------------------------
% *******begin subsection***************
\subsubsection{\DbgSecSt{\StPart}{cncCustomRequestHwl}}
\index{Программный интерфейс ПЛК!Управление УЧПУ!Функции cncCustomRequestHwl}
\label{sec: cncCustomRequestHwl}

\begin{pHeader}
   Синтаксис:      & \RightHandText{void cncCustomRequestHwl (int request);}\\
    Аргумент(ы):    & \RightHandText{Целое знаковое число} \\   Аргумент(ы):    & \RightHandText{int request ~-- идентификатор команды пользователя} \\%    Возвращаемое значение:       & \RightHandText{Нет} \\    
    Файл объявления:             & \RightHandText{include/cnc/cnc.h} \\
\end{pHeader}

Функция обработки пользовательских команд в режиме дискретных перемещений УЧПУ.  Аргументом функции является команда пользователя.

Реализуется пользователем.
% *******end subsection*****************
%--------------------------------------------------------
% *******begin subsection***************
\subsubsection{\DbgSecSt{\StPart}{cncCustomRequestRepos}}
\index{Программный интерфейс ПЛК!Управление УЧПУ!Функция cncCustomRequestRepos}
\label{sec: cncCustomRequestRepos}

\begin{pHeader}
    Синтаксис:      & \RightHandText{void cncCustomRequestRepos (int request);}\\
   Аргумент(ы):    & \RightHandText{int request ~-- идентификатор команды пользователя} \\
%    Возвращаемое значение:       & \RightHandText{Нет} \\    
    Файл объявления:             & \RightHandText{include/cnc/cnc.h} \\
\end{pHeader}

Функция обработки пользовательских команд в режиме возврата на контур УЧПУ.  Аргументом функции является команда пользователя. 

Реализуется пользователем.
% *******end subsection*****************
%--------------------------------------------------------
% *******begin subsection***************
\subsubsection{\DbgSecSt{\StPart}{cncManualCanChangeOverride}}
\index{Программный интерфейс ПЛК!Управление УЧПУ!Функция cncManualCanChangeOverride}
\label{sec: cncManualCanChangeOverride}

\begin{pHeader}
    Синтаксис:      & \RightHandText{int cncManualCanChangeOverride();}\\
    Аргумент(ы):    & \RightHandText{Нет} \\
%    Возвращаемое значение:       & \RightHandText{Целое знаковое число} \\
    Файл объявления:             & \RightHandText{include/cnc/cnc.h} \\
\end{pHeader}

Функция выполняет запрос на разрешение применения коррекции подачи. \killoverfullbefore

Возвращает 1, если коррекция разрешена, и 0 в противном случае.

Реализуется пользователем.
% *******end subsection*****************
%--------------------------------------------------------
% *******begin subsection***************
\subsubsection{\DbgSecSt{\StPart}{shutdown}}
\index{Программный интерфейс ПЛК!Управление УЧПУ!Функция shutdown}
\label{sec:shutdown}

\begin{pHeader}
    Синтаксис:      & \RightHandText{void shutdown();}\\
   Аргумент(ы):    & \RightHandText {нет} \\  
%    Возвращаемое значение:       & \RightHandText{Нет} \\
    Файл объявления:             & \RightHandText{sys/sys.h} \\      
\end{pHeader}

Функция вызывает выключение УЧПУ. \killoverfullbefore

Является системной.
% *******end subsection*****************
%--------------------------------------------------------
% *******begin subsection***************
\subsubsection{\DbgSecSt{\StPart}{reset}}
\index{Программный интерфейс ПЛК!Управление УЧПУ!Функция reset}
\label{sec:reset}

\begin{pHeader}
    Синтаксис:      & \RightHandText{void reset();}\\
   Аргумент(ы):    & \RightHandText {нет} \\  
%    Возвращаемое значение:       & \RightHandText{Нет} \\
    Файл объявления:             & \RightHandText{sys/sys.h} \\      
\end{pHeader}

Функция вызывает перезагрузку УЧПУ, которая эквивалентна выключению и последующему включению питания. \killoverfullbefore

Является системной.
% *******end subsection*****************
%--------------------------------------------------------
% *******begin subsection***************
\subsubsection{\DbgSecSt{\StPart}{reinitialize}}
\index{Программный интерфейс ПЛК!Управление УЧПУ!Функция reinitialize}
\label{sec:reinitialize}

\begin{pHeader}
    Синтаксис:      & \RightHandText{void reinitialize();}\\
   Аргумент(ы):    & \RightHandText {нет} \\  
%    Возвращаемое значение:       & \RightHandText{Нет} \\
    Файл объявления:             & \RightHandText{sys/sys.h} \\      
\end{pHeader}

Функция вызывает сброс параметров УЧПУ до заводских. \killoverfullbefore

Является системной.
% *******end subsection*****************

% *******end section*****************

%--------------------------------------------------------
                 % API: управление УЧПУ
	%--------------------------------------------------------
% *******begin section***************
\section{\DbgSecSt{\StPart}{Управление станком}}
%--------------------------------------------------------
\subsection{\DbgSecSt{\StPart}{Типы данных}}

% *******begin subsection***************
\subsubsection{\DbgSecSt{\StPart}{MTState}}
\index{Программный интерфейс ПЛК!Управление станком!Перечисление MTState}
\label{sec:MTState}

\begin{fHeader}
    Тип данных:            & \RightHandText{Перечисление MTState}\\
    Файл объявления:             & \RightHandText{include/cnc/mt.h} \\
\end{fHeader}

Перечисление определяет идентификаторы состояний станка.

\begin{MyTableTwoColAllCntr}{Перечисление MTState}{tbl:MTState}{|m{0.38\linewidth}|m{0.57\linewidth}|}{Идентификатор}{Описание}
\hline mtNotReady &  Станок выключен  \\
\hline mtStartOn &  Начало включения \\
\hline mtDriveOn &  Включение приводов \\
\hline mtWaitDriveOn &  Ожидание включения приводов \\
\hline mtOthersMotorOn & Включение вспомогательных моторов \\
\hline mtReady & Станок включен \\
\hline mtStartOff & Начало выключения \\
\hline mtOthersMotorOff & Выключение вспомогательных моторов \\
\hline mtAxisStop & Останов осей и шпинделя \\
\hline mtAxisWaitStop & Ожидание останова осей и шпинделя \\
\hline mtDriveOff & Выключение приводов \\
\hline mtAbort & Аварийное торможение \\
\hline mtPhaseRef & Фазировка \\
\hline mtWaitPhaseRef  & Ожидание фазировки \\
\hline mtWaitOff & Ожидание выключения питания станка \\
\hline mtWaitAbsPos & Ожидание данных от абсолютного ДОС \\
\end{MyTableTwoColAllCntr}
% *******end subsection***************
%--------------------------------------------------------
% *******begin subsection***************
\subsubsection{\DbgSecSt{\StPart}{MTCNCRequests}}
\index{Программный интерфейс ПЛК!Управление станком!Перечисление MTCNCRequests}
\label{sec:MTCNCRequests}

\begin{fHeader}
    Тип данных:            & \RightHandText{Перечисление MTCNCRequests}\\
    Файл объявления:             & \RightHandText{include/cnc/mt.h} \\
\end{fHeader}

Перечисление определяет идентификаторы команд управления станком. 

Начальный номер блока пользовательских команд  (mtcncCommandStart) равен 1000, конечный (mtcncCommandEnd) ~-- 1999. \killoverfullbefore

Начальный номер блока пользовательских команд движения (mtcncMoveCommandStart) равен 2000, конечный (mtcncMoveCommandEnd) ~-- 2999. \killoverfullbefore

\begin{MyTableTwoColAllCntr}{Перечисление MTCNCRequests}{tbl:MTCNCRequests}{|m{0.38\linewidth}|m{0.57\linewidth}|}{Идентификатор}{Описание}
\hline mtcncNone &  Нет команды  \\
\hline mtcncPowerOn  &  Включение станка \\
\hline mtcncPowerOff  &  Выключение станка \\
\hline mtcncEmergencyStop  &  Аварийный останов  \\
\hline mtcncReset  &  Сброс в начальное состояние \\
\hline mtcncStart  & Запуск операции в текущем режиме \\
\hline mtcncStop &  Останов операции в текущем режиме  \\
\hline mtcncCncOff &  Выключение УЧПУ \\

\hline mtcncActivateManual  & Включение ручного режима  \\
\hline mtcncActivateHandwheel  & Включение режима дискретных перемещений \\
\hline mtcncActivateRef &  Включение режима выезда в нулевую точку \\
\hline mtcncActivateMDI &  Включение режима преднабора \\
\hline mtcncActivateAuto & Включение  автоматического режима \\
\hline mtcncActivateRepos &  Включение режима возврата на контур \\

\hline mtcncToggleStep &  Покадровая отработка УП \\
\hline mtcncToggleRepos &  Возврат на контур \\
\hline mtcncToggleVirtual &  Отработка УП в виртуальном режиме \\
\hline mtcncToggleOptionalSkip & Отработка УП с программным пропуском кадров \\
\hline mtcncToggleOptionalStop & Отработка УП с опциональным остановом \\
\hline mtcncSelectSpeed1 & Выбор первой скорости/дискреты \newline безразмерных/дискретных
перемещений  \\
\hline mtcncSelectSpeed2 & Выбор второй скорости/дискреты \newline безразмерных/дискретных
перемещений \\
\hline mtcncSelectSpeed3 & Выбор третьей скорости/дискреты \newline безразмерных/дискретных перемещений  \\
\hline mtcncSelectSpeed4 & Выбор четвёртой скорости/дискреты \newline безразмерных/дискретных перемещений \\
\hline mtcncSelectRapid & Перемещение на скорости быстрого хода \\
\hline mtcncDryRun & Пробная подача   \\
\hline mtcncReducedRapid &  Уменьшенная подача быстрого хода \\
\hline mtcncMoveLock & Отработка УП с блокировкой движения \\
\hline mtcncAlarmCancel & Сброс ошибок \\

\hline mtcncCommandStart & Начальный номер блока пользовательских команд \\
\hline mtcncCommandEnd &  Конечный номер блока пользовательских команд \\

\hline mtcncMoveCommandStart & Начальный номер блока пользовательских команд движения \\
\hline mtcncMoveCommandEnd & Конечный номер блока пользовательских команд движения  \\

\end{MyTableTwoColAllCntr}
% *******end subsection***************
%--------------------------------------------------------
\begin{comment}
% *******begin subsection***************
\subsubsection{\DbgSecSt{\StPart}{UsedPult}}
\index{Программный интерфейс ПЛК!Управление станком!Перечисление UsedPult}
\label{sec:UsedPult}

\begin{fHeader}
    Тип данных:            & \RightHandText{Перечисление UsedPult}\\
    Файл объявления:             & \RightHandText{include/cnc/mt.h} \\
\end{fHeader}

Перечисление определяет идентификаторы терминальных устройств.

\begin{MyTableThreeColAllCntr}{Перечисление UsedPult}{tbl:UsedPult}{|m{0.33\linewidth}|m{0.22\linewidth}|m{0.45\linewidth}|}{Идентификатор}{Значение}{Описание}
\hline opertor & \centering{0} & Пульт оператора  \\
\hline portable & \centering{1} & Переносной пульт \\
\end{MyTableThreeColAllCntr}
% *******end subsection***************
\end{comment}
%--------------------------------------------------------
% *******begin subsection***************
\subsubsection{\DbgSecSt{\StPart}{MTDesc}}
\index{Программный интерфейс ПЛК!Управление станком!Структура MTDesc}
\label{sec:MTDesc}

\begin{fHeader}
    Тип данных:            & \RightHandText{Структура MTDesc}\\
    Файл объявления:             & \RightHandText{include/cnc/mt.h} \\
\end{fHeader}

Структура определяет данные станка.

\begin{MyTableThreeColAllCntr}{Структура MTDesc}{tbl:MTDesc}{|m{0.33\linewidth}|m{0.222\linewidth}|m{0.45\linewidth}|}{Элемент}{Тип}{Описание}
\hline State & \centering{int} &  Состояние автомата включения/выключения станка  \\
\hline IN & \centering{\hyperlink{IO_union}{MTInputs}} & Входы плат входов\\
\hline OUT & \centering{\hyperlink{IO_union}{MTOutputs}} & Выходы плат реле\\
\hline PultIn & \centering{\hyperlink{IO_union}{PultInputs}} & Входы пульта оператора \\
\hline PultOut & \centering{\hyperlink{IO_union}{PultOutputs}} & Выходы пульта оператора \\
\hline PortablePultIn & \centering{\hyperlink{IO_union}{PortablePultInputs}} &  Входы переносного пульта \\

\hline timerState & \centering{\myreftosec{Timer}} & Таймер состояния \\
\hline timerReset & \centering{\myreftosec{Timer}} & Таймер сброса \\
\hline timerScan & \centering{\myreftosec{Timer}} & Таймер выполнения операции \\

\hline ncNotReadyReq & \centering{Битовое поле:1} & Запрос готовности системы \\
\hline ncFollowUpReq & \centering{Битовое поле:1} & Запрос восстановления после ошибки  \\
\hline ncStopReq & \centering{Битовое поле:1} & Запрос немедленного останова УП или движения  \\
\hline ncStopAtEndReq & \centering{Битовое поле:1} & Запрос останова в конце текущего кадра\\
%\hline usedPult & \centering{int} &   \\
%\hline corrFTemp & \centering{double} &   \\
\end{MyTableThreeColAllCntr}
% *******end subsection***************
%-------------------------------------------------------------------
% *******begin subsection***************
\subsection{\DbgSecSt{\StPart}{Функции}}

% *******begin subsection***************
\subsubsection{\DbgSecSt{\StPart}{systemPlcActive}}
\index{Программный интерфейс ПЛК!Управление станком!Функция systemPlcActive}
\label{sec:systemPlcActive}

\begin{pHeader}
    Синтаксис:      & \RightHandText{int systemPlcActive();}\\
    Аргумент(ы):    & \RightHandText{Нет} \\    
%    Возвращаемое значение:       & \RightHandText{Целое знаковое число} \\ 
    Файл объявления:             & \RightHandText{include/cnc/mt.h} \\       
\end{pHeader}

Функция возвращает 1, если нет ошибок программ ПЛК, и 0 в противном случае.

Реализуется пользователем.

% *******end section*****************
%-------------------------------------------------------------------
% *******begin subsection***************
\subsubsection{\DbgSecSt{\StPart}{hasEmergencyStopRequest}}
\index{Программный интерфейс ПЛК!Управление станком!Функция hasEmergencyStopRequest}
\label{sec:hasEmergencyStopRequest}

\begin{pHeader}
    Синтаксис:      & \RightHandText{int hasEmergencyStopRequest();}\\
    Аргумент(ы):    & \RightHandText{Нет} \\    
%    Возвращаемое значение:       & \RightHandText{Целое знаковое число} \\ 
    Файл объявления:             & \RightHandText{include/cnc/mt.h} \\       
\end{pHeader}

Функция возвращает 1, если есть запрос аварийного останова, и 0 в противном случае.

Реализуется пользователем.
% *******end section*****************
\clearpage
\begin{comment}
%-------------------------------------------------------------------
% *******begin subsection***************
\subsubsection{\DbgSecSt{\StPart}{int hasEmergencyStopMt()}}
\index{Программный интерфейс ПЛК!Управление станком!Функция int hasEmergencyStopMt()}
\label{sec:hasEmergencyStopMt}

\begin{pHeader}
%    Синтаксис:      & \RightHandText{void InitCnc();}\\
    Аргумент(ы):    & \RightHandText{Нет} \\    
%    Возвращаемое значение:       & \RightHandText{Целое знаковое число} \\ 
    Файл объявления:             & \RightHandText{include/cnc/mt.h} \\       
\end{pHeader}


% *******end section*****************
\end{comment}
%-------------------------------------------------------------------
% *******begin subsection***************
\subsubsection{\DbgSecSt{\StPart}{mtControlRequest}}
\index{Программный интерфейс ПЛК!Управление станком!Функция mtControlRequest}
\label{sec:mtControlRequest}

\begin{pHeader}
    Синтаксис:      & \RightHandText{void mtControlRequest();}\\
    Аргумент(ы):    & \RightHandText{Нет} \\    
%    Возвращаемое значение:       & \RightHandText{Нет} \\ 
    Файл объявления:             & \RightHandText{include/cnc/mt.h} \\
\end{pHeader}

Функция добавляет команды в очередь.

Реализуется пользователем.
% *******end section*****************
%-------------------------------------------------------------------
% *******begin subsection***************
\subsubsection{\DbgSecSt{\StPart}{mtUpdateCNCIndication}}
\index{Программный интерфейс ПЛК!Управление станком!Функция mtUpdateCNCIndication}
\label{sec:mtUpdateCNCIndication}

\begin{pHeader}
    Синтаксис:      & \RightHandText{void mtUpdateCNCIndication();}\\
    Аргумент(ы):    & \RightHandText{Нет} \\    
%    Возвращаемое значение:       & \RightHandText{Нет} \\ 
    Файл объявления:             & \RightHandText{include/cnc/mt.h} \\       
\end{pHeader}

Функция обновляет индикацию пульта оператора.

Реализуется пользователем.
% *******end section*****************
%-------------------------------------------------------------------
                  % API: управление станком
	% *******begin section***************
\section{\DbgSecSt{\StPart}{Обработка ошибок}}
%--------------------------------------------------------
\subsection{\DbgSecSt{\StPart}{Типы данных}}

% *******begin subsection***************
\subsubsection{\DbgSecSt{\StPart}{DriveErrors}}
\index{Программный интерфейс ПЛК!Обработка ошибок!Объединение DriveErrors}
\label{sec:DriveErrors}

\begin{fHeader}
    Тип данных:            & \RightHandText{Объединение DriveErrors}\\
    Файл объявления:             & \RightHandText{include/cnc/errors.h} \\
\end{fHeader}

Объединение определяет ошибки и режим работы сервоусилителя. \killoverfullbefore 

\clearpage

\begin{MyTableThreeColAllCntr}{Объединение DriveErrors}{tbl:DriveErrors}{|m{0.33\linewidth}|m{0.22\linewidth}|m{0.45\linewidth}|}{Элемент}{Тип}{Описание}
\hline struct \{ 
\newline
protocol & \newline \centering{unsigned:1} & \newline Ошибка протокола  \\
\hhline{~} ampNotReady & \centering{unsigned:1} & Нет готовности \\
\hhline{~} ampFault & \centering{unsigned:1} & Сервоусилитель в состоянии ошибки \\
\hhline{~} i2tFault & \centering{unsigned:1} &  Ошибка i2t \\
\hhline{~} crc & \centering{unsigned:1} & Ошибка контрольной суммы \\
\hhline{~} igbtFault & \centering{unsigned:1} & Ошибка IGBT модуля \\
\hhline{~} igbtTempFault & \centering{unsigned:1} & Превышение температуры IGBT модуля \\
\hhline{~} highDCFault & \centering{unsigned:1} & Повышенное напряжение в ЗПТ \\
\hhline{~} lowDCFault & \centering{unsigned:1} & Пониженное напряжение в ЗПТ \\
\hhline{~} linkFault & \centering{unsigned:1} & Ошибка связи \\
\hhline{~} brakeOnLowFault & \centering{unsigned:1} & Сигнал на открытие тормозного транзистора в состоянии L (не в слежении) \\
\hhline{~} brakeOnHighFault & \centering{unsigned:1} & Сигнал на открытие тормозного транзистора в состоянии H (не в слежении) \\
\hhline{~} brakeFault & \centering{unsigned:1} & Недостаточная мощность тормозного резистора \\
\hhline{~} currentOutFault & \centering{unsigned:1} & Измеренный ток в фазе в отсечке \\
\hhline{~} adcFault & \centering{unsigned:1} & Ошибка АЦП \\
\hhline{~} pwmShortFault & \centering{unsigned:1} & Период сигнала ШИМ  меньше 50 мкс \\
\hhline{~} pwmLongFault & \centering{unsigned:1} & Период сигнала ШИМ больше 400 мкс \\
\hhline{~} reserved & \centering{unsigned:8} & Резерв \\
\hhline{~} ampState & \centering{unsigned:2} & Состояние сервоусилителя \\
\hhline{~} errorCode \} & \centering{unsigned:4} & Текущий код ошибки \\
\hline errors & \centering{unsigned} & Переменная, содержащая все битовые поля \\
\end{MyTableThreeColAllCntr}
% *******end subsection***************

Поле \texttt{ampState} является 2-битным и содержит коды состояния сервоусилителя:
\begin{itemize}
\item 0 ~-- не подано высокое напряжение; \killoverfullbefore
\item 1 ~-- подано высокое напряжение; \killoverfullbefore
\item 2 ~-- сервоусилитель в слежении. \killoverfullbefore \BL
\end{itemize} 

Поле \texttt{errorCode} является 4-битным и и содержит текущий код ошибки сервоусилителя:
\begin{itemize}
\item 0 ~-- нет ошибок; \killoverfullbefore
\item 1 ~-- ошибка IGBT модуля; \killoverfullbefore
\item 2 ~-- превышение температуры IGBT модуля; \killoverfullbefore 
\item 3 ~-- повышенное напряжение в ЗПТ;
\item 4 ~-- пониженное напряжение в ЗПТ;
\item 5 ~-- ошибка связи;
\item 6 ~-- сигнал на открытие тормозного транзистора в состоянии L (не в слежении);
\item 7 ~-- сигнал на открытие тормозного транзистора в состоянии H (не в слежении);
\item 8 ~-- недостаточная мощность тормозного резистора;
\item 9 ~-- измеренный ток в фазе в отсечке;
\item 10 ~-- ошибка АЦП;
\item 11 ~-- период сигнала ШИМ меньше 50 мкс;
\item 12 ~-- Период сигнала ШИМ больше 400 мкс.\BL
\end{itemize} 
%--------------------------------------------------------
% *******begin subsection***************
\subsubsection{\DbgSecSt{\StPart}{EncoderErrors}}
\index{Программный интерфейс ПЛК!Обработка ошибок!Объединение EncoderErrors}
\label{sec:EncoderErrors}

\begin{fHeader}
    Тип данных:            & \RightHandText{Объединение EncoderErrors}\\
    Файл объявления:             & \RightHandText{include/cnc/errors.h} \\
\end{fHeader}

Объединение определяет ошибки ДОС.

\begin{MyTableThreeColAllCntr}{Объединение EncoderErrors}{tbl:EncoderErrors}{|m{0.33\linewidth}|m{0.22\linewidth}|m{0.45\linewidth}|}{Элемент}{Тип}{Описание}
\hline struct \{ 
\newline
encFault & \newline \centering{unsigned:1} & \newline Комбинированная ошибка датчика  \\
\hhline{~} decode & \centering{unsigned:1} & Ошибка декодирования \\
\hhline{~} sumOfSqr & \centering{unsigned:1} & Неверная сумма квадратов каналов синусно-косинусного датчика \\
\hhline{~} faultN & \centering{unsigned:1} &  Сигнал FAULT\_N \\
\hhline{~} adc & \centering{unsigned:1} & Ошибка АЦП \\
\hhline{~} lineA & \centering{unsigned:1} & Ошибка канала A \\
\hhline{~} lineB & \centering{unsigned:1} & Ошибка канала B \\
\hhline{~} lineC & \centering{unsigned:1} & Ошибка канала C \\
\hhline{~} power & \centering{unsigned:1} & Ошибка питания \\
\hhline{~} serialDataNotReady & \centering{unsigned:1} & Ошибка последовательного ДОС \\
\hhline{~} warningBiSS & \centering{unsigned:1} & Предупреждение ДОС BiSS \\
\hhline{~} faultBiSS & \centering{unsigned:1} & Ошибка ДОС BiSS \\
\hhline{~} statusEnDat \} & \centering{unsigned:1} & Ошибка статуса ДОС с протоколом EnDat \\
\hline errors & \centering{unsigned} & Переменная, содержащая все битовые поля \\
\end{MyTableThreeColAllCntr}
% *******end subsection***************
%--------------------------------------------------------
% *******begin subsection***************
\subsubsection{\DbgSecSt{\StPart}{IOErrors}}
\index{Программный интерфейс ПЛК!Обработка ошибок!Объединение IOErrors}
\label{sec:IOErrors}

\begin{fHeader}
    Тип данных:            & \RightHandText{Объединение IOErrors}\\
    Файл объявления:             & \RightHandText{include/cnc/errors.h} \\
\end{fHeader}

Объединение определяет ошибки последовательного интерфейса плат входов/выходов.

\begin{MyTableThreeColAllCntr}{Объединение IOErrors}{tbl:IOErrors}{|m{0.33\linewidth}|m{0.22\linewidth}|m{0.45\linewidth}|}{Элемент}{Тип}{Описание}
\hline struct \{ 
\newline
parity & \newline \centering{unsigned:1} & \newline Ошибка четности \\
\hhline{~} protocol & \centering{unsigned:1} & Ошибка протокола \\
\hhline{~} crc & \centering{unsigned:1} & Ошибка контрольной суммы \\
\hhline{~} watchdog \} & \centering{unsigned:1} & Срабатывание сторожевого таймера \\
\hline errors & \centering{unsigned} & Переменная, содержащая все битовые поля \\
\end{MyTableThreeColAllCntr}
% *******end subsection***************
%--------------------------------------------------------

% *******begin subsection***************
\subsubsection{\DbgSecSt{\StPart}{MotorErrors}}
\index{Программный интерфейс ПЛК!Обработка ошибок!Объединение MotorErrors}
\label{sec:MotorErrors}

\begin{fHeader}
    Тип данных:            & \RightHandText{Объединение MotorErrors}\\
    Файл объявления:             & \RightHandText{include/cnc/errors.h} \\
\end{fHeader}

Объединение определяет ошибки приводов.

\begin{MyTableThreeColAllCntr}{Объединение MotorErrors}{tbl:MotorErrors}{|m{0.33\linewidth}|m{0.22\linewidth}|m{0.45\linewidth}|}{Элемент}{Тип}{Описание}
\hline struct \{ 
\newline phaseref & \newline \centering{unsigned:1} & \newline Не выполнена фазировка \\
\hhline{~} home & \centering{unsigned:1} & Не выполнен поиск нулевой точки \\
\hhline{~} homeError & \centering{unsigned:1} & Произошла ошибка при поиске нулевой точки \\
\hhline{~} openLoop & \centering{unsigned:1} & Двигатель не в слежении \\

\hhline{~} encoder & \centering{unsigned:1} & Ошибка ДОС \\
\hhline{~} plusLimit & \centering{unsigned:1} & Срабатывание аппаратного ограничителя в положительном направлении \\
\hhline{~} minusLimit & \centering{unsigned:1} & Срабатывание аппаратного ограничителя в отрицательном направлении\\
\hhline{~} swPlusLimit & \centering{unsigned:1} & Срабатывание программного ограничителя в положительном направлении \\
\hhline{~} swMinusLimit & \centering{unsigned:1} & Срабатывание программного ограничителя в отрицательном направлении \\
\hhline{~} folError & \centering{unsigned:1} & Критическая ошибка слежения \\
\hhline{~} folErrorWarning & \centering{unsigned:1} & Предупредительная ошибка слежения \\
\hhline{~} temperature & \centering{unsigned:1} & Перегрев двигателя \\
\hhline{~} tempWarning & \centering{unsigned:1} & Предупреждение о перегреве двигателя \\
\hhline{~} auxFault & \centering{unsigned:1} & Внешняя ошибка \\
\hhline{~} pos2Error & \centering{unsigned:1} & Ошибка рассогласования датчиков положения и скорости \\
\hhline{~} pos2Warning & \centering{unsigned:1} & Предупреждение рассогласования датчиков положения и скорости \\
\hhline{~} phasePosError & \centering{unsigned:1} & Ошибка рассогласования датчиков положения и коммутации \\
\hhline{~} phasePosWarning \} & \centering{unsigned:1} & Предупреждение рассогласования датчиков положения и коммутации \\
\hline errors & \centering{unsigned} & Переменная, содержащая все битовые поля \\
\end{MyTableThreeColAllCntr}
% *******end subsection***************\\

%--------------------------------------------------------
% *******begin subsection***************
\subsubsection{\DbgSecSt{\StPart}{AxisErrors}}
\index{Программный интерфейс ПЛК!Обработка ошибок!Объединение AxisErrors}
\label{sec:AxisErrors}

\begin{fHeader}
    Тип данных:            & \RightHandText{Объединение AxisErrors}\\
    Файл объявления:             & \RightHandText{include/cnc/errors.h} \\
\end{fHeader}

Объединение определяет ошибки оси.

\begin{MyTableThreeColAllCntr}{Объединение AxisErrors}{tbl:AxisErrors}{|m{0.33\linewidth}|m{0.22\linewidth}|m{0.45\linewidth}|}{Элемент}{Тип}{Описание}
\hline struct \{ 
\newline abortTimeout & \newline \centering{unsigned:1} & \newline Истекло время операции аварийного торможения \\
\hhline{~} activateTimeout & \centering{unsigned:1} & Истекло время операции включения в слежение \\
\hhline{~} phaseRefTimeout & \centering{unsigned:1} & Истекло время операции фазировки \\
\hhline{~} deactivateTimeout \} & \centering{unsigned:1} & Истекло время операции выключения \\
\hline errors & \centering{unsigned} & Переменная, содержащая все битовые поля \\
\end{MyTableThreeColAllCntr}
% *******end subsection***************

%--------------------------------------------------------
% *******begin subsection***************
\subsubsection{\DbgSecSt{\StPart}{SpindleErrors}}
\index{Программный интерфейс ПЛК!Обработка ошибок!Объединение SpindleErrors}
\label{sec:SpindleErrors}

\begin{fHeader}
    Тип данных:            & \RightHandText{Объединение SpindleErrors}\\
    Файл объявления:             & \RightHandText{include/cnc/errors.h} \\
\end{fHeader}

Объединение определяет ошибки шпинделя.

\begin{MyTableThreeColAllCntr}{Объединение SpindleErrors}{tbl:SpindleErrors}{|m{0.33\linewidth}|m{0.22\linewidth}|m{0.45\linewidth}|}{Элемент}{Тип}{Описание}
\hline struct \{ 
\newline  abortTimeout & \newline \centering{unsigned:1} &  \newline Истекло время операции аварийного торможения \\
\hhline{~} activateTimeout & \centering{unsigned:1} & Истекло время операции включения в слежение \\
\hhline{~} phaseRefTimeout & \centering{unsigned:1} & Истекло время операции фазировки \\
\hhline{~} deactivateTimeout & \centering{unsigned:1} & Истекло время операции выключения \\
\hhline{~} speedTimeout & \centering{unsigned:1} & Истекло время выхода на заданную скорость \\
\hhline{~} stopTimeout & \centering{unsigned:1} & Истекло время операции останова \\
\hhline{~} homeTimeout & \centering{unsigned:1} & Истекло время операции поиска нулевой точки  \\
\hhline{~} positionTimeout \} & \centering{unsigned:1} & Истекло время выхода в заданное положение \\
\hline errors & \centering{unsigned} & Переменная, содержащая все битовые поля \\
\end{MyTableThreeColAllCntr}
% *******end subsection***************
%--------------------------------------------------------
% *******begin subsection***************
\subsubsection{\DbgSecSt{\StPart}{ChannelErrors}}
\index{Программный интерфейс ПЛК!Обработка ошибок!Объединение ChannelErrors}
\label{sec:ChannelErrors}

\begin{fHeader}
    Тип данных:            & \RightHandText{Объединение ChannelErrors}\\
    Файл объявления:             & \RightHandText{include/cnc/errors.h} \\
\end{fHeader}

Объединение определяет ошибки канала управления.

\begin{MyTableThreeColAllCntr}{Объединение ChannelErrors}{tbl:ChannelErrors}{|m{0.33\linewidth}|m{0.22\linewidth}|m{0.45\linewidth}|}{Элемент}{Тип}{Описание}
\hline struct \{ 
\newline phaseRefTimeout  & \newline \centering{unsigned:1} &  \newline Истекло время операции фазировки \\
\hhline{~} driveOnTimeout  & \centering{unsigned:1} & Истекло время ожидания включения сервоусилителя \\
\hhline{~} driveOffTimeout  & \centering{unsigned:1} & Истекло время ожидания выключения сервоусилителя \\
\hhline{~} abortTimeout & \centering{unsigned:1} & Истекло время операции аварийного торможения \\
\hhline{~} stopTimeout & \centering{unsigned:1} & Истекло время операции останова \\
\hhline{~} homeTimeout & \centering{unsigned:1} & Истекло время операции поиска нулевой точки \\
\hhline{~} homeError & \centering{unsigned:1} & Ошибка поиска нулевой точки \\

\hhline{~} startWithoutHome & \centering{unsigned:1} & Попытка запуска без определения нулевой точки \\
\hhline{~} cannotStart & \centering{unsigned:1} & Ошибка запуска программы \\

\hhline{~} progStopOk & \centering{unsigned:1} & УП выполнена \\
\hhline{~} progStopAbort & \centering{unsigned:1} & УП прервана \\
\hhline{~} progStopSyncError & \centering{unsigned:1} & Ошибка присвоения в буфере синхронных переменных \\
\hhline{~} progStopBufferError & \centering{unsigned:1} & Ошибка в буфере программы движения \\
\hhline{~} progStopCCMove & \centering{unsigned:1} & Неверный кадр в режиме коррекции инструмента \\
\hhline{~} progStopLinToPvt & \centering{unsigned:1} & Ошибка при преобразовании линейного движения в сплайн или pvt-движение \\
\hhline{~} progStopCCLeadOut & \centering{unsigned:1} & Неверный кадр при отмене режима коррекции инструмента \\
\hhline{~} progStopCCLeadIn & \centering{unsigned:1} &  Неверный кадр при активации режима коррекции инструмента \\
\hhline{~} progStopCCBufSize & \centering{unsigned:1} & Недостаточный размер буфера в режиме коррекции инструмента \\
\hhline{~} progStopPvt & \centering{unsigned:1} & Ошибка расчёта pvt-движения\\
\hhline{~} progStopCCFeed & \centering{unsigned:1} & Неверное указание подачи в режиме коррекции инструмента \\
\hhline{~} progStopCCDir & \centering{unsigned:1} & Смена направления движения в режиме коррекции инструмента \\
\hhline{~} progStopNoSolve & \centering{unsigned:1} & Невозможно рассчитать движение в режиме коррекции инструмента \\
\hhline{~} progStopCC3NdotT & \centering{unsigned:1} &  Ошибка расчёта точки резания в режиме трёхмерной коррекции инструмента \\
\hhline{~} progStopCCDist & \centering{unsigned:1} & Невозможно предотвратить «перерез» в режиме коррекции инструмента \\
\hhline{~} progStopCCNoIntersect & \centering{unsigned:1} & Невозможно найти пересечение траекторий в режиме коррекции инструмента \\
\hhline{~} progStopCCNoMoves & \centering{unsigned:1} & Между активацией и отменой  
режима коррекции инструмента кадры без команд движения \\

\hhline{~} progStopRunTime & \centering{unsigned:1} &  Недостаточное время для расчёта движения \\
\hhline{~} progStopInPos & \centering{unsigned:1} & Истекло время ожидания состояния «в позиции» \\
\hhline{~} progStopSoftLimit & \centering{unsigned:1} & Срабатывание программного ограничения \\
\hhline{~} progStopRadiusX & \centering{unsigned:1} & Срабатывание ограничения величины радиальной ошибки в режиме кругового движения \\
\hhline{~} progStopRadiusXX & \centering{unsigned:1} & Срабатывание ограничения величины радиальной ошибки в режиме кругового движения в расширенной системе координат \\
\hhline{~} progPausedM00 & \centering{unsigned:1} & УП временно остановлена по команде М00 или М01 \\
\hhline{~} cycleInvalidArgs & \centering{unsigned:1} & Неверные аргументы функции постоянного цикла \\
\hhline{~} seekingBlock & \centering{unsigned:1} & Поиск кадра \\
\hhline{~} seekBlockFound & \centering{unsigned:1} & Кадр найден \\
\hhline{~} seekBlockNotFound \} & \centering{unsigned:1} & Кадр не найден \\
\hline errors & \centering{unsigned} & Переменная, содержащая все битовые поля \\
\end{MyTableThreeColAllCntr}
% *******end subsection***************
%--------------------------------------------------------
% *******begin subsection***************
\subsubsection{\DbgSecSt{\StPart}{NCErrors}}
\index{Программный интерфейс ПЛК!Обработка ошибок!Объединение NCErrors}
\label{sec:NCErrors}

\begin{fHeader}
    Тип данных:            & \RightHandText{Объединение NCErrors}\\
    Файл объявления:             & \RightHandText{include/cnc/errors.h} \\
\end{fHeader}

Объединение определяет системные ошибки.

\begin{MyTableThreeColAllCntr}{Объединение NCErrors}{tbl:NCErrors}{|m{0.33\linewidth}|m{0.22\linewidth}|m{0.45\linewidth}|}{Элемент}{Тип}{Описание}
\hline struct \{ 
\newline  factory & \newline \centering{unsigned:1} &  \newline Ошибка загрузки системных параметров, используются параметры по умолчанию \\
\hhline{~} userFactory & \centering{unsigned:1} &  Ошибка загрузки параметров пользователя, пользовательские переменные не определены \\
\hhline{~} swClock & \centering{unsigned:1} & Отсутствует аппаратный источник частоты \\
\hhline{~} bgWdt & \centering{unsigned:1} & Срабатывание сторожевого таймера фонового режима \\
\hhline{~} rtWdt & \centering{unsigned:1} &  Срабатывание сторожевого таймера реального времени \\
\hhline{~} sysPlcFault & \centering{unsigned:1} & Ошибка выполнения системных программ ПЛК \\
\hhline{~} hmiWatchdog \} & \centering{unsigned:1} & Срабатывание сторожевого таймера связи с пультом оператора \\
\hline errors & \centering{unsigned} & Переменная, содержащая все битовые поля \\
\end{MyTableThreeColAllCntr}
% *******end subsection***************
\begin{comment}
%--------------------------------------------------------
% *******begin subsection***************
\subsubsection{\DbgSecSt{\StPart}{MachineErrors}}
\index{Программный интерфейс ПЛК!Обработка ошибок!Структура MachineErrors}
\label{sec:MachineErrors}

\begin{fHeader}
    Тип данных:            & \RightHandText{Структура MachineErrors}\\
    Файл объявления:             & \RightHandText{include/cnc/errors.h} \\
\end{fHeader}

Структура определяет ошибки станка.

% *******end subsection***************
\end{comment}
%--------------------------------------------------------
% *******begin subsection***************
\subsubsection{\DbgSecSt{\StPart}{Errors}}
\index{Программный интерфейс ПЛК!Обработка ошибок!Структура Errors}
\label{sec:Errors}

\begin{fHeader}
    Тип данных:            & \RightHandText{Структура Errors}\\
    Файл объявления:             & \RightHandText{include/cnc/errors.h} \\
\end{fHeader}

Структура содержит данные о системных ошибках, об ошибках станка, каналов управления, осей, шпинделей, приводов, ДОС, сервоусилителей и плат входов/выходов.

\begin{MyTableThreeColAllCntr}{Структура Errors}{tbl:Errors}{|m{0.38\linewidth}|m{0.22\linewidth}|m{0.4\linewidth}|}{Элемент}{Тип}{Описание}
\hline machine & \centering{\hyperlink{Machine_Errors}{MachineErrors}} & Ошибки станка \\
\hline nc & \centering{\myreftosec{NCErrors}} & Системные ошибки  \\
\hline channel [ЧИСЛО\_КАНАЛОВ] & \centering{\myreftosec{ChannelErrors}} & Ошибки каналов управления \\
\hline axes [ЧИСЛО\_ОСЕЙ] & \centering{\myreftosec{AxisErrors}} & Ошибки осей \\
\hline spindles [ЧИСЛО\_ШПИНДЕЛЕЙ] & \centering{\myreftosec{SpindleErrors}} & Ошибки шпинделей \\
\hline motors [ЧИСЛО\_ДВИГАТЕЛЕЙ+1] & \centering{\myreftosec{MotorErrors}} & Ошибки приводов \\
\hline encoders [ЧИСЛО\_ДОС] & \centering{\myreftosec{EncoderErrors}} & Ошибки ДОС \\
\hline drive [ЧИСЛО\_ДВИГАТЕЛЕЙ+1] & \centering{\myreftosec{DriveErrors}} & Ошибки сервоусилителей \\
\hline io [ЧИСЛО\_ПЛАТ\_ВХ/ВЫХ] & \centering{\myreftosec{IOErrors}} & Ошибки плат входов/выходов \\
\end{MyTableThreeColAllCntr}
% *******end subsection***************
%--------------------------------------------------------
% *******begin subsection***************
\subsubsection{\DbgSecSt{\StPart}{ErrorReaction}}
\index{Программный интерфейс ПЛК!Обработка ошибок!Перечисление ErrorReaction}
\label{sec:ErrorReaction}

\begin{fHeader}
    Тип данных:            & \RightHandText{Перечисление ErrorReaction}\\
    Файл объявления:             & \RightHandText{include/cnc/errors.h} \\
\end{fHeader}

Перечисление определяет идентификаторы типов реакций на ошибки.

\begin{MyTableTwoColAllCntr}{Перечисление ErrorReaction}{tbl:ErrorReaction}{|m{0.38\linewidth}|m{0.57\linewidth}|}{Идентификатор}{Описание}
\hline reactNone &  Нет реакции \\
% \hline reactLocal &    \\
\hline reactFollowUp  &  Восстановление после ошибки \\
\hline reactStopProgram  & Прервано выполнение программы \\
\hline reactNCNotReady  & Нет готовности системы \\
\hline reactChannelNotReady  & Нет готовности канала \\
\hline reactStartDisable  & Запрет запуска программы в канале \\
\hline reactNeedHome  & Необходим повторный поиск нулевой точки для осей в канале \\
\hline reactShowAlarm  & Показать сообщение об ошибке \\
\hline reactStop &  Останов осей \\
\hline reactStopAtEnd  & Останов осей в конце блока \\
\hline reactAutoOnly  & Фиксация ошибки только в автоматическом режиме \\
\hline reactWarning  & Показать предупреждение \\
\end{MyTableTwoColAllCntr}
% *******end subsection***************
%--------------------------------------------------------
% *******begin subsection***************
\subsubsection{\DbgSecSt{\StPart}{ErrorClear}}
\index{Программный интерфейс ПЛК!Обработка ошибок!Перечисление ErrorClear}
\label{sec:ErrorClear}

\begin{fHeader}
    Тип данных:            & \RightHandText{Перечисление ErrorClear}\\
    Файл объявления:             & \RightHandText{include/cnc/errors.h} \\
\end{fHeader}

Перечисление определяет идентификаторы типов сброса ошибок. 

Самый низкий приоритет  имеет автоматический сброс (\texttt{clearSelf}), самый высокий приоритет ~-- сброс по включению питания (\texttt{clearPowerOn}). \killoverfullbefore

\begin{MyTableTwoColAllCntr}{Перечисление ErrorClear}{tbl:ErrorClear}{|m{0.38\linewidth}|m{0.57\linewidth}|}{Идентификатор}{Описание}
\hline clearSelf &  Автоматический сброс \\
\hline clearCancel &  Сброс из оболочки, отменой текущего режима работы или перезапуском УП \\
\hline clearNCStart &  Сброс отменой текущего режима работы или перезапуском УП \\
\hline clearReset &  Сброс отменой текущего режима работы \\
\hline clearNCReset &  Сброс перезагрузкой системы \\
\hline clearPowerOn &  Сброс по включению питания \\
\end{MyTableTwoColAllCntr}
% *******end subsection***************
%--------------------------------------------------------
% *******begin subsection***************
\subsubsection{\DbgSecSt{\StPart}{DriveErrorReaction}}
\index{Программный интерфейс ПЛК!Обработка ошибок!Перечисление DriveErrorReaction}
\label{sec:DriveErrorReaction}

\begin{fHeader}
    Тип данных:            & \RightHandText{Перечисление DriveErrorReaction}\\
    Файл объявления:             & \RightHandText{include/cnc/errors.h} \\
\end{fHeader}

Перечисление определяет идентификаторы типов реакции на ошибки сервоусилителя.

\begin{MyTableTwoColAllCntr}{Перечисление DriveErrorReaction}{tbl:DriveErrorReaction}{|m{0.38\linewidth}|m{0.57\linewidth}|}{Идентификатор}{Описание}
\hline dreactNone &  Нет реакции \\
\hline dreactOFF1 &  Останов и выключение с задержкой в режиме слежения, иначе выключение \\
\hline dreactOFF1delayed  &  Пауза, останов и выключение в режиме слежения, иначе пауза и выключение \\
\hline dreactOFF2 &  Выключение \\
\hline dreactOFF3 &  Аварийное торможение и выключение с задержкой в режиме слежения, иначе выключение \\
\hline dreactSTOP2 & Аварийное торможение и сохранение режима слежения \\
\hline dreactIASC\_DCBRK  & Для синхронного - закоротить обмотки, для асинхронного - торможение постоянным током \\
\hline dreactENC & Настраивается (по умолчанию dreactOFF2) \\
\end{MyTableTwoColAllCntr}
% *******end subsection***************

\begin{comment}
%--------------------------------------------------------
% *******begin subsection***************
\subsubsection{\DbgSecSt{\StPart}{DriveErrorAcknowledge}}
\index{Программный интерфейс ПЛК!Обработка ошибок!Перечисление DriveErrorAcknowledge}
\label{sec:DriveErrorAcknowledge}

\begin{fHeader}
    Тип данных:            & \RightHandText{Перечисление DriveErrorAcknowledge}\\
    Файл объявления:             & \RightHandText{include/cnc/errors.h} \\
\end{fHeader}

Перечисление определяет идентификаторы подтверждений ошибок сервоусилителя.

\begin{MyTableTwoColAllCntr}{Перечисление DriveErrorAcknowledge}{tbl:DriveErrorAcknowledge}{|m{0.38\linewidth}|m{0.57\linewidth}|}{Идентификатор}{Описание}
\hline dackPowerOn &    \\
\hline dackImmediately  &   \\
\hline dackDisable &    \\
\end{MyTableTwoColAllCntr}
% *******end subsection***************
\end{comment}

%--------------------------------------------------------
% *******begin subsection***************
\subsubsection{\DbgSecSt{\StPart}{ErrorDescription}}
\index{Программный интерфейс ПЛК!Обработка ошибок!Структура ErrorDescription}
\label{sec:ErrorDescription}

\begin{fHeader}
    Тип данных:            & \RightHandText{Структура ErrorDescription}\\
    Файл объявления:             & \RightHandText{include/cnc/errors.h} \\
\end{fHeader}

Структура определяет параметры описания ошибки.

\begin{MyTableThreeColAllCntr}{Структура ErrorDescription}{tbl:ErrorDescription}{|m{0.33\linewidth}|m{0.22\linewidth}|m{0.45\linewidth}|}{Элемент}{Тип}{Описание}
\hline id & \centering{unsigned} & Номер ошибки в категории \\
\hline reaction & \centering{unsigned} & Тип реакции \\
\hline clear & \centering{unsigned} & Тип сброса \\
\end{MyTableThreeColAllCntr}
% *******end subsection***************
%--------------------------------------------------------
% *******begin subsection***************
\subsubsection{\DbgSecSt{\StPart}{ErrorRequests}}
\index{Программный интерфейс ПЛК!Обработка ошибок!Структура ErrorRequests}
\label{sec:ErrorRequests}

\begin{fHeader}
    Тип данных:            & \RightHandText{Структура ErrorRequests}\\
    Файл объявления:             & \RightHandText{include/cnc/errors.cfg} \\
\end{fHeader}

Структура определяет флаги действий системы, которые вызываются согласно реакциям на ошибки в перечислении \myreftosec{ErrorReaction}.

\begin{MyTableThreeColAllCntr}{Структура ErrorDescription}{tbl:ErrorDescription}{|m{0.33\linewidth}|m{0.22\linewidth}|m{0.45\linewidth}|}{Элемент}{Тип}{Описание}
\hline ncNotReady & \centering{int:1} &  Нет готовности системы \\
\hline ncStop & \centering{int:1} & Останов осей \\
\hline ncStopAtEnd & \centering{int:1} & Останов осей в конце блока \\
\hline ncFollowUp & \centering{int:1} &  Восстановление после ошибки \\
\hline channelNotReady & \centering{int:1} & Нет готовности канала \\
\hline startDisable & \centering{int:1} & Запрет запуска программы в канале \\
\hline needHome & \centering{int:1} & Необходим повторный поиск нулевой точки для осей в канале \\
\end{MyTableThreeColAllCntr}
% *******end subsection***************
%--------------------------------------------------------
% *******begin subsection***************
\subsection{\DbgSecSt{\StPart}{Функции и макросы}}

%--------------------------------------------------------
% *******begin subsection***************
\subsubsection{\DbgSecSt{\StPart}{errorSetScan}}
\index{Программный интерфейс ПЛК!Обработка ошибок!Функция errorSetScan}
\label{sec:errorSetScan}

\begin{pHeader}
    Синтаксис:      & \RightHandText{int errorSetScan (unsigned curInput, unsigned input,}\\
    & \RightHandText {const ErrorDescription \&desc, ErrorClear request);} \\
 Аргумент(ы):    & \RightHandText{unsigned curInput ~-- флаг ошибки,} \\ 
 & \RightHandText{unsigned input ~-- состояние соответствующего входа ошибки,} \\   
 & \RightHandText {const \myreftosec{ErrorDescription} \&desc ~-- описание ошибки} \\
    & \RightHandText {\myreftosec{ErrorClear} request ~-- идентификатор запроса на сброс ошибки} \\
%    Возвращаемое значение:       & \RightHandText{Целое знаковое число} \\
    Файл объявления:             & \RightHandText{include/cnc/errors.h} \\      
\end{pHeader}

Функция возвращает 1 (наличие ошибки), если состояние соответствующего входа ошибки отлично от 0. 

Если состояние соответствующего входа равно 0 и уровень сброса ошибки в структуре описания ошибки меньше или равен значению идентификатора запроса на сброс ошибки, то функция возвращает 0 (ошибка сброшена). \killoverfullbefore

Если состояние соответствующего входа равно 0 и уровень сброса ошибки в структуре описания ошибки больше значения идентификатора запроса на сброс ошибки, то функция возвращает текущее значение флага ошибки. \killoverfullbefore

Является системной.
% *******end subsection*****************
%--------------------------------------------------------
% *******begin subsection***************
\subsubsection{\DbgSecSt{\StPart}{errorScanSet}}
\index{Программный интерфейс ПЛК!Обработка ошибок!Макрос errorScanSet}
\label{sec:errorScanSet}

\begin{pHeader}
    Синтаксис:      & \RightHandText{errorScanSet (error, input, desc, request)}\\
 Аргумент(ы):    & \RightHandText{error ~-- флаг ошибки,} \\ 
 & \RightHandText{input ~-- состояние соответствующего входа,} \\   
 & \RightHandText {desc ~-- описание ошибки (переменная типа \myreftosec{ErrorDescription}),} \\
    & \RightHandText {request ~-- идентификатор запроса на сброс ошибки (переменная \newline типа \myreftosec{ErrorClear})} \\
%    Возвращаемое значение:       & \RightHandText{Целое знаковое число} \\
    Файл объявления:             & \RightHandText{include/cnc/errors.h} \\      
\end{pHeader}

Макрос \myreftosec{errorScanSet} вызывает функцию \myreftosec{errorSetScan} и присваивает возвращаемое значение аргументу \texttt{error} (флагу ошибки).

Макрос обновляет флаг выбранной ошибки в зависимости от состояния соответствующего входа и заданного идентификатора запроса на сброс ошибки.

Является системной.
% *******end subsection*****************
%--------------------------------------------------------
% *******begin subsection***************
\subsubsection{\DbgSecSt{\StPart}{errorScanRequest}}
\index{Программный интерфейс ПЛК!Обработка ошибок!Функция errorScanRequest}
\label{sec:errorScanRequest}

\begin{pHeader}
    Синтаксис:      & \RightHandText{void errorScanRequest (ErrorClear request);}\\
    Аргумент(ы):    & \RightHandText{\myreftosec{ErrorClear} request ~-- идентификатор запроса на сброс ошибки} \\    
%    Возвращаемое значение:       & \RightHandText{Нет} \\ 
    Файл объявления:             & \RightHandText{include/cnc/errors.h} \\
\end{pHeader}

Функция выполняет вызовы макроса \myreftosec{errorScanSet} для обновления флагов ошибок станка, УЧПУ, каналов управления, осей, шпинделей, приводов, сервоусилителей, ДОС и последовательного интерфейса плат входов/выходов. \killoverfullbefore

Является системной.
% *******end section*****************
%--------------------------------------------------------
% *******begin subsection***************
\subsubsection{\DbgSecSt{\StPart}{errorScan}}
\index{Программный интерфейс ПЛК!Обработка ошибок!Функция errorScan}
\label{sec:errorScan}

\begin{pHeader}
    Синтаксис:      & \RightHandText{void errorScan();}\\
    Аргумент(ы):    & \RightHandText{Нет} \\   
%    Возвращаемое значение:       & \RightHandText{Нет} \\
    Файл объявления:             & \RightHandText{include/cnc/errors.h} \\      
\end{pHeader}

Функция выполняет вызов \myreftosec{errorScanRequest} с аргументом \texttt{clearSelf} (см. \myreftosec{ErrorClear}) для обновления флагов ошибок с запросом автоматического сброса ошибок. \killoverfullbefore

Является системной.
% *******end subsection*****************
%--------------------------------------------------------
% *******begin subsection***************
\subsubsection{\DbgSecSt{\StPart}{errorReaction}}
\index{Программный интерфейс ПЛК!Обработка ошибок!Функция errorReaction}
\label{sec:errorReaction}

\begin{pHeader}
    Синтаксис:      & \RightHandText{void errorReaction(unsigned input, const ErrorDescription \&desc);}\\
    Аргумент(ы):    & \RightHandText{unsigned input ~-- флаг ошибки,} \\   
     & \RightHandText{const \myreftosec{ErrorDescription} \&desc ~-- описание ошибки} \\
%    Возвращаемое значение:       & \RightHandText{Нет} \\
    Файл объявления:             & \RightHandText{include/cnc/errors.h} \\      
\end{pHeader}

%Устанавливает флаги действий системы (см. \myreftosec{ErrorRequests}), которые вызываются реакциями на возникшую ошибку.

Устанавливает флаги действий системы (см. \myreftosec{ErrorRequests}) согласно реакциям на возникшую ошибку.

Является системной.
% *******end subsection*****************
%--------------------------------------------------------
% *******begin subsection***************
\subsubsection{\DbgSecSt{\StPart}{errorsMachineScan}}
\index{Программный интерфейс ПЛК!Обработка ошибок!Функция errorsMachineScan}
\label{sec:errorsMachineScan}

\begin{pHeader}
    Синтаксис:      & \RightHandText{void errorsMachineScan (int request);}\\
  Аргумент(ы):    & \RightHandText{int request ~-- идентификатор запроса на сброс ошибки} \\   
%    Возвращаемое значение:       & \RightHandText{Нет} \\
    Файл объявления:             & \RightHandText{include/cnc/errors.h} \\
\end{pHeader}

Функция выполняет вызовы макроса \myreftosec{errorScanSet} для обновления флагов ошибок станка. Вызывается из  \myreftosec{errorScanRequest}. \killoverfullbefore

Реализуется пользователем.
% *******end subsection*****************
%--------------------------------------------------------
% *******begin subsection***************
\subsubsection{\DbgSecSt{\StPart}{errorsMachineReaction}}
\index{Программный интерфейс ПЛК!Обработка ошибок!Функция errorsMachineReaction}
\label{sec:errorsMachineReaction}

\begin{pHeader}
    Синтаксис:      & \RightHandText{void errorsMachineReaction();}\\
    Аргумент(ы):    & \RightHandText{Нет} \\   
%    Возвращаемое значение:       & \RightHandText{Нет} \\
    Файл объявления:             & \RightHandText{include/cnc/errors.h} \\
\end{pHeader}

%Функция обрабатывает ошибки станка.
Функция выполняет вызовы функции \myreftosec{errorReaction} для ошибок станка.

%Устанавливает ошибочные состояния системы (см. \myreftosec{ErrorRequests}), которые вызываются реакциями на ошибки станка.

Реализуется пользователем. 
% *******end subsection*****************
%-------------------------------------------------------------------
% *******begin subsection***************
\subsubsection{\DbgSecSt{\StPart}{encoderScanErrors}}
\index{Программный интерфейс ПЛК!Датчики обратной связи!Функция encoderScanErrors}
\label{sec:encoderScanErrors}

\begin{pHeader}
    Синтаксис:      & \RightHandText{void encoderScanErrors(ErrorClear request);}\\
    Аргумент(ы):    & \RightHandText{\myreftosec{ErrorClear} request ~-- идентификатор типа сброса ошибки} \\    
%    Возвращаемое значение:       & \RightHandText{Нет} \\ 
    Файл объявления:             & \RightHandText{include/func/enc.h} \\       
\end{pHeader}

Функция выполняет вызовы макроса \myreftosec{errorScanSet} для обновления флагов ошибок ДОС. Вызывается из  \myreftosec{errorScanRequest}. \killoverfullbefore

Является системной.
% *******end section*****************

%-------------------------------------------------------------------
% *******begin subsection***************
\subsubsection{\DbgSecSt{\StPart}{encoderErrorsReaction}}
\index{Программный интерфейс ПЛК!Датчики обратной связи!Функция encoderErrorsReaction}
\label{sec:encoderErrorsReaction}

\begin{pHeader}
    Синтаксис:      & \RightHandText{void encoderErrorsReaction();}\\
    Аргумент(ы):    & \RightHandText{Нет} \\    
%    Возвращаемое значение:       & \RightHandText{Нет} \\ 
    Файл объявления:             & \RightHandText{include/func/enc.h} \\       
\end{pHeader}

Функция выполняет вызовы функции \myreftosec{errorReaction} для ошибок ДОС.

Является системной.
% *******end section*****************

%--------------------------------------------------------
% *******begin subsection***************
\subsubsection{\DbgSecSt{\StPart}{ampScanErrors}}
\index{Программный интерфейс ПЛК!Обработка ошибок!Функция ampScanErrors}
\label{sec:ampScanErrors}

\begin{pHeader}
    Синтаксис:      & \RightHandText{void ampScanErrors(int motor, int servo, int chan, ErrorClear request,}\\
     & \RightHandText{int isaxis, int id);}\\
  Аргумент(ы):    & \RightHandText{int motor ~-- номер связанного с осью двигателя,} \\   
    & \RightHandText {int servo ~-- номер платы управления,}  \\ 
    & \RightHandText {int chan ~-- номер канала,}  \\ 
    & \RightHandText{\myreftosec{ErrorClear} request ~-- идентификатор типа сброса ошибки,} \\ 
    & \RightHandText {int isaxis ~-- флаг оси (1) или шпинделя (0),}  \\ 
    & \RightHandText {int id ~-- номер оси или шпинделя}  \\ 
%    Возвращаемое значение:       & \RightHandText{Нет} \\
    Файл объявления:             & \RightHandText{include/func/amp\_fault.h} \\      
\end{pHeader}

Функция выполняет вызовы макроса \myreftosec{errorScanSet} для обновления флагов ошибок сервоусилителей. Вызывается из  \myreftosec{errorScanRequest}. \killoverfullbefore

Является системной. 
% *******end subsection*****************
%--------------------------------------------------------
% *******begin subsection***************
\subsubsection{\DbgSecSt{\StPart}{ampErrorsReaction}}
\index{Программный интерфейс ПЛК!Обработка ошибок!Функция ampErrorsReaction}
\label{sec:ampErrorsReaction}

\begin{pHeader}
    Синтаксис:      & \RightHandText{void ampErrorsReaction(int motor);}\\
    Аргумент(ы):    & \RightHandText{int motor ~-- номер связанного с осью двигателя} \\   
%    Возвращаемое значение:       & \RightHandText{Нет} \\
    Файл объявления:             & \RightHandText{include/func/amp\_fault.h} \\      
\end{pHeader}

%Функция обрабатывает ошибки сервоусилителей.
Функция выполняет вызовы функции \myreftosec{errorReaction} для ошибок сервоусилителей.

Является системной.  
% *******end subsection*****************
%--------------------------------------------------------
% *******begin subsection***************
\subsubsection{\DbgSecSt{\StPart}{ioScanErrors}}
\index{Программный интерфейс ПЛК!Обработка ошибок!Функция ioScanErrors}
\label{sec:ioScanErrors}

\begin{pHeader}
    Синтаксис:      & \RightHandText{void ioScanErrors(int ioNum, int servo, int io, ErrorClear request)}\\
    Аргумент(ы):    & \RightHandText{int ioNum ~-- номер платы входов/выходов,} \\ 
    & \RightHandText {int servo ~-- номер платы управления,} \\
    & \RightHandText {int io ~-- номер входа/выхода,} \\           
    & \RightHandText {\myreftosec{ErrorClear} request ~-- идентификатор типа сброса ошибки} \\  
%    Возвращаемое значение:       & \RightHandText{Нет} \\
    Файл объявления:             & \RightHandText{include/func/io.h} \\      
\end{pHeader}

Функция выполняет вызовы макроса \myreftosec{errorScanSet} для обновления флагов ошибок последовательного интерфейса плат входов/выходов. Вызывается из  \myreftosec{errorScanRequest}. \killoverfullbefore

Является системной. 
% *******end subsection*****************
%--------------------------------------------------------
% *******begin subsection***************
\subsubsection{\DbgSecSt{\StPart}{ioErrorsReaction}}
\index{Программный интерфейс ПЛК!Обработка ошибок!Функция ioErrorsReaction}
\label{sec:ioErrorsReaction}

\begin{pHeader}
    Синтаксис:      & \RightHandText{void ioErrorsReaction(int ioNum, int io, int servo);}\\
    Аргумент(ы):    & \RightHandText{int ioNum ~-- номер платы входов/выходов,} \\   
    & \RightHandText {int io ~-- номер входа/выхода,} \\  
    & \RightHandText {int servo ~-- номер платы управления} \\    
%    Возвращаемое значение:       & \RightHandText{Нет} \\
    Файл объявления:             & \RightHandText{include/func/io.h} \\      
\end{pHeader}

%Функция обрабатывает ошибки последовательного интерфейса плат входов/выходов.
Функция выполняет вызовы функции \myreftosec{errorReaction} для ошибок последовательного интерфейса плат входов/выходов.

Является системной.  
% *******end subsection*****************
%-------------------------------------------------------------------

              % API: обработка ошибок
    %--------------------------------------------------------
% *******begin section***************
\section{\DbgSecSt{\StPart}{Управление осями}}
%--------------------------------------------------------
\subsection{\DbgSecSt{\StPart}{Типы данных}}

% *******begin subsection***************
\subsubsection{\DbgSecSt{\StPart}{AxisStates}}
\index{Программный интерфейс ПЛК!Управление осями!Перечисление AxisStates}
\label{sec:AxisStates}

\begin{fHeader}
    Тип данных:            & \RightHandText{Перечисление AxisStates}\\
    Файл объявления:             & \RightHandText{include/func/axis.h} \\
\end{fHeader}

Перечисление определяет идентификаторы состояний оси.

\begin{MyTableTwoColAllCntr}{Перечисление AxisStates}{tbl:AxisStates}{|m{0.38\linewidth}|m{0.57\linewidth}|}{Идентификатор}{Описание}
\hline axisInactive &  Ось выключена  \\
\hline axisActive &  Ось находится в слежении \\
\hline axisJoggingPlus & Толчковое движение в положительном направлении \\
\hline axisJoggingMinus & Толчковое движение в отрицательном направлении \\
\hline axisJoggingTo & Толчковое движение в заданное положение или на заданное расстояние \\
\hline axisStopping & Останов оси \\
\hline axisHomeWaitHW & Ожидание применения аппаратных настроек выезда в нулевую точку \\
\hline axisHoming & Выезд в нулевую точку \\
\hline axisIndexWaitHW & Ожидание применения аппаратных настроек поиска индексной метки \\
\hline axisIndexing & Поиск индексной метки \\
\hline axisAborting & Аварийное торможение \\
\hline axisWaitActivate & Ожидание включения \\
\hline axisWaitDeactivate & Ожидание выключения \\
\hline axisWaitPhaseRef  & Ожидание фазировки \\
\end{MyTableTwoColAllCntr}
% *******end subsection***************
%--------------------------------------------------------
% *******begin subsection***************
\subsubsection{\DbgSecSt{\StPart}{AxisCommands}}
\index{Программный интерфейс ПЛК!Управление осями!Перечисление AxisCommands}
\label{sec:AxisCommands}

\begin{fHeader}
    Тип данных:            & \RightHandText{Перечисление AxisCommands}\\
    Файл объявления:             & \RightHandText{include/func/axis.h} \\
\end{fHeader}

Перечисление определяет идентификаторы команд управления осями. 

\begin{MyTableTwoColAllCntr}{Перечисление AxisCommands}{tbl:AxisCommands}{|m{0.38\linewidth}|m{0.57\linewidth}|}{Идентификатор}{Описание}
\hline axisCmdIdle &  Нет команды  \\
\hline axisCmdKill  &  Выключить ось \\
\hline axisCmdActivate  &  Включить ось в слежение \\
\hline axisCmdDeactivate  &  Выключить ось  \\
\hline axisCmdJogPlus  &  Выполнить толчковое движение в положительном направлении \\
\hline axisCmdJogMinus  & Выполнить толчковое движение в отрицательном направлении \\
\hline axisCmdJogStop & Выполнить останов \\
\hline axisCmdJogRet & Вернуться в сохраненную позицию \\
\hline axisCmdInc  & Выполнить толчковое движение на заданное расстояние \\
\hline axisCmdAbs & Выполнить толчковое движение в заданное положение \\
\hline axisCmdHome & Выполнить движение в нулевую точку \\
\hline axisCmdIndex & Выполнить движение до индексной метки \\
\hline axisCmdPhaseRef & Выполнить фазировку \\
\hline axisCmdAbort & Выполнить аварийное выключение \\
\end{MyTableTwoColAllCntr}
% *******end subsection***************

\clearpage

%--------------------------------------------------------
% *******begin subsection***************
\subsubsection{\DbgSecSt{\StPart}{AxisAbortMode}}
\index{Программный интерфейс ПЛК!Управление осями!Перечисление AxisAbortMode}
\label{sec:AxisAbortMode}

\begin{fHeader}
    Тип данных:            & \RightHandText{Перечисление AxisAbortMode}\\
    Файл объявления:             & \RightHandText{include/func/axis.h} \\
\end{fHeader}

Перечисление определяет идентификаторы действий по команде аварийного останова (ABORT).

\begin{MyTableTwoColAllCntr}{Перечисление AxisAbortMode}{tbl:AxisAbortMode}{|m{0.38\linewidth}|m{0.57\linewidth}|}{Идентификатор}{Описание}
\hline axisAbortStop & Останов категории 2 (аварийно затормозить и оставаться в слежении) \\
\hline axisAbortAndKill  & Останов категории 1 (аварийно затормозить и выключить) \\
\hline axisAbortKill & Останов категории 0 (выключить) \\
\end{MyTableTwoColAllCntr}
% *******end subsection***************
%--------------------------------------------------------
% *******begin subsection***************
\subsubsection{\DbgSecSt{\StPart}{AxisConfig}}
\index{Программный интерфейс ПЛК!Управление осями!Структура AxisConfig}
\label{sec:AxisConfig}

\begin{fHeader}
    Тип данных:            & \RightHandText{Структура AxisConfig}\\
    Файл объявления:             & \RightHandText{include/func/axis.h} \\
\end{fHeader}

Структура определяет настройки оси.

\begin{MyTableThreeColAllCntr}{Структура AxisConfig}{tbl:AxisConfig}{|m{0.33\linewidth}|m{0.22\linewidth}|m{0.45\linewidth}|}{Элемент}{Тип}{Описание}
\hline servo & \centering{unsigned} & Номер платы управления ($0\div3$)\\
\hline chan & \centering{unsigned} & Номер канала ($0\div7$)\\
\hline motor & \centering{unsigned} & Номер связанного с осью двигателя ($0\div31$)\\
\hline homeOrder & \centering{unsigned} & Порядок выезда в нулевую точку \\
\hline needDKill & \centering{unsigned:1} & Требуется задержка перед отключением \\
\hline needPhaseRef & \centering{unsigned:1} & Требуется фазировка \\

\hline needHome & \centering{unsigned:1} & Требуется выезд в нулевую точку \\
\hline needIndex & \centering{unsigned:1} & Требуется позиционирование по индексной метке \\
\hline hasAbsPos & \centering{unsigned:1} & Установлен абсолютный датчик \\

\hline abortMode & \centering{unsigned:2} & Реакции на команду аварийного выключения (см. \myreftosec{AxisAbortMode}) \\
\hline homeCaptCtrl & \centering{unsigned:4} & Настройка захвата положения для выезда в нулевую точку по входу (флагу) \\
\hline indexCaptCtrl & \centering{unsigned:4} & Настройка захвата положения для выезда в нулевую точку по индексной метке ДОС \\
\hline needPosRef & \centering{unsigned:1} & Требуется позиционирование при включении станка \\
\hline refAxis & \centering{unsigned:5} & Координата для оси \\
\hline rotaryAxis & \centering{unsigned:1} & Вращающаяся ось с периодом 360 \\

\hline reserved & \centering{unsigned:10} & Резерв \\
\hline homeVel & \centering{double} & Скорость и направление выезда в нулевую точку \\
\hline indexVel & \centering{double} & Скорость и направление поиска индексной метки \\
\hline homeOffset & \centering{double} & Смещение нулевой точки относительно позиции ДОС \\
\hline indexOffset & \centering{double} & Смещение индексной метки относительно позиции ДОС \\
\hline homeOfsVel & \centering{double} & Скорость движения в позицию смещения нулевой точки (не используется) \\
\hline indexOfsVel & \centering{double} & Скорость движения в позицию смещения индексной метки (не используется) \\
\hline minPos & \centering{double} & Программное ограничение в отрицательном направлении (для абсолютного ДОС настраивается в дискретах датчика, определённых в энкодерной таблице) \\
\hline maxPos & \centering{double} &  Программное ограничение в положительном направлении (для абсолютного ДОС настраивается в дискретах датчика, определённых в энкодерной таблице) \\
\hline defaultTa & \centering{double} & Время в мс ускорения/замедления (при значении больше 0) или коэффициент, обратный величине амплитуды ускорения/замедления (при значении меньше 0) по умолчанию \\
\hline defaultTs & \centering{double} & Время в мс (при значении больше 0) или коэффициент, обратный значению амплитуды рывка (при значении меньше 0), для каждой половины S-кривой профиля ускорения по умолчанию \\
\hline manualTa & \centering{double} & Время в мс ускорения/замедления (при значении больше 0) или коэффициент, обратный величине амплитуды ускорения/замедления (при значении меньше 0) в ручном режиме \\
\hline manualTs & \centering{double} & Время в мс (при значении больше 0) или коэффициент, обратный значению амплитуды рывка (при значении меньше 0), для каждой половины S-кривой профиля ускорения в ручном режиме \\
\hline hwlTa & \centering{double} & Время в мс ускорения/замедления (при значении больше 0) или коэффициент, обратный величине амплитуды ускорения/замедления (при значении меньше 0) в режиме дискретных перемещений \\
\hline hwlTs & \centering{double} & Время в мс (при значении больше 0) или коэффициент, обратный значению амплитуды рывка (при значении меньше 0), для каждой половины S-кривой профиля ускорения в режиме дискретных перемещений \\
\hline homeTa & \centering{double} & Время в мс ускорения/замедления (при значении больше 0) или коэффициент, обратный величине амплитуды ускорения/замедления (при значении меньше 0) в режиме выезда в нулевую точку \\
\hline homeTs & \centering{double} &  Время в мс (при значении больше 0) или коэффициент, обратный значению амплитуды рывка (при значении меньше 0), для каждой половины S-кривой профиля ускорения в режиме выезда в нулевую точку \\
\hline autoTa & \centering{double} & Время в мс ускорения/замедления (при значении больше 0) или коэффициент, обратный величине амплитуды ускорения/замедления (при значении меньше 0) в автоматическом режиме \\
\hline autoTs & \centering{double} &  Время в мс (при значении больше 0) или коэффициент, обратный значению амплитуды рывка (при значении меньше 0), для каждой половины S-кривой профиля ускорения в автоматическом режиме \\
\hline encRes & \centering{double} &  Число дискрет датчика на оборот\\
\end{MyTableThreeColAllCntr}
% *******end subsection***************

Поля \texttt{homeCaptCtrl} и \texttt{indexCaptCtrl} являются 4-битными и содержат настройки захвата положения для выезда в нулевую точку:
\begin{itemize}
\item биты 0 и 1 определяют тип захвата положения (0 ~-- непосредственный захват, 1 ~--  по индексному сигналу датчика, 2 ~-- захват по флагу, 3 ~-- по флагу и индексному сигналу); \killoverfullbefore
\item бит 2 управляет инверсией индексного сигнала ДОС (0 ~-- не инвертировать, 1 ~-- инвертировать); \killoverfullbefore
\item бит 3 управляет инверсией флага при захвате положения (0 ~-- не инвертировать, 1 ~-- инвертировать). \killoverfullbefore \BL
\end{itemize} 

%--------------------------------------------------------
% *******begin subsection***************
\subsubsection{\DbgSecSt{\StPart}{Axis}}
\index{Программный интерфейс ПЛК!Управление осями!Структура Axis}
\label{sec:Axis}

\begin{fHeader}
    Тип данных:            & \RightHandText{Структура Axis}\\
    Файл объявления:             & \RightHandText{include/func/axis.h} \\
\end{fHeader}

Структура определяет состояние, параметры и данные оси.

\begin{MyTableThreeColAllCntr}{Структура Axis}{tbl:Axis}{|m{0.3\linewidth}|m{0.25\linewidth}|m{0.45\linewidth}|}{Элемент}{Тип}{Описание}
\hline state & \centering{\myreftosec{AxisStates}} & Текущее состояние \\
\hline command & \centering{\myreftosec{AxisCommands}} & Текущая команда \\
\hline statePreHome & \centering{\myreftosec{AxisStates}} & Состояние перед выездом в нулевую точку \\
\hline followup & \centering{unsigned:1} & Восстановление после ошибки \\
\hline phaseRefComplete & \centering{unsigned:1} & Фазировка выполнена \\
\hline phaseRefError & \centering{unsigned:1} & Ошибка фазировки \\
\hline homeComplete & \centering{unsigned:1} & Выполнен выезд в нулевую точку \\
\hline homeErrorFlag & \centering{unsigned:1} & Ошибка выезда в нулевую точку \\
\hline posRefComplete & \centering{unsigned:1} & Позиционирование при включении станка выполнено \\
\hline timer & \centering{\myreftosec{Timer}} & Таймер \\
\hline JogValue & \centering{double} & Значение заданной позиции для толчкового перемещения \\
\hline IncStep & \centering{double} & Значение заданного расстояния для толчкового перемещения \\
\hline platform & \centering{\hyperlink{Axis_Platform_Control}{AxisPlatformControl}} & Пользовательские параметры и переменные оси \\
\end{MyTableThreeColAllCntr}

\hypertarget{Axis_Platform_Control}{Структура} \texttt{AxisPlatformControl} является пользовательской и служит для введения дополнительных параметров и переменных оси.\killoverfullbefore

Если структура \texttt{AxisPlatformControl} задана пользователем, то должен быть определён идентификатор \texttt{AXES\_PLATFORM\_CONTROL\_DEFINED}: \texttt{\#define AXES\_PLATFORM\_CONTROL\_DEFINED}. \killoverfullbefore
% *******end subsection***************
%--------------------------------------------------------
% *******begin subsection***************
\subsubsection{\DbgSecSt{\StPart}{AxesControl}}
\index{Программный интерфейс ПЛК!Управление осями!Структура AxesControl}
\label{sec:AxesControl}

\begin{fHeader}
    Тип данных:            & \RightHandText{Структура AxesControl}\\
    Файл объявления:             & \RightHandText{include/func/axis.h} \\
\end{fHeader}

Структура определяет состояние, параметры и данные осей.

\begin{MyTableThreeColAllCntr}{Структура AxesControl}{tbl:AxesControl}{|m{0.3\linewidth}|m{0.25\linewidth}|m{0.45\linewidth}|}{Элемент}{Тип}{Описание}
\hline homeState & \centering{\myreftosec{HomeStates}} & Состояние выезда в нулевую точку\\
\hline homeComplete & \centering{unsigned:1} & Выполнен выезд в нулевую точку \\
\hline homeErrorFlag & \centering{unsigned:1} & Ошибка выезда в нулевую точку \\
\hline homeStage & \centering{int} & Этап выезда в нулевую точку \\
\hline axis[ЧИСЛО\_ОСЕЙ] & \centering{\myreftosec{Axis}} & Данные осей \\
\hline timerHome & \centering{\myreftosec{Timer}} & Таймер для задержек переключений состояний в режиме выезда в нулевую точку\\
\hline platform & \centering{\hyperlink{Axes_Platform_Control}{AxesPlatformControl}} & Пользовательские параметры и переменные осей \\
\hline saveSpeed & \centering{int} & Сохранённая скорость с пульта оператора \\
\hline activeAxis & \centering{int} & Номер активной оси \\
\end{MyTableThreeColAllCntr}

\hypertarget{Axes_Platform_Control}{Структура} \texttt{AxesPlatformControl} является пользовательской и служит для введения дополнительных параметров и переменных осей.\killoverfullbefore

Если структура \texttt{AxesPlatformControl} задана пользователем, то должен быть определён идентификатор \texttt{AXES\_PLATFORM\_CONTROL\_DEFINED}: \texttt{\#define AXES\_PLATFORM\_CONTROL\_DEFINED}.\killoverfullbefore
% *******end subsection***************
%-------------------------------------------------------------------
% *******begin subsection***************
\subsection{\DbgSecSt{\StPart}{Функции}}

% *******begin subsection***************
\subsubsection{\DbgSecSt{\StPart}{axesForceKill}}
\index{Программный интерфейс ПЛК!Управление осями!Функция axesForceKill}
\label{sec:axesForceKill}

\begin{pHeader}
    Синтаксис:      & \RightHandText{void axesForceKill();}\\
    Аргумент(ы):    & \RightHandText{Нет} \\    
%    Возвращаемое значение:       & \RightHandText{Нет} \\ 
    Файл объявления:             & \RightHandText{include/func/axis.h} \\       
\end{pHeader}

Функция вызывает принудительное выключение всех осей.

Является системной.
% *******end section*****************
%-------------------------------------------------------------------
% *******begin subsection***************
\subsubsection{\DbgSecSt{\StPart}{axisForceKill}}
\index{Программный интерфейс ПЛК!Управление осями!Функция axisForceKill}
\label{sec:axisForceKill}

\begin{pHeader}
    Синтаксис:      & \RightHandText{void axisForceKill(unsigned axis);}\\
    Аргумент(ы):    & \RightHandText{unsigned axis ~-- номер оси} \\    
%    Возвращаемое значение:       & \RightHandText{Нет} \\ 
    Файл объявления:             & \RightHandText{include/func/axis.h} \\       
\end{pHeader}

Функция вызывает принудительное выключение оси, номер которой является аргументом функции.

Является системной.
% *******end section*****************

%-------------------------------------------------------------------
% *******begin subsection***************
\subsubsection{\DbgSecSt{\StPart}{axesDeactivate}}
\index{Программный интерфейс ПЛК!Управление осями!Функция axesDeactivate}
\label{sec:axesDeactivate}

\begin{pHeader}
    Синтаксис:      & \RightHandText{void axesDeactivate();}\\
    Аргумент(ы):    & \RightHandText{Нет} \\    
%    Возвращаемое значение:       & \RightHandText{Нет} \\ 
    Файл объявления:             & \RightHandText{include/func/axis.h} \\       
\end{pHeader}

Функция вызывает выключение всех осей.

Является системной.
% *******end section*****************
%-------------------------------------------------------------------
% *******begin subsection***************
\subsubsection{\DbgSecSt{\StPart}{axesActivate}}
\index{Программный интерфейс ПЛК!Управление осями!Функция axesActivate}
\label{sec:axesActivate}

\begin{pHeader}
    Синтаксис:      & \RightHandText{void axesActivate();}\\
    Аргумент(ы):    & \RightHandText{Нет} \\    
%    Возвращаемое значение:       & \RightHandText{Нет} \\ 
    Файл объявления:             & \RightHandText{include/func/axis.h} \\
\end{pHeader}

Функция вызывает включение в слежение всех осей.

Является системной.
% *******end section*****************

%-------------------------------------------------------------------
% *******begin subsection***************
\subsubsection{\DbgSecSt{\StPart}{axesInactive}}
\index{Программный интерфейс ПЛК!Управление осями!Функция axesInactive}
\label{sec:axesInactive}

\begin{pHeader}
    Синтаксис:      & \RightHandText{int axesInactive();}\\
    Аргумент(ы):    & \RightHandText{Нет} \\    
%    Возвращаемое значение:       & \RightHandText{Целое знаковое число} \\ 
    Файл объявления:             & \RightHandText{include/func/axis.h} \\       
\end{pHeader}

Функция возвращает 1, если хотя бы одна ось не находится в слежении, и 0 в противном случае.

Является системной.
% *******end section*****************
%-------------------------------------------------------------------
% *******begin subsection***************
\subsubsection{\DbgSecSt{\StPart}{axesActive}}
\index{Программный интерфейс ПЛК!Управление осями!Функция axesActive}
\label{sec:axesActive}

\begin{pHeader}
    Синтаксис:      & \RightHandText{int axesActive();}\\
    Аргумент(ы):    & \RightHandText{Нет} \\    
%    Возвращаемое значение:       & \RightHandText{Целое знаковое число} \\ 
    Файл объявления:             & \RightHandText{include/func/axis.h} \\       
\end{pHeader}

Функция возвращает 1, если все оси находятся в слежении, и 0 в противном случае.

Является системной.
% *******end section*****************
%-------------------------------------------------------------------
% *******begin subsection***************
\subsubsection{\DbgSecSt{\StPart}{axesPhaseRefComplete}}
\index{Программный интерфейс ПЛК!Управление осями!Функция axesPhaseRefComplete}
\label{sec:axesPhaseRefComplete}

\begin{pHeader}
    Синтаксис:      & \RightHandText{int axesPhaseRefComplete();}\\
    Аргумент(ы):    & \RightHandText{Нет} \\    
%    Возвращаемое значение:       & \RightHandText{Целое знаковое число} \\ 
    Файл объявления:             & \RightHandText{include/func/axis.h} \\       
\end{pHeader}

Функция возвращает 1, если фазировка выполнена для всех осей, и 0 в противном случае.

Является системной.
% *******end section*****************
%-------------------------------------------------------------------
% *******begin subsection***************
\subsubsection{\DbgSecSt{\StPart}{axesPhaseRef}}
\index{Программный интерфейс ПЛК!Управление осями!Функция axesPhaseRef}
\label{sec:axesPhaseRef}

\begin{pHeader}
    Синтаксис:      & \RightHandText{int axesPhaseRef();}\\
    Аргумент(ы):    & \RightHandText{Нет} \\    
%    Возвращаемое значение:       & \RightHandText{Целое знаковое число} \\ 
    Файл объявления:             & \RightHandText{include/func/axis.h} \\
\end{pHeader}

Функция возвращает 1, если хотя бы одна ось требует фазировки, и 0 в противном случае. Для оси, фазировка которой не выполнена, даётся команда фазировки.

Является системной.
% *******end section*****************
%-------------------------------------------------------------------
% *******begin subsection***************
\subsubsection{\DbgSecSt{\StPart}{axesAborted}}
\index{Программный интерфейс ПЛК!Управление осями!Функция axesAborted}
\label{sec:axesAborted}

\begin{pHeader}
    Синтаксис:      & \RightHandText{int axesAborted();}\\
    Аргумент(ы):    & \RightHandText{Нет} \\    
%    Возвращаемое значение:       & \RightHandText{Целое знаковое число} \\ 
    Файл объявления:             & \RightHandText{include/func/axis.h} \\       
\end{pHeader}

Функция возвращает 1, если все оси аварийно остановлены, и 0 в противном случае.

Является системной.
% *******end section*****************
%-------------------------------------------------------------------
% *******begin subsection***************
\subsubsection{\DbgSecSt{\StPart}{axisStopped}}
\index{Программный интерфейс ПЛК!Управление осями!Функция axisStopped}
\label{sec:axisStopped}

\begin{pHeader}
    Синтаксис:      & \RightHandText{int axisStopped(unsigned axis);}\\
    Аргумент(ы):    & \RightHandText{unsigned axis ~-- номер оси} \\    
%    Возвращаемое значение:       & \RightHandText{Целое знаковое число} \\
    Файл объявления:             & \RightHandText{include/func/axis.h} \\
\end{pHeader}

Функция возвращает 1, если ось остановлена (ось в слежении имеет равную нулю заданную скорость и находится в позиции), и 0 в противном случае.  

Является системной.
% *******end section*****************
%-------------------------------------------------------------------
% *******begin subsection***************
\subsubsection{\DbgSecSt{\StPart}{axesStopped}}
\index{Программный интерфейс ПЛК!Управление осями!Функция axesStopped}
\label{sec:axesStopped}

\begin{pHeader}
    Синтаксис:      & \RightHandText{int axesStopped();}\\
    Аргумент(ы):    & \RightHandText{Нет} \\    
%    Возвращаемое значение:       & \RightHandText{Целое знаковое число} \\ 
    Файл объявления:             & \RightHandText{include/func/axis.h} \\       
\end{pHeader}

Функция возвращает 1, если все оси остановлены (оси в слежении имеют равную нулю заданную скорость и находятся в позиции), и 0 в противном случае.

Является системной.
% *******end section*****************
%-------------------------------------------------------------------
% *******begin subsection***************
\subsubsection{\DbgSecSt{\StPart}{axesAbortAll}}
\index{Программный интерфейс ПЛК!Управление осями!Функция axesAbortAll}
\label{sec:axesAbortAll}

\begin{pHeader}
    Синтаксис:      & \RightHandText{void axesAbortAll();}\\
    Аргумент(ы):    & \RightHandText{Нет} \\    
%    Возвращаемое значение:       & \RightHandText{Нет} \\ 
    Файл объявления:             & \RightHandText{include/func/axis.h} \\
\end{pHeader}

Функция вызывает аварийное выключение всех осей.

Является системной.
% *******end section*****************
%-------------------------------------------------------------------
% *******begin subsection***************
\subsubsection{\DbgSecSt{\StPart}{axesStopAll}}
\index{Программный интерфейс ПЛК!Управление осями!Функция axesStopAll}
\label{sec:axesStopAll}

\begin{pHeader}
    Синтаксис:      & \RightHandText{void axesStopAll();}\\
    Аргумент(ы):    & \RightHandText{Нет} \\    
%    Возвращаемое значение:       & \RightHandText{Нет} \\ 
    Файл объявления:             & \RightHandText{include/func/axis.h} \\
\end{pHeader}

Функция вызывает останов всех осей при толчковых перемещениях.

Является системной.
% *******end section*****************
%--------------------------------------------------------
% *******begin subsection***************
\subsubsection{\DbgSecSt{\StPart}{axesRet}}
\index{Программный интерфейс ПЛК!Управление осями!Функция axesRet}
\label{sec:axesRet}

\begin{pHeader}
    Синтаксис:      & \RightHandText{void axesRet();}\\
    Аргумент(ы):    & \RightHandText{Нет} \\    
%    Возвращаемое значение:       & \RightHandText{Нет} \\ 
    Файл объявления:             & \RightHandText{include/func/axis.h} \\
\end{pHeader}

Функция вызывает перемещение осей в сохраненную позицию при толчковых перемещениях.

Является системной.
% *******end section*****************

%--------------------------------------------------------
% *******begin subsection***************
\subsubsection{\DbgSecSt{\StPart}{axisIndexInit}}
\index{Программный интерфейс ПЛК!Управление осями!Функция axisIndexInit}
\label{sec:axisIndexInit}

\begin{pHeader}
    Синтаксис:      & \RightHandText{void axisIndexInit(unsigned axis);}\\
    Аргумент(ы):    & \RightHandText{unsigned axis ~-- номер оси} \\    
%    Возвращаемое значение:       & \RightHandText{Нет} \\ 
    Файл объявления:             & \RightHandText{include/func/axis.h} \\
\end{pHeader}

%Функция выполняет запрос выезда оси, номер которой является аргументом функции, в нулевую точку по индексной метке. 

Функция выполняет инициализацию параметров поиска индексной метки для оси, номер которой является аргументом функции.

Является системной.
% *******end section*****************

%--------------------------------------------------------
% *******begin subsection***************
\subsubsection{\DbgSecSt{\StPart}{axisPosition}}
\index{Программный интерфейс ПЛК!Управление осями!Функция axisPosition}
\label{sec:axisPosition}

\begin{pHeader}
    Синтаксис:      & \RightHandText{double axisPosition(unsigned axis);}\\
    Аргумент(ы):    & \RightHandText{unsigned axis ~-- номер оси} \\    
%    Возвращаемое значение:       & \RightHandText{Число с плавающей запятой двойной точности} \\ 
    Файл объявления:             & \RightHandText{include/func/axis.h} \\
\end{pHeader}

Функция возвращает заданную позицию оси, номер которой является аргументом функции.

Возвращаемое значение измеряется в единицах \texttt{encRes} (см. структуру \myreftosec{AxisConfig}). Для вращающихся осей возвращаемое значение ~-- остаток от деления на 360. \killoverfullbefore

Является системной.
% *******end section*****************
%--------------------------------------------------------
% *******begin subsection***************
\subsubsection{\DbgSecSt{\StPart}{axesFollowup}}
\index{Программный интерфейс ПЛК!Управление осями!Функция axesFollowup}
\label{sec:axesFollowup}

\begin{pHeader}
    Синтаксис:      & \RightHandText{void axesFollowup();}\\
    Аргумент(ы):    & \RightHandText{Нет} \\    
%    Возвращаемое значение:       & \RightHandText{Нет} \\ 
    Файл объявления:             & \RightHandText{include/func/axis.h} \\
\end{pHeader}

Функция устанавливает для всех осей флаг <<Восстановление после ошибки>> (см. структуру \myreftosec{Axis}).\killoverfullbefore

Является системной.
% *******end section*****************
%--------------------------------------------------------
% *******begin subsection***************
\subsubsection{\DbgSecSt{\StPart}{initAxis}}
\index{Программный интерфейс ПЛК!Управление осями!Функция initAxis}
\label{sec:initAxis}

\begin{pHeader}
    Синтаксис:      & \RightHandText{void initAxis(int axis);}\\
    Аргумент(ы):    & \RightHandText{int axis ~-- номер оси} \\    
%    Возвращаемое значение:       & \RightHandText{Нет} \\ 
    Файл объявления:             & \RightHandText{include/func/axis.h} \\
\end{pHeader}

Функция выполняет инициализацию оси, номер которой является аргументом функции, параметрами по умолчанию. 

Является системной.
% *******end section*****************
%--------------------------------------------------------
% *******begin subsection***************
\subsubsection{\DbgSecSt{\StPart}{initAxes}}
\index{Программный интерфейс ПЛК!Управление осями!Функция initAxes}
\label{sec:initAxes}

\begin{pHeader}
    Синтаксис:      & \RightHandText{void initAxes();}\\
    Аргумент(ы):    & \RightHandText{Нет} \\    
%    Возвращаемое значение:       & \RightHandText{Нет} \\ 
    Файл объявления:             & \RightHandText{include/func/axis.h} \\
\end{pHeader}

Функция выполняет инициализацию осей параметрами по умолчанию. 

Является системной.
% *******end section*****************
%--------------------------------------------------------
% *******begin subsection***************
\subsubsection{\DbgSecSt{\StPart}{axisInitPlatform}}
\index{Программный интерфейс ПЛК!Управление осями!Функция axisInitPlatform}
\label{sec:axisInitPlatform}

\begin{pHeader}
    Синтаксис:      & \RightHandText{void axisInitPlatform(int axis);}\\
    Аргумент(ы):    & \RightHandText{int axis ~-- номер оси} \\    
%    Возвращаемое значение:       & \RightHandText{Нет} \\ 
    Файл объявления:             & \RightHandText{include/func/axis.h} \\
\end{pHeader}

Функция выполняет инициализацию параметров оси, номер которой является аргументом функции, пользовательскими значениями, в том числе структуры \hyperlink{Axis_Platform_Control}{AxisPlatformControl}. \killoverfullbefore

Реализуется пользователем.
% *******end section*****************
%--------------------------------------------------------
% *******begin subsection***************
\subsubsection{\DbgSecSt{\StPart}{axesInitPlatform}}
\index{Программный интерфейс ПЛК!Управление осями!Функция axesInitPlatform}
\label{sec:axesInitPlatform}

\begin{pHeader}
    Синтаксис:      & \RightHandText{void axesInitPlatform();}\\
    Аргумент(ы):    & \RightHandText{Нет} \\    
%    Возвращаемое значение:       & \RightHandText{Нет} \\ 
    Файл объявления:             & \RightHandText{include/func/axis.h} \\
\end{pHeader}

Функция выполняет инициализацию параметров осей пользовательскими значениями, в том числе структуры \hyperlink{Axes_Platform_Control}{AxesPlatformControl}. \killoverfullbefore

Реализуется пользователем.
% *******end section*****************
%--------------------------------------------------------
% *******begin subsection***************
\subsubsection{\DbgSecSt{\StPart}{axesAbsPosRead}}
\index{Программный интерфейс ПЛК!Управление осями!Функция axesAbsPosRead}
\label{sec:axesAbsPosRead}

\begin{pHeader}
    Синтаксис:      & \RightHandText{void axesAbsPosRead();}\\
    Аргумент(ы):    & \RightHandText{Нет} \\    
%    Возвращаемое значение:       & \RightHandText{Нет} \\ 
    Файл объявления:             & \RightHandText{include/func/axis.h} \\
\end{pHeader}

Функция выполняет запрос чтения данных абсолютных ДОС для осей.

Является системной.
% *******end section*****************
%-------------------------------------------------------------------
% *******begin subsection***************
\subsubsection{\DbgSecSt{\StPart}{axesAbsPosReadComplete}}
\index{Программный интерфейс ПЛК!Управление осями!Функция axesAbsPosReadComplete}
\label{sec:axesAbsPosReadComplete}

\begin{pHeader}
    Синтаксис:      & \RightHandText{int axesAbsPosReadComplete();}\\
    Аргумент(ы):    & \RightHandText{Нет} \\    
%    Возвращаемое значение:       & \RightHandText{Целое знаковое число} \\ 
    Файл объявления:             & \RightHandText{include/func/axis.h} \\       
\end{pHeader}

Функция возвращает 1, если чтение данных абсолютных ДОС для осей завершено, и 0 в противном случае.

Является системной.
% *******end section*****************
%-------------------------------------------------------------------
% *******begin subsection***************
\subsubsection{\DbgSecSt{\StPart}{axisRefPosComplete}}
\index{Программный интерфейс ПЛК!Управление осями!Функция axisRefPosComplete}
\label{sec:axisRefPosComplete}

\begin{pHeader}
    Синтаксис:      & \RightHandText{int axisRefPosComplete(unsigned axis);}\\
    Аргумент(ы):   & \RightHandText{unsigned axis ~-- номер оси}\\
%    Возвращаемое значение:       & \RightHandText{Целое знаковое число} \\ 
    Файл объявления:             & \RightHandText{include/func/axis.h} \\       
\end{pHeader}

Функция возвращает 1, если требуется и выполнено позиционирование оси, номер которой является аргументом функции, при включении станка или запуске программы, и 0 в противном случае. \killoverfullbefore

Является системной.
% *******end section*****************
%-------------------------------------------------------------------
% *******begin subsection***************
\subsubsection{\DbgSecSt{\StPart}{axesRefPosComplete}}
\index{Программный интерфейс ПЛК!Управление осями!Функция axesRefPosComplete}
\label{sec:axesRefPosComplete}

\begin{pHeader}
    Синтаксис:      & \RightHandText{int axesRefPosComplete();}\\
    Аргумент(ы):    & \RightHandText{Нет} \\    
%    Возвращаемое значение:       & \RightHandText{Целое знаковое число} \\ 
    Файл объявления:             & \RightHandText{include/func/axis.h} \\       
\end{pHeader}

Функция возвращает 1, если требуется и выполнено позиционирование всех осей при включении станка или запуске программы, и 0 в противном случае. \killoverfullbefore

Является системной.
% *******end section*****************
%-------------------------------------------------------------------
% *******begin subsection***************
\subsubsection{\DbgSecSt{\StPart}{axisAtRefPos}}
\index{Программный интерфейс ПЛК!Управление осями!Функция axisAtRefPos}
\label{sec:axisAtRefPos}

\begin{pHeader}
    Синтаксис:      & \RightHandText{int axisAtRefPos(unsigned axis);}\\
    Аргумент(ы):    & \RightHandText{unsigned axis ~-- номер оси} \\    
%%    Возвращаемое значение:       & \RightHandText{Целое знаковое число} \\ 
    Файл объявления:             & \RightHandText{include/func/axis.h} \\       
\end{pHeader}

Функция возвращает 1, если ось, номер которой является аргументом функции, находится в первой референтной позиции, и 0 в противном случае. \killoverfullbefore

Является системной.
% *******end section*****************
%-------------------------------------------------------------------
% *******begin subsection***************
\subsubsection{\DbgSecSt{\StPart}{axesAtRefPos}}
\index{Программный интерфейс ПЛК!Управление осями!Функция axesAtRefPos}
\label{sec:axesAtRefPos}

\begin{pHeader}
    Синтаксис:      & \RightHandText{int axesAtRefPos();}\\
    Аргумент(ы):    & \RightHandText{Нет} \\    
%%    Возвращаемое значение:       & \RightHandText{Целое знаковое число} \\ 
    Файл объявления:             & \RightHandText{include/func/axis.h} \\       
\end{pHeader}

Функция возвращает 1, если все оси находятся в первой референтной позиции, и 0 в противном случае. 

Является системной.
% *******end section*****************
%--------------------------------------------------------
                % API: управление осями
    %--------------------------------------------------------
% *******begin section***************
\section{\DbgSecSt{\StPart}{Управление шпинделями}}
%--------------------------------------------------------
\subsection{\DbgSecSt{\StPart}{Типы данных}}

% *******begin subsection***************
\subsubsection{\DbgSecSt{\StPart}{SpindleStates}}
\index{Программный интерфейс ПЛК!Управление шпинделями!Перечисление SpindleStates}
\label{sec:SpindleStates}

\begin{fHeader}
    Тип данных:            & \RightHandText{Перечисление SpindleStates}\\
    Файл объявления:             & \RightHandText{include/func/spnd.h} \\
\end{fHeader}

Перечисление определяет идентификаторы состояний шпинделя.

\begin{MyTableTwoColAllCntr}{Перечисление SpindleStates}{tbl:SpindleStates}{|m{0.38\linewidth}|m{0.57\linewidth}|}{Идентификатор}{Описание}
\hline spndInactive & Шпиндель выключен  \\
\hline spndActive &  Шпиндель находится в слежении \\
\hline spndCW & Вращение по часовой стрелке\\
\hline spndCCW & Вращение против часовой стрелки \\
\hline spndStopping & Останов шпинделя \\
\hline spndHomeWaitHW & Ожидание применения аппаратных настроек выезда в нулевую точку \\
\hline spndHoming & Выезд в нулевую точку \\
\hline spndIndexWaitHW & Ожидание применения аппаратных настроек поиска индексной метки \\
\hline spndIndexing & Поиск индексной метки \\
\hline spndAborting & Аварийное торможение \\
\hline spndWaitActivate & Ожидание включения \\
\hline spndWaitDeactivate & Ожидание выключения \\
\hline spndWaitPhaseRef & Ожидание фазировки \\
\end{MyTableTwoColAllCntr}
% *******end subsection***************
%--------------------------------------------------------
% *******begin subsection***************
\subsubsection{\DbgSecSt{\StPart}{SpindleCommands}}
\index{Программный интерфейс ПЛК!Управление шпинделями!Перечисление SpindleCommands}
\label{sec:SpindleCommands}

\begin{fHeader}
    Тип данных:            & \RightHandText{Перечисление SpindleCommands}\\
    Файл объявления:             & \RightHandText{include/func/spnd.h} \\
\end{fHeader}

Перечисление определяет идентификаторы команд управления шпинделями. 

\begin{MyTableTwoColAllCntr}{Перечисление SpindleCommands}{tbl:SpindleCommands}{|m{0.38\linewidth}|m{0.57\linewidth}|}{Идентификатор}{Описание}
\hline spndCmdIdle &  Нет команды  \\
\hline spndCmdKill  &  Выключить шпиндель \\
\hline spndCmdActivate  &  Включить шпиндель в слежение \\
\hline spndCmdDeactivate  &  Выключить шпиндель  \\
\hline spndCmdCW &  Выполнить вращение по часовой стрелке \\
\hline spndCmdCCW & Выполнить вращение против часовой стрелки \\
\hline spndCmdStop & Выполнить останов \\
\hline spndCmdInc  & Выполнить поворот на заданный угол \\
\hline spndCmdAbs & Выполнить поворот в заданное положение \\
\hline spndCmdHome & Выполнить движение в нулевую точку \\
\hline spndCmdIndex & Выполнить движение до индексной метки \\
\hline spndCmdPhaseRef & Выполнить фазировку \\
\hline spndCmdAbort & Выполнить аварийное выключение \\
\end{MyTableTwoColAllCntr}
% *******end subsection***************
%--------------------------------------------------------
% *******begin subsection***************
\subsubsection{\DbgSecSt{\StPart}{SpindleStage}}
\index{Программный интерфейс ПЛК!Управление шпинделями!Структура SpindleStage}
\label{sec:SpindleStage}

\begin{fHeader}
    Тип данных:            & \RightHandText{Структура SpindleStage} \\
    Файл объявления:             & \RightHandText{include/func/spnd.h} \\
\end{fHeader}

Структура определяет настройки ступенчатого разгона шпинделя.

\begin{MyTableThreeColAllCntr}{Структура SpindleStage}{tbl:SpindleStage}{|m{0.33\linewidth}|m{0.22\linewidth}|m{0.45\linewidth}|}{Элемент}{Тип}{Описание}
\hline Vmin & \centering{double} & Минимальная скорость \\
\hline Vmax & \centering{doubled} & Максимальная скорость \\
\hline Amax & \centering{double} & Максимальное ускорение \\
\hline Jmax & \centering{double} & Максимальный рывок \\
\end{MyTableThreeColAllCntr}
% *******end subsection***************
%--------------------------------------------------------
%--------------------------------------------------------
% *******begin subsection***************
\subsubsection{\DbgSecSt{\StPart}{SpindleConfig}}
\index{Программный интерфейс ПЛК!Управление шпинделями!Структура SpindleConfig}
\label{sec:SpindleConfig}

\begin{fHeader}
    Тип данных:            & \RightHandText{Структура SpindleConfig} \\
    Файл объявления:             & \RightHandText{include/func/spnd.h} \\
\end{fHeader}

Структура определяет настройки шпинделя.

\begin{MyTableThreeColAllCntr}{Структура SpindleConfig}{tbl:SpindleConfig}{|m{0.35\linewidth}|m{0.2\linewidth}|m{0.45\linewidth}|}{Элемент}{Тип}{Описание}
\hline servo & \centering{unsigned} & Номер платы управления ($0\div3$) \\
\hline chan & \centering{unsigned} & Номер канала ($0\div7$) \\
\hline motor & \centering{unsigned} & Номер связанного с осью двигателя ($0\div31$)\\
\hline homeOrder & \centering{unsigned} & Порядок выезда в нулевую точку \\
\hline needDKill & \centering{Битовое поле:1} & Требуется задержка перед отключением \\
\hline needPhaseRef & \centering{Битовое поле:1} & Требуется фазировка \\

\hline needHome & \centering{Битовое поле:1} & Требуется выезд в нулевую точку \\
\hline needIndex & \centering{Битовое поле:1} & Требуется позиционирование по индексной метке \\
\hline hasAbsPos & \centering{Битовое поле:1} & Установлен абсолютный датчик \\

\hline abortMode & \centering{Битовое поле:2} & Реакции на команду аварийного выключения (см. \myreftosec{AxisAbortMode})\\
\hline homeCaptCtrl & \centering{Битовое поле:4} & Настройка CaptCtrl для выезда в нулевую точку по входу (флагу) \\
\hline indexCaptCtrl & \centering{Битовое поле:4} & Настройка CaptCtrl для выезда в нулевую точку по индексной метке \\
\hline killAfterStop & \centering{Битовое поле:1} &  Выключение после останова \\
\hline reserved & \centering{Битовое поле:16} &  Резерв \\
\hline homeVel & \centering{double} & Скорость выезда в нулевую точку \\
\hline indexVel & \centering{double} & Скорость поиска индексной метки \\
\hline homeOffset & \centering{double} & Смещение нулевой точки относительно позиции ДОС\\
\hline indexOffset & \centering{double} & Смещение индексной метки относительно позиции ДОС\\
\hline homeOfsVel & \centering{double} & Скорость движения в позицию смещения нулевой точки (не используется)\\
\hline indexOfsVel & \centering{double} & Скорость движения в позицию смещения индексной метки (не используется)\\
\hline minPos & \centering{double} & Программное ограничение в отрицательном направлении (для абсолютного ДОС настраивается в дискретах датчика, определённых в энкодерной таблице) \\
\hline maxPos & \centering{double} &  Программное ограничение в положительном направлении (для абсолютного ДОС настраивается в дискретах датчика, определённых в энкодерной таблице) \\
\hline spinEncRes & \centering{double} & Число дискрет датчика на оборот \\
\hline atSpeedBand & \centering{double} & Амплитуда зоны ошибки заданной скорости \\
\hline stages\newline[ЧИСЛО\_СТУПЕНЕЙ\_РАЗГОНА] & \centering{\myreftosec{SpindleStage}} & Параметры ступенчатого разгона \\
\end{MyTableThreeColAllCntr}
% *******end subsection***************

Поля \texttt{homeCaptCtrl} и \texttt{indexCaptCtrl} являются 4-битными и содержат настройки захвата положения для выезда в нулевую точку:
\begin{itemize}
\item биты 0 и 1 определяют тип захвата положения (0 ~-- непосредственный захват, 1 ~--  по индексному сигналу датчика, 2 ~-- захват по флагу, 3 ~-- по флагу и индексному сигналу); \killoverfullbefore
\item бит 2 управляет инверсией индексного сигнала ДОС (0 ~-- не инвертировать, 1 ~-- инвертировать); \killoverfullbefore
\item бит 3 управляет инверсией флага при захвате положения (0 ~-- не инвертировать, 1 ~-- инвертировать). \killoverfullbefore \BL
\end{itemize} 
%--------------------------------------------------------
% *******begin subsection***************
\subsubsection{\DbgSecSt{\StPart}{Spindle}}
\index{Программный интерфейс ПЛК!Управление шпинделями!Структура Spindle}
\label{sec:Spindle}

\begin{fHeader}
    Тип данных:            & \RightHandText{Структура Spindle}\\
    Файл объявления:             & \RightHandText{include/func/spnd.h} \\
\end{fHeader}

Структура определяет состояние, параметры и данные шпинделя.

\begin{MyTableThreeColAllCntr}{Структура Spindle}{tbl:Spindle}{|m{0.31\linewidth}|m{0.29\linewidth}|m{0.4\linewidth}|}{Элемент}{Тип}{Описание}
\hline state & \centering{\myreftosec{SpindleStates}} & Текущее состояние \\
\hline command & \centering{\myreftosec{SpindleCommands}} & Текущая команда \\
\hline commandAfterActivate & \centering{\myreftosec{SpindleCommands}} & Команда после включения \\
\hline followup & \centering{Битовое поле:1} & Восстановление после ошибки \\
\hline phaseRefComplete & \centering{Битовое поле:1} & Фазировка выполнена \\
\hline phaseRefError & \centering{Битовое поле:1} & Ошибка фазировки \\
\hline homeComplete & \centering{Битовое поле:1} & Выполнен выезд в нулевую точку \\
\hline homeErrorFlag & \centering{Битовое поле:1} & Ошибка выезда в нулевую точку \\
\hline atSpeed & \centering{Битовое поле:1} & Заданная скорость достигнута \\
\hline timer & \centering{\myreftosec{Timer}} & Таймер \\
\hline SpeedValue & \centering{double} &  Значение скорости \\
\hline SpeedOverride & \centering{double} & Значение корректора скорости \\
\hline spinStage & \centering{unsigned} & Номер ступени разгона \\
\hline platform & \centering{\hyperlink{Spindle_Platform_Control}{SpindlePlatformControl}} & Пользовательские параметры и переменные шпинделя \\
\end{MyTableThreeColAllCntr}

\hypertarget{Spindle_Platform_Control}{Структура} \texttt{SpindlePlatformControl} является пользовательской и служит для введения дополнительных параметров и переменных шпинделя.\killoverfullbefore

Если структура \texttt{SpindlePlatformControl} задана пользователем, то должен быть определён идентификатор \texttt{SPINDLE\_PLATFORM\_CONTROL\_DEFINED}: \texttt{\#define SPINDLE\_PLATFORM\_CONTROL\_DEFINED}. \killoverfullbefore
% *******end subsection***************
%--------------------------------------------------------
% *******begin subsection***************
\subsubsection{\DbgSecSt{\StPart}{SpindleControl}}
\index{Программный интерфейс ПЛК!Управление шпинделями!Структура SpindleControl}
\label{sec:SpindleControl}

\begin{fHeader}
    Тип данных:            & \RightHandText{Структура SpindleControl}\\
    Файл объявления:             & \RightHandText{include/func/spnd.h} \\
\end{fHeader}

Структура определяет состояние, параметры и данные шпинделей. 

\begin{MyTableThreeColAllCntr}{Структура SpindleControl}{tbl:SpindleControl}{|m{0.31\linewidth}|m{0.29\linewidth}|m{0.4\linewidth}|}{Элемент}{Тип}{Описание}
\hline homeState & \centering{\myreftosec{HomeStates}} & Состояние выезда в нулевую точку\\
\hline homeComplete & \centering{Битовое поле:1} & Выполнен выезд в нулевую точку \\
\hline homeErrorFlag & \centering{Битовое поле:1} & Ошибка выезда в нулевую точку \\
\hline homeStage & \centering{int} & Этап выезда в нулевую точку \\
\hline spin[ЧИСЛО\_ШПИНДЕЛЕЙ] & \centering{\myreftosec{Spindle}} & Данные шпинделя(-ей) \\
\hline timerHome & \centering{\myreftosec{Timer}} & Таймер для задержек переключений состояний в режиме выезда в нулевую точку\\
\hline platform & \centering{\hyperlink{Spindles_Platform_Control}{SpindlesPlatformControl}} & Пользовательские параметры и переменные шпинделей \\
\end{MyTableThreeColAllCntr}

\hypertarget{Spindles_Platform_Control}{Структура} \texttt{SpindlesPlatformControl} является пользовательской и служит для введения дополнительных параметров и переменных шпинделей.\killoverfullbefore

Если структура \texttt{SpindlesPlatformControl} задана пользователем, то должен быть определён идентификатор \texttt{SPINDLE\_PLATFORM\_CONTROL\_DEFINED}: \texttt{\#define SPINDLE\_PLATFORM\_CONTROL\_DEFINED}. \killoverfullbefore
% *******end subsection***************

%-------------------------------------------------------------------
% *******begin subsection***************
\subsection{\DbgSecSt{\StPart}{Функции}}

% *******begin subsection***************
\subsubsection{\DbgSecSt{\StPart}{spinsForceKill}}
\index{Программный интерфейс ПЛК!Управление шпинделями!Функция spinsForceKill}
\label{sec:spinsForceKill}

\begin{pHeader}
    Синтаксис:      & \RightHandText{void spinsForceKill();}\\
    Аргумент(ы):    & \RightHandText{Нет} \\    
%    Возвращаемое значение:       & \RightHandText{Нет} \\ 
    Файл объявления:             & \RightHandText{include/func/spnd.h} \\       
\end{pHeader}

Функция вызывает принудительное выключение всех шпинделей.

Является системной.
% *******end section*****************

%-------------------------------------------------------------------
% *******begin subsection***************
\subsubsection{\DbgSecSt{\StPart}{spinForceKill}}
\index{Программный интерфейс ПЛК!Управление шпинделями!Функция spinForceKill}
\label{sec:spinForceKill}

\begin{pHeader}
    Синтаксис:      & \RightHandText{void spinForceKill(unsigned spin);}\\
    Аргумент(ы):    & \RightHandText{unsigned spin ~-- номер шпинделя} \\ 
%    Возвращаемое значение:       & \RightHandText{Нет} \\ 
    Файл объявления:             & \RightHandText{include/func/spnd.h} \\       
\end{pHeader}

Функция вызывает принудительное выключение шпинделя, номер которого является аргументом функции.

Является системной.
% *******end section*****************
%-------------------------------------------------------------------
% *******begin subsection***************
\subsubsection{\DbgSecSt{\StPart}{spinsDeactivate}}
\index{Программный интерфейс ПЛК!Управление шпинделями!Функция spinsDeactivate}
\label{sec:spinsDeactivate}

\begin{pHeader}
    Синтаксис:      & \RightHandText{void spinsDeactivate();}\\
    Аргумент(ы):    & \RightHandText{Нет} \\    
%    Возвращаемое значение:       & \RightHandText{Нет} \\ 
    Файл объявления:             & \RightHandText{include/func/spnd.h} \\       
\end{pHeader}

Функция вызывает выключение всех шпинделей.

Является системной.
% *******end section*****************
%-------------------------------------------------------------------
% *******begin subsection***************
\subsubsection{\DbgSecSt{\StPart}{spinsActivate}}
\index{Программный интерфейс ПЛК!Управление шпинделями!Функция spinsActivate}
\label{sec:spinsActivate}

\begin{pHeader}
    Синтаксис:      & \RightHandText{void spinsActivate();}\\
    Аргумент(ы):    & \RightHandText{Нет} \\    
%    Возвращаемое значение:       & \RightHandText{Нет} \\ 
    Файл объявления:             & \RightHandText{include/func/spnd.h} \\
\end{pHeader}

Функция вызывает включение в слежение всех шпинделей.

Является системной.
% *******end section*****************
\clearpage
%-------------------------------------------------------------------
% *******begin subsection***************
\subsubsection{\DbgSecSt{\StPart}{spinsInactive}}
\index{Программный интерфейс ПЛК!Управление шпинделями!Функция spinsInactive}
\label{sec:spinsInactive}

\begin{pHeader}
    Синтаксис:      & \RightHandText{int spinsInactive();}\\
    Аргумент(ы):    & \RightHandText{Нет} \\    
%    Возвращаемое значение:       & \RightHandText{Целое знаковое число} \\ 
    Файл объявления:             & \RightHandText{include/func/spnd.h} \\       
\end{pHeader}

Функция возвращает 1, если хотя бы один шпиндель не находится в слежении, и 0 в противном случае.

Является системной.
% *******end section*****************
%-------------------------------------------------------------------
% *******begin subsection***************
\subsubsection{\DbgSecSt{\StPart}{spinsActive}}
\index{Программный интерфейс ПЛК!Управление шпинделями!Функция spinsActive}
\label{sec:spinsActive}

\begin{pHeader}
    Синтаксис:      & \RightHandText{int spinsActive();}\\
    Аргумент(ы):    & \RightHandText{Нет} \\    
%    Возвращаемое значение:       & \RightHandText{Целое знаковое число} \\ 
    Файл объявления:             & \RightHandText{include/func/spnd.h} \\       
\end{pHeader}

Функция возвращает 1, если все шпиндели находятся в слежении, и 0 в противном случае.

Является системной.
% *******end section*****************
%-------------------------------------------------------------------
% *******begin subsection***************
\subsubsection{\DbgSecSt{\StPart}{spinsPhaseRefComplete}}
\index{Программный интерфейс ПЛК!Управление шпинделями!Функция spinsPhaseRefComplete}
\label{sec:spinsPhaseRefComplete}

\begin{pHeader}
    Синтаксис:      & \RightHandText{int spinsPhaseRefComplete();}\\
    Аргумент(ы):    & \RightHandText{Нет} \\    
%    Возвращаемое значение:       & \RightHandText{Целое знаковое число} \\ 
    Файл объявления:             & \RightHandText{include/func/spnd.h} \\       
\end{pHeader}

Функция возвращает 1, если фазировка выполнена для всех шпинделей, и 0 в противном случае.

Является системной.
% *******end section*****************
%-------------------------------------------------------------------
% *******begin subsection***************
\subsubsection{\DbgSecSt{\StPart}{spinsPhaseRef}}
\index{Программный интерфейс ПЛК!Управление шпинделями!Функция spinsPhaseRef}
\label{sec:spinsPhaseRef}

\begin{pHeader}
    Синтаксис:      & \RightHandText{int spinsPhaseRef();}\\
    Аргумент(ы):    & \RightHandText{Нет} \\    
%    Возвращаемое значение:       & \RightHandText{Целое знаковое число} \\ 
    Файл объявления:             & \RightHandText{include/func/spnd.h} \\       
\end{pHeader}

Функция возвращает 1, если хотя бы один шпиндель требует фазировки, и 0 в противном случае. Для шпинделя, фазировка которого не выполнена, даётся команда фазировки.

Является системной.
% *******end section*****************

%--------------------------------------------------------
% *******begin subsection***************
\subsubsection{\DbgSecSt{\StPart}{spinAborted}}
\index{Программный интерфейс ПЛК!Управление шпинделями!Функция spinAborted}
\label{sec:spinAborted}

\begin{pHeader}
    Синтаксис:      & \RightHandText{int spinAborted(unsigned spin);}\\
    Аргумент(ы):    & \RightHandText{unsigned spin ~-- номер шпинделя} \\ 
%    Возвращаемое значение:       & \RightHandText{Целое знаковое число} \\ 
    Файл объявления:             & \RightHandText{include/func/spnd.h} \\
\end{pHeader}

Функция возвращает 1, если шпиндель, номер которого является аргументом функции, аварийно остановлен, и 0 в противном случае.  

Является системной.
% *******end section*****************
%-------------------------------------------------------------------
% *******begin subsection***************
\subsubsection{\DbgSecSt{\StPart}{spinsAborted}}
\index{Программный интерфейс ПЛК!Управление шпинделями!Функция spinsAborted}
\label{sec:spinsAborted}

\begin{pHeader}
    Синтаксис:      & \RightHandText{int spinsAborted();}\\
    Аргумент(ы):    & \RightHandText{Нет} \\    
%    Возвращаемое значение:       & \RightHandText{Целое знаковое число} \\ 
    Файл объявления:             & \RightHandText{include/func/spnd.h} \\       
\end{pHeader}

Функция возвращает 1, если все шпиндели аварийно остановлены, и 0 в противном случае.

Является системной.
% *******end section*****************
%--------------------------------------------------------
% *******begin subsection***************
\subsubsection{\DbgSecSt{\StPart}{spinIsStopped}}
\index{Программный интерфейс ПЛК!Управление шпинделями!Функция spinIsStopped}
\label{sec:spinIsStopped}

\begin{pHeader}
    Синтаксис:      & \RightHandText{int spinIsStopped(unsigned spin);}\\
    Аргумент(ы):    & \RightHandText{unsigned spin ~-- номер шпинделя} \\ 
%    Возвращаемое значение:       & \RightHandText{Целое знаковое число} \\ 
    Файл объявления:             & \RightHandText{include/func/spnd.h} \\
\end{pHeader}

Функция возвращает 1, если шпиндель, номер которого является аргументом функции, остановлен (в слежении имеет равную нулю заданную скорость), и 0 в противном случае.  

Является системной.
% *******end section*****************
%-------------------------------------------------------------------
% *******begin subsection***************
\subsubsection{\DbgSecSt{\StPart}{spinsStopped}}
\index{Программный интерфейс ПЛК!Управление шпинделями!Функция spinsStopped}
\label{sec:spinsStopped}

\begin{pHeader}
    Синтаксис:      & \RightHandText{int spinsStopped();}\\
    Аргумент(ы):    & \RightHandText{Нет} \\    
%    Возвращаемое значение:       & \RightHandText{Целое знаковое число} \\ 
    Файл объявления:             & \RightHandText{include/func/spnd.h} \\
\end{pHeader}

Функция возвращает 1, если все шпиндели остановлены (в слежении имеют равную нулю заданную скорость), и 0 в противном случае.

Является системной.
% *******end section*****************

%-------------------------------------------------------------------
% *******begin subsection***************
\subsubsection{\DbgSecSt{\StPart}{spinsAbortAll}}
\index{Программный интерфейс ПЛК!Управление шпинделями!Функция spinsAbortAll}
\label{sec:spinsAbortAll}

\begin{pHeader}
    Синтаксис:      & \RightHandText{void spinsAbortAll();}\\
    Аргумент(ы):    & \RightHandText{Нет} \\    
%    Возвращаемое значение:       & \RightHandText{Нет} \\ 
    Файл объявления:             & \RightHandText{include/func/spnd.h} \\
\end{pHeader}

Функция вызывает аварийное выключение всех шпинделей.

Является системной.
% *******end section*****************
%-------------------------------------------------------------------
% *******begin subsection***************
\subsubsection{\DbgSecSt{\StPart}{spinsStopAll}}
\index{Программный интерфейс ПЛК!Управление шпинделями!Функция spinsStopAll}
\label{sec:spinsStopAll}

\begin{pHeader}
    Синтаксис:      & \RightHandText{void spinsStopAll();}\\
    Аргумент(ы):    & \RightHandText{Нет} \\    
%    Возвращаемое значение:       & \RightHandText{Нет} \\ 
    Файл объявления:             & \RightHandText{include/func/spnd.h} \\
\end{pHeader}

Функция вызывает останов всех шпинделей при толчковых перемещениях.

Является системной.
% *******end section*****************
%--------------------------------------------------------
% *******begin subsection***************
\subsubsection{\DbgSecSt{\StPart}{spinAtSpeed}}
\index{Программный интерфейс ПЛК!Управление шпинделями!Функция spinAtSpeed}
\label{sec:spinAtSpeed}

\begin{pHeader}
    Синтаксис:      & \RightHandText{int spinAtSpeed(unsigned spin);}\\
    Аргумент(ы):    & \RightHandText{unsigned spin ~-- номер шпинделя} \\ 
%    Возвращаемое значение:       & \RightHandText{Целое знаковое число} \\ 
    Файл объявления:             & \RightHandText{include/func/spnd.h} \\
\end{pHeader}

Функция возвращает 1, если шпиндель, номер которого является аргументом функции, имеет скорость равную заданной, и 0 в противном случае. 

Является системной.
% *******end section*****************
%--------------------------------------------------------
% *******begin subsection***************
\subsubsection{\DbgSecSt{\StPart}{spinPosition}}
\index{Программный интерфейс ПЛК!Управление шпинделями!Функция spinPosition}
\label{sec:spinPosition}

\begin{pHeader}
    Синтаксис:      & \RightHandText{double spinPosition(unsigned spin);}\\
    Аргумент(ы):    & \RightHandText{unsigned spin ~-- номер шпинделя} \\ 
%    Возвращаемое значение:       & \RightHandText{Число с плавающей запятой двойной точности} \\ 
    Файл объявления:             & \RightHandText{include/func/spnd.h} \\
\end{pHeader}

Функция возвращает заданную позицию шпинделя, номер которого является аргументом функции.

Возвращаемое значение измеряется в единицах \texttt{spinEncRes} (см. структуру \myreftosec{SpindleConfig}). \killoverfullbefore

Является системной.
% *******end section*****************
%--------------------------------------------------------
% *******begin subsection***************
\subsubsection{\DbgSecSt{\StPart}{spinSpeedCommand}}
\index{Программный интерфейс ПЛК!Управление шпинделями!Функция spinSpeedCommand}
\label{sec:spinSpeedCommand}

\begin{pHeader}
    Синтаксис:      & \RightHandText{void spinSpeedCommand(unsigned spin, double speed,}\\
    & \RightHandText{int direction);}\\
    Аргумент(ы):    & \RightHandText{unsigned spin ~-- номер шпинделя, } \\ 
    & \RightHandText{double speed ~--  скорость,} \\   
    & \RightHandText{int direction ~-- направление вращения} \\   
%    Возвращаемое значение:       & \RightHandText{Нет} \\ 
    Файл объявления:             & \RightHandText{include/func/spnd.h} \\
\end{pHeader}

Функция задаёт скорость и направление вращения шпинделя, номер которого является аргументом функции. 

Является системной.
% *******end section*****************
%--------------------------------------------------------
% *******begin subsection***************
\subsubsection{\DbgSecSt{\StPart}{spinCurStage}}
\index{Программный интерфейс ПЛК!Управление шпинделями!Функция spinCurStage}
\label{sec:spinCurStage}

\begin{pHeader}
    Синтаксис:      & \RightHandText{unsigned spinCurStage(unsigned spin);}\\
    Аргумент(ы):    & \RightHandText{unsigned spin ~-- номер шпинделя} \\ 
%    Возвращаемое значение:       & \RightHandText{Целое беззнаковое число} \\ 
    Файл объявления:             & \RightHandText{include/func/spnd.h} \\
\end{pHeader}

Функция возвращает номер ступени разгона шпинделя, номер которого является аргументом функции. 

Является системной.
% *******end section*****************
\clearpage

%--------------------------------------------------------
% *******begin subsection***************
\subsubsection{\DbgSecSt{\StPart}{spinNeedChangeStage}}
\index{Программный интерфейс ПЛК!Управление шпинделями!Функция spinNeedChangeStage}
\label{sec:spinNeedChangeStage}

\begin{pHeader}
    Синтаксис:      & \RightHandText{int spinNeedChangeStage(unsigned spin);}\\
    Аргумент(ы):    & \RightHandText{unsigned spin ~-- номер шпинделя} \\ 
%    Возвращаемое значение:       & \RightHandText{Целое знаковое число} \\ 
    Файл объявления:             & \RightHandText{include/func/spnd.h} \\
\end{pHeader}

Функция возвращает 1, если требуется смена ступени разгона шпинделя, номер которого является аргументом функции, и 0 в противном случае.

Является системной.
% *******end section*****************
%--------------------------------------------------------
% *******begin subsection***************
\subsubsection{\DbgSecSt{\StPart}{spinsFollowup}}
\index{Программный интерфейс ПЛК!Управление шпинделями!Функция spinsFollowup}
\label{sec:spinsFollowup}

\begin{pHeader}
    Синтаксис:      & \RightHandText{void spinsFollowup();}\\
    Аргумент(ы):    & \RightHandText{Нет} \\    
%    Возвращаемое значение:       & \RightHandText{Нет} \\ 
    Файл объявления:             & \RightHandText{include/func/axis.h} \\
\end{pHeader}

Функция устанавливает для всех шпинделей флаг <<Восстановление после ошибки>> (см. структуру \myreftosec{Spindle}).

Является системной.
% *******end section*****************

%--------------------------------------------------------
% *******begin subsection***************
\subsubsection{\DbgSecSt{\StPart}{initSpindle}}
\index{Программный интерфейс ПЛК!Управление шпинделями!Функция initSpindle}
\label{sec:initSpindle}

\begin{pHeader}
    Синтаксис:      & \RightHandText{void initSpindle(int spin);}\\
    Аргумент(ы):    & \RightHandText{int spin ~-- номер шпинделя} \\ 
%    Возвращаемое значение:       & \RightHandText{Нет} \\ 
    Файл объявления:             & \RightHandText{include/func/axis.h} \\
\end{pHeader}

Функция выполняет инициализацию шпинделя, номер которого является аргументом функции, параметрами по умолчанию. 

Является системной.
% *******end section*****************

%--------------------------------------------------------
% *******begin subsection***************
\subsubsection{\DbgSecSt{\StPart}{initSpindles}}
\index{Программный интерфейс ПЛК!Управление шпинделями!Функция initSpindles}
\label{sec:initSpindles}

\begin{pHeader}
    Синтаксис:      & \RightHandText{void initSpindles();}\\
    Аргумент(ы):    & \RightHandText{Нет} \\    
%    Возвращаемое значение:       & \RightHandText{Нет} \\ 
    Файл объявления:             & \RightHandText{include/func/axis.h} \\
\end{pHeader}

Функция выполняет инициализацию шпинделей параметрами по умолчанию. 

Является системной.
% *******end section*****************
%--------------------------------------------------------
% *******begin subsection***************
\subsubsection{\DbgSecSt{\StPart}{spinInitPlatform}}
\index{Программный интерфейс ПЛК!Управление шпинделями!Функция spinInitPlatform}
\label{sec:spinInitPlatform}

\begin{pHeader}
    Синтаксис:      & \RightHandText{void spinInitPlatform(int spin);}\\
    Аргумент(ы):    & \RightHandText{int spin ~-- номер шпинделя} \\    
%    Возвращаемое значение:       & \RightHandText{Нет} \\ 
    Файл объявления:             & \RightHandText{include/func/axis.h} \\
\end{pHeader}

Функция выполняет инициализацию параметров шпинделя, номер которого является аргументом функции, пользовательскими значениями, в том числе структуры \hyperlink{Spindle_Platform_Control}{SpindlePlatformControl}. \killoverfullbefore 

Реализуется пользователем.
% *******end section*****************
%--------------------------------------------------------
% *******begin subsection***************
\subsubsection{\DbgSecSt{\StPart}{spinsInitPlatform}}
\index{Программный интерфейс ПЛК!Управление шпинделями!Функция spinsInitPlatform}
\label{sec:spinsInitPlatform}

\begin{pHeader}
    Синтаксис:      & \RightHandText{void spinsInitPlatform();}\\
    Аргумент(ы):    & \RightHandText{Нет} \\    
%%    Возвращаемое значение:       & \RightHandText{Нет} \\ 
    Файл объявления:             & \RightHandText{include/func/axis.h} \\
\end{pHeader}

Функция выполняет инициализацию параметров шпинделей пользовательскими значениями, в том числе структуры \hyperlink{Spindles_Platform_Control}{SpindlesPlatformControl}. \killoverfullbefore 

Реализуется пользователем.
% *******end section*****************
             % API: управление шпинделями
    %--------------------------------------------------------
% *******begin section***************
\section{\DbgSecSt{\StPart}{Датчики обратной связи}}
%--------------------------------------------------------
\subsection{\DbgSecSt{\StPart}{Типы данных}}

% *******begin subsection***************
\subsubsection{\DbgSecSt{\StPart}{EncType}}
\index{Программный интерфейс ПЛК!Датчики обратной связи!Перечисление EncType}
\label{sec:EncType}

\begin{fHeader}
    Тип данных:            & \RightHandText{Перечисление EncType}\\
    Файл объявления:             & \RightHandText{include/func/enc.h} \\
\end{fHeader}

Перечисление определяет идентификаторы типов датчиков обратной связи.

\begin{MyTableTwoColAllCntr}{Перечисление EncType}{tbl:EncType}{|m{0.38\linewidth}|m{0.57\linewidth}|}{Идентификатор}{Описание}
\hline encNone & Нет \\
\hline encIncrement & Инкрементальный ДОС \\
\hline encSinCos & Синусно-косинусный ДОС \\
\hline encEnDat & ДОС с интерфейсом EnDat 2.2\\
\hline encBiSS & ДОС с интерфейсом BiSS \\
\end{MyTableTwoColAllCntr}
% *******end subsection***************
%--------------------------------------------------------
% *******begin subsection***************
\subsubsection{\DbgSecSt{\StPart}{EncConfig}}
\index{Программный интерфейс ПЛК!Датчики обратной связи!Структура EncConfig}
\label{sec:EncConfig}

\begin{fHeader}
    Тип данных:            & \RightHandText{Структура EncConfig} \\
    Файл объявления:             & \RightHandText{include/func/enc.h} \\
\end{fHeader}

Структура определяет параметры датчика.

\begin{MyTableThreeColAllCntr}{Структура EncConfig}{tbl:EncConfig}{|m{0.33\linewidth}|m{0.22\linewidth}|m{0.45\linewidth}|}{Элемент}{Тип}{Описание}
\hline servo & \centering{unsigned} & Номер платы (0$\div$3) \\
\hline chan & \centering{unsigned} & Номер канала (0$\div$7) \\
\hline type & \centering{unsigned} & Тип датчика (см. \myreftosec{EncType})\\
\end{MyTableThreeColAllCntr}
% *******end subsection***************
%--------------------------------------------------------
\begin{comment}
% *******begin subsection***************
\subsection{\DbgSecSt{\StPart}{Функции}}

% *******begin subsection***************
\subsubsection{\DbgSecSt{\StPart}{encoderScanErrors}}
\index{Программный интерфейс ПЛК!Датчики обратной связи!Функция encoderScanErrors}
\label{sec:encoderScanErrors}

\begin{pHeader}
    Синтаксис:      & \RightHandText{void encoderScanErrors(ErrorClear request);}\\
    Аргумент(ы):    & \RightHandText{\myreftosec{ErrorClear} request ~-- идентификатор типа сброса ошибки} \\    
%    Возвращаемое значение:       & \RightHandText{Нет} \\ 
    Файл объявления:             & \RightHandText{include/func/enc.h} \\       
\end{pHeader}

 

Является системной.
% *******end section*****************

%-------------------------------------------------------------------
% *******begin subsection***************
\subsubsection{\DbgSecSt{\StPart}{encoderErrorsReaction}}
\index{Программный интерфейс ПЛК!Датчики обратной связи!Функция encoderErrorsReaction}
\label{sec:encoderErrorsReaction}

\begin{pHeader}
    Синтаксис:      & \RightHandText{void encoderErrorsReaction();}\\
    Аргумент(ы):    & \RightHandText{Нет} \\    
%    Возвращаемое значение:       & \RightHandText{Нет} \\ 
    Файл объявления:             & \RightHandText{include/func/enc.h} \\       
\end{pHeader}



Является системной.
% *******end section*****************
\end{comment}

%--------------------------------------------------------
                 % API: ДОС
    %--------------------------------------------------------
% *******begin section***************
\section{\DbgSecSt{\StPart}{Реферирование осей}}
%--------------------------------------------------------
\subsection{\DbgSecSt{\StPart}{Типы данных}}

% *******begin subsection***************
\subsubsection{\DbgSecSt{\StPart}{HomeStates}}
\index{Программный интерфейс ПЛК!Реферирование осей!Перечисление HomeStates}
\label{sec:HomeStates}

\begin{fHeader}
    Тип данных:            & \RightHandText{Перечисление HomeStates}\\
    Файл объявления:             & \RightHandText{include/func/home.h} \\
\end{fHeader}

Перечисление определяет идентификаторы состояний выезда в нулевую точку. \killoverfullbefore

\begin{MyTableTwoColAllCntr}{Перечисление HomeStates}{tbl:HomeStates}{|m{0.38\linewidth}|m{0.57\linewidth}|}{Идентификатор}{Описание}
\hline homeReady & Выезд в ноль не выполнен \\
\hline homeStart &  Начало выезда в нулевую точку \\
\hline homeWaitStage & Ожидание этапа выезда в нулевую точку \\
\hline homeComplete & Выезд в нулевую точку выполнен \\
\hline homeError & Ошибка выезда в нулевую точку \\
\end{MyTableTwoColAllCntr}
% *******end subsection***************
%--------------------------------------------------------
% *******begin subsection***************
\subsection{\DbgSecSt{\StPart}{Функции}}

% *******begin subsection***************
\subsubsection{\DbgSecSt{\StPart}{isHomeComplete}}
\index{Программный интерфейс ПЛК!Реферирование осей!Функция isHomeComplete}
\label{sec:isHomeComplete}

\begin{pHeader}
    Синтаксис:      & \RightHandText{int isHomeComplete();}\\
    Аргумент(ы):    & \RightHandText{Нет} \\    
%    Возвращаемое значение:       & \RightHandText{Целое знаковое число} \\ 
    Файл объявления:             & \RightHandText{include/func/home.h} \\       
\end{pHeader}

Функция возвращает 1, если завершено реферирование осей и шпинделей, и 0 в противном случае. 

Реализуется пользователем.
% *******end section*****************
%-------------------------------------------------------------------
% *******begin subsection***************
\subsubsection{\DbgSecSt{\StPart}{isHoming}}
\index{Программный интерфейс ПЛК!Реферирование осей!Функция isHoming}
\label{sec:isHoming}

\begin{pHeader}
    Синтаксис:      & \RightHandText{int isHoming();}\\
    Аргумент(ы):    & \RightHandText{Нет} \\    
%    Возвращаемое значение:       & \RightHandText{Целое знаковое число} \\ 
    Файл объявления:             & \RightHandText{include/func/home.h} \\       
\end{pHeader}

Функция возвращает 1, если выполняется реферирование осей и шпинделей, и 0 в противном случае.

Реализуется пользователем.
% *******end section*****************
%-------------------------------------------------------------------
% *******begin subsection***************
\subsubsection{\DbgSecSt{\StPart}{startHoming}}
\index{Программный интерфейс ПЛК!Реферирование осей!Функция startHoming}
\label{sec:startHoming}

\begin{pHeader}
    Синтаксис:      & \RightHandText{void startHoming();}\\
    Аргумент(ы):    & \RightHandText{Нет} \\    
%    Возвращаемое значение:       & \RightHandText{Нет} \\ 
    Файл объявления:             & \RightHandText{include/func/home.h} \\       
\end{pHeader}

Функция инициирует начало реферирование осей и шпинделей.

Реализуется пользователем.
% *******end section*****************
%-------------------------------------------------------------------
% *******begin subsection***************
\subsubsection{\DbgSecSt{\StPart}{isHomingError}}
\index{Программный интерфейс ПЛК!Реферирование осей!Функция isHomingError}
\label{sec:isHomingError}

\begin{pHeader}
    Синтаксис:      & \RightHandText{int isHomingError();}\\
    Аргумент(ы):    & \RightHandText{Нет} \\    
%    Возвращаемое значение:       & \RightHandText{Целое знаковое число} \\ 
    Файл объявления:             & \RightHandText{include/func/home.h} \\       
\end{pHeader}

Функция возвращает 1, если произошла ошибка при реферировании осей или шпинделей, и 0 в противном случае.

Реализуется пользователем.
% *******end section*****************
%-------------------------------------------------------------------
% *******begin subsection***************
\subsubsection{\DbgSecSt{\StPart}{homeCancel}}
\index{Программный интерфейс ПЛК!Реферирование осей!Функция homeCancel}
\label{sec:homeCancel}

\begin{pHeader}
    Синтаксис:      & \RightHandText{void homeCancel();}\\
    Аргумент(ы):    & \RightHandText{Нет} \\    
%    Возвращаемое значение:       & \RightHandText{Нет} \\ 
    Файл объявления:             & \RightHandText{include/func/home.h} \\       
\end{pHeader}

Функция выполняет останов реферирования осей и шпинделей.

Реализуется пользователем.
% *******end section*****************
%--------------------------------------------------------
                % API: выезд в 0
    %--------------------------------------------------------
% *******begin section***************
\section{\DbgSecSt{\StPart}{Состояние управляющей программы}}
%--------------------------------------------------------
% *******begin subsection***************
\subsection{\DbgSecSt{\StPart}{Функции}}

% *******begin subsection***************
\subsubsection{\DbgSecSt{\StPart}{csProgramRunning}}
\index{Программный интерфейс ПЛК!Состояние управляющей программы!Функция csProgramRunning}
\label{sec:csProgramRunning}

\begin{pHeader}
    Синтаксис:      & \RightHandText{int csProgramRunning(int cs);}\\
    Аргумент(ы):    & \RightHandText{int cs ~-- номер координатной системы} \\    
%    Возвращаемое значение:       & \RightHandText{Целое знаковое число} \\ 
    Файл объявления:             & \RightHandText{include/func/cs.h} \\       
\end{pHeader}

Функция возвращает 1, если выполняется УП, и 0 в противном случае. 

Является системной.
% *******end section*****************
%-------------------------------------------------------------------
% *******begin subsection***************
\subsubsection{\DbgSecSt{\StPart}{csProgramHolding}}
\index{Программный интерфейс ПЛК!Состояние управляющей программы!Функция csProgramHolding}
\label{sec:csProgramHolding}

\begin{pHeader}
    Синтаксис:      & \RightHandText{int csProgramHolding(int cs);}\\
    Аргумент(ы):    & \RightHandText{int cs ~-- номер координатной системы} \\
%    Возвращаемое значение:       & \RightHandText{Целое знаковое число} \\ 
    Файл объявления:             & \RightHandText{include/func/cs.h} \\       
\end{pHeader}

Функция возвращает 1, если произведён приостанов подачи, и 0 в противном случае. 

Является системной.
% *******end section*****************
%-------------------------------------------------------------------
% *******begin subsection***************
\subsubsection{\DbgSecSt{\StPart}{csProgramStarting}}
\index{Программный интерфейс ПЛК!Состояние управляющей программы!Функция csProgramStarting}
\label{sec:csProgramStarting}

\begin{pHeader}
    Синтаксис:      & \RightHandText{int csProgramStarting(int cs);}\\
    Аргумент(ы):    & \RightHandText{int cs ~-- номер координатной системы} \\  
%    Возвращаемое значение:       & \RightHandText{Целое знаковое число} \\ 
    Файл объявления:             & \RightHandText{include/func/cs.h} \\       
\end{pHeader}

Функция возвращает 1, если УП начинает выполняться, и 0 в противном случае. 

Является системной.
% *******end section*****************
%-------------------------------------------------------------------
% *******begin subsection***************
\subsubsection{\DbgSecSt{\StPart}{csProgramPaused}}
\index{Программный интерфейс ПЛК!Состояние управляющей программы!Функция csProgramPaused}
\label{sec:csProgramPaused}

\begin{pHeader}
    Синтаксис:      & \RightHandText{int csProgramPaused(int cs);}\\
    Аргумент(ы):    & \RightHandText{int cs ~-- номер координатной системы} \\  
%    Возвращаемое значение:       & \RightHandText{Целое знаковое число} \\ 
    Файл объявления:             & \RightHandText{include/func/cs.h} \\       
\end{pHeader}

Функция возвращает 1, если УП временно приостановлена, и 0 в противном случае. 

Является системной.
% *******end section*****************
%-------------------------------------------------------------------
% *******begin subsection***************
\subsubsection{\DbgSecSt{\StPart}{csProgramStopped}}
\index{Программный интерфейс ПЛК!Состояние управляющей программы!Функция csProgramStopped}
\label{sec:csProgramStopped}

\begin{pHeader}
    Синтаксис:      & \RightHandText{int csProgramStopped(int cs);}\\
    Аргумент(ы):    & \RightHandText{int cs ~-- номер координатной системы} \\    
%    Возвращаемое значение:       & \RightHandText{Целое знаковое число} \\ 
    Файл объявления:             & \RightHandText{include/func/cs.h} \\       
\end{pHeader}

Функция возвращает 1, если УП остановлена, и 0 в противном случае. 

Является системной.
% *******end section*****************

                  % API: координатные системы
    %--------------------------------------------------------
% *******begin section***************
\section{\DbgSecSt{\StPart}{Очередь команд}}
%--------------------------------------------------------

\subsection{\DbgSecSt{\StPart}{Типы данных}}
%--------------------------------------------------------
% *******begin subsection***************
\subsubsection{\DbgSecSt{\StPart}{CommandRequest}}
\index{Программный интерфейс ПЛК!Очередь команд!Структура CommandRequest}
\label{sec:CommandRequest}

\begin{fHeader}
    Тип данных:            & \RightHandText{Структура CommandRequest}\\
    Файл объявления:             & \RightHandText{include/func/commands.h} \\
\end{fHeader}

Структура определяет параметры команды.

\begin{MyTableThreeColAllCntr}{Структура CommandRequest}{tbl:CommandRequest}{|m{0.3\linewidth}|m{0.25\linewidth}|m{0.45\linewidth}|}{Элемент}{Тип}{Описание}
\hline command & \centering{unsigned} & Идентификатор команды \\
\hline prio & \centering{unsigned} & Приоритет \\
\hline next & \centering{int} & Номер следующей команды в очереди \\
\end{MyTableThreeColAllCntr}
% *******end subsection***************
%--------------------------------------------------------
% *******begin subsection***************
\subsubsection{\DbgSecSt{\StPart}{CommandQueue}}
\index{Программный интерфейс ПЛК!Очередь команд!Структура CommandQueue}
\label{sec:CommandQueue}

\begin{fHeader}
    Тип данных:            & \RightHandText{Структура CommandQueue}\\
    Файл объявления:             & \RightHandText{include/func/commands.h} \\
\end{fHeader}

Структура определяет параметры очереди команд.

\begin{MyTableThreeColAllCntr}{Структура CommandQueue}{tbl:CommandQueue}{|m{0.41\linewidth}|m{0.24\linewidth}|m{0.35\linewidth}|}{Элемент}{Тип}{Описание}
\hline used & \centering{int} & Первый используемый номер команды в очереди \\
\hline free & \centering{int} & Первый неиспользуемый номер команды в очереди  \\
\hline queue[РАЗМЕР\_ОЧЕРЕДИ\_КОМАНД] & \centering{\myreftosec{CommandRequest}} & Массив команд \\
\end{MyTableThreeColAllCntr}
% *******end subsection***************
%--------------------------------------------------------
% *******begin subsection***************
\subsection{\DbgSecSt{\StPart}{Функции}}

% *******begin subsection***************
\subsubsection{\DbgSecSt{\StPart}{commandsInit}}
\index{Программный интерфейс ПЛК!Очередь команд!Функция commandsInit}
\label{sec:commandsInit}

\begin{pHeader}
    Синтаксис:      & \RightHandText{void commandsInit(struct CommandQueue \&queue);}\\
   Аргумент(ы):    & \RightHandText{struct \myreftosec{CommandQueue} \&queue ~-- очередь команд} \\    
%    Возвращаемое значение:       & \RightHandText{Нет} \\ 
    Файл объявления:             & \RightHandText{include/func/commands.h} \\       
\end{pHeader}

Функция инициализирует очередь команд. 

Является системной.
% *******end section*****************

%-------------------------------------------------------------------
% *******begin subsection***************
\subsubsection{\DbgSecSt{\StPart}{commandPush}}
\index{Программный интерфейс ПЛК!Очередь команд!Функция commandPush}
\label{sec:commandPush}

\begin{pHeader}
    Синтаксис:      & \RightHandText{int commandPush(struct CommandQueue \&queue,}\\
      & \RightHandText{unsigned command, unsigned prio);}\\
    Аргумент(ы):    & \RightHandText{struct \myreftosec{CommandQueue} \&queue ~-- очередь команд,} \\
    & \RightHandText{unsigned command ~-- идентификатор команды, } \\   
    & \RightHandText{unsigned prio ~-- приоритет команды} \\  
%    Возвращаемое значение:       & \RightHandText{Целое знаковое число} \\ 
    Файл объявления:             & \RightHandText{include/func/commands.h} \\
\end{pHeader}

Функция помещает команду с учётом заданного приоритета в очередь команд.\killoverfullbefore

Функция возвращает номер команды, если она помещена в очередь, и -1 в случае ошибки.

Является системной.
% *******end section*****************
%-------------------------------------------------------------------
% *******begin subsection***************
\subsubsection{\DbgSecSt{\StPart}{commandPop}}
\index{Программный интерфейс ПЛК!Очередь команд!Функция commandPop}
\label{sec:commandPop}

\begin{pHeader}
    Синтаксис:      & \RightHandText{unsigned commandPop(struct CommandQueue \&queue);}\\
    Аргумент(ы):    & \RightHandText{struct \myreftosec{CommandQueue} \&queue ~-- очередь команд} \\ 
%    Возвращаемое значение:       & \RightHandText{Целое беззнаковое число} \\ 
    Файл объявления:             & \RightHandText{include/func/commands.h} \\
\end{pHeader}

Функция возвращает идентификатор команды, которая должна быть выполнена с учётом приоритета, из очереди команд. 

Является системной.
% *******end section*****************

%-------------------------------------------------------------------
% *******begin subsection***************
\subsubsection{\DbgSecSt{\StPart}{commandFlush}}
\index{Программный интерфейс ПЛК!Очередь команд!Функция commandFlush}
\label{sec:commandFlush}

\begin{pHeader}
    Синтаксис:      & \RightHandText{void commandFlush(struct CommandQueue \&queue);}\\
   Аргумент(ы):    & \RightHandText{struct \myreftosec{CommandQueue} \&queue ~-- очередь команд} \\    
%    Возвращаемое значение:       & \RightHandText{Нет} \\ 
    Файл объявления:             & \RightHandText{include/func/commands.h} \\
\end{pHeader}

Функция очищает очередь команд. 

Является системной.
% *******end section*****************

            % API: команды
%    %--------------------------------------------------------
% *******begin section***************
\section{\DbgSecSt{\StPart}{Вспомогательные функции}}
%--------------------------------------------------------

% *******begin subsection***************
\subsection{\DbgSecSt{\StPart}{detectEdgeRise}}
\index{Программный интерфейс ПЛК!Вспомогательные функции!Функция detectEdgeRise}
\label{sec:detectEdgeRise}

\begin{pHeader}
    Синтаксис:      & \RightHandText{int detectEdgeRise(int \&detector, int input);}\\
    Аргумент(ы):    & \RightHandText{int \&detector ~-- предыдущее входное значение} \\    
    & \RightHandText{int input ~--  входное значение} \\ 
    Файл объявления:             & \RightHandText{include/func/misc.h} \\       
\end{pHeader}

Функция служит для детектирования изменения с 0 на 1 (детектирования фронта) входной величины. \killoverfullbefore

Функция возвращает 0, если входное значение не изменилось и осталось равным 0, и 1, если входное значение стало отличным от 0. \killoverfullbefore

Является системной.
% *******end subsection*****************
%-------------------------------------------------------------------
% *******begin subsection***************
\subsection{\DbgSecSt{\StPart}{detectEdgeFall}}
\index{Программный интерфейс ПЛК!Вспомогательные функции!Функция detectEdgeFall}
\label{sec:detectEdgeFall}

\begin{pHeader}
    Синтаксис:      & \RightHandText{int detectEdgeFall(int \&detector, int input);}\\
    Аргумент(ы):    & \RightHandText{int \&detector ~-- предыдущее входное значение} \\    
    & \RightHandText{int input ~--  входное значение} \\ 
    Файл объявления:             & \RightHandText{include/func/misc.h} \\       
\end{pHeader}

Функция служит для детектирования изменения с 1 на 0 (детектирования спада) входной величины. \killoverfullbefore

Функция возвращает 0, если входное значение не изменилось и осталось равным 1, и 1, если входное значение стало равным 0. \killoverfullbefore

Является системной.
% *******end subsection*****************
%-------------------------------------------------------------------
% *******begin subsection***************
\subsection{\DbgSecSt{\StPart}{initPulsedTimer}}
\index{Программный интерфейс ПЛК!Вспомогательные функции!Функция initPulsedTimer}
\label{sec:initPulsedTimer}

\begin{pHeader}
    Синтаксис:      & \RightHandText{void initPulsedTimer();}\\
    Аргумент(ы):    & \RightHandText{Нет} \\  
%    Возвращаемое значение:       & \RightHandText{Целое знаковое число} \\ 
    Файл объявления:             & \RightHandText{include/func/misc.h} \\       
\end{pHeader}

Функция инициализации периодического (импульсного) таймера. \killoverfullbefore
%который срабатывает (возвращает 1) через заданный интервал. 

Является системной.
% *******end subsection*****************
%-------------------------------------------------------------------
% *******begin subsection***************
\subsection{\DbgSecSt{\StPart}{timerSc}}
\index{Программный интерфейс ПЛК!Вспомогательные функции!Функция timerSc}
\label{sec:timerSc}

\begin{pHeader}
    Синтаксис:      & \RightHandText{int timerSc(int period);}\\
    Аргумент(ы):    & \RightHandText{int period ~-- период таймера} \\  
%    Возвращаемое значение:       & \RightHandText{Целое знаковое число} \\ 
    Файл объявления:             & \RightHandText{include/func/misc.h} \\
\end{pHeader}

Функция периодического (импульсного) таймера ~-- таймера, выходное значение которого периодически переключается с 0 на 1 и обратно через интервал, равный половине периода таймера. Период таймера задаётся в периодах сервоцикла (1 период сервоцикла равен 400 мс). Так, например, интервал 1 c соответствует значению периода таймера равному 2500. \killoverfullbefore
%сброса таймера (установки 0)

Функция возвращает 1, если с момента переключения таймера с 1 на 0 истёк интервал, больший или равный половине периода, и 0 в противном случае. \killoverfullbefore 

Является системной.
% *******end subsection*****************
%-------------------------------------------------------------------
% *******end section*****************

                % API: вспомогательные функции    
    %--------------------------------------------------------
% *******begin section***************
\section{\DbgSecSt{\StPart}{Управление движением}}
%--------------------------------------------------------

\subsection{\DbgSecSt{\StPart}{Типы данных}}
%--------------------------------------------------------
% *******begin subsection***************
\subsubsection{\DbgSecSt{\StPart}{XYZ}}
\index{Программный интерфейс ПЛК!Управление движением!Структура XYZ}
\label{sec:XYZ}

\begin{fHeader}
    Тип данных:            & \RightHandText{Структура XYZ}\\
    Файл объявления:             & \RightHandText{sys/sys.h} \\
\end{fHeader}

Структура определяет координаты по осям декартовой системы координат.

\begin{MyTableThreeColAllCntr}{Структура XYZ}{tbl:XYZ}{|m{0.3\linewidth}|m{0.25\linewidth}|m{0.45\linewidth}|}{Элемент}{Тип}{Описание}
\hline \qquad X & \centering{double} & Координата по оси Х  \\
\hline \qquad Y & \centering{double} & Координата по оси Y  \\
\hline \qquad Z & \centering{double} & Координата по оси Z  \\
\end{MyTableThreeColAllCntr}
% *******end subsection***************
%--------------------------------------------------------
% *******begin subsection***************
\subsubsection{\DbgSecSt{\StPart}{SpindleTimeBase}}
\index{Программный интерфейс ПЛК!Управление движением!Перечисление SpindleTimeBase}
\label{sec:SpindleTimeBase}

\begin{fHeader}
    Тип данных:            & \RightHandText{Перечисление SpindleTimeBase}\\
    Файл объявления:             & \RightHandText{sys/sys.h} \\
\end{fHeader}

Перечисление определяет идентификаторы временной развёртки шпинделя.

\begin{MyTableTwoColAllCntr}{Перечисление SpindleTimeBase}{tbl:SpindleTimeBase}{|m{0.38\linewidth}|m{0.57\linewidth}|}{Идентификатор}{Описание}
\hline spinUseCSTimebase & Временная развёртка указанной координатной системы \\
\hline spinUseCS0TimeBase  & Временная развёртка координатной системы \newline № 0 \\
\hline spinUseFixedTimeBase & 100\% фиксированная временная развёртка \\
\end{MyTableTwoColAllCntr}
% *******end subsection***************
%--------------------------------------------------------
% *******begin subsection***************
\subsubsection{\DbgSecSt{\StPart}{Timer}}
\index{Программный интерфейс ПЛК!Управление движением!Структура Timer}
\label{sec:Timer}

\begin{fHeader}
    Тип данных:            & \RightHandText{Структура Timer}\\
    Файл объявления:             & \RightHandText{sys/sys.h} \\
\end{fHeader}

Структура определяет параметры таймера.

\begin{MyTableThreeColAllCntr}{Структура Timer}{tbl:Timer}{|m{0.41\linewidth}|m{0.24\linewidth}|m{0.35\linewidth}|}{Элемент}{Тип}{Описание}
\hline start & \centering{int} & Начальное значение счётчика таймера \\
\hline timeout & \centering{int} & Интервал \\
\end{MyTableThreeColAllCntr}
% *******end subsection***************
%--------------------------------------------------------
% *******begin subsection***************
\subsubsection{\DbgSecSt{\StPart}{MotorDefinition}}
\index{Программный интерфейс ПЛК!Управление движением!Структура MotorDefinition}
\label{sec:MotorDefinition}

\begin{fHeader}
    Тип данных:            & \RightHandText{Структура MotorDefinition}\\
    Файл объявления:             & \RightHandText{sys/sys.h} \\
\end{fHeader}

Структура определяет параметры привязки двигателя к оси координатной системы.

\begin{MyTableThreeColAllCntr}{Структура MotorDefinition}{tbl:MotorDefinition}{|m{0.41\linewidth}|m{0.24\linewidth}|m{0.35\linewidth}|}{Элемент}{Тип}{Описание}
\hline A, B, C, U, V, W, X, Y, Z \newline  XA, XB, XC, XD, XE, XF, XG, XH \newline XL, XM, XN, XO, XP, XQ, XR, XS \newline XT, XU, XV, XW, XX, XY, XZ & \centering{double} & Масштабные коэффициенты,
связывающие положение двигателя и координаты осей (число дискрет перемещения для двигателя на одну единицу величины перемещения по оси) \\
\hline Ofs & \centering{double} & Смещение между нулевой точкой двигателя и нулевой позицией оси \\
\end{MyTableThreeColAllCntr}
% *******end subsection***************
%--------------------------------------------------------
% *******begin subsection***************
\subsubsection{\DbgSecSt{\StPart}{Pos}}
\index{Программный интерфейс ПЛК!Управление движением!Объединение Pos}
\label{sec:Pos}

\begin{fHeader}
    Тип данных:            & \RightHandText{Объединение Pos}\\
    Файл объявления:             & \RightHandText{sys/sys.h} \\
\end{fHeader}

Объединение определяет данные перемещения для различных режимов движения.

\begin{MyTableThreeColAllCntr}{Объединение Pos}{tbl:Pos}{|m{0.33\linewidth}|m{0.22\linewidth}|m{0.45\linewidth}|}{Элемент}{Тип}{Описание}
\hline struct \{ 
\newline
A, B, C, U, V, W, X, Y, Z & \newline \centering{double} & \newline Координаты по осям \\
\hhline{~} I, J, K & \centering{double} & Компоненты вектора \\
\hhline{~} R \} & \centering{double} & Радиус \\
\hline struct \{ 
\newline
Axis[9] & \newline \centering{double} & \newline Координаты по осям \\
\hhline{~} Vec[6] \} & \centering{double} & Компоненты вектора, радиус \\
\end{MyTableThreeColAllCntr}

\begin{comment}
\begin{MyTableThreeColAllCntr}{Объединение Pos}{tbl:Pos}{|m{0.41\linewidth}|m{0.24\linewidth}|m{0.35\linewidth}|}{Элемент}{Тип}{Описание}
\hline struct \{ 
\newline double A, B, C, U, V, W, X, Y, Z; 
\newline double I, J, K; 
\newline double R; 
\newline \}
& \centering{структура} & 
\newline Координаты по осям, 
\newline компоненты вектора, 
\newline радиус \newline \\
\hline struct \{ 
\newline double Axis[9]; 
\newline double Vec[6]; 
\newline \} 
& \centering{структура} &  
\newline Координаты по осям, 
\newline компоненты вектора, радиус \newline \\
\end{MyTableThreeColAllCntr}
\end{comment}

% *******end subsection***************
%--------------------------------------------------------
% *******begin subsection***************
\subsubsection{\DbgSecSt{\StPart}{JogTarget}}
\index{Программный интерфейс ПЛК!Управление движением!Структура JogTarget}
\label{sec:JogTarget}

\begin{fHeader}
    Тип данных:            & \RightHandText{Структура JogTarget}\\
    Файл объявления:             & \RightHandText{sys/sys.h} \\
\end{fHeader}

Структура определяет координаты и смещения для толчковых перемещений.

\begin{MyTableThreeColAllCntr}{Структура JogTarget}{tbl:JogTarget}{|m{0.41\linewidth}|m{0.24\linewidth}|m{0.35\linewidth}|}{Элемент}{Тип}{Описание}
\hline pos[32] & \centering{double} & Координаты \\
\hline offset[32] & \centering{double} & Смещения \\
\end{MyTableThreeColAllCntr}
% *******end subsection***************
%--------------------------------------------------------
% *******begin subsection***************
\subsubsection{\DbgSecSt{\StPart}{Vec}}
\index{Программный интерфейс ПЛК!Управление движением!Объединение Vec}
\label{sec:Vec}

\begin{fHeader}
    Тип данных:            & \RightHandText{Объединение Vec}\\
    Файл объявления:             & \RightHandText{sys/sys.h} \\
\end{fHeader}

Объединение определяет компоненты вектора.

\begin{MyTableThreeColAllCntr}{Объединение Pos}{tbl:Pos}{|m{0.33\linewidth}|m{0.22\linewidth}|m{0.45\linewidth}|}{Элемент}{Тип}{Описание}
\hline struct \{ 
\newline I, J, K \} & \newline \centering{double} & \newline Компоненты вектора \\
\hline V[3] & \centering{double} & Компоненты вектора \\
\end{MyTableThreeColAllCntr}

\begin{comment}
\begin{MyTableThreeColAllCntr}{Объединение Pos}{tbl:Pos}{|m{0.41\linewidth}|m{0.24\linewidth}|m{0.35\linewidth}|}{Элемент}{Тип}{Описание}
\hline struct \{ 
\newline double I, J, K;
\newline \} 
& \centering{структура} & 
\newline Компоненты вектора \newline \\
\hline V[3] & \centering{double} & Компоненты вектора \\
\end{MyTableThreeColAllCntr}
\end{comment}

% *******end subsection***************
%--------------------------------------------------------
% *******begin subsection***************
\subsection{\DbgSecSt{\StPart}{Функции и макросы}}

% *******begin subsection***************
\subsubsection{\DbgSecSt{\StPart}{kill}}
\index{Программный интерфейс ПЛК!Управление движением!Функция kill}
\label{sec:kill}

\begin{pHeader}
    Синтаксис:      & \RightHandText{int kill(int motor);}\\
   Аргумент(ы):  & \RightHandText{int motor ~-- номер двигателя} \\ 
%    Возвращаемое значение:       & \RightHandText{Нет} \\ 
    Файл объявления:             & \RightHandText{sys/sys.h} \\       
\end{pHeader}

Функция вызывает снятие управления и полное отключение двигателя, номер которого определяется аргументом функции, с последующим остановом в режиме свободного выбега (категория останова 0).\killoverfullbefore

 Возвращаемое значение равно 0 при отсутствии ошибок и отлично от 0 в противном случае.\killoverfullbefore

Является системной.
% *******end section*****************
%--------------------------------------------------------
% *******begin subsection***************
\subsubsection{\DbgSecSt{\StPart}{killMulti}}
\index{Программный интерфейс ПЛК!Управление движением!Функция killMulti}
\label{sec:killMulti}

\begin{pHeader}
    Синтаксис:      & \RightHandText{int killMulti(int motors);}\\
   Аргумент(ы):  & \RightHandText{int motors ~-- номера двигателей} \\ 
%    Возвращаемое значение:       & \RightHandText{Нет} \\ 
    Файл объявления:             & \RightHandText{sys/sys.h} \\       
\end{pHeader}

Функция снятие управления и полное отключение двигателей, номера которых определяются аргументом функции, с последующим остановом в режиме свободного выбега (категория останова 0). \killoverfullbefore

Аргумент функции – битовое поле, в котором номера установленных битов (значения которых равны 1) соответствуют номерам отключаемых двигателей.\killoverfullbefore

 Возвращаемое значение равно 0 при отсутствии ошибок и отлично от 0 в противном случае.\killoverfullbefore

Является системной.
% *******end section*****************
%--------------------------------------------------------
% *******begin subsection***************
\subsubsection{\DbgSecSt{\StPart}{dkill}}
\index{Программный интерфейс ПЛК!Управление движением!Функция dkill}
\label{sec:dkill}

\begin{pHeader}
    Синтаксис:      & \RightHandText{int dkill(int motor);}\\
   Аргумент(ы):  & \RightHandText{int motor ~-- номер двигателя} \\ 
%    Возвращаемое значение:       & \RightHandText{Нет} \\ 
    Файл объявления:             & \RightHandText{sys/sys.h} \\       
\end{pHeader}

Функция вызывает снятие управления и полное отключение двигателя, номер которого определяется аргументом функции, с задержкой на включение тормоза (категория останова 0).\killoverfullbefore

 Возвращаемое значение равно 0 при отсутствии ошибок и отлично от 0 в противном случае.\killoverfullbefore

Является системной.
% *******end section*****************
%--------------------------------------------------------
% *******begin subsection***************
\subsubsection{\DbgSecSt{\StPart}{dkillMulti}}
\index{Программный интерфейс ПЛК!Управление движением!Функция dkillMulti}
\label{sec:dkillMulti}

\begin{pHeader}
    Синтаксис:      & \RightHandText{int dkillMulti(int motors);}\\
   Аргумент(ы):  & \RightHandText{int motors ~-- номера двигателей} \\ 
%    Возвращаемое значение:       & \RightHandText{Нет} \\ 
    Файл объявления:             & \RightHandText{sys/sys.h} \\       
\end{pHeader}

Функция вызывает снятие управления и полное отключение двигателей, номера которых определяются аргументом функции, с задержкой на включение тормоза (категория останова 0). \killoverfullbefore

Аргумент функции – битовое поле, в котором номера установленных битов (значения которых равны 1) соответствуют номерам отключаемых двигателей.\killoverfullbefore

 Возвращаемое значение равно 0 при отсутствии ошибок и отлично от 0 в противном случае.\killoverfullbefore

Является системной.
% *******end section*****************
%--------------------------------------------------------
% *******begin subsection***************
\subsubsection{\DbgSecSt{\StPart}{abortMotor}}
\index{Программный интерфейс ПЛК!Управление движением!Функция abortMotor}
\label{sec:aborMotort}

\begin{pHeader}
    Синтаксис:      & \RightHandText{int abortMotor(int motor);}\\
   Аргумент(ы):  & \RightHandText{int motor ~-- номер двигателя} \\ 
%    Возвращаемое значение:       & \RightHandText{Нет} \\ 
    Файл объявления:             & \RightHandText{sys/sys.h} \\       
\end{pHeader}

Функция выполняет управляемый аварийный останов двигателя, номер которого определяется аргументом функции. После останова двигатель либо выключается (категория останова 1) либо остается в слежении (категория останова 2).\killoverfullbefore

 Возвращаемое значение равно 0 при отсутствии ошибок и отлично от 0 в противном случае.\killoverfullbefore

Является системной.
% *******end section*****************
%--------------------------------------------------------
% *******begin subsection***************
\subsubsection{\DbgSecSt{\StPart}{abortMotorMulti}}
\index{Программный интерфейс ПЛК!Управление движением!Функция abortMotorMulti}
\label{sec:abortMotorMulti}

\begin{pHeader}
    Синтаксис:      & \RightHandText{int abortMotorMulti(int motors);}\\
   Аргумент(ы):  & \RightHandText{int motors ~-- номера двигателей} \\ 
%    Возвращаемое значение:       & \RightHandText{Нет} \\ 
    Файл объявления:             & \RightHandText{sys/sys.h} \\       
\end{pHeader}

Функция выполняет управляемый аварийный останов двигателей, номера которых определяются аргументом функции. После останова двигатели либо выключаются (категория останова 1) либо остаются в слежении (категория останова 2). \killoverfullbefore

Аргумент функции – битовое поле, в котором номера установленных битов (значения которых равны 1) соответствуют номерам останавливаемых двигателей. \killoverfullbefore

Возвращаемое значение равно 0 при отсутствии ошибок и отлично от 0 в противном случае.\killoverfullbefore

Является системной.
% *******end section*****************
%--------------------------------------------------------
% *******begin subsection***************
\subsubsection{\DbgSecSt{\StPart}{adisableMotor}}
\index{Программный интерфейс ПЛК!Управление движением!Функция adisableMotor}
\label{sec:adisableMotor}

\begin{pHeader}
    Синтаксис:      & \RightHandText{int adisableMotor(int motor);}\\
    Аргумент(ы):    & \RightHandText{int motor ~--  номер двигателя} \\   
%    Возвращаемое значение:       & \RightHandText{Нет} \\
    Файл объявления:             & \RightHandText{sys/sys.h} \\      
\end{pHeader}

Функция выполняет управляемый аварийный останов двигателя, номер которого определяется аргументом функции, с последующим отключением с задержкой на включение тормоза (категория останова 1).\killoverfullbefore

Возвращаемое значение равно 0 при отсутствии ошибок и отлично от 0 в противном случае.

Является системной. 
% *******end subsection*****************
%--------------------------------------------------------
% *******begin subsection***************
\subsubsection{\DbgSecSt{\StPart}{adisableMotorMulti}}
\index{Программный интерфейс ПЛК!Управление движением!Функция adisableMotorMulti}
\label{sec:adisableMotorMulti}

\begin{pHeader}
    Синтаксис:      & \RightHandText{int adisableMotorMulti(int motors);}\\
    Аргумент(ы):    & \RightHandText{int motors ~--  номера двигателей} \\   
%    Возвращаемое значение:       & \RightHandText{Нет} \\
    Файл объявления:             & \RightHandText{sys/sys.h} \\      
\end{pHeader}

Функция выполняет управляемый аварийный останов двигателей, номера которых определяются аргументом функции, с последующим их отключением с задержкой на включение тормоза (категория останова 1). \killoverfullbefore

Аргумент функции – битовое поле, в котором номера установленных битов (значения которых равны 1) соответствуют номерам останавливаемых двигателей. \killoverfullbefore

Возвращаемое значение равно 0 при отсутствии ошибок и отлично от 0 в противном случае.

Является системной. 
% *******end subsection*****************
%--------------------------------------------------------
% *******begin subsection***************
\subsubsection{\DbgSecSt{\StPart}{assignMotor}}
\index{Программный интерфейс ПЛК!Управление движением!Функция assignMotor}
\label{sec:assignMotor}

\begin{pHeader}
    Синтаксис:      & \RightHandText{int assignMotor(int motor, const MotorDefinition \&def);}\\
    Аргумент(ы):    & \RightHandText{int motor ~-- номер двигателя,} \\ 
     & \RightHandText {const \myreftosec{MotorDefinition} \&def ~-- параметры привязки двигателя к оси} \\  
%    Возвращаемое значение:       & \RightHandText{Нет} \\
    Файл объявления:             & \RightHandText{sys/sys.h} \\      
\end{pHeader}

Функция выполняет привязку двигателя, номер которого определяется аргументом функции, к оси координатной системы.\killoverfullbefore

 Возвращаемое значение равно 0 при отсутствии ошибок и отлично от 0 в противном случае.\killoverfullbefore

Является системной. 
% *******end subsection*****************
%--------------------------------------------------------
% *******begin subsection***************
\subsubsection{\DbgSecSt{\StPart}{assignMotorInverse}}
\index{Программный интерфейс ПЛК!Управление движением!Функция assignMotorInverse}
\label{sec:assignMotorInverse}

\begin{pHeader}
    Синтаксис:      & \RightHandText{int assignMotorInverse(int motor);}\\
    Аргумент(ы):    & \RightHandText{int motor ~-- номер двигателя} \\  
%    Возвращаемое значение:       & \RightHandText{Нет} \\
    Файл объявления:             & \RightHandText{sys/sys.h} \\      
\end{pHeader}

Функция выполняет привязку двигателя, номер которого определяется аргументом функции, к оси  инверсной кинематики.\killoverfullbefore

 Возвращаемое значение равно 0 при отсутствии ошибок и отлично от 0 в противном случае.\killoverfullbefore

Является системной. 
% *******end subsection*****************
%--------------------------------------------------------
% *******begin subsection***************
\subsubsection{\DbgSecSt{\StPart}{assignMotorSpindle}}
\index{Программный интерфейс ПЛК!Управление движением!Функция assignMotorSpindle}
\label{sec:assignMotorSpindle}

\begin{pHeader}
    Синтаксис:      & \RightHandText{int assignMotorSpindle(int motor, SpindleTimeBase mode);}\\
    Аргумент(ы):    & \RightHandText{int motor ~-- номер двигателя,} \\ 
    & \RightHandText {\myreftosec{SpindleTimeBase} mode ~-- идентификатор временной развёртки} \\   
%    Возвращаемое значение:       & \RightHandText{Нет} \\
    Файл объявления:             & \RightHandText{sys/sys.h} \\      
\end{pHeader}

Функция выполняет привязку двигателя, номер которого определяется аргументом функции, к шпиндельной оси.\killoverfullbefore

 Возвращаемое значение равно 0 при отсутствии ошибок и отлично от 0 в противном случае.\killoverfullbefore

Является системной. 
% *******end subsection*****************
%--------------------------------------------------------
% *******begin subsection***************
\subsubsection{\DbgSecSt{\StPart}{unassignMotor}}
\index{Программный интерфейс ПЛК!Управление движением!Функция unassignMotor}
\label{sec:unassignMotor}

\begin{pHeader}
    Синтаксис:      & \RightHandText{int unassignMotor(int motor);}\\
    Аргумент(ы):    & \RightHandText{int motor ~-- номер двигателя} \\   
%    Возвращаемое значение:       & \RightHandText{Нет} \\
    Файл объявления:             & \RightHandText{sys/sys.h} \\      
\end{pHeader}

Функция выполняет отвязку двигателя, номер которого определяется аргументом функции, от оси (обнуляет масштабирующие коэффициенты, связывающие положение двигателя и координаты осей).\killoverfullbefore

 Возвращаемое значение равно 0 при отсутствии ошибок и отлично от 0 в противном случае.\killoverfullbefore

Является системной. 
% *******end subsection*****************
%--------------------------------------------------------
% *******begin subsection***************
\subsubsection{\DbgSecSt{\StPart}{phaseref}}
\index{Программный интерфейс ПЛК!Управление движением!Функция phaseref}
\label{sec:phaseref}

\begin{pHeader}
    Синтаксис:      & \RightHandText{int phaseref(int motor);}\\
    Аргумент(ы):    & \RightHandText{int motor ~-- номер двигателя} \\   
%    Возвращаемое значение:       & \RightHandText{Нет} \\
    Файл объявления:             & \RightHandText{sys/sys.h} \\      
\end{pHeader}

Функция вызывает выполнение фазировки двигателем, номер которого определяется аргументом функции.\killoverfullbefore

 Возвращаемое значение равно 0 при отсутствии ошибок и отлично от 0 в противном случае.\killoverfullbefore

Является системной. 
% *******end subsection*****************
%--------------------------------------------------------
% *******begin subsection***************
\subsubsection{\DbgSecSt{\StPart}{phaserefMulti}}
\index{Программный интерфейс ПЛК!Управление движением!Функция phaserefMulti}
\label{sec:phaserefMulti}

\begin{pHeader}
    Синтаксис:      & \RightHandText{int phaserefMulti(int motors);}\\
   Аргумент(ы):  & \RightHandText{int motors ~-- номера двигателей} \\ 
%    Возвращаемое значение:       & \RightHandText{Нет} \\ 
    Файл объявления:             & \RightHandText{sys/sys.h} \\       
\end{pHeader}

Функция вызывает выполнение фазировки двигателями, номера которых определяются аргументом функции. \killoverfullbefore

Аргумент функции – битовое поле, в котором номера установленных битов (значения которых равны 1) соответствуют номерам двигателей.\killoverfullbefore

 Возвращаемое значение равно 0 при отсутствии ошибок и отлично от 0 в противном случае.\killoverfullbefore

Является системной.
% *******end section*****************
%--------------------------------------------------------
% *******begin subsection***************
\subsubsection{\DbgSecSt{\StPart}{home}}
\index{Программный интерфейс ПЛК!Управление движением!Функция home}
\label{sec:home}

\begin{pHeader}
    Синтаксис:      & \RightHandText{int home(int motor);}\\
    Аргумент(ы):    & \RightHandText{int motor ~-- номер двигателя} \\   
%    Возвращаемое значение:       & \RightHandText{Нет} \\
    Файл объявления:             & \RightHandText{sys/sys.h} \\      
\end{pHeader}

Функция вызывает выполнение поиска нулевой точки двигателем, номер которого определяется аргументом функции.\killoverfullbefore

 Возвращаемое значение равно 0 при отсутствии ошибок и отлично от 0 в противном случае.\killoverfullbefore

Является системной. 
% *******end subsection*****************
%--------------------------------------------------------
% *******begin subsection***************
\subsubsection{\DbgSecSt{\StPart}{homeMulti}}
\index{Программный интерфейс ПЛК!Управление движением!Функция homeMulti}
\label{sec:homeMulti}

\begin{pHeader}
    Синтаксис:      & \RightHandText{int homeMulti(int motors);}\\
   Аргумент(ы):  & \RightHandText{int motors ~-- номера двигателей} \\ 
%    Возвращаемое значение:       & \RightHandText{Нет} \\ 
    Файл объявления:             & \RightHandText{sys/sys.h} \\       
\end{pHeader}

Функция вызывает выполнение поиска нулевой точки двигателями, номера которых определяются аргументом функции. \killoverfullbefore

Аргумент функции – битовое поле, в котором номера установленных битов (значения которых равны 1) соответствуют номерам двигателей.\killoverfullbefore

 Возвращаемое значение равно 0 при отсутствии ошибок и отлично от 0 в противном случае.\killoverfullbefore

Является системной.
% *******end section*****************
%--------------------------------------------------------
% *******begin subsection***************
\subsubsection{\DbgSecSt{\StPart}{homez}}
\index{Программный интерфейс ПЛК!Управление движением!Функция homez}
\label{sec:homez}

\begin{pHeader}
    Синтаксис:      & \RightHandText{int homez(int motor);}\\
    Аргумент(ы):    & \RightHandText{int motor ~-- номер двигателя} \\   
%    Возвращаемое значение:       & \RightHandText{Нет} \\
    Файл объявления:             & \RightHandText{sys/sys.h} \\      
\end{pHeader}

Функция вызывает установку новой позиции нулевой точки для двигателя, номер которого определяется аргументом функции.\killoverfullbefore

 Возвращаемое значение равно 0 при отсутствии ошибок и отлично от 0 в противном случае.\killoverfullbefore

Является системной. 
% *******end subsection*****************
%--------------------------------------------------------
% *******begin subsection***************
\subsubsection{\DbgSecSt{\StPart}{homezMulti}}
\index{Программный интерфейс ПЛК!Управление движением!Функция homezMulti}
\label{sec:homezMulti}

\begin{pHeader}
    Синтаксис:      & \RightHandText{int homezMulti(int motors);}\\
   Аргумент(ы):  & \RightHandText{int motors ~-- номера двигателей} \\ 
%    Возвращаемое значение:       & \RightHandText{Нет} \\ 
    Файл объявления:             & \RightHandText{sys/sys.h} \\       
\end{pHeader}

Функция вызывает установку новой позиции нулевой точки для двигателей, номера которых определяются аргументом функции. \killoverfullbefore

Аргумент функции – битовое поле, в котором номера установленных битов (значения которых равны 1) соответствуют номерам двигателей.\killoverfullbefore

 Возвращаемое значение равно 0 при отсутствии ошибок и отлично от 0 в противном случае.\killoverfullbefore

Является системной.
% *******end section*****************
%--------------------------------------------------------
% *******begin subsection***************
\subsubsection{\DbgSecSt{\StPart}{jogPlus}}
\index{Программный интерфейс ПЛК!Управление движением!Функция jogPlus}
\label{sec:jogPlus}

\begin{pHeader}
    Синтаксис:      & \RightHandText{int jogPlus(int motor);}\\
    Аргумент(ы):    & \RightHandText{int motor ~-- номер двигателя} \\   
%    Возвращаемое значение:       & \RightHandText{Нет} \\
    Файл объявления:             & \RightHandText{sys/sys.h} \\      
\end{pHeader}

Функция вызывает толчковое перемещение в положительном направлении двигателем, номер которого определяется аргументом функции.\killoverfullbefore

 Возвращаемое значение равно 0 при отсутствии ошибок и отлично от 0 в противном случае.\killoverfullbefore

Является системной. 
% *******end subsection*****************
%--------------------------------------------------------
% *******begin subsection***************
\subsubsection{\DbgSecSt{\StPart}{jogMotorsPlus}}
\index{Программный интерфейс ПЛК!Управление движением!Функция jogMotorsPlus}
\label{sec:jogMotorsPlus}

\begin{pHeader}
    Синтаксис:      & \RightHandText{int jogMotorsPlus(int motors);}\\
    Аргумент(ы):    & \RightHandText{int motors ~-- номера двигателей} \\   
%    Возвращаемое значение:       & \RightHandText{Нет} \\
    Файл объявления:             & \RightHandText{sys/sys.h} \\      
\end{pHeader}

Функция вызывает толчковое перемещение в положительном направлении двигателями, номера которых определяются аргументом функции. \killoverfullbefore

Аргумент функции – битовое поле, в котором номера установленных битов (значения которых равны 1) соответствуют номерам двигателей.\killoverfullbefore

 Возвращаемое значение равно 0 при отсутствии ошибок и отлично от 0 в противном случае.\killoverfullbefore

Является системной. 
% *******end subsection*****************
%--------------------------------------------------------
% *******begin subsection***************
\subsubsection{\DbgSecSt{\StPart}{jogMinus}}
\index{Программный интерфейс ПЛК!Управление движением!Функция jogMinus}
\label{sec:jogMinus}

\begin{pHeader}
    Синтаксис:      & \RightHandText{int jogMinus(int motor);}\\
    Аргумент(ы):    & \RightHandText{int motor ~-- номер двигателя} \\   
%    Возвращаемое значение:       & \RightHandText{Нет} \\
    Файл объявления:             & \RightHandText{sys/sys.h} \\      
\end{pHeader}

Функция вызывает толчковое перемещение в отрицательном направлении двигателем, номер которого определяется аргументом функции.\killoverfullbefore

 Возвращаемое значение равно 0 при отсутствии ошибок и отлично от 0 в противном случае.\killoverfullbefore

Является системной. 
% *******end subsection*****************
%--------------------------------------------------------
% *******begin subsection***************
\subsubsection{\DbgSecSt{\StPart}{jogMotorsMinus}}
\index{Программный интерфейс ПЛК!Управление движением!Функция jogMotorsMinus}
\label{sec:jogMotorsMinus}

\begin{pHeader}
    Синтаксис:      & \RightHandText{int jogMotorsMinus(int motors);}\\
    Аргумент(ы):    & \RightHandText{int motors ~-- номера двигателей} \\   
%    Возвращаемое значение:       & \RightHandText{Нет} \\
    Файл объявления:             & \RightHandText{sys/sys.h} \\      
\end{pHeader}

Функция вызывает толчковое перемещение в отрицательном направлении двигателями, номера которых определяются аргументом функции.\killoverfullbefore

 Аргумент функции – битовое поле, в котором номера установленных битов (значения которых равны 1) соответствуют номерам двигателей.\killoverfullbefore

 Возвращаемое значение равно 0 при отсутствии ошибок и отлично от 0 в противном случае.\killoverfullbefore

Является системной. 
% *******end subsection*****************
%--------------------------------------------------------
% *******begin subsection***************
\subsubsection{\DbgSecSt{\StPart}{jogStop}}
\index{Программный интерфейс ПЛК!Управление движением!Функция jogStop}
\label{sec:jogStop}

\begin{pHeader}
    Синтаксис:      & \RightHandText{int jogStop(int motor);}\\
    Аргумент(ы):    & \RightHandText{int motor ~-- номер двигателя} \\   
%    Возвращаемое значение:       & \RightHandText{Нет} \\
    Файл объявления:             & \RightHandText{sys/sys.h} \\      
\end{pHeader}

Функция вызывает останов толчкового перемещение двигателя, номер которого определяется аргументом функции. \killoverfullbefore

Возвращаемое значение равно 0 при отсутствии ошибок и отлично от 0 в противном случае.\killoverfullbefore

Является системной. 
% *******end subsection*****************
%--------------------------------------------------------
% *******begin subsection***************
\subsubsection{\DbgSecSt{\StPart}{jogMotorsStop}}
\index{Программный интерфейс ПЛК!Управление движением!Функция jogMotorsStop}
\label{sec:jogMotorsStop}

\begin{pHeader}
    Синтаксис:      & \RightHandText{int jogMotorsStop(int motors);}\\
    Аргумент(ы):    & \RightHandText{int motors ~-- номера двигателей} \\   
%    Возвращаемое значение:       & \RightHandText{Нет} \\
    Файл объявления:             & \RightHandText{sys/sys.h} \\      
\end{pHeader}

Функция вызывает останов толчкового перемещения двигателей, номера которых определяются аргументом функции.\killoverfullbefore

 Аргумент функции – битовое поле, в котором номера установленных битов (значения которых равны 1) соответствуют номерам двигателей.\killoverfullbefore

 Возвращаемое значение равно 0 при отсутствии ошибок и отлично от 0 в противном случае.\killoverfullbefore

Является системной. 
% *******end subsection*****************
%--------------------------------------------------------
% *******begin subsection***************
\subsubsection{\DbgSecSt{\StPart}{jogTo}}
\index{Программный интерфейс ПЛК!Управление движением!Функция jogTo}
\label{sec:jogTo}

\begin{pHeader}
    Синтаксис:      & \RightHandText{int jogTo(int motor, double target);}\\
    Аргумент(ы):    & \RightHandText{int motor ~-- номер двигателя,} \\   
     & \RightHandText{double target ~-- заданная позиция} \\ 
%    Возвращаемое значение:       & \RightHandText{Нет} \\
    Файл объявления:             & \RightHandText{sys/sys.h} \\      
\end{pHeader}

Функция вызывает толчковое движение в заданную позицию относительно
нулевой точки двигателя, номер которого определяется аргументом функции.\killoverfullbefore

 Возвращаемое значение равно 0 при отсутствии ошибок и отлично от 0 в противном случае.\killoverfullbefore

Является системной. 
% *******end subsection*****************
%--------------------------------------------------------
% *******begin subsection***************
\subsubsection{\DbgSecSt{\StPart}{jogMotorsTo}}
\index{Программный интерфейс ПЛК!Управление движением!Функция jogMotorsTo}
\label{sec:jogMotorsTo}

\begin{pHeader}
    Синтаксис:      & \RightHandText{int jogMotorsTo(JogTarget target);}\\
    Аргумент(ы):    & \RightHandText{\myreftosec{JogTarget} target ~-- заданные позиции} \\   
%    Возвращаемое значение:       & \RightHandText{Нет} \\
    Файл объявления:             & \RightHandText{sys/sys.h} \\      
\end{pHeader}

Функция вызывает толчковое движение в заданные позиции относительно
нулевой точки двигателей, номера которых определяются аргументом функции.\killoverfullbefore

 Аргумент функции ~-- структура \myreftosec{JogTarget}, в которой номера ячеек массива со значениями, отличными от \texttt{NAN}, соответствуют номерам двигателей, а сами значения ячеек являются заданными позициями. \killoverfullbefore

Возвращаемое значение равно 0 при отсутствии ошибок и отлично от 0 в противном случае.\killoverfullbefore

Является системной. 
% *******end subsection*****************
%--------------------------------------------------------
% *******begin subsection***************
\subsubsection{\DbgSecSt{\StPart}{jogRelToCmd}}
\index{Программный интерфейс ПЛК!Управление движением!Функция jogRelToCmd}
\label{sec:jogRelToCmd}

\begin{pHeader}
    Синтаксис:      & \RightHandText{int jogRelToCmd(int motor, double target);}\\
    Аргумент(ы):    & \RightHandText{int motor ~-- номер двигателя,} \\   
     & \RightHandText{double target ~-- заданное расстояние} \\ 
%    Возвращаемое значение:       & \RightHandText{Нет} \\
    Файл объявления:             & \RightHandText{sys/sys.h} \\      
\end{pHeader}

Функция вызывает толчковое движение на заданное расстояние относительно текущей программной позиции двигателя, номер которого определяется аргументом функции.\killoverfullbefore

 Возвращаемое значение равно 0 при отсутствии ошибок и отлично от 0 в противном случае.\killoverfullbefore

Является системной. 
% *******end subsection*****************
%--------------------------------------------------------
% *******begin subsection***************
\subsubsection{\DbgSecSt{\StPart}{jogMotorsRelToCmd}}
\index{Программный интерфейс ПЛК!Управление движением!Функция jogMotorsRelToCmd}
\label{sec:jogMotorsRelToCmd}

\begin{pHeader}
    Синтаксис:      & \RightHandText{int jogMotorsRelToCmd(JogTarget target);}\\
    Аргумент(ы):    & \RightHandText{\myreftosec{JogTarget} target ~-- заданные расстояния} \\   
%    Возвращаемое значение:       & \RightHandText{Нет} \\
    Файл объявления:             & \RightHandText{sys/sys.h} \\      
\end{pHeader}

Функция вызывает толчковое движение на заданные расстояния относительно текущей программной позиции двигателей, номера которых определяются аргументом функции.\killoverfullbefore

 Аргумент функции ~-- структура \myreftosec{JogTarget}, в которой номера ячеек массива со значениями, отличными от \texttt{NAN}, соответствуют номерам двигателей, а сами значения ячеек являются заданными расстояниями.\killoverfullbefore

 Возвращаемое значение равно 0 при отсутствии ошибок и отлично от 0 в противном случае.\killoverfullbefore

Является системной. 
% *******end subsection*****************
%--------------------------------------------------------
% *******begin subsection***************
\subsubsection{\DbgSecSt{\StPart}{jogRelToAct}}
\index{Программный интерфейс ПЛК!Управление движением!Функция jogRelToAct}
\label{sec:jogRelToAct}

\begin{pHeader}
    Синтаксис:      & \RightHandText{int jogRelToAct(int motor, double target);}\\
    Аргумент(ы):    & \RightHandText{int motor ~-- номер двигателя,} \\   
     & \RightHandText{double target ~-- заданное расстояние} \\ 
%    Возвращаемое значение:       & \RightHandText{Нет} \\
    Файл объявления:             & \RightHandText{sys/sys.h} \\      
\end{pHeader}

Функция вызывает толчковое движение на заданное расстояние относительно текущей фактической позиции двигателя, номер которого определяется аргументом функции.\killoverfullbefore

 Возвращаемое значение равно 0 при отсутствии ошибок и отлично от 0 в противном случае.\killoverfullbefore

Является системной. 
% *******end subsection*****************
%--------------------------------------------------------
% *******begin subsection***************
\subsubsection{\DbgSecSt{\StPart}{jogMotorsRelToAct}}
\index{Программный интерфейс ПЛК!Управление движением!Функция jogMotorsRelToAct}
\label{sec:jogMotorsRelToAct}

\begin{pHeader}
    Синтаксис:      & \RightHandText{int jogMotorsRelToAct(JogTarget target);}\\
    Аргумент(ы):    & \RightHandText{\myreftosec{JogTarget} target ~-- заданные расстояния} \\   
%    Возвращаемое значение:       & \RightHandText{Нет} \\
    Файл объявления:             & \RightHandText{sys/sys.h} \\      
\end{pHeader}

Функция вызывает толчковое движение на заданные расстояния относительно текущей фактической позиции двигателей, номера которых определяются аргументом функции.\killoverfullbefore

 Аргумент функции ~-- структура \myreftosec{JogTarget}, в которой номера ячеек массива со значениями, отличными от \texttt{NAN}, соответствуют номерам двигателей, а сами значения ячеек являются заданными расстояниями.\killoverfullbefore

 Возвращаемое значение равно 0 при отсутствии ошибок и отлично от 0 в противном случае.\killoverfullbefore

Является системной. 
% *******end subsection*****************
%--------------------------------------------------------
% *******begin subsection***************
\subsubsection{\DbgSecSt{\StPart}{jogRet}}
\index{Программный интерфейс ПЛК!Управление движением!Функция jogRet}
\label{sec:jogRet}

\begin{pHeader}
    Синтаксис:      & \RightHandText{int jogRet(int motor);}\\
    Аргумент(ы):    & \RightHandText{int motor ~-- номер двигателя} \\ 
%    Возвращаемое значение:       & \RightHandText{Нет} \\
    Файл объявления:             & \RightHandText{sys/sys.h} \\      
\end{pHeader}

Функция вызывает толчковое движение в сохранённую позицию двигателем, номер которого определяется аргументом функции. \killoverfullbefore

Возвращаемое значение равно 0 при отсутствии ошибок и отлично от 0 в противном случае.\killoverfullbefore

Является системной. 
% *******end subsection*****************
%--------------------------------------------------------
% *******begin subsection***************
\subsubsection{\DbgSecSt{\StPart}{jogMotorsRet}}
\index{Программный интерфейс ПЛК!Управление движением!Функция jogMotorsRet}
\label{sec:jogMotorsRet}

\begin{pHeader}
    Синтаксис:      & \RightHandText{int jogMotorsRet(int motors);}\\
    Аргумент(ы):    & \RightHandText{int motors ~-- номера двигателей} \\   
%    Возвращаемое значение:       & \RightHandText{Нет} \\
    Файл объявления:             & \RightHandText{sys/sys.h} \\      
\end{pHeader}

Функция вызывает толчковое движение в сохранённую позицию двигателями, номера которых определяются аргументом функции.\killoverfullbefore

 Аргумент функции – битовое поле, в котором номера установленных битов (значения которых равны 1) соответствуют номерам двигателей.\killoverfullbefore

 Возвращаемое значение равно 0 при отсутствии ошибок и отлично от 0 в противном случае.\killoverfullbefore

Является системной. 
% *******end subsection*****************
%--------------------------------------------------------
% *******begin subsection***************
\subsubsection{\DbgSecSt{\StPart}{jogToSave}}
\index{Программный интерфейс ПЛК!Управление движением!Функция jogToSave}
\label{sec:jogToSave}

\begin{pHeader}
    Синтаксис:      & \RightHandText{int jogToSave(int motor, double target);}\\
    Аргумент(ы):    & \RightHandText{int motor ~-- номер двигателя,} \\   
     & \RightHandText{double target ~-- заданная позиция} \\ 
%    Возвращаемое значение:       & \RightHandText{Нет} \\
    Файл объявления:             & \RightHandText{sys/sys.h} \\      
\end{pHeader}

Функция вызывает толчковое движение в заданную позицию относительно
нулевой точки двигателя, номер которого определяется аргументом функции, и сохранение значения конечного положения. \killoverfullbefore

Возвращаемое значение равно 0 при отсутствии ошибок и отлично от 0 в противном случае.\killoverfullbefore

Является системной. 
% *******end subsection*****************
%--------------------------------------------------------
% *******begin subsection***************
\subsubsection{\DbgSecSt{\StPart}{jogMotorsToSave}}
\index{Программный интерфейс ПЛК!Управление движением!Функция jogMotorsToSave}
\label{sec:jogMotorsToSave}

\begin{pHeader}
    Синтаксис:      & \RightHandText{int jogMotorsToSave(JogTarget target);}\\
    Аргумент(ы):    & \RightHandText{\myreftosec{JogTarget} target ~-- заданные позиции} \\   
%    Возвращаемое значение:       & \RightHandText{Нет} \\
    Файл объявления:             & \RightHandText{sys/sys.h} \\      
\end{pHeader}

Функция вызывает толчковое движение в заданные позиции относительно
нулевой точки двигателей, номера которых определяются аргументом функции, и сохранение значений конечного положения.\killoverfullbefore

 Аргумент функции ~-- структура \myreftosec{JogTarget}, в которой номера ячеек массива со значениями, отличными от \texttt{NAN}, соответствуют номерам двигателей, а сами значения ячеек являются заданными позициями. \killoverfullbefore

Возвращаемое значение равно 0 при отсутствии ошибок и отлично от 0 в противном случае.\killoverfullbefore

Является системной. 
% *******end subsection*****************
%--------------------------------------------------------
% *******begin subsection***************
\subsubsection{\DbgSecSt{\StPart}{absAxes}}
\index{Программный интерфейс ПЛК!Управление движением!Функция absAxes}
\label{sec:absAxes}

\begin{pHeader}
    Синтаксис:      & \RightHandText{int absAxes(unsigned axes);}\\
    Аргумент(ы):    & \RightHandText{unsigned axes ~--  номера осей} \\   
%    Возвращаемое значение:       & \RightHandText{Нет} \\
    Файл объявления:             & \RightHandText{sys/sys.h} \\
\end{pHeader}

Функция устанавливает абсолютный режим перемещений для осей, номера которых определяются аргументом функции. В данном режиме программируется величина конечного положения.\killoverfullbefore

Аргумент функции – битовое поле, в котором номера установленных битов (значения которых равны 1) соответствуют номерам осей.\killoverfullbefore

Возвращаемое значение равно 0 при отсутствии ошибок и отлично от 0 в противном случае.\killoverfullbefore

Является системной. 
% *******end subsection*****************
%--------------------------------------------------------
% *******begin subsection***************
\subsubsection{\DbgSecSt{\StPart}{incAxes}}
\index{Программный интерфейс ПЛК!Управление движением!Функция incAxes}
\label{sec:incAxes}

\begin{pHeader}
    Синтаксис:      & \RightHandText{int incAxes(unsigned axes);}\\
    Аргумент(ы):    & \RightHandText{unsigned axes ~-- номера осей} \\   
%    Возвращаемое значение:       & \RightHandText{Нет} \\
    Файл объявления:             & \RightHandText{sys/sys.h} \\      
\end{pHeader}

Функция устанавливает относительный режим перемещений для осей, номера которых определяются аргументом функции. В данном режиме программируется величина перемещения от текущего положения.\killoverfullbefore

Аргумент функции – битовое поле, в котором номера установленных битов (значения которых равны 1) соответствуют номерам осей.\killoverfullbefore

Возвращаемое значение равно 0 при отсутствии ошибок и отлично от 0 в противном случае.\killoverfullbefore

Является системной. 
% *******end subsection*****************
%--------------------------------------------------------
% *******begin subsection***************
\subsubsection{\DbgSecSt{\StPart}{absVectors}}
\index{Программный интерфейс ПЛК!Управление движением!Функция absVectors}
\label{sec:absVectors}

\begin{pHeader}
    Синтаксис:      & \RightHandText{int absVectors(unsigned vectors);}\\
    Аргумент(ы):    & \RightHandText{unsigned vectors ~-- номера векторов} \\   
%    Возвращаемое значение:       & \RightHandText{Нет} \\
    Файл объявления:             & \RightHandText{sys/sys.h} \\      
\end{pHeader}

Функция устанавливает абсолютный режим для задающих центр окружности компонент вектора, номера которых определяются аргументом функции. В данном режиме компоненты  I, J, K, II, JJ, KK параллельные осям X, Y, Z, XX, XY, XZ соответственно, определяют расстояние от начала координат до центра окружности. \killoverfullbefore

Аргумент функции – битовое поле, в котором номера установленных битов (значения которых равны 1) соответствуют номерам векторов.\killoverfullbefore

Возвращаемое значение равно 0 при отсутствии ошибок и отлично от 0 в противном случае.\killoverfullbefore

Является системной. 
% *******end subsection*****************
%--------------------------------------------------------
% *******begin subsection***************
\subsubsection{\DbgSecSt{\StPart}{incVectors}}
\index{Программный интерфейс ПЛК!Управление движением!Функция incVectors}
\label{sec:incVectors}

\begin{pHeader}
    Синтаксис:      & \RightHandText{int incVectors(unsigned vectors);}\\
    Аргумент(ы):    & \RightHandText{unsigned vectors ~--  номера векторов} \\   
%    Возвращаемое значение:       & \RightHandText{Нет} \\
    Файл объявления:             & \RightHandText{sys/sys.h} \\      
\end{pHeader}

Функция устанавливает относительный режим для задающих центр окружности компонент вектора, номера которых определяются аргументом функции. В данном режиме компоненты  I, J, K, II, JJ, KK параллельные осям X, Y, Z, XX, XY, XZ соответственно, определяют расстояние от начальной точки перемещения до центра окружности. \killoverfullbefore

Аргумент функции – битовое поле, в котором номера установленных битов (значения которых равны 1) соответствуют номерам векторов.\killoverfullbefore

Возвращаемое значение равно 0 при отсутствии ошибок и отлично от 0 в противном случае.\killoverfullbefore

Является системной. 
% *******end subsection*****************
%--------------------------------------------------------
% *******begin subsection***************
\subsubsection{\DbgSecSt{\StPart}{frax}}
\index{Программный интерфейс ПЛК!Управление движением!Функция frax}
\label{sec:frax}

\begin{pHeader}
    Синтаксис:      & \RightHandText{int frax(unsigned axes);}\\
    Аргумент(ы):    & \RightHandText{unsigned axes ~-- номера осей} \\   
%    Возвращаемое значение:       & \RightHandText{Нет} \\
    Файл объявления:             & \RightHandText{sys/sys.h} \\      
\end{pHeader}

Функция определяет, какие оси должны быть задействованы в расчёте подачи в основной декартовой системы координат (X/Y/Z). 

Аргумент функции – битовое поле, в котором номера установленных битов (значения которых равны 1) соответствуют номерам осей.\killoverfullbefore

Возвращаемое значение равно 0 при отсутствии ошибок и отлично от 0 в противном случае.\killoverfullbefore

Является системной. 
% *******end subsection*****************
%--------------------------------------------------------
% *******begin subsection***************
\subsubsection{\DbgSecSt{\StPart}{frax2}}
\index{Программный интерфейс ПЛК!Управление движением!Функция frax2}
\label{sec:frax2}

\begin{pHeader}
    Синтаксис:      & \RightHandText{int frax2(unsigned axes);}\\
    Аргумент(ы):    & \RightHandText{unsigned axes ~-- номера осей} \\   
%    Возвращаемое значение:       & \RightHandText{Нет} \\
    Файл объявления:             & \RightHandText{sys/sys.h} \\      
\end{pHeader}

Функция определяет, какие оси должны быть задействованы в расчёте подачи в расширенной декартовой системы координат (XX/XY/XZ). 

Аргумент функции – битовое поле, в котором номера установленных битов (значения которых равны 1) соответствуют номерам осей.\killoverfullbefore

Возвращаемое значение равно 0 при отсутствии ошибок и отлично от 0 в противном случае.\killoverfullbefore

Является системной. 
% *******end subsection*****************
%--------------------------------------------------------
% *******begin subsection***************
\subsubsection{\DbgSecSt{\StPart}{nofrax}}
\index{Программный интерфейс ПЛК!Управление движением!Функция nofrax}
\label{sec:nofrax}

\begin{pHeader}
    Синтаксис:      & \RightHandText{int nofrax();}\\
    Аргумент(ы):    & \RightHandText{нет} \\   
%    Возвращаемое значение:       & \RightHandText{Нет} \\
    Файл объявления:             & \RightHandText{sys/sys.h} \\      
\end{pHeader}

Функция отменяет выбор осей, задействованных в расчёте подачи в основной декартовой системы координат (X/Y/Z). \killoverfullbefore

Возвращаемое значение равно 0 при отсутствии ошибок и отлично от 0 в противном случае.\killoverfullbefore

Является системной. 
% *******end subsection*****************
%--------------------------------------------------------
% *******begin subsection***************
\subsubsection{\DbgSecSt{\StPart}{nofrax2}}
\index{Программный интерфейс ПЛК!Управление движением!Функция nofrax2}
\label{sec:nofrax2}

\begin{pHeader}
    Синтаксис:      & \RightHandText{int nofrax2();}\\
    Аргумент(ы):    & \RightHandText{нет} \\   
%    Возвращаемое значение:       & \RightHandText{Нет} \\
    Файл объявления:             & \RightHandText{sys/sys.h} \\      
\end{pHeader}

Функция отменяет выбор осей, задействованных в расчёте подачи в расширенной декартовой системы координат (XX/XY/XZ). \killoverfullbefore

Возвращаемое значение равно 0 при отсутствии ошибок и отлично от 0 в противном случае.\killoverfullbefore

Является системной. 
% *******end subsection*****************
%--------------------------------------------------------
% *******begin subsection***************
\subsubsection{\DbgSecSt{\StPart}{delay}}
\index{Программный интерфейс ПЛК!Управление движением!Функция delay}
\label{sec:delay}

\begin{pHeader}
    Синтаксис:      & \RightHandText{int delay(double time);}\\
    Аргумент(ы):    & \RightHandText{double time ~-- время останова} \\   
%    Возвращаемое значение:       & \RightHandText{Нет} \\
    Файл объявления:             & \RightHandText{sys/sys.h} \\      
\end{pHeader}

Функция останавливает движение всех осей в координатной системе, в которой выполняется УП, на заданное время (удержание программной позиции в течение заданного времени).\killoverfullbefore 

Время останова, измеряемое в мс, включает в себя половину времени торможения и ускорения, не прерывает расчеты в буфере опережающего просмотра  и масштабируется в зависимости от временной развёртки (например при увеличении значения временной развертки на 50\% фактическое время останова в 2 раза превысит заданное). \killoverfullbefore

Возвращаемое значение равно 0 при отсутствии ошибок и отлично от 0 в противном случае.\killoverfullbefore

Является системной. 
% *******end subsection*****************
%--------------------------------------------------------
% *******begin subsection***************
\subsubsection{\DbgSecSt{\StPart}{dwell}}
\index{Программный интерфейс ПЛК!Управление движением!Функция dwell}
\label{sec:dwell}

\begin{pHeader}
    Синтаксис:      & \RightHandText{int dwell(double time);}\\
    Аргумент(ы):    & \RightHandText{double time ~-- время задержки} \\   
%    Возвращаемое значение:       & \RightHandText{Нет} \\
    Файл объявления:             & \RightHandText{sys/sys.h} \\      
\end{pHeader}

Функция останавливает движение всех осей в координатной системе, в которой выполняется УП, на заданное время (удержание программной позиции в течение заданного времени).\killoverfullbefore 

%Функция устанавливает фиксированную временную задержку, которая автоматически вставляется между двумя последовательными программными перемещениями, если их сопряжение не производится.\killoverfullbefore

Время задержки, измеряемое в мс, не учитывает время торможения и ускорения, прерывает расчеты в буфере опережающего просмотра и не зависит от временной развёртки. \killoverfullbefore

Возвращаемое значение равно 0 при отсутствии ошибок и отлично от 0 в противном случае.\killoverfullbefore

Является системной. 
% *******end subsection*****************
%--------------------------------------------------------
% *******begin subsection***************
\subsubsection{\DbgSecSt{\StPart}{setF}}
\index{Программный интерфейс ПЛК!Управление движением!Функция setF}
\label{sec:setF}

\begin{pHeader}
    Синтаксис:      & \RightHandText{int setF(double feedrate);}\\
    Аргумент(ы):    & \RightHandText{double feedrate ~-- величина скорости подачи} \\   
%    Возвращаемое значение:       & \RightHandText{Нет} \\
    Файл объявления:             & \RightHandText{sys/sys.h} \\      
\end{pHeader}

Функция устанавливает скорость подачи, величина которой является аргументом функции.\killoverfullbefore

Возвращаемое значение равно 0 при отсутствии ошибок и отлично от 0 в противном случае.\killoverfullbefore

Является системной. 
% *******end subsection*****************
%--------------------------------------------------------
% *******begin subsection***************
\subsubsection{\DbgSecSt{\StPart}{setS}}
\index{Программный интерфейс ПЛК!Управление движением!Функция setS}
\label{sec:setS}

\begin{pHeader}
    Синтаксис:      & \RightHandText{int setS(double spindle);}\\
    Аргумент(ы):    & \RightHandText{double spindle ~-- величина скорости шпинделя} \\   
%    Возвращаемое значение:       & \RightHandText{Нет} \\
    Файл объявления:             & \RightHandText{sys/sys.h} \\      
\end{pHeader}

Функция устанавливает скорость шпинделя, величина которой является аргументом функции.\killoverfullbefore

Возвращаемое значение равно 0 при отсутствии ошибок и отлично от 0 в противном случае.\killoverfullbefore

Является системной. 
% *******end subsection*****************
%--------------------------------------------------------
% *******begin subsection***************
\subsubsection{\DbgSecSt{\StPart}{ta}}
\index{Программный интерфейс ПЛК!Управление движением!Функция ta}
\label{sec:ta}

\begin{pHeader}
    Синтаксис:      & \RightHandText{int ta(double time);}\\
    Аргумент(ы):    & \RightHandText{double time ~-- время ускорения} \\   
%    Возвращаемое значение:       & \RightHandText{Нет} \\
    Файл объявления:             & \RightHandText{sys/sys.h} \\      
\end{pHeader}

Функция устанавливает время заданного ускорения для программных линейных или круговых движений (время ускорения S-кривой), величина которого является аргументом функции. Оно используется как время начального ускорения после останова в начале последовательности сопряжённых перемещений и при переходах между последовательными перемещениями.\killoverfullbefore

Если данное время больше, чем заданное функцией \texttt{int ts(double time)}, то общее время ускорения будет равно сумме этих времён.\killoverfullbefore

Если данное время меньше, чем заданное функцией \texttt{int ts(double time)}, то общее время ускорения (торможения) будет равно удвоенному значению аргумента функции \texttt{int ts(double time)}.\killoverfullbefore

Возвращаемое значение равно 0 при отсутствии ошибок и отлично от 0 в противном случае.\killoverfullbefore

Является системной. 
% *******end subsection*****************
%--------------------------------------------------------
% *******begin subsection***************
\subsubsection{\DbgSecSt{\StPart}{td}}
\index{Программный интерфейс ПЛК!Управление движением!Функция td}
\label{sec:td}

\begin{pHeader}
    Синтаксис:      & \RightHandText{int td(double time);}\\
    Аргумент(ы):    & \RightHandText{double time ~-- время торможения} \\   
%    Возвращаемое значение:       & \RightHandText{Нет} \\
    Файл объявления:             & \RightHandText{sys/sys.h} \\      
\end{pHeader}

Функция устанавливает время заданного ускорения торможения для программных линейных или круговых движений (время торможения S-кривой), величина которого является аргументом функции. Оно используется как время конечного ускорения торможения до останова в конце последовательности сопряжённых перемещений.\killoverfullbefore

Если данное время больше, чем заданное функцией \texttt{int ts(double time)}, то общее время ускорения будет равно сумме этих времён.\killoverfullbefore

Если данное время меньше, чем заданное функцией \texttt{int ts(double time)}, то общее время ускорения (торможения) будет равно удвоенному значению аргумента функции \texttt{int ts(double time)}.\killoverfullbefore

Возвращаемое значение равно 0 при отсутствии ошибок и отлично от 0 в противном случае.\killoverfullbefore

Является системной. 
% *******end subsection*****************
%--------------------------------------------------------
% *******begin subsection***************
\subsubsection{\DbgSecSt{\StPart}{tm}}
\index{Программный интерфейс ПЛК!Управление движением!Функция tm}
\label{sec:tm}

\begin{pHeader}
    Синтаксис:      & \RightHandText{int tm(double time);}\\
    Аргумент(ы):    & \RightHandText{double time ~-- время или модуль вектора скорости подачи} \\   
%    Возвращаемое значение:       & \RightHandText{Нет} \\
    Файл объявления:             & \RightHandText{sys/sys.h} \\      
\end{pHeader}

Функция устанавливает время или модуль вектора скорости подачи для линейных или круговых движений. \killoverfullbefore

Если значение аргумента больше нуля, то оно определяет время движения в мс. При этом скорость движения будет такой, чтобы перемещение было выполнено за указанное время.\killoverfullbefore

Если значение аргумента меньше нуля, то оно определяет модуль вектора скорости. При этом время движения будет таким, чтобы перемещение было выполнено с указанной скоростью.\killoverfullbefore

Возвращаемое значение равно 0 при отсутствии ошибок и отлично от 0 в противном случае.\killoverfullbefore

Является системной. 
% *******end subsection*****************
%--------------------------------------------------------
% *******begin subsection***************
\subsubsection{\DbgSecSt{\StPart}{ts}}
\index{Программный интерфейс ПЛК!Управление движением!Функция ts}
\label{sec:ts}

\begin{pHeader}
    Синтаксис:      & \RightHandText{int ts(double time);}\\
    Аргумент(ы):    & \RightHandText{double time ~-- время разгона/торможения S-кривой} \\   
%    Возвращаемое значение:       & \RightHandText{Нет} \\
    Файл объявления:             & \RightHandText{sys/sys.h} \\      
\end{pHeader}

Функция устанавливает для каждой половины заданной S-кривой время ускорения для программных линейных или круговых движений. Оно используется как время начального ускорения после останова в начале последовательности сопряжённых перемещений, при переходах между последовательными перемещениями и конечного ускорения торможения до останова в конце последовательности.\killoverfullbefore

Возвращаемое значение равно 0 при отсутствии ошибок и отлично от 0 в противном случае.\killoverfullbefore

Является системной. 
% *******end subsection*****************
%--------------------------------------------------------
% *******begin subsection***************
\subsubsection{\DbgSecSt{\StPart}{abort}}
\index{Программный интерфейс ПЛК!Управление движением!Функция abort}
\label{sec:abort}

\begin{pHeader}
    Синтаксис:      & \RightHandText{int abort(int cs);}\\
   Аргумент(ы):  & \RightHandText{int cs ~-- номер координатной системы} \\ 
%    Возвращаемое значение:       & \RightHandText{Нет} \\ 
    Файл объявления:             & \RightHandText{sys/sys.h} \\       
\end{pHeader}

%Функция выполняет управляемый аварийный останов координатной системы, номер которой определяется аргументом функции. 

Функция выполняет прерывание программы движения для координатной системы, номер которой определяется аргументом функции, а также управляемый аварийный останов двигателей в заданной координатной системе. После останова двигатели либо выключаются (категория останова 1) либо остаются в слежении (категория останова 2).\killoverfullbefore\killoverfullbefore

Возвращаемое значение равно 0 при отсутствии ошибок и отлично от 0 в противном случае.\killoverfullbefore

Является системной.
% *******end section*****************
%--------------------------------------------------------
% *******begin subsection***************
\subsubsection{\DbgSecSt{\StPart}{abortMulti}}
\index{Программный интерфейс ПЛК!Управление движением!Функция abortMulti}
\label{sec:abortMulti}

\begin{pHeader}
    Синтаксис:      & \RightHandText{int abortMulti(int cs);}\\
   Аргумент(ы):  & \RightHandText{int cs ~-- номера координатных систем} \\ 
%    Возвращаемое значение:       & \RightHandText{Нет} \\ 
    Файл объявления:             & \RightHandText{sys/sys.h} \\       
\end{pHeader}

%Функция выполняет управляемый аварийный останов координатных систем. 
Функция выполняет прерывание программ движения для заданных координатных систем, а также управляемый аварийный останов соответствующих двигателей. После останова двигатели либо выключаются (категория останова 1) либо остаются в слежении (категория останова 2). \killoverfullbefore

Аргумент функции – битовое поле, в котором номера установленных битов (значения которых равны 1) соответствуют номерам координатных систем.\killoverfullbefore

Возвращаемое значение равно 0 при отсутствии ошибок и отлично от 0 в противном случае.\killoverfullbefore

Является системной.
% *******end section*****************
%--------------------------------------------------------
% *******begin subsection***************
\subsubsection{\DbgSecSt{\StPart}{adisable}}
\index{Программный интерфейс ПЛК!Управление движением!Функция adisable}
\label{sec:adisable}

\begin{pHeader}
    Синтаксис:      & \RightHandText{int adisable(int cs);}\\
    Аргумент(ы):    & \RightHandText{int cs ~--  номер координатной системы} \\   
%    Возвращаемое значение:       & \RightHandText{Нет} \\
    Файл объявления:             & \RightHandText{sys/sys.h} \\      
\end{pHeader}

Функция выполняет прерывание программы движения для координатной системы, номер которой определяется аргументом функции, а также управляемый аварийный останов  двигателей в заданной координатной системе с последующим их отключением с задержкой на включение тормоза (категория останова 1).\killoverfullbefore

Возвращаемое значение равно 0 при отсутствии ошибок и отлично от 0 в противном случае.\killoverfullbefore

Является системной. 
% *******end subsection*****************
%--------------------------------------------------------
% *******begin subsection***************
\subsubsection{\DbgSecSt{\StPart}{adisableMulti}}
\index{Программный интерфейс ПЛК!Управление движением!Функция adisableMulti}
\label{sec:adisableMulti}

\begin{pHeader}
    Синтаксис:      & \RightHandText{int adisableMulti(int cs);}\\
    Аргумент(ы):    & \RightHandText{int cs ~-- номера координатных систем} \\   
%    Возвращаемое значение:       & \RightHandText{Нет} \\
    Файл объявления:             & \RightHandText{sys/sys.h} \\      
\end{pHeader}

Функция выполняет прерывание программ движения для заданных координатных систем, а также управляемый аварийный останов соответствующих двигателей с последующим их отключением с задержкой на включение тормоза (категория останова 1). \killoverfullbefore

Аргумент функции – битовое поле, в котором номера установленных битов (значения которых равны 1) соответствуют номерам координатных систем.\killoverfullbefore

Возвращаемое значение равно 0 при отсутствии ошибок и отлично от 0 в противном случае.\killoverfullbefore

Является системной. 
% *******end subsection*****************
%--------------------------------------------------------
% *******begin subsection***************
\subsubsection{\DbgSecSt{\StPart}{disable}}
\index{Программный интерфейс ПЛК!Управление движением!Функция disable}
\label{sec:disable}

\begin{pHeader}
    Синтаксис:      & \RightHandText{int disable(int cs);}\\
    Аргумент(ы):    & \RightHandText{int cs ~--  номер координатной системы} \\   
%    Возвращаемое значение:       & \RightHandText{Нет} \\
    Файл объявления:             & \RightHandText{sys/sys.h} \\      
\end{pHeader}

Функция выполняет прерывание программы движения для координатной системы, номер которой определяется аргументом функции, а также снятие управления и полное отключение двигателей в заданной координатной системе с их последующим остановом в режиме свободного выбега (категория останова 0).\killoverfullbefore

Возвращаемое значение равно 0 при отсутствии ошибок и отлично от 0 в противном случае.\killoverfullbefore

Является системной. 
% *******end subsection*****************
%--------------------------------------------------------
% *******begin subsection***************
\subsubsection{\DbgSecSt{\StPart}{disableMulti}}
\index{Программный интерфейс ПЛК!Управление движением!Функция disableMulti}
\label{sec:disableMulti}

\begin{pHeader}
    Синтаксис:      & \RightHandText{int disableMulti(int cs);}\\
    Аргумент(ы):    & \RightHandText{int cs ~-- номера координатных систем} \\   
%    Возвращаемое значение:       & \RightHandText{Нет} \\
    Файл объявления:             & \RightHandText{sys/sys.h} \\      
\end{pHeader}

Функция выполняет прерывание программ движения для заданных координатных систем, а также снятие управления и полное отключение соответствующих двигателей с их последующим остановом в режиме свободного выбега (категория останова 0). \killoverfullbefore

Аргумент функции – битовое поле, в котором номера установленных битов (значения которых равны 1) соответствуют номерам координатных систем.\killoverfullbefore

Возвращаемое значение равно 0 при отсутствии ошибок и отлично от 0 в противном случае.\killoverfullbefore

Является системной. 
% *******end subsection*****************
%--------------------------------------------------------
% *******begin subsection***************
\subsubsection{\DbgSecSt{\StPart}{ddisable}}
\index{Программный интерфейс ПЛК!Управление движением!Функция ddisable}
\label{sec:ddisable}

\begin{pHeader}
    Синтаксис:      & \RightHandText{int ddisable(int cs);}\\
    Аргумент(ы):    & \RightHandText{int cs ~--  номер координатной системы} \\   
%    Возвращаемое значение:       & \RightHandText{Нет} \\
    Файл объявления:             & \RightHandText{sys/sys.h} \\      
\end{pHeader}

Функция выполняет прерывание программы движения для координатной системы, номер которой определяется аргументом функции, а также снятие управления и полное отключение двигателей в заданной координатной системе с задержкой на включение тормоза (категория останова 0).\killoverfullbefore

Возвращаемое значение равно 0 при отсутствии ошибок и отлично от 0 в противном случае.\killoverfullbefore

Является системной. 
% *******end subsection*****************
%--------------------------------------------------------
% *******begin subsection***************
\subsubsection{\DbgSecSt{\StPart}{ddisableMulti}}
\index{Программный интерфейс ПЛК!Управление движением!Функция ddisableMulti}
\label{sec:ddisableMulti}

\begin{pHeader}
    Синтаксис:      & \RightHandText{int ddisableMulti(int cs);}\\
    Аргумент(ы):    & \RightHandText{int cs ~-- номера координатных систем} \\   
%    Возвращаемое значение:       & \RightHandText{Нет} \\
    Файл объявления:             & \RightHandText{sys/sys.h} \\      
\end{pHeader}

Функция выполняет прерывание программ движения для заданных координатных систем, а также снятие управления и полное отключение соответствующих двигателей с задержкой на включение тормоза (категория останова 0). \killoverfullbefore

Аргумент функции – битовое поле, в котором номера установленных битов (значения которых равны 1) соответствуют номерам координатных систем.\killoverfullbefore

Возвращаемое значение равно 0 при отсутствии ошибок и отлично от 0 в противном случае.\killoverfullbefore

Является системной. 
% *******end subsection*****************
%--------------------------------------------------------
% *******begin subsection***************
\subsubsection{\DbgSecSt{\StPart}{enable}}
\index{Программный интерфейс ПЛК!Управление движением!Функция enable}
\label{sec:enable}

\begin{pHeader}
    Синтаксис:      & \RightHandText{int enable(int cs);}\\
    Аргумент(ы):    & \RightHandText{int cs ~--  номер координатной системы} \\   
%    Возвращаемое значение:       & \RightHandText{Нет} \\
    Файл объявления:             & \RightHandText{sys/sys.h} \\      
\end{pHeader}

Функция выполняет включение двигателей в слежение (включение и замыкание контура положения) в координатной системе, номер которой определяется аргументом функции. В слежение будут включены двигатели, которые находятся в отключенном состоянии или работающие в режиме контура тока/момента. \killoverfullbefore

Возвращаемое значение равно 0 при отсутствии ошибок и отлично от 0 в противном случае.\killoverfullbefore

Является системной. 
% *******end subsection*****************
%--------------------------------------------------------
% *******begin subsection***************
\subsubsection{\DbgSecSt{\StPart}{enableMulti}}
\index{Программный интерфейс ПЛК!Управление движением!Функция enableMulti}
\label{sec:enableMulti}

\begin{pHeader}
    Синтаксис:      & \RightHandText{int enableMulti(int cs);}\\
    Аргумент(ы):    & \RightHandText{int cs ~-- номера координатных систем} \\   
%    Возвращаемое значение:       & \RightHandText{Нет} \\
    Файл объявления:             & \RightHandText{sys/sys.h} \\      
\end{pHeader}

Функция выполняет включение двигателей в слежение (включение и замыкание контура положения) в заданных координатных системах. В слежение будут включены двигатели, которые находятся в отключенном состоянии или работающие в режиме контура тока/момента. \killoverfullbefore

Аргумент функции – битовое поле, в котором номера установленных битов (значения которых равны 1) соответствуют номерам координатных систем.\killoverfullbefore

Возвращаемое значение равно 0 при отсутствии ошибок и отлично от 0 в противном случае.\killoverfullbefore

Является системной. 
% *******end subsection*****************
%--------------------------------------------------------
% *******begin subsection***************
\subsubsection{\DbgSecSt{\StPart}{hold}}
\index{Программный интерфейс ПЛК!Управление движением!Функция hold}
\label{sec:hold}

\begin{pHeader}
    Синтаксис:      & \RightHandText{int hold(int cs);}\\
    Аргумент(ы):    & \RightHandText{int cs ~--  номер координатной системы} \\   
%    Возвращаемое значение:       & \RightHandText{Нет} \\
    Файл объявления:             & \RightHandText{sys/sys.h} \\      
\end{pHeader}

Функция приостанавливает выполнение УП в координатной системе, номер которой определяется аргументом функции, уменьшая значение временной развертки координатной системы до 0.\killoverfullbefore

Возвращаемое значение равно 0 при отсутствии ошибок и отлично от 0 в противном случае.\killoverfullbefore

Является системной. 
% *******end subsection*****************
%--------------------------------------------------------
% *******begin subsection***************
\subsubsection{\DbgSecSt{\StPart}{holdMulti}}
\index{Программный интерфейс ПЛК!Управление движением!Функция holdMulti}
\label{sec:holdMulti}

\begin{pHeader}
    Синтаксис:      & \RightHandText{int holdMulti(int cs);}\\
    Аргумент(ы):    & \RightHandText{int cs ~-- номера координатных систем} \\   
%    Возвращаемое значение:       & \RightHandText{Нет} \\
    Файл объявления:             & \RightHandText{sys/sys.h} \\      
\end{pHeader}

Функция приостанавливает выполнение УП в заданных координатных системах, уменьшая значение их временной развертки до 0.\killoverfullbefore

Аргумент функции – битовое поле, в котором номера установленных битов (значения которых равны 1) соответствуют номерам координатных систем.\killoverfullbefore

Возвращаемое значение равно 0 при отсутствии ошибок и отлично от 0 в противном случае.\killoverfullbefore

Является системной. 
% *******end subsection*****************
%--------------------------------------------------------
% *******begin subsection***************
\subsubsection{\DbgSecSt{\StPart}{pause}}
\index{Программный интерфейс ПЛК!Управление движением!Функция pause}
\label{sec:pause}

\begin{pHeader}
    Синтаксис:      & \RightHandText{int pause(int cs);}\\
    Аргумент(ы):    & \RightHandText{int cs ~--  номер координатной системы} \\   
%    Возвращаемое значение:       & \RightHandText{Нет} \\
    Файл объявления:             & \RightHandText{sys/sys.h} \\      
\end{pHeader}

Функция временно останавливает выполнение УП в координатной системе, номер которой определяется аргументом функции, в конце последнего вычисленного перемещения.\killoverfullbefore

Возвращаемое значение равно 0 при отсутствии ошибок и отлично от 0 в противном случае.\killoverfullbefore

Является системной. 
% *******end subsection*****************
%--------------------------------------------------------
% *******begin subsection***************
\subsubsection{\DbgSecSt{\StPart}{pauseMulti}}
\index{Программный интерфейс ПЛК!Управление движением!Функция pauseMulti}
\label{sec:pauseMulti}

\begin{pHeader}
    Синтаксис:      & \RightHandText{int pauseMulti(int cs);}\\
    Аргумент(ы):    & \RightHandText{int cs ~-- номера координатных систем} \\   
%    Возвращаемое значение:       & \RightHandText{Нет} \\
    Файл объявления:             & \RightHandText{sys/sys.h} \\      
\end{pHeader}

Функция временно останавливает выполнение УП в заданных координатных системах в конце последнего рассчитанного перемещения. \killoverfullbefore

Аргумент функции – битовое поле, в котором номера установленных битов (значения которых равны 1) соответствуют номерам координатных систем.\killoverfullbefore

Возвращаемое значение равно 0 при отсутствии ошибок и отлично от 0 в противном случае.\killoverfullbefore

Является системной. 
% *******end subsection*****************
%--------------------------------------------------------
% *******begin subsection***************
\subsubsection{\DbgSecSt{\StPart}{resume}}
\index{Программный интерфейс ПЛК!Управление движением!Функция resume}
\label{sec:resume}

\begin{pHeader}
    Синтаксис:      & \RightHandText{int resume(int cs);}\\
    Аргумент(ы):    & \RightHandText{int cs ~--  номер координатной системы} \\   
%    Возвращаемое значение:       & \RightHandText{Нет} \\
    Файл объявления:             & \RightHandText{sys/sys.h} \\      
\end{pHeader}

Функция возобновляет выполнение временно остановленных УП в координатной системе, номер которой определяется аргументом функции, начиная с точки останова. \killoverfullbefore

Возвращаемое значение равно 0 при отсутствии ошибок и отлично от 0 в противном случае.\killoverfullbefore

Является системной. 
% *******end subsection*****************
%--------------------------------------------------------
% *******begin subsection***************
\subsubsection{\DbgSecSt{\StPart}{resumeMulti}}
\index{Программный интерфейс ПЛК!Управление движением!Функция resumeMulti}
\label{sec:resumeMulti}

\begin{pHeader}
    Синтаксис:      & \RightHandText{int resumeMulti(int cs);}\\
    Аргумент(ы):    & \RightHandText{int cs ~-- номера координатных систем} \\   
%    Возвращаемое значение:       & \RightHandText{Нет} \\
    Файл объявления:             & \RightHandText{sys/sys.h} \\      
\end{pHeader}

Функция возобновляет выполнение временно остановленных УП в заданных координатных системах, начиная с точки останова. \killoverfullbefore

Аргумент функции – битовое поле, в котором номера установленных битов (значения которых равны 1) соответствуют номерам координатных систем.\killoverfullbefore

Возвращаемое значение равно 0 при отсутствии ошибок и отлично от 0 в противном случае.\killoverfullbefore

Является системной. 
% *******end subsection*****************
%--------------------------------------------------------
% *******begin subsection***************
\subsubsection{\DbgSecSt{\StPart}{run}}
\index{Программный интерфейс ПЛК!Управление движением!Функция run}
\label{sec:run}

\begin{pHeader}
    Синтаксис:      & \RightHandText{int resume(int cs);}\\
    Аргумент(ы):    & \RightHandText{int cs ~--  номер координатной системы} \\   
%    Возвращаемое значение:       & \RightHandText{Нет} \\
    Файл объявления:             & \RightHandText{sys/sys.h} \\      
\end{pHeader}

Функция вызывает выполнение УП в координатной системе, номер которой определяется аргументом функции. Если выполнение УП было остановлено с помощью функций \myreftosec{hold}, \myreftosec{pause} или \myreftosec{step}, УП начнёт выполняться с той точки, где она была остановлена. Для перехода в начало УП следует предварительно вызвать функцию \myreftosec{begin}. \killoverfullbefore

Возвращаемое значение равно 0 при отсутствии ошибок и отлично от 0 в противном случае.\killoverfullbefore

Является системной. 
% *******end subsection*****************
%--------------------------------------------------------
% *******begin subsection***************
\subsubsection{\DbgSecSt{\StPart}{runMulti}}
\index{Программный интерфейс ПЛК!Управление движением!Функция runMulti}
\label{sec:runMulti}

\begin{pHeader}
    Синтаксис:      & \RightHandText{int runMulti(int cs);}\\
    Аргумент(ы):    & \RightHandText{int cs ~-- номера координатных систем} \\   
%    Возвращаемое значение:       & \RightHandText{Нет} \\
    Файл объявления:             & \RightHandText{sys/sys.h} \\      
\end{pHeader}

Функция вызывает выполнение УП в заданных координатных системах. Если выполнение УП было остановлено с помощью функций \myreftosec{holdMulti}, \myreftosec{pauseMulti} или \myreftosec{stepMulti}, УП начнёт выполняться с той точки, где она была остановлена. Для перехода в начало УП следует предварительно вызвать функцию \myreftosec{beginMulti}. \killoverfullbefore

Аргумент функции – битовое поле, в котором номера установленных битов (значения которых равны 1) соответствуют номерам координатных систем.\killoverfullbefore

Возвращаемое значение равно 0 при отсутствии ошибок и отлично от 0 в противном случае.\killoverfullbefore

Является системной. 
% *******end subsection*****************
%--------------------------------------------------------
% *******begin subsection***************
\subsubsection{\DbgSecSt{\StPart}{begin}}
\index{Программный интерфейс ПЛК!Управление движением!Функция begin}
\label{sec:begin}

\begin{pHeader}
    Синтаксис:      & \RightHandText{int begin(int cs, double prog);}\\
    Аргумент(ы):    & \RightHandText{int cs ~--  номер координатной системы,} \\   
      & \RightHandText{double prog ~-- номер программы движения} \\
    Файл объявления:             & \RightHandText{sys/sys.h} \\      
\end{pHeader}

Функция устанавливает программу движения для координатной системы, номер которой определяется аргументом функции, и вызывает переход в начало заданной программы. \killoverfullbefore

Возвращаемое значение равно 0 при отсутствии ошибок и отлично от 0 в противном случае.\killoverfullbefore

Является системной. 
% *******end subsection*****************
%--------------------------------------------------------
% *******begin subsection***************
\subsubsection{\DbgSecSt{\StPart}{beginMulti}}
\index{Программный интерфейс ПЛК!Управление движением!Функция beginMulti}
\label{sec:beginMulti}

\begin{pHeader}
    Синтаксис:      & \RightHandText{int beginMulti(int cs, double prog);}\\
    Аргумент(ы):    & \RightHandText{int cs ~-- номера координатных систем,} \\  
      & \RightHandText{double prog ~--  номер программы движения} \\
%    Возвращаемое значение:       & \RightHandText{Нет} \\
    Файл объявления:             & \RightHandText{sys/sys.h} \\      
\end{pHeader}

Функция устанавливает программу движения для координатных систем, номера которых определяются аргументом функции, и вызывает переход в начало заданной программы. \killoverfullbefore

Первый аргумент функции \texttt{cs} ~-- битовое поле, в котором номера установленных битов (значения которых равны 1) соответствуют номерам координатных систем. Второй аргумент \texttt{prog} ~-- номер программы движения. \killoverfullbefore

Возвращаемое значение равно 0 при отсутствии ошибок и отлично от 0 в противном случае.\killoverfullbefore

Является системной. 
% *******end subsection*****************
%--------------------------------------------------------
% *******begin subsection***************
\subsubsection{\DbgSecSt{\StPart}{start}}
\index{Программный интерфейс ПЛК!Управление движением!Функция start}
\label{sec:start}

\begin{pHeader}
    Синтаксис:      & \RightHandText{int start(int cs, double prog);}\\
    Аргумент(ы):    & \RightHandText{int cs ~--  номер координатной системы,} \\   
      & \RightHandText{double prog ~-- номер программы движения } \\
    Файл объявления:             & \RightHandText{sys/sys.h} \\      
\end{pHeader}

Функция устанавливает программу движения для координатной системы, номер которой определяется аргументом функции, вызывает переход в начало заданной программы и последующее её выполнение. \killoverfullbefore

Возвращаемое значение равно 0 при отсутствии ошибок и отлично от 0 в противном случае.\killoverfullbefore

Является системной. 
% *******end subsection*****************
%--------------------------------------------------------
% *******begin subsection***************
\subsubsection{\DbgSecSt{\StPart}{startMulti}}
\index{Программный интерфейс ПЛК!Управление движением!Функция startMulti}
\label{sec:startMulti}

\begin{pHeader}
    Синтаксис:      & \RightHandText{int startMulti(int cs, double prog);}\\
    Аргумент(ы):    & \RightHandText{int cs ~-- номера координатных систем,} \\  
      & \RightHandText{double prog ~-- номер программы движения} \\
%    Возвращаемое значение:       & \RightHandText{Нет} \\
    Файл объявления:             & \RightHandText{sys/sys.h} \\      
\end{pHeader}

Функция устанавливает программу движения для координатных систем, номера которых определяются аргументом функции, вызывает переход в начало заданной программы и последующее её выполнение. \killoverfullbefore

Первый аргумент функции \texttt{cs} ~-- битовое поле, в котором номера установленных битов (значения которых равны 1) соответствуют номерам координатных систем. Второй аргумент \texttt{prog} ~-- номер программы движения. \killoverfullbefore

Возвращаемое значение равно 0 при отсутствии ошибок и отлично от 0 в противном случае.\killoverfullbefore

Является системной. 
% *******end subsection*****************
%--------------------------------------------------------
% *******begin subsection***************
\subsubsection{\DbgSecSt{\StPart}{step}}
\index{Программный интерфейс ПЛК!Управление движением!Функция step}
\label{sec:step}

\begin{pHeader}
    Синтаксис:      & \RightHandText{int step(int cs);}\\
    Аргумент(ы):    & \RightHandText{int cs ~--  номер координатной системы} \\   
    Файл объявления:             & \RightHandText{sys/sys.h} \\      
\end{pHeader}

Функция вызывает пошаговое выполнение УП в координатной системе, номер которой определяется аргументом функции. \killoverfullbefore

Возвращаемое значение равно 0 при отсутствии ошибок и отлично от 0 в противном случае.\killoverfullbefore

Является системной. 
% *******end subsection*****************
%--------------------------------------------------------
% *******begin subsection***************
\subsubsection{\DbgSecSt{\StPart}{stepMulti}}
\index{Программный интерфейс ПЛК!Управление движением!Функция stepMulti}
\label{sec:stepMulti}

\begin{pHeader}
    Синтаксис:      & \RightHandText{int stepMulti(int cs);}\\
    Аргумент(ы):    & \RightHandText{int cs ~-- номера координатных систем} \\  
%    Возвращаемое значение:       & \RightHandText{Нет} \\
    Файл объявления:             & \RightHandText{sys/sys.h} \\
\end{pHeader}

Функция вызывает пошаговое выполнение УП в заданных координатных системах. \killoverfullbefore

Аргумент функции – битовое поле, в котором номера установленных битов (значения которых равны 1) соответствуют номерам координатных систем.\killoverfullbefore

Возвращаемое значение равно 0 при отсутствии ошибок и отлично от 0 в противном случае.\killoverfullbefore

Является системной. 
% *******end subsection*****************
%--------------------------------------------------------
% *******begin subsection***************
\subsubsection{\DbgSecSt{\StPart}{stop}}
\index{Программный интерфейс ПЛК!Управление движением!Функция stop}
\label{sec:stop}

\begin{pHeader}
    Синтаксис:      & \RightHandText{int stop(int cs);}\\
    Аргумент(ы):    & \RightHandText{int cs ~--  номер координатной системы} \\   
    Файл объявления:             & \RightHandText{sys/sys.h} \\      
\end{pHeader}

Функция вызывает останов выполнения УП в координатной системе, номер которой определяется аргументом функции, позволяя завершить уже рассчитанные перемещения. После останова программы выполняется переход в её начало.  \killoverfullbefore

Возвращаемое значение равно 0 при отсутствии ошибок и отлично от 0 в противном случае.\killoverfullbefore

Является системной. 
% *******end subsection*****************
%--------------------------------------------------------
% *******begin subsection***************
\subsubsection{\DbgSecSt{\StPart}{stopMulti}}
\index{Программный интерфейс ПЛК!Управление движением!Функция stopMulti}
\label{sec:stopMulti}

\begin{pHeader}
    Синтаксис:      & \RightHandText{int stopMulti(int cs);}\\
    Аргумент(ы):    & \RightHandText{int cs ~-- номера координатных систем} \\  
%    Возвращаемое значение:       & \RightHandText{Нет} \\
    Файл объявления:             & \RightHandText{sys/sys.h} \\      
\end{pHeader}

Функция вызывает останов выполнения УП в заданных координатных системах, позволяя завершить уже рассчитанные перемещения. После останова программы выполняется переход в её начало.  \killoverfullbefore

Аргумент функции – битовое поле, в котором номера установленных битов (значения которых равны 1) соответствуют номерам координатных систем.\killoverfullbefore

Возвращаемое значение равно 0 при отсутствии ошибок и отлично от 0 в противном случае.\killoverfullbefore

Является системной. 
% *******end subsection*****************
%--------------------------------------------------------
% *******begin subsection***************
\subsubsection{\DbgSecSt{\StPart}{suspend}}
\index{Программный интерфейс ПЛК!Управление движением!Функция suspend}
\label{sec:suspend}

\begin{pHeader}
    Синтаксис:      & \RightHandText{int suspend(int cs);}\\
    Аргумент(ы):    & \RightHandText{int cs ~--  номер координатной системы} \\   
    Файл объявления:             & \RightHandText{sys/sys.h} \\      
\end{pHeader}

Функция является аналогичной \myreftosec{pause}.\killoverfullbefore

Является системной. 
% *******end subsection*****************
%--------------------------------------------------------
% *******begin subsection***************
\subsubsection{\DbgSecSt{\StPart}{suspendMulti}}
\index{Программный интерфейс ПЛК!Управление движением!Функция suspendMulti}
\label{sec:suspendMulti}

\begin{pHeader}
    Синтаксис:      & \RightHandText{int suspendMulti(int cs);}\\
    Аргумент(ы):    & \RightHandText{int cs ~--  номера координатных систем} \\   
    Файл объявления:             & \RightHandText{sys/sys.h} \\      
\end{pHeader}

Функция является аналогичной \myreftosec{pauseMulti}.\killoverfullbefore

Является системной. 
% *******end subsection*****************
%--------------------------------------------------------
% *******begin subsection***************
\subsubsection{\DbgSecSt{\StPart}{bstart}}
\index{Программный интерфейс ПЛК!Управление движением!Функция bstart}
\label{sec:bstart}

\begin{pHeader}
    Синтаксис:      & \RightHandText{int bstart();}\\
    Аргумент(ы):    & \RightHandText{нет} \\   
    Файл объявления:             & \RightHandText{sys/sys.h} \\      
\end{pHeader}

Функция указывает начало части программы, которая должна быть выполнена за один «шаг». Выполнение будет продолжаться до вызова функции \myreftosec{bstop}. \killoverfullbefore

Возвращаемое значение равно 0 при отсутствии ошибок и отлично от 0 в противном случае.\killoverfullbefore

Является системной. 
% *******end subsection*****************
%--------------------------------------------------------
% *******begin subsection***************
\subsubsection{\DbgSecSt{\StPart}{bstop}}
\index{Программный интерфейс ПЛК!Управление движением!Функция bstop}
\label{sec:bstop}

\begin{pHeader}
    Синтаксис:      & \RightHandText{int bstop();}\\
    Аргумент(ы):    & \RightHandText{нет} \\   
    Файл объявления:             & \RightHandText{sys/sys.h} \\      
\end{pHeader}

Функция указывает окончание части программы, которая должна быть выполнена за один «шаг». \killoverfullbefore

Возвращаемое значение равно 0 при отсутствии ошибок и отлично от 0 в противном случае.\killoverfullbefore

Является системной. 
% *******end subsection*****************
%--------------------------------------------------------
% *******begin subsection***************
\subsubsection{\DbgSecSt{\StPart}{dread}}
\index{Программный интерфейс ПЛК!Управление движением!Функция dread}
\label{sec:dread}

\begin{pHeader}
    Синтаксис:      & \RightHandText{int dread(int cs, double *p);}\\
    Аргумент(ы):    & \RightHandText{int cs ~--  номер координатной системы} \\   
     & \RightHandText {double *p ~-- указатель на массив} \\  
    Файл объявления:             & \RightHandText{sys/sys.h} \\      
\end{pHeader}

Функция выполняет расчёт и запись в массив заданной позиции для активных (имеющих определение) осей в координатной системе, номер которой определяется аргументом функции. Заданная позиция оси рассчитывается на основе заданной позиции двигателя, выражения определения оси и действующей матрицы преобразований.\killoverfullbefore

Первый аргумент функции \texttt{cs} – номер координатной системы. Второй
аргумент \mbox{\texttt{*p} ~-–} указатель на массив типа \texttt{double}, который должен содержать не менее 32 значений.\killoverfullbefore

Возвращаемое значение равно 0 при отсутствии ошибок и отлично от 0 в противном случае.\killoverfullbefore

Является системной. 
% *******end subsection*****************
%--------------------------------------------------------
% *******begin subsection***************
\subsubsection{\DbgSecSt{\StPart}{pread}}
\index{Программный интерфейс ПЛК!Управление движением!Функция pread}
\label{sec:pread}

\begin{pHeader}
    Синтаксис:      & \RightHandText{int pread(int cs, double *p);}\\
    Аргумент(ы):    & \RightHandText{int cs ~--  номер координатной системы} \\   
     & \RightHandText {double *p ~-- указатель на массив} \\  
    Файл объявления:             & \RightHandText{sys/sys.h} \\      
\end{pHeader}

Функция выполняет расчёт и запись в массив текущей позиции для активных (имеющих определение) осей в координатной системе, номер которой определяется аргументом функции. Текущая позиция оси рассчитывается на основе текущей позиции двигателя, выражения определения оси и действующей матрицы преобразований.\killoverfullbefore

Первый аргумент функции \texttt{cs} – номер координатной системы. Второй
аргумент \mbox{\texttt{*p} ~-–} указатель на массив типа \texttt{double}, который должен содержать не менее 32 значений.\killoverfullbefore

Возвращаемое значение равно 0 при отсутствии ошибок и отлично от 0 в противном случае.\killoverfullbefore

Является системной. 
% *******end subsection*****************
%--------------------------------------------------------
% *******begin subsection***************
\subsubsection{\DbgSecSt{\StPart}{tread}}
\index{Программный интерфейс ПЛК!Управление движением!Функция tread}
\label{sec:tread}

\begin{pHeader}
    Синтаксис:      & \RightHandText{int tread(int cs, double *p);}\\
    Аргумент(ы):    & \RightHandText{int cs ~--  номер координатной системы} \\   
     & \RightHandText {double *p ~-- указатель на массив} \\  
    Файл объявления:             & \RightHandText{sys/sys.h} \\      
\end{pHeader}

Функция выполняет расчёт и запись в массив конечной позиции выполняемого перемещения (выполняемого кадра) для активных (имеющих определение) осей в координатной системе, номер которой определяется аргументом функции.  \killoverfullbefore

Первый аргумент функции \texttt{cs} – номер координатной системы. Второй
аргумент \mbox{\texttt{*p} ~-–} указатель на массив типа \texttt{double}, который должен содержать не менее 32 значений.\killoverfullbefore

Возвращаемое значение равно 0 при отсутствии ошибок и отлично от 0 в противном случае.\killoverfullbefore

Является системной. 
% *******end subsection*****************
%--------------------------------------------------------
% *******begin subsection***************
\subsubsection{\DbgSecSt{\StPart}{dtogread}}
\index{Программный интерфейс ПЛК!Управление движением!Функция dtogread}
\label{sec:dtogread}

\begin{pHeader}
    Синтаксис:      & \RightHandText{int dtogread(int cs, double *p);}\\
    Аргумент(ы):    & \RightHandText{int cs ~--  номер координатной системы} \\   
     & \RightHandText {double *p ~-- указатель на массив} \\  
    Файл объявления:             & \RightHandText{sys/sys.h} \\      
\end{pHeader}

Функция выполняет расчёт и запись в массив остатка пути выполняемого перемещения (выполняемого кадра) для активных (имеющих определение) осей в координатной системе, номер которой определяется аргументом функции. Остаток пути рассчитывается как разность конечной позиции выполняемого перемещения и текущего значения заданной позиции оси. Значение заданной позиции оси рассчитывается на основе заданной позиции двигателя, выражения определения оси и действующей матрицы преобразований. При активной коррекции инструмента значение конечной позиции выполняемого перемещения смещается на величину коррекции. \killoverfullbefore

Первый аргумент функции \texttt{cs} – номер координатной системы. Второй
аргумент \mbox{\texttt{*p} ~-–} указатель на массив типа \texttt{double}, который должен содержать не менее 32 значений.\killoverfullbefore

Возвращаемое значение равно 0 при отсутствии ошибок и отлично от 0 в противном случае.\killoverfullbefore

Является системной. 
% *******end subsection*****************
%--------------------------------------------------------
% *******begin subsection***************
\subsubsection{\DbgSecSt{\StPart}{fread}}
\index{Программный интерфейс ПЛК!Управление движением!Функция fread}
\label{sec:fread}

\begin{pHeader}
    Синтаксис:      & \RightHandText{int fread(int cs, double *f);}\\
    Аргумент(ы):    & \RightHandText{int cs ~--  номер координатной системы} \\   
     & \RightHandText {double *f ~-- указатель на массив} \\  
    Файл объявления:             & \RightHandText{sys/sys.h} \\      
\end{pHeader}

Функция выполняет расчёт и запись в массив ошибки слежения для активных (имеющих определение) осей в координатной системе, номер которой определяется аргументом функции. Ошибка слежения оси рассчитывается на основе ошибки слежения двигателя, выражения определения оси и действующей матрицы преобразований.\killoverfullbefore

Первый аргумент функции \texttt{cs} – номер координатной системы. Второй
аргумент \mbox{\texttt{*f} ~-–} указатель на массив типа \texttt{double}, который должен содержать не менее 32 значений.

Возвращаемое значение равно 0 при отсутствии ошибок и отлично от 0 в противном случае.\killoverfullbefore

Является системной. 
% *******end subsection*****************
%--------------------------------------------------------
% *******begin subsection***************
\subsubsection{\DbgSecSt{\StPart}{vread}}
\index{Программный интерфейс ПЛК!Управление движением!Функция vread}
\label{sec:vread}

\begin{pHeader}
    Синтаксис:      & \RightHandText{int vread(int cs, double *v);}\\
    Аргумент(ы):    & \RightHandText{int cs ~--  номер координатной системы} \\   
     & \RightHandText {double *f ~-- указатель на массив} \\  
    Файл объявления:             & \RightHandText{sys/sys.h} \\      
\end{pHeader}

Функция выполняет расчёт и запись в массив текущей скорости (усреднённой за 16 сервоциклов) для активных (имеющих определение) осей в координатной системе, номер которой определяется аргументом функции. Скорость измеряется в единицах измерения перемещения по оси за мс.\killoverfullbefore

Первый аргумент функции \texttt{cs} – номер координатной системы. Второй
аргумент \mbox{\texttt{*v} ~-–} указатель на массив типа \texttt{double}, который должен содержать не менее 32 значений.\killoverfullbefore

Возвращаемое значение равно 0 при отсутствии ошибок и отлично от 0 в противном случае.\killoverfullbefore

Является системной. 
% *******end subsection*****************
%--------------------------------------------------------
% *******begin subsection***************
\subsubsection{\DbgSecSt{\StPart}{pset}}
\index{Программный интерфейс ПЛК!Управление движением!Функция pset}
\label{sec:pset}

\begin{pHeader}
    Синтаксис:      & \RightHandText{int pset(const Pos \&pos);}\\
    Аргумент(ы):    & \RightHandText {const \myreftosec{Pos} \&pos ~-- данные позиций} \\  
%    Возвращаемое значение:       & \RightHandText{Нет} \\
    Файл объявления:             & \RightHandText{sys/sys.h} \\
\end{pHeader}

Функция переопределяет текущую позицию осей и связанных с ними двигателей. Значения текущих заданных позиций становятся равными указанным значениям, поэтому никакого движения не происходит. Функция изменяет позицию нулевой точки двигателей (вводит смещение) и, таким образом, программные пределы перемещения и таблицу компенсаций.  \killoverfullbefore

Аргумент функции ~-- структура \myreftosec{Pos}, в которой номера ячеек массива со значениями, отличными от \texttt{NAN}, соответствуют номерам осей, а сами значения ячеек являются текущими позициями. \killoverfullbefore

Возвращаемое значение равно 0 при отсутствии ошибок и отлично от 0 в противном случае.\killoverfullbefore

Является системной. 
% *******end subsection*****************
%--------------------------------------------------------
% *******begin subsection***************
\subsubsection{\DbgSecSt{\StPart}{pstore}}
\index{Программный интерфейс ПЛК!Управление движением!Функция pstore}
\label{sec:pstore}

\begin{pHeader}
    Синтаксис:      & \RightHandText{int pstore();}\\
    Аргумент(ы):    & \RightHandText {нет} \\  
%    Возвращаемое значение:       & \RightHandText{Нет} \\
    Файл объявления:             & \RightHandText{sys/sys.h} \\
\end{pHeader}

Функция сохраняет смещения, которые вызваны \myreftosec{pset}.  \killoverfullbefore

Возвращаемое значение равно 0 при отсутствии ошибок и отлично от 0 в противном случае.\killoverfullbefore

Является системной. 
% *******end subsection*****************
%--------------------------------------------------------
% *******begin subsection***************
\subsubsection{\DbgSecSt{\StPart}{pload}}
\index{Программный интерфейс ПЛК!Управление движением!Функция pload}
\label{sec:pload}

\begin{pHeader}
    Синтаксис:      & \RightHandText{int pload();}\\
    Аргумент(ы):    & \RightHandText {нет} \\  
%    Возвращаемое значение:       & \RightHandText{Нет} \\
    Файл объявления:             & \RightHandText{sys/sys.h} \\
\end{pHeader}

Функция загружает смещения, которые сохранены \myreftosec{pstore}. \killoverfullbefore

Действие функции аналогично \myreftosec{pset}.\killoverfullbefore

Возвращаемое значение равно 0 при отсутствии ошибок и отлично от 0 в противном случае.\killoverfullbefore

Является системной. 
% *******end subsection*****************
%--------------------------------------------------------
% *******begin subsection***************
\subsubsection{\DbgSecSt{\StPart}{pclear}}
\index{Программный интерфейс ПЛК!Управление движением!Функция pclear}
\label{sec:pclear}

\begin{pHeader}
    Синтаксис:      & \RightHandText{int pclear();}\\
    Аргумент(ы):    & \RightHandText {нет} \\  
%    Возвращаемое значение:       & \RightHandText{Нет} \\
    Файл объявления:             & \RightHandText{sys/sys.h} \\
\end{pHeader}

Функция устанавливает текущую позицию осей и связанных с ними двигателей, равной 0. Значения текущих заданных позиций становятся равными указанным значениям, поэтому никакого движения не происходит. Функция изменяет позицию нулевой точки двигателей (вводит смещение) и, таким образом, программные пределы перемещения и таблицу компенсаций.  \killoverfullbefore

Действие функции аналогично \myreftosec{pset} с аргументами 0 или \myreftosec{homez} для двигателей.\killoverfullbefore

Возвращаемое значение равно 0 при отсутствии ошибок и отлично от 0 в противном случае.\killoverfullbefore

Является системной. 
% *******end subsection*****************
%--------------------------------------------------------
% *******begin subsection***************
\subsubsection{\DbgSecSt{\StPart}{pmatch}}
\index{Программный интерфейс ПЛК!Управление движением!Функция pmatch}
\label{sec:pmatch}

\begin{pHeader}
    Синтаксис:      & \RightHandText{int pmatch();}\\
    Аргумент(ы):    & \RightHandText {нет} \\  
%    Возвращаемое значение:       & \RightHandText{Нет} \\
    Файл объявления:             & \RightHandText{sys/sys.h} \\
\end{pHeader}

Функция вызывает расчёт начальных позиций осей в координатной системе, чтобы они соответствовали текущим заданным позициям двигателей. Расчёт производится посредством выражений, обратных выражениям определений осей, или прямых кинематических преобразований. \killoverfullbefore

Возвращаемое значение равно 0 при отсутствии ошибок и отлично от 0 в противном случае.\killoverfullbefore

Является системной. 
% *******end subsection*****************
%--------------------------------------------------------
% *******begin subsection***************
\subsubsection{\DbgSecSt{\StPart}{move}}
\index{Программный интерфейс ПЛК!Управление движением!Функция move}
\label{sec:move}

\begin{pHeader}
    Синтаксис:      & \RightHandText{int move(const Pos \&pos);}\\
    Аргумент(ы):    & \RightHandText {const \myreftosec{Pos} \&pos ~-- данные перемещения} \\  
%    Возвращаемое значение:       & \RightHandText{Нет} \\
    Файл объявления:             & \RightHandText{sys/sys.h} \\      
\end{pHeader}

Функция выполняет перемещение в указанное аргументом функции положение в установленном режиме движения.\killoverfullbefore

Возвращаемое значение равно 0 при отсутствии ошибок и отлично от 0 в противном случае.\killoverfullbefore

Является системной. 
% *******end subsection*****************
%--------------------------------------------------------
% *******begin subsection***************
\subsubsection{\DbgSecSt{\StPart}{rapidmove}}
\index{Программный интерфейс ПЛК!Управление движением!Функция rapidmove}
\label{sec:rapidmove}

\begin{pHeader}
    Синтаксис:      & \RightHandText{int rapidmove(const Pos \&pos);}\\
    Аргумент(ы):    & \RightHandText {const \myreftosec{Pos} \&pos ~-- данные перемещения} \\  
%    Возвращаемое значение:       & \RightHandText{Нет} \\
    Файл объявления:             & \RightHandText{sys/sys.h} \\      
\end{pHeader}

Функция выполняет быстрое перемещение в указанное аргументом функции положение.\killoverfullbefore

 Возвращаемое значение равно 0 при отсутствии ошибок и отлично от 0 в противном случае.\killoverfullbefore

Является системной. 
% *******end subsection*****************
%--------------------------------------------------------
% *******begin subsection***************
\subsubsection{\DbgSecSt{\StPart}{rapid}}
\index{Программный интерфейс ПЛК!Управление движением!Функция rapid}
\label{sec:rapid}

\begin{pHeader}
    Синтаксис:      & \RightHandText{int rapid();}\\
    Аргумент(ы):    & \RightHandText {Нет} \\  
%    Возвращаемое значение:       & \RightHandText{Нет} \\
    Файл объявления:             & \RightHandText{sys/sys.h} \\      
\end{pHeader}

Функция возвращает 1, если активен режим быстрых перемещений, и 0 в противном случае.\killoverfullbefore

Является системной. 
% *******end subsection*****************
%--------------------------------------------------------
% *******begin subsection***************
\subsubsection{\DbgSecSt{\StPart}{linearmove}}
\index{Программный интерфейс ПЛК!Управление движением!Функция linearmove}
\label{sec:linearmove}

\begin{pHeader}
    Синтаксис:      & \RightHandText{int linearmove(const Pos \&pos);}\\
    Аргумент(ы):    & \RightHandText {const \myreftosec{Pos} \&pos ~-- данные перемещения} \\  
%    Возвращаемое значение:       & \RightHandText{Нет} \\
    Файл объявления:             & \RightHandText{sys/sys.h} \\      
\end{pHeader}

Функция выполняет линейное перемещение в указанное аргументом функции положение.\killoverfullbefore

 Возвращаемое значение равно 0 при отсутствии ошибок и отлично от 0 в противном случае.\killoverfullbefore

Является системной. 
% *******end subsection*****************
%--------------------------------------------------------
% *******begin subsection***************
\subsubsection{\DbgSecSt{\StPart}{linear}}
\index{Программный интерфейс ПЛК!Управление движением!Функция linear}
\label{sec:linear}

\begin{pHeader}
    Синтаксис:      & \RightHandText{int linear();}\\
    Аргумент(ы):    & \RightHandText {Нет} \\  
%    Возвращаемое значение:       & \RightHandText{Нет} \\
    Файл объявления:             & \RightHandText{sys/sys.h} \\      
\end{pHeader}

Функция возвращает 1, если активен режим линейной интерполяции, и 0 в противном случае.\killoverfullbefore

Является системной. 
% *******end subsection*****************
%--------------------------------------------------------
% *******begin subsection***************
\subsubsection{\DbgSecSt{\StPart}{cir1move}}
\index{Программный интерфейс ПЛК!Управление движением!Функция cir1move}
\label{sec:cir1move}

\begin{pHeader}
    Синтаксис:      & \RightHandText{int cir1move(const Pos \&pos);}\\
    Аргумент(ы):    & \RightHandText {const \myreftosec{Pos} \&pos ~-- данные перемещения} \\  
%    Возвращаемое значение:       & \RightHandText{Нет} \\
    Файл объявления:             & \RightHandText{sys/sys.h} \\      
\end{pHeader}

Функция выполняет круговое перемещение по часовой стрелке в указанное аргументом функции положение. \killoverfullbefore

Возвращаемое значение равно 0 при отсутствии ошибок и отлично от 0 в противном случае.\killoverfullbefore

Является системной. 
%--------------------------------------------------------
% *******begin subsection***************
\subsubsection{\DbgSecSt{\StPart}{cir2move}}
\index{Программный интерфейс ПЛК!Управление движением!Функция cir2move}
\label{sec:cir2move}

\begin{pHeader}
    Синтаксис:      & \RightHandText{int cir2move(const Pos \&pos);}\\
    Аргумент(ы):    & \RightHandText {const \myreftosec{Pos} \&pos ~-- данные перемещения} \\  
%    Возвращаемое значение:       & \RightHandText{Нет} \\
    Файл объявления:             & \RightHandText{sys/sys.h} \\      
\end{pHeader}

Функция выполняет круговое перемещение против часовой стрелки в указанное аргументом функции положение. \killoverfullbefore

Возвращаемое значение равно 0 при отсутствии ошибок и отлично от 0 в противном случае. \killoverfullbefore

Является системной.
% *******end subsection*****************
%--------------------------------------------------------
% *******begin subsection***************
\subsubsection{\DbgSecSt{\StPart}{circle1}}
\index{Программный интерфейс ПЛК!Управление движением!Функция circle1}
\label{sec:circle1}

\begin{pHeader}
    Синтаксис:      & \RightHandText{int circle1();}\\
    Аргумент(ы):    & \RightHandText {нет} \\  
%    Возвращаемое значение:       & \RightHandText{Нет} \\
    Файл объявления:             & \RightHandText{sys/sys.h} \\      
\end{pHeader}

Функция устанавливает режим круговой интерполяции по часовой стрелке для основной декартовой системы координат (X/Y/Z). \killoverfullbefore

Возвращаемое значение равно 0 при отсутствии ошибок и отлично от 0 в противном случае. \killoverfullbefore

Является системной.
% *******end subsection*****************
%--------------------------------------------------------
% *******begin subsection***************
\subsubsection{\DbgSecSt{\StPart}{circle2}}
\index{Программный интерфейс ПЛК!Управление движением!Функция circle2}
\label{sec:circle2}

\begin{pHeader}
    Синтаксис:      & \RightHandText{int circle2();}\\
    Аргумент(ы):    & \RightHandText {нет} \\  
%    Возвращаемое значение:       & \RightHandText{Нет} \\
    Файл объявления:             & \RightHandText{sys/sys.h} \\      
\end{pHeader}

Функция устанавливает режим круговой интерполяции против часовой стрелки для основной декартовой системы координат (X/Y/Z). \killoverfullbefore

Возвращаемое значение равно 0 при отсутствии ошибок и отлично от 0 в противном случае. \killoverfullbefore

Является системной.
% *******end subsection*****************
%--------------------------------------------------------
% *******begin subsection***************
\subsubsection{\DbgSecSt{\StPart}{circle3}}
\index{Программный интерфейс ПЛК!Управление движением!Функция circle3}
\label{sec:circle3}

\begin{pHeader}
    Синтаксис:      & \RightHandText{int circle3();}\\
    Аргумент(ы):    & \RightHandText {нет} \\  
%    Возвращаемое значение:       & \RightHandText{Нет} \\
    Файл объявления:             & \RightHandText{sys/sys.h} \\      
\end{pHeader}

Функция устанавливает режим круговой интерполяции по часовой стрелке для расширенной декартовой системы координат (XX/XY/XZ). \killoverfullbefore

Возвращаемое значение равно 0 при отсутствии ошибок и отлично от 0 в противном случае. \killoverfullbefore

Является системной.
% *******end subsection*****************
%--------------------------------------------------------
% *******begin subsection***************
\subsubsection{\DbgSecSt{\StPart}{circle4}}
\index{Программный интерфейс ПЛК!Управление движением!Функция circle4}
\label{sec:circle4}

\begin{pHeader}
    Синтаксис:      & \RightHandText{int circle4();}\\
    Аргумент(ы):    & \RightHandText {нет} \\  
%    Возвращаемое значение:       & \RightHandText{Нет} \\
    Файл объявления:             & \RightHandText{sys/sys.h} \\      
\end{pHeader}

Функция устанавливает режим круговой интерполяции против часовой стрелки для расширенной декартовой системы координат (XX/XY/XZ). \killoverfullbefore

Возвращаемое значение равно 0 при отсутствии ошибок и отлично от 0 в противном случае. \killoverfullbefore

Является системной.
% *******end subsection*****************
%--------------------------------------------------------
% *******begin subsection***************
\subsubsection{\DbgSecSt{\StPart}{pvt}}
\index{Программный интерфейс ПЛК!Управление движением!Функция pvt}
\label{sec:pvt}

\begin{pHeader}
    Синтаксис:      & \RightHandText{int pvt(double time);}\\
    Аргумент(ы):    & \RightHandText {double time ~-- время перемещения} \\  
    Файл объявления:             & \RightHandText{sys/sys.h} \\      
\end{pHeader}

Функция устанавливает режим движения с заданными положением, скоростью и временем (pvt-движение). Если данный режим уже установлен, то изменяется время перемещения. Если установлен другой режим движения (линейная или круговая интерполяция, быстрые перемещения, сплайновая интерполяция), то будет выполнен выход из него. Время перемещения измеряется в мс. \killoverfullbefore

Возвращаемое значение равно 0 при отсутствии ошибок и отлично от 0 в противном случае. \killoverfullbefore

Является системной.
% *******end subsection*****************
%--------------------------------------------------------
% *******begin subsection***************
\subsubsection{\DbgSecSt{\StPart}{spline}}
\index{Программный интерфейс ПЛК!Управление движением!Функция spline}
\label{sec:spline}

\begin{pHeader}
    Синтаксис:      & \RightHandText{int spline(double time);}\\
    Аргумент(ы):    & \RightHandText {double time ~-- время сегмента перемещения} \\  
    Файл объявления:             & \RightHandText{sys/sys.h} \\      
\end{pHeader}

Функция устанавливает режим сплайновой интерполяции. Если данный режим уже установлен, то изменяется время сегмента перемещения. Если установлен другой режим движения (линейная или круговая интерполяция, быстрые перемещения, pvt-движение), то будет выполнен выход из него. Время сегмента перемещения измеряется в мс. \killoverfullbefore

Возвращаемое значение равно 0 при отсутствии ошибок и отлично от 0 в противном случае. \killoverfullbefore

Является системной.
% *******end subsection*****************
%--------------------------------------------------------
% *******begin subsection***************
\subsubsection{\DbgSecSt{\StPart}{ccmode1}}
\index{Программный интерфейс ПЛК!Управление движением!Функция ccmode1}
\label{sec:ccmode1}

\begin{pHeader}
    Синтаксис:      & \RightHandText{int ccmode1();}\\
    Аргумент(ы):    & \RightHandText {нет} \\  
    Файл объявления:             & \RightHandText{sys/sys.h} \\
\end{pHeader}

Функция отменяет двухмерную и трёхмерную коррекцию радиуса инструмента, уменьшая её постепенно на последующем линейном перемещении. Является эквивалентом G40. \killoverfullbefore

Возвращаемое значение равно 0 при отсутствии ошибок и отлично от 0 в противном случае. \killoverfullbefore

Является системной.
% *******end subsection*****************
%--------------------------------------------------------
% *******begin subsection***************
\subsubsection{\DbgSecSt{\StPart}{ccmode2}}
\index{Программный интерфейс ПЛК!Управление движением!Функция ccmode2}
\label{sec:ccmode2}

\begin{pHeader}
    Синтаксис:      & \RightHandText{int ccmode2();}\\
    Аргумент(ы):    & \RightHandText {нет} \\  
    Файл объявления:             & \RightHandText{sys/sys.h} \\
\end{pHeader}

Функция включает двухмерную коррекцию радиуса инструмента влево, вводя её постепенно на последующем линейном перемещении. Является эквивалентом G41. \killoverfullbefore

Возвращаемое значение равно 0 при отсутствии ошибок и отлично от 0 в противном случае. \killoverfullbefore

Является системной.
% *******end subsection*****************
%--------------------------------------------------------
% *******begin subsection***************
\subsubsection{\DbgSecSt{\StPart}{ccmode3}}
\index{Программный интерфейс ПЛК!Управление движением!Функция ccmode3}
\label{sec:ccmode3}

\begin{pHeader}
    Синтаксис:      & \RightHandText{int ccmode3();}\\
    Аргумент(ы):    & \RightHandText {нет} \\  
    Файл объявления:             & \RightHandText{sys/sys.h} \\
\end{pHeader}

Функция включает двухмерную коррекцию радиуса инструмента вправо, вводя её постепенно на последующем линейном перемещении. Является эквивалентом G42. \killoverfullbefore

Возвращаемое значение равно 0 при отсутствии ошибок и отлично от 0 в противном случае. \killoverfullbefore

Является системной.
% *******end subsection*****************
%--------------------------------------------------------
% *******begin subsection***************
\subsubsection{\DbgSecSt{\StPart}{ccmode4}}
\index{Программный интерфейс ПЛК!Управление движением!Функция ccmode4}
\label{sec:ccmode4}

\begin{pHeader}
    Синтаксис:      & \RightHandText{int ccmode4();}\\
    Аргумент(ы):    & \RightHandText {нет} \\  
    Файл объявления:             & \RightHandText{sys/sys.h} \\
\end{pHeader}

Функция включает трёхмерную коррекцию радиуса инструмента, вводя её постепенно на последующем линейном перемещении. \killoverfullbefore

Возвращаемое значение равно 0 при отсутствии ошибок и отлично от 0 в противном случае. \killoverfullbefore

Является системной.
% *******end subsection*****************

%The tool-orientation vector can subsequently be specified by the txyz program command, and the surface-normal vector can subsequently be specified by the nxyz program command.
%--------------------------------------------------------
% *******begin subsection***************
\subsubsection{\DbgSecSt{\StPart}{ccr}}
\index{Программный интерфейс ПЛК!Управление движением!Функция ccr}
\label{sec:ccr}

\begin{pHeader}
    Синтаксис:      & \RightHandText{int ccr(double r);}\\
    Аргумент(ы):    & \RightHandText {double r ~-- радиус инструмента} \\  
    Файл объявления:             & \RightHandText{sys/sys.h} \\      
\end{pHeader}

Функция задаёт величину радиуса инструмента для двухмерной коррекции. Траектория центра инструмента будет смещена на данное расстояние перпендикулярно  запрограммированной траектории в заданной плоскости коррекции. \killoverfullbefore

Возвращаемое значение равно 0 при отсутствии ошибок и отлично от 0 в противном случае. \killoverfullbefore

Является системной.
% *******end subsection*****************
%--------------------------------------------------------
% *******begin subsection***************
\subsubsection{\DbgSecSt{\StPart}{txyz}}
\index{Программный интерфейс ПЛК!Управление движением!Функция txyz}
\label{sec:txyz}

\begin{pHeader}
    Синтаксис:      & \RightHandText{int txyz(Vec v);}\\
    Аргумент(ы):    & \RightHandText {\myreftosec{Vec} v ~-- вектор ориентации инструмента} \\  
    Файл объявления:             & \RightHandText{sys/sys.h} \\      
\end{pHeader}

Функция задаёт вектор ориентации инструмента для трёхмерной коррекции. Компоненты вектора I, J, K параллельны осям X, Y, Z соответственно. Геометрическая сумма компонент определяет направление вектора (от основания к концу или от конца к основанию), длина вектора не имеет значения. \killoverfullbefore

Возвращаемое значение равно 0 при отсутствии ошибок и отлично от 0 в противном случае. \killoverfullbefore

Является системной.
% *******end subsection*****************
%--------------------------------------------------------
% *******begin subsection***************
\subsubsection{\DbgSecSt{\StPart}{txyzScale}}
\index{Программный интерфейс ПЛК!Управление движением!Функция txyzScale}
\label{sec:txyzScale}

\begin{pHeader}
    Синтаксис:      & \RightHandText{int txyzScale(double s);}\\
    Аргумент(ы):    & \RightHandText {double s ~-- масштабный коэффициент} \\  
    Файл объявления:             & \RightHandText{sys/sys.h} \\      
\end{pHeader}

Функция задаёт масштабный коэффициент для вектора подачи и радиуса инструмента при двухмерной коррекции, отличный от коэффициентов масштабирования матрицы преобразования.  Предназначен для сохранения подачи и радиуса в непреобразованных единицах измерения перемещения по осям. \killoverfullbefore

Возвращаемое значение равно 0 при отсутствии ошибок и отлично от 0 в противном случае. \killoverfullbefore

Является системной.
% *******end subsection*****************
%--------------------------------------------------------
% *******begin subsection***************
\subsubsection{\DbgSecSt{\StPart}{nxyz}}
\index{Программный интерфейс ПЛК!Управление движением!Функция nxyz}
\label{sec:nxyz}

\begin{pHeader}
    Синтаксис:      & \RightHandText{int nxyz(const Vec \&v);}\\
    Аргумент(ы):    & \RightHandText {const \myreftosec{Vec} v ~-- вектор нормали к поверхности} \\  
    Файл объявления:             & \RightHandText{sys/sys.h} \\      
\end{pHeader}

Функция задаёт вектор нормали к поверхности для трёхмерной коррекции. Компоненты вектора I, J, K параллельны осям X, Y, Z соответственно. Геометрическая сумма компонент определяет направление вектора от поверхности детали к инструменту, длина вектора не имеет значения. Вектор должен быть определен в базовых машинных координатах. \killoverfullbefore

Возвращаемое значение равно 0 при отсутствии ошибок и отлично от 0 в противном случае. \killoverfullbefore

Является системной.
% *******end subsection*****************
%--------------------------------------------------------
% *******begin subsection***************
\subsubsection{\DbgSecSt{\StPart}{normal}}
\index{Программный интерфейс ПЛК!Управление движением!Функция normal}
\label{sec:normal}

\begin{pHeader}
    Синтаксис:      & \RightHandText{int normal(const Vec \&v);}\\
    Аргумент(ы):    & \RightHandText {const \myreftosec{Vec} v ~-- вектор нормали к рабочей плоскости} \\  
    Файл объявления:             & \RightHandText{sys/sys.h} \\
\end{pHeader}

Функция задаёт вектор нормали к рабочей плоскости (перпендикулярный рабочей плоскости) для круговой интерполяции, двухмерной коррекции. Компоненты вектора I, J, K параллельны осям X, Y, Z соответственно. Геометрическая сумма компонент определяет направление вектора и, следовательно, положение рабочей плоскости. От ориентации вектора зависит направление перемещения по дуге окружности и коррекции инструмента (используется правосторонняя система координат). Длина вектора не имеет значения.  \killoverfullbefore

Возвращаемое значение равно 0 при отсутствии ошибок и отлично от 0 в противном случае. \killoverfullbefore

Является системной.
% *******end subsection*****************
%--------------------------------------------------------
% *******begin subsection***************
\subsubsection{\DbgSecSt{\StPart}{tsel}}
\index{Программный интерфейс ПЛК!Управление движением!Функция tsel}
\label{sec:tsel}

\begin{pHeader}
    Синтаксис:      & \RightHandText{int tsel(int id);}\\
    Аргумент(ы):    & \RightHandText {int id ~-- номер матрицы преобразования} \\  
    Файл объявления:             & \RightHandText{sys/sys.h} \\
\end{pHeader}

Функция задаёт номер активной матрицы преобразования для координатной системы УП. Диапазон действительных номеров ~-- 0$\div$255. Значение номера, равное -1, отменяет выбор всех матриц преобразования.\killoverfullbefore

Возвращаемое значение равно 0 при отсутствии ошибок и отлично от 0 в противном случае. \killoverfullbefore

Является системной.
% *******end subsection*****************
%--------------------------------------------------------
% *******begin subsection***************
\subsubsection{\DbgSecSt{\StPart}{enablePLC}}
\index{Программный интерфейс ПЛК!Управление движением!Функция enablePLC}
\label{sec:enablePLC}

\begin{pHeader}
    Синтаксис:      & \RightHandText{int enablePLC(int plc);}\\
    Аргумент(ы):    & \RightHandText {int plc ~-- номер программы ПЛК} \\  
%    Возвращаемое значение:       & \RightHandText{Нет} \\
    Файл объявления:             & \RightHandText{sys/sys.h} \\      
\end{pHeader}

Функция вызывает выполнение программы ПЛК, номер которой (от 0 до 31) определяется аргументом функции. Выполнение стартует с начала программы. \killoverfullbefore

Возвращаемое значение равно 0 при отсутствии ошибок и отлично от 0 в противном случае. \killoverfullbefore

Является системной.
% *******end subsection*****************
%--------------------------------------------------------
% *******begin subsection***************
\subsubsection{\DbgSecSt{\StPart}{enablePLCs}}
\index{Программный интерфейс ПЛК!Управление движением!Функция enablePLCs}
\label{sec:enablePLCs}

\begin{pHeader}
    Синтаксис:      & \RightHandText{int enablePLCs(int plc);}\\
    Аргумент(ы):    & \RightHandText {int plc ~-- номера программ ПЛК} \\  
%    Возвращаемое значение:       & \RightHandText{Нет} \\
    Файл объявления:             & \RightHandText{sys/sys.h} \\      
\end{pHeader}

Функция вызывает выполнение программ ПЛК, номера которых (от 0 до 31) определяются аргументом функции. Аргумент функции – битовое поле, в котором номера установленных битов (значения которых равны 1) соответствуют номерам программ ПЛК. Выполнение стартует с начала программы. \killoverfullbefore

Возвращаемое значение равно 0 при отсутствии ошибок и отлично от 0 в противном случае. \killoverfullbefore

Является системной.
% *******end subsection*****************
%--------------------------------------------------------
% *******begin subsection***************
\subsubsection{\DbgSecSt{\StPart}{pausePLC}}
\index{Программный интерфейс ПЛК!Управление движением!Функция pausePLC}
\label{sec:pausePLC}

\begin{pHeader}
    Синтаксис:      & \RightHandText{int pausePLC(int plc);}\\
    Аргумент(ы):    & \RightHandText {int plc ~-- номер программы ПЛК} \\  
%    Возвращаемое значение:       & \RightHandText{Нет} \\
    Файл объявления:             & \RightHandText{sys/sys.h} \\      
\end{pHeader}

Функция вызывает временный останов программы ПЛК, номер которой (от 0 до 31) определяется аргументом функции. \killoverfullbefore

Возвращаемое значение равно 0 при отсутствии ошибок и отлично от 0 в противном случае. \killoverfullbefore

Является системной.
% *******end subsection*****************
%--------------------------------------------------------
% *******begin subsection***************
\subsubsection{\DbgSecSt{\StPart}{pausePLCs}}
\index{Программный интерфейс ПЛК!Управление движением!Функция pausePLCs}
\label{sec:pausePLCs}

\begin{pHeader}
    Синтаксис:      & \RightHandText{int pausePLCs(int plc);}\\
    Аргумент(ы):    & \RightHandText {int plc ~-- номера программ ПЛК} \\  
%    Возвращаемое значение:       & \RightHandText{Нет} \\
    Файл объявления:             & \RightHandText{sys/sys.h} \\      
\end{pHeader}

Функция вызывает временный останов программ ПЛК, номера которых (от 0 до 31) определяются аргументом функции. Аргумент функции – битовое поле, в котором номера установленных битов (значения которых равны 1) соответствуют номерам программ ПЛК.\killoverfullbefore

 Возвращаемое значение равно 0 при отсутствии ошибок и отлично от 0 в противном случае. \killoverfullbefore

Является системной.
% *******end subsection*****************
%--------------------------------------------------------
% *******begin subsection***************
\subsubsection{\DbgSecSt{\StPart}{resumePLC}}
\index{Программный интерфейс ПЛК!Управление движением!Функция resumePLC}
\label{sec:resumePLC}

\begin{pHeader}
    Синтаксис:      & \RightHandText{int resumePLC(int plc);}\\
    Аргумент(ы):    & \RightHandText {int plc ~-- номер программы ПЛК} \\  
%    Возвращаемое значение:       & \RightHandText{Нет} \\
    Файл объявления:             & \RightHandText{sys/sys.h} \\      
\end{pHeader}

Функция вызывает возобновление выполнения программы ПЛК, номер которой (от 0 до 31) определяется аргументом функции. \killoverfullbefore

Возвращаемое значение равно 0 при отсутствии ошибок и отлично от 0 в противном случае. \killoverfullbefore

Является системной.
% *******end subsection*****************
%--------------------------------------------------------
% *******begin subsection***************
\subsubsection{\DbgSecSt{\StPart}{resumePLCs}}
\index{Программный интерфейс ПЛК!Управление движением!Функция resumePLCs}
\label{sec:resumePLCs}

\begin{pHeader}
    Синтаксис:      & \RightHandText{int resumePLCs(int plc);}\\
    Аргумент(ы):    & \RightHandText {int plc ~-- номера программ ПЛК} \\  
%    Возвращаемое значение:       & \RightHandText{Нет} \\
    Файл объявления:             & \RightHandText{sys/sys.h} \\      
\end{pHeader}

Функция вызывает возобновление выполнения программ ПЛК, номера которых (от 0 до 31) определяются аргументом функции. Аргумент функции – битовое поле, в котором номера установленных битов (значения которых равны 1) соответствуют номерам программ ПЛК.\killoverfullbefore

 Возвращаемое значение равно 0 при отсутствии ошибок и отлично от 0 в противном случае. \killoverfullbefore

Является системной.
% *******end subsection*****************
%--------------------------------------------------------
% *******begin subsection***************
\subsubsection{\DbgSecSt{\StPart}{disablePLC}}
\index{Программный интерфейс ПЛК!Управление движением!Функция disablePLC}
\label{sec:disablePLC}

\begin{pHeader}
    Синтаксис:      & \RightHandText{int disablePLC(int plc);}\\
    Аргумент(ы):    & \RightHandText {int plc ~-- номер программы ПЛК} \\  
%    Возвращаемое значение:       & \RightHandText{Нет} \\
    Файл объявления:             & \RightHandText{sys/sys.h} \\      
\end{pHeader}

Функция вызывает отмену выполнения программы ПЛК, номер которой (от 0 до 31) определяется аргументом функции. Возвращаемое значение равно 0 при отсутствии ошибок и отлично от 0 в противном случае. \killoverfullbefore

Является системной.
% *******end subsection*****************
%--------------------------------------------------------
% *******begin subsection***************
\subsubsection{\DbgSecSt{\StPart}{disablePLCs}}
\index{Программный интерфейс ПЛК!Управление движением!Функция disablePLCs}
\label{sec:disablePLCs}

\begin{pHeader}
    Синтаксис:      & \RightHandText{int disablePLCs(int plc);}\\
    Аргумент(ы):    & \RightHandText {int plc ~-- номера программ ПЛК} \\  
%    Возвращаемое значение:       & \RightHandText{Нет} \\
    Файл объявления:             & \RightHandText{sys/sys.h} \\      
\end{pHeader}

Функция вызывает отмену выполнения программ ПЛК, номера которых (от 0 до 31) определяются аргументом функции. Аргумент функции – битовое поле, в котором номера установленных битов (значения которых равны 1) соответствуют номерам программ ПЛК.\killoverfullbefore

 Возвращаемое значение равно 0 при отсутствии ошибок и отлично от 0 в противном случае. \killoverfullbefore

Является системной.
% *******end subsection*****************
%--------------------------------------------------------
% *******begin subsection***************
\subsubsection{\DbgSecSt{\StPart}{stepPLC}}
\index{Программный интерфейс ПЛК!Управление движением!Функция stepPLC}
\label{sec:stepPLC}

\begin{pHeader}
    Синтаксис:      & \RightHandText{int stepPLC(int plc);}\\
    Аргумент(ы):    & \RightHandText {int plc ~-- номер программы ПЛК} \\  
%    Возвращаемое значение:       & \RightHandText{Нет} \\
    Файл объявления:             & \RightHandText{sys/sys.h} \\      
\end{pHeader}

Функция вызывает пошаговое выполнение программы ПЛК, номер которой (от 0 до 31) определяется аргументом функции. \killoverfullbefore

Возвращаемое значение равно 0 при отсутствии ошибок и отлично от 0 в противном случае. \killoverfullbefore

Является системной.
% *******end subsection*****************
%--------------------------------------------------------
% *******begin subsection***************
\subsubsection{\DbgSecSt{\StPart}{stepPLCs}}
\index{Программный интерфейс ПЛК!Управление движением!Функция stepPLCs}
\label{sec:stepPLCs}

\begin{pHeader}
    Синтаксис:      & \RightHandText{int stepPLCs(int plc);}\\
    Аргумент(ы):    & \RightHandText {int plc ~-- номера программ ПЛК} \\  
%    Возвращаемое значение:       & \RightHandText{Нет} \\
    Файл объявления:             & \RightHandText{sys/sys.h} \\      
\end{pHeader}

Функция вызывает пошаговое выполнение программ ПЛК, номера которых (от 0 до 31) определяются аргументом функции. Аргумент функции – битовое поле, в котором номера установленных битов (значения которых равны 1) соответствуют номерам программ ПЛК.\killoverfullbefore

 Возвращаемое значение равно 0 при отсутствии ошибок и отлично от 0 в противном случае. \killoverfullbefore

Является системной.
% *******end subsection*****************
%--------------------------------------------------------
% *******begin subsection***************
\subsubsection{\DbgSecSt{\StPart}{timerStart}}
\index{Программный интерфейс ПЛК!Управление движением!Макрос timerStart}
\label{sec:timerStart}

\begin{pHeader}
    Синтаксис:      & \RightHandText{timerStart(timer, timeoutVal);}\\
   Аргумент(ы):    & \RightHandText {timer ~-- переменная типа \myreftosec{Timer}} \\  
    & \RightHandText{timeoutVal ~-- интервал срабатывания} \\
    Файл объявления:             & \RightHandText{sys/sys.h} \\      
\end{pHeader}

Макрос запускает таймер, инициализируя переменную \texttt{timer}: полю timer.start присваивается текущее значение системного счётчика, полю timer.timeout ~-- значение интервала срабатывания.\killoverfullbefore

Интервал срабатывания таймера задаётся в периодах сервоцикла (1 период сервоцикла равен 400 мс). Так, например, 1 c соответствует значению интервала равному 2500. \killoverfullbefore

Является системной.
% *******end subsection*****************
%--------------------------------------------------------
% *******begin subsection***************
\subsubsection{\DbgSecSt{\StPart}{timerTimeout}}
\index{Программный интерфейс ПЛК!Управление движением!Макрос timerTimeout}
\label{sec:timerTimeout}

\begin{pHeader}
    Синтаксис:      & \RightHandText{timerTimeout(timer);}\\
   Аргумент(ы):    & \RightHandText {timer ~-- переменная типа \myreftosec{Timer}} \\  
%    Возвращаемое значение:       & \RightHandText{Нет} \\
    Файл объявления:             & \RightHandText{sys/sys.h} \\      
\end{pHeader}

Макрос возвращает 0, если не истёк заданный интервал срабатывания, и значение, отличное от 0, в противном случае.\killoverfullbefore

Является системной.
%--------------------------------------------------------
% *******begin subsection***************
\subsubsection{\DbgSecSt{\StPart}{timerLeft}}
\index{Программный интерфейс ПЛК!Управление движением!Макрос timerLeft}
\label{sec:timerLeft}

\begin{pHeader}
    Синтаксис:      & \RightHandText{timerLeft(timer);}\\
   Аргумент(ы):    & \RightHandText {timer ~-- переменная типа \myreftosec{Timer}} \\  
%    Возвращаемое значение:       & \RightHandText{Нет} \\
    Файл объявления:             & \RightHandText{sys/sys.h} \\      
\end{pHeader}

Макрос возвращает число периодов сервоцикла, оставшихся до срабатывания таймера.\killoverfullbefore

Является системной.
%--------------------------------------------------------
% *******begin subsection***************
\subsubsection{\DbgSecSt{\StPart}{timerPassed}}
\index{Программный интерфейс ПЛК!Управление движением!Макрос timerPassed}
\label{sec:timerPassed}

\begin{pHeader}
    Синтаксис:      & \RightHandText{timerPassed(timer);}\\
   Аргумент(ы):    & \RightHandText {timer ~-- переменная типа \myreftosec{Timer}} \\  
%    Возвращаемое значение:       & \RightHandText{Нет} \\
    Файл объявления:             & \RightHandText{sys/sys.h} \\      
\end{pHeader}

Макрос возвращает число периодов сервоцикла, прошедших с момента запуска таймера.\killoverfullbefore

Является системной.
% *******end subsection*****************
%--------------------------------------------------------
% *******begin subsection***************
\subsubsection{\DbgSecSt{\StPart}{clearGather}}
\index{Программный интерфейс ПЛК!Управление движением!Функция clearGather}
\label{sec:clearGather}

\begin{pHeader}
    Синтаксис:      & \RightHandText{int clearGather();}\\
   Аргумент(ы):    & \RightHandText {нет} \\  
%    Возвращаемое значение:       & \RightHandText{Нет} \\
    Файл объявления:             & \RightHandText{sys/sys.h} \\
\end{pHeader}

Функция очищает буфер данных сервопрерываний. \killoverfullbefore

Является системной.
% *******end subsection*****************
%--------------------------------------------------------
% *******begin subsection***************
\subsubsection{\DbgSecSt{\StPart}{clearPhaseGather}}
\index{Программный интерфейс ПЛК!Управление движением!Функция clearPhaseGather}
\label{sec:clearPhaseGather}

\begin{pHeader}
    Синтаксис:      & \RightHandText{int clearPhaseGather();}\\
   Аргумент(ы):    & \RightHandText {нет} \\  
%    Возвращаемое значение:       & \RightHandText{Нет} \\
    Файл объявления:             & \RightHandText{sys/sys.h} \\
\end{pHeader}

Функция очищает буфер данных фазных прерываний. \killoverfullbefore

Является системной.
% *******end subsection*****************
%--------------------------------------------------------
% *******begin subsection***************
\subsubsection{\DbgSecSt{\StPart}{shutdown}}
\index{Программный интерфейс ПЛК!Управление движением!Функция shutdown}
\label{sec:shutdown}

\begin{pHeader}
    Синтаксис:      & \RightHandText{void shutdown();}\\
   Аргумент(ы):    & \RightHandText {нет} \\  
%    Возвращаемое значение:       & \RightHandText{Нет} \\
    Файл объявления:             & \RightHandText{sys/sys.h} \\      
\end{pHeader}

Функция вызывает выключение УЧПУ. \killoverfullbefore

Является системной.
% *******end subsection*****************
%--------------------------------------------------------
% *******begin subsection***************
\subsubsection{\DbgSecSt{\StPart}{reset}}
\index{Программный интерфейс ПЛК!Управление движением!Функция reset}
\label{sec:reset}

\begin{pHeader}
    Синтаксис:      & \RightHandText{void reset();}\\
   Аргумент(ы):    & \RightHandText {нет} \\  
%    Возвращаемое значение:       & \RightHandText{Нет} \\
    Файл объявления:             & \RightHandText{sys/sys.h} \\      
\end{pHeader}

Функция вызывает перезагрузку УЧПУ, которая эквивалентна выключению и последующему включению питания. \killoverfullbefore

Является системной.
% *******end subsection*****************
%--------------------------------------------------------
% *******begin subsection***************
\subsubsection{\DbgSecSt{\StPart}{reinitialize}}
\index{Программный интерфейс ПЛК!Управление движением!Функция reinitialize}
\label{sec:reinitialize}

\begin{pHeader}
    Синтаксис:      & \RightHandText{void reinitialize();}\\
   Аргумент(ы):    & \RightHandText {нет} \\  
%    Возвращаемое значение:       & \RightHandText{Нет} \\
    Файл объявления:             & \RightHandText{sys/sys.h} \\      
\end{pHeader}

Функция вызывает сброс параметров УЧПУ до заводских. \killoverfullbefore

Является системной.
% *******end subsection*****************
%--------------------------------------------------------        % API: управление движением - базовые функции
    %--------------------------------------------------------
% *******begin section***************
\section{\DbgSecSt{\StPart}{Переменные и буфера}}

% *******begin subsection***************
\subsection{\DbgSecSt{\StPart}{Функции}}
%--------------------------------------------------------
% *******begin subsection***************
\subsubsection{\DbgSecSt{\StPart}{detectEdgeRise}}
\index{Программный интерфейс ПЛК!Переменные и буфера!Функция detectEdgeRise}
\label{sec:detectEdgeRise}

\begin{pHeader}
    Синтаксис:      & \RightHandText{int detectEdgeRise(int \&detector, int input);}\\
    Аргумент(ы):    & \RightHandText{int \&detector ~-- сохранённое входное значение} \\    
    & \RightHandText{int input ~--  текущее входное значение} \\ 
    Файл объявления:             & \RightHandText{include/func/misc.h} \\       
\end{pHeader}

Функция служит для детектирования изменения с 0 на 1 (детектирования фронта) входной величины. \killoverfullbefore

Функция возвращает 0, если входное значение не изменилось и осталось равным 0, и 1, если входное значение стало отличным от 0. \killoverfullbefore

Является системной.
% *******end subsection*****************
%-------------------------------------------------------------------
% *******begin subsection***************
\subsubsection{\DbgSecSt{\StPart}{detectEdgeFall}}
\index{Программный интерфейс ПЛК!Переменные и буфера!Функция detectEdgeFall}
\label{sec:detectEdgeFall}

\begin{pHeader}
    Синтаксис:      & \RightHandText{int detectEdgeFall(int \&detector, int input);}\\
    Аргумент(ы):    & \RightHandText{int \&detector ~-- сохранённое входное значение} \\    
    & \RightHandText{int input ~--  текущее входное значение} \\ 
    Файл объявления:             & \RightHandText{include/func/misc.h} \\       
\end{pHeader}

Функция служит для детектирования изменения с 1 на 0 (детектирования спада) входной величины. \killoverfullbefore

Функция возвращает 0, если входное значение не изменилось и осталось равным 1, и 1, если входное значение стало равным 0. \killoverfullbefore

Является системной.
% *******end subsection*****************
%--------------------------------------------------------
% *******begin subsection***************
\subsubsection{\DbgSecSt{\StPart}{syncset}}
\index{Программный интерфейс ПЛК!Переменные и буфера!Функция syncset}
\label{sec:syncset}

\begin{pHeader}
    Синтаксис:      & \RightHandText{void syncset(void *ptr, int value);}\\
   Аргумент(ы):    & \RightHandText {void *ptr ~-- указатель на переменную} \\  
  & \RightHandText{int value ~-- присваиваемое значение} \\
    Файл объявления:             & \RightHandText{sys/sys.h} \\      
\end{pHeader}

Функция выполняет синхронное присваивание значения типа \texttt{int}, адресуемой указателем переменной. Синхронное присваивание осуществляется в момент начала следующего перемещения.

Первый аргумент функции \texttt{*ptr} ~-- указатель, ссылающийся на переменную. Второй аргумент \texttt{value} ~-– присваиваемое значение.\killoverfullbefore

Является системной.
% *******end subsection*****************
%--------------------------------------------------------
% *******begin subsection***************
\subsubsection{\DbgSecSt{\StPart}{syncsetf}}
\index{Программный интерфейс ПЛК!Переменные и буфера!Функция syncsetf}
\label{sec:syncsetf}

\begin{pHeader}
    Синтаксис:      & \RightHandText{void syncsetf(void *ptr, float value);}\\
   Аргумент(ы):    & \RightHandText {void *ptr ~-- указатель на переменную} \\  
  & \RightHandText{float value ~-- присваиваемое значение} \\
    Файл объявления:             & \RightHandText{sys/sys.h} \\      
\end{pHeader}

Синхронное присваивание значения типа \texttt{float}, адресуемой указателем переменной. Синхронное присваивание осуществляется в момент начала следующего перемещения.

Первый аргумент функции \texttt{*ptr} ~-- указатель, ссылающийся на переменную. Второй аргумент \texttt{value} ~-– присваиваемое значение.\killoverfullbefore

Является системной.
% *******end subsection*****************
%--------------------------------------------------------
% *******begin subsection***************
\subsubsection{\DbgSecSt{\StPart}{syncsetd}}
\index{Программный интерфейс ПЛК!Переменные и буфера!Функция syncsetd}
\label{sec:syncsetd}

\begin{pHeader}
    Синтаксис:      & \RightHandText{void syncsetd(void *ptr, double value);}\\
   Аргумент(ы):    & \RightHandText {void *ptr ~-- указатель на переменную} \\  
  & \RightHandText{double value ~-- присваиваемое значение} \\
    Файл объявления:             & \RightHandText{sys/sys.h} \\      
\end{pHeader}

Синхронное присваивание значения типа \texttt{double}, адресуемой указателем переменной. Синхронное присваивание осуществляется в момент начала следующего перемещения.

Первый аргумент функции \texttt{*ptr} ~-- указатель, ссылающийся на переменную. Второй аргумент \texttt{value} ~-– присваиваемое значение.\killoverfullbefore

Является системной.
% *******end subsection*****************
%--------------------------------------------------------
% *******begin subsection***************
\subsubsection{\DbgSecSt{\StPart}{usave}}
\index{Программный интерфейс ПЛК!Переменные и буфера!Функция usave}
\label{sec:usave}

\begin{pHeader}
    Синтаксис:      & \RightHandText{void usave(void *ptr);}\\
   Аргумент(ы):    & \RightHandText {void *ptr ~-- указатель на переменную} \\  
%    Возвращаемое значение:       & \RightHandText{Нет} \\
    Файл объявления:             & \RightHandText{sys/sys.h} \\      
\end{pHeader}

Функция выполняет сохранение пользовательской переменной, на которую ссылается указатель. Пользовательская переменная должна быть объявлена посредством макроса \texttt{\#define USER\_SAVE(name)}. \killoverfullbefore

Является системной.
% *******end subsection*****************
%--------------------------------------------------------
% *******begin subsection***************
\subsubsection{\DbgSecSt{\StPart}{clearGather}}
\index{Программный интерфейс ПЛК!Переменные и буфера!Функция clearGather}
\label{sec:clearGather}

\begin{pHeader}
    Синтаксис:      & \RightHandText{int clearGather();}\\
   Аргумент(ы):    & \RightHandText {нет} \\  
%    Возвращаемое значение:       & \RightHandText{Нет} \\
    Файл объявления:             & \RightHandText{sys/sys.h} \\
\end{pHeader}

Функция очищает буфер данных сервопрерываний. \killoverfullbefore

Является системной.
% *******end subsection*****************
%--------------------------------------------------------
% *******begin subsection***************
\subsubsection{\DbgSecSt{\StPart}{clearPhaseGather}}
\index{Программный интерфейс ПЛК!Переменные и буфера!Функция clearPhaseGather}
\label{sec:clearPhaseGather}

\begin{pHeader}
    Синтаксис:      & \RightHandText{int clearPhaseGather();}\\
   Аргумент(ы):    & \RightHandText {нет} \\  
%    Возвращаемое значение:       & \RightHandText{Нет} \\
    Файл объявления:             & \RightHandText{sys/sys.h} \\
\end{pHeader}

Функция очищает буфер данных фазных прерываний. \killoverfullbefore

Является системной.
% *******end subsection*****************
%--------------------------------------------------------            % API: переменные и буфера
    %--------------------------------------------------------
% *******begin section***************
\section{\DbgSecSt{\StPart}{Таймеры}}
%--------------------------------------------------------
\subsection{\DbgSecSt{\StPart}{Типы данных}}
%--------------------------------------------------------

% *******begin subsection***************
\subsubsection{\DbgSecSt{\StPart}{Timer}}
\index{Программный интерфейс ПЛК!Таймеры!Структура Timer}
\label{sec:Timer}

\begin{fHeader}
    Тип данных:            & \RightHandText{Структура Timer}\\
    Файл объявления:             & \RightHandText{sys/sys.h} \\
\end{fHeader}

Структура определяет параметры таймера.

\begin{MyTableThreeColAllCntr}{Структура Timer}{tbl:Timer}{|m{0.41\linewidth}|m{0.24\linewidth}|m{0.35\linewidth}|}{Элемент}{Тип}{Описание}
\hline start & \centering{int} & Начальное значение счётчика таймера \\
\hline timeout & \centering{int} & Интервал \\
\end{MyTableThreeColAllCntr}
% *******end subsection***************

% *******begin subsection***************
\subsection{\DbgSecSt{\StPart}{Функции и макросы}}

%--------------------------------------------------------
% *******begin subsection***************
\subsubsection{\DbgSecSt{\StPart}{timerStart}}
\index{Программный интерфейс ПЛК!Таймеры!Макрос timerStart}
\label{sec:timerStart}

\begin{pHeader}
    Синтаксис:      & \RightHandText{timerStart(timer, timeoutVal);}\\
   Аргумент(ы):    & \RightHandText {timer ~-- переменная типа \myreftosec{Timer}} \\  
    & \RightHandText{timeoutVal ~-- интервал срабатывания} \\
    Файл объявления:             & \RightHandText{sys/sys.h} \\      
\end{pHeader}

Макрос запускает таймер, инициализируя переменную \texttt{timer}: полю timer.start присваивается текущее значение системного счётчика, полю timer.timeout ~-- значение интервала срабатывания.\killoverfullbefore

Интервал срабатывания таймера задаётся в периодах сервоцикла (1 период сервоцикла равен 400 мс). Так, например, 1 c соответствует значению интервала равному 2500. \killoverfullbefore

Является системной.
% *******end subsection*****************
%--------------------------------------------------------
% *******begin subsection***************
\subsubsection{\DbgSecSt{\StPart}{timerTimeout}}
\index{Программный интерфейс ПЛК!Таймеры!Макрос timerTimeout}
\label{sec:timerTimeout}

\begin{pHeader}
    Синтаксис:      & \RightHandText{timerTimeout(timer);}\\
   Аргумент(ы):    & \RightHandText {timer ~-- переменная типа \myreftosec{Timer}} \\  
%    Возвращаемое значение:       & \RightHandText{Нет} \\
    Файл объявления:             & \RightHandText{sys/sys.h} \\      
\end{pHeader}

Макрос возвращает 0, если не истёк заданный интервал срабатывания, и значение, отличное от 0, в противном случае.\killoverfullbefore

Является системной.
%--------------------------------------------------------
% *******begin subsection***************
\subsubsection{\DbgSecSt{\StPart}{timerLeft}}
\index{Программный интерфейс ПЛК!Таймеры!Макрос timerLeft}
\label{sec:timerLeft}

\begin{pHeader}
    Синтаксис:      & \RightHandText{timerLeft(timer);}\\
   Аргумент(ы):    & \RightHandText {timer ~-- переменная типа \myreftosec{Timer}} \\  
%    Возвращаемое значение:       & \RightHandText{Нет} \\
    Файл объявления:             & \RightHandText{sys/sys.h} \\      
\end{pHeader}

Макрос возвращает число периодов сервоцикла, оставшихся до срабатывания таймера.\killoverfullbefore

Является системной.
%--------------------------------------------------------
% *******begin subsection***************
\subsubsection{\DbgSecSt{\StPart}{timerPassed}}
\index{Программный интерфейс ПЛК!Таймеры!Макрос timerPassed}
\label{sec:timerPassed}

\begin{pHeader}
    Синтаксис:      & \RightHandText{timerPassed(timer);}\\
   Аргумент(ы):    & \RightHandText {timer ~-- переменная типа \myreftosec{Timer}} \\  
%    Возвращаемое значение:       & \RightHandText{Нет} \\
    Файл объявления:             & \RightHandText{sys/sys.h} \\      
\end{pHeader}

Макрос возвращает число периодов сервоцикла, прошедших с момента запуска таймера.\killoverfullbefore

Является системной.
% *******end subsection*****************

%-------------------------------------------------------------------
% *******begin subsection***************
\subsubsection{\DbgSecSt{\StPart}{initPulsedTimer}}
\index{Программный интерфейс ПЛК!Таймеры!Функция initPulsedTimer}
\label{sec:initPulsedTimer}

\begin{pHeader}
    Синтаксис:      & \RightHandText{void initPulsedTimer();}\\
    Аргумент(ы):    & \RightHandText{Нет} \\  
%    Возвращаемое значение:       & \RightHandText{Целое знаковое число} \\ 
    Файл объявления:             & \RightHandText{include/func/misc.h} \\       
\end{pHeader}

Функция инициализации периодического (импульсного) таймера. \killoverfullbefore
%который срабатывает (возвращает 1) через заданный интервал. 

Является системной.
% *******end subsection*****************
%-------------------------------------------------------------------
% *******begin subsection***************
\subsubsection{\DbgSecSt{\StPart}{timerSc}}
\index{Программный интерфейс ПЛК!Таймеры!Функция timerSc}
\label{sec:timerSc}

\begin{pHeader}
    Синтаксис:      & \RightHandText{int timerSc(int period);}\\
    Аргумент(ы):    & \RightHandText{int period ~-- период таймера} \\  
%    Возвращаемое значение:       & \RightHandText{Целое знаковое число} \\ 
    Файл объявления:             & \RightHandText{include/func/misc.h} \\
\end{pHeader}

Функция периодического (импульсного) таймера ~-- таймера, выходное значение которого периодически переключается с 0 на 1 и обратно через интервал, равный половине периода таймера. Период таймера задаётся в периодах сервоцикла (1 период сервоцикла равен 400 мс). Так, например, интервал 1 c соответствует значению периода таймера равному 2500. \killoverfullbefore
%сброса таймера (установки 0)

Функция возвращает 1, если с момента переключения таймера с 1 на 0 истёк интервал, больший или равный половине периода, и 0 в противном случае. \killoverfullbefore 

Является системной.
% *******end subsection*****************
%-------------------------------------------------------------------
              % API: таймеры
    %--------------------------------------------------------
% *******begin section***************
\section{\DbgSecSt{\StPart}{Управление программами ПЛК}}
%--------------------------------------------------------

% *******begin subsection***************
\subsection{\DbgSecSt{\StPart}{Функции}}

%--------------------------------------------------------
% *******begin subsection***************
\subsubsection{\DbgSecSt{\StPart}{enablePLC}}
\index{Программный интерфейс ПЛК!Управление программами ПЛК!Функция enablePLC}
\label{sec:enablePLC}

\begin{pHeader}
    Синтаксис:      & \RightHandText{int enablePLC(int plc);}\\
    Аргумент(ы):    & \RightHandText {int plc ~-- номер программы ПЛК} \\  
%    Возвращаемое значение:       & \RightHandText{Нет} \\
    Файл объявления:             & \RightHandText{sys/sys.h} \\      
\end{pHeader}

Функция вызывает выполнение программы ПЛК, номер которой (от 0 до 31) определяется аргументом функции. Выполнение стартует с начала программы. \killoverfullbefore

Возвращаемое значение равно 0 при отсутствии ошибок и отлично от 0 в противном случае. \killoverfullbefore

Является системной.
% *******end subsection*****************
%--------------------------------------------------------
% *******begin subsection***************
\subsubsection{\DbgSecSt{\StPart}{enablePLCs}}
\index{Программный интерфейс ПЛК!Управление программами ПЛК!Функция enablePLCs}
\label{sec:enablePLCs}

\begin{pHeader}
    Синтаксис:      & \RightHandText{int enablePLCs(int plc);}\\
    Аргумент(ы):    & \RightHandText {int plc ~-- номера программ ПЛК} \\  
%    Возвращаемое значение:       & \RightHandText{Нет} \\
    Файл объявления:             & \RightHandText{sys/sys.h} \\      
\end{pHeader}

Функция вызывает выполнение программ ПЛК, номера которых (от 0 до 31) определяются аргументом функции. Аргумент функции – битовое поле, в котором номера установленных битов (значения которых равны 1) соответствуют номерам программ ПЛК. Выполнение стартует с начала программы. \killoverfullbefore

Возвращаемое значение равно 0 при отсутствии ошибок и отлично от 0 в противном случае. \killoverfullbefore

Является системной.
% *******end subsection*****************
%--------------------------------------------------------
% *******begin subsection***************
\subsubsection{\DbgSecSt{\StPart}{pausePLC}}
\index{Программный интерфейс ПЛК!Управление программами ПЛК!Функция pausePLC}
\label{sec:pausePLC}

\begin{pHeader}
    Синтаксис:      & \RightHandText{int pausePLC(int plc);}\\
    Аргумент(ы):    & \RightHandText {int plc ~-- номер программы ПЛК} \\  
%    Возвращаемое значение:       & \RightHandText{Нет} \\
    Файл объявления:             & \RightHandText{sys/sys.h} \\      
\end{pHeader}

Функция вызывает временный останов программы ПЛК, номер которой (от 0 до 31) определяется аргументом функции. \killoverfullbefore

Возвращаемое значение равно 0 при отсутствии ошибок и отлично от 0 в противном случае. \killoverfullbefore

Является системной.
% *******end subsection*****************
%--------------------------------------------------------
% *******begin subsection***************
\subsubsection{\DbgSecSt{\StPart}{pausePLCs}}
\index{Программный интерфейс ПЛК!Управление программами ПЛК!Функция pausePLCs}
\label{sec:pausePLCs}

\begin{pHeader}
    Синтаксис:      & \RightHandText{int pausePLCs(int plc);}\\
    Аргумент(ы):    & \RightHandText {int plc ~-- номера программ ПЛК} \\  
%    Возвращаемое значение:       & \RightHandText{Нет} \\
    Файл объявления:             & \RightHandText{sys/sys.h} \\      
\end{pHeader}

Функция вызывает временный останов программ ПЛК, номера которых (от 0 до 31) определяются аргументом функции. Аргумент функции – битовое поле, в котором номера установленных битов (значения которых равны 1) соответствуют номерам программ ПЛК.\killoverfullbefore

 Возвращаемое значение равно 0 при отсутствии ошибок и отлично от 0 в противном случае. \killoverfullbefore

Является системной.
% *******end subsection*****************
%--------------------------------------------------------
% *******begin subsection***************
\subsubsection{\DbgSecSt{\StPart}{resumePLC}}
\index{Программный интерфейс ПЛК!Управление программами ПЛК!Функция resumePLC}
\label{sec:resumePLC}

\begin{pHeader}
    Синтаксис:      & \RightHandText{int resumePLC(int plc);}\\
    Аргумент(ы):    & \RightHandText {int plc ~-- номер программы ПЛК} \\  
%    Возвращаемое значение:       & \RightHandText{Нет} \\
    Файл объявления:             & \RightHandText{sys/sys.h} \\      
\end{pHeader}

Функция вызывает возобновление выполнения программы ПЛК, номер которой (от 0 до 31) определяется аргументом функции. \killoverfullbefore

Возвращаемое значение равно 0 при отсутствии ошибок и отлично от 0 в противном случае. \killoverfullbefore

Является системной.
% *******end subsection*****************

%--------------------------------------------------------
% *******begin subsection***************
\subsubsection{\DbgSecSt{\StPart}{resumePLCs}}
\index{Программный интерфейс ПЛК!Управление программами ПЛК!Функция resumePLCs}
\label{sec:resumePLCs}

\begin{pHeader}
    Синтаксис:      & \RightHandText{int resumePLCs(int plc);}\\
    Аргумент(ы):    & \RightHandText {int plc ~-- номера программ ПЛК} \\  
%    Возвращаемое значение:       & \RightHandText{Нет} \\
    Файл объявления:             & \RightHandText{sys/sys.h} \\      
\end{pHeader}

Функция вызывает возобновление выполнения программ ПЛК, номера которых (от 0 до 31) определяются аргументом функции. Аргумент функции – битовое поле, в котором номера установленных битов (значения которых равны 1) соответствуют номерам программ ПЛК.\killoverfullbefore

 Возвращаемое значение равно 0 при отсутствии ошибок и отлично от 0 в противном случае. \killoverfullbefore

Является системной.
% *******end subsection*****************
%--------------------------------------------------------
% *******begin subsection***************
\subsubsection{\DbgSecSt{\StPart}{disablePLC}}
\index{Программный интерфейс ПЛК!Управление программами ПЛК!Функция disablePLC}
\label{sec:disablePLC}

\begin{pHeader}
    Синтаксис:      & \RightHandText{int disablePLC(int plc);}\\
    Аргумент(ы):    & \RightHandText {int plc ~-- номер программы ПЛК} \\  
%    Возвращаемое значение:       & \RightHandText{Нет} \\
    Файл объявления:             & \RightHandText{sys/sys.h} \\      
\end{pHeader}

Функция вызывает отмену выполнения программы ПЛК, номер которой (от 0 до 31) определяется аргументом функции. Возвращаемое значение равно 0 при отсутствии ошибок и отлично от 0 в противном случае. \killoverfullbefore

Является системной.
% *******end subsection*****************
%--------------------------------------------------------
% *******begin subsection***************
\subsubsection{\DbgSecSt{\StPart}{disablePLCs}}
\index{Программный интерфейс ПЛК!Управление программами ПЛК!Функция disablePLCs}
\label{sec:disablePLCs}

\begin{pHeader}
    Синтаксис:      & \RightHandText{int disablePLCs(int plc);}\\
    Аргумент(ы):    & \RightHandText {int plc ~-- номера программ ПЛК} \\  
%    Возвращаемое значение:       & \RightHandText{Нет} \\
    Файл объявления:             & \RightHandText{sys/sys.h} \\      
\end{pHeader}

Функция вызывает отмену выполнения программ ПЛК, номера которых (от 0 до 31) определяются аргументом функции. Аргумент функции – битовое поле, в котором номера установленных битов (значения которых равны 1) соответствуют номерам программ ПЛК.\killoverfullbefore

 Возвращаемое значение равно 0 при отсутствии ошибок и отлично от 0 в противном случае. \killoverfullbefore

Является системной.
% *******end subsection*****************
%--------------------------------------------------------
% *******begin subsection***************
\subsubsection{\DbgSecSt{\StPart}{stepPLC}}
\index{Программный интерфейс ПЛК!Управление программами ПЛК!Функция stepPLC}
\label{sec:stepPLC}

\begin{pHeader}
    Синтаксис:      & \RightHandText{int stepPLC(int plc);}\\
    Аргумент(ы):    & \RightHandText {int plc ~-- номер программы ПЛК} \\  
%    Возвращаемое значение:       & \RightHandText{Нет} \\
    Файл объявления:             & \RightHandText{sys/sys.h} \\      
\end{pHeader}

Функция вызывает пошаговое выполнение программы ПЛК, номер которой (от 0 до 31) определяется аргументом функции. \killoverfullbefore

Возвращаемое значение равно 0 при отсутствии ошибок и отлично от 0 в противном случае. \killoverfullbefore

Является системной.
% *******end subsection*****************

%--------------------------------------------------------
% *******begin subsection***************
\subsubsection{\DbgSecSt{\StPart}{stepPLCs}}
\index{Программный интерфейс ПЛК!Управление программами ПЛК!Функция stepPLCs}
\label{sec:stepPLCs}

\begin{pHeader}
    Синтаксис:      & \RightHandText{int stepPLCs(int plc);}\\
    Аргумент(ы):    & \RightHandText {int plc ~-- номера программ ПЛК} \\  
%    Возвращаемое значение:       & \RightHandText{Нет} \\
    Файл объявления:             & \RightHandText{sys/sys.h} \\      
\end{pHeader}

Функция вызывает пошаговое выполнение программ ПЛК, номера которых (от 0 до 31) определяются аргументом функции. Аргумент функции – битовое поле, в котором номера установленных битов (значения которых равны 1) соответствуют номерам программ ПЛК.\killoverfullbefore

 Возвращаемое значение равно 0 при отсутствии ошибок и отлично от 0 в противном случае. \killoverfullbefore

Является системной.
% *******end subsection*****************
%--------------------------------------------------------                % API: управление программами plc
%    %--------------------------------------------------------
% *******begin section***************
\section{\DbgSecSt{\StPart}{Управление движением при фрезерной обработке}}
%--------------------------------------------------------

\subsection{\DbgSecSt{\StPart}{Типы данных}}
%--------------------------------------------------------
% *******begin subsection***************
\subsubsection{\DbgSecSt{\StPart}{Tool}}
\index{Программный интерфейс ПЛК!Управление движением при фрезерной  обработке!Структура Tool}
\label{sec:Tool}

\begin{fHeader}
    Тип данных:            & \RightHandText{Структура Tool}\\
    Файл объявления:             & \RightHandText{include/motion/mill/motion.h} \\
\end{fHeader}

Структура определяет параметры инструмента.

\begin{MyTableThreeColAllCntr}{Структура Tool}{tbl:Tool}{|m{0.3\linewidth}|m{0.25\linewidth}|m{0.45\linewidth}|}{Элемент}{Тип}{Описание}
\hline X & \centering{double} &   \\
\hline Y & \centering{double} &  \\
\hline L & \centering{double} & Длина \\
\hline R & \centering{double} & Радиус \\
\end{MyTableThreeColAllCntr}
% *******end subsection***************
%--------------------------------------------------------
% *******begin subsection***************
\subsubsection{\DbgSecSt{\StPart}{CSTransform}}
\index{Программный интерфейс ПЛК!Управление движением при фрезерной  обработке!Структура CSTransform}
\label{sec:CSTransform}

\begin{fHeader}
    Тип данных:            & \RightHandText{Структура CSTransform}\\
    Файл объявления:             & \RightHandText{include/motion/mill/motion.h} \\
\end{fHeader}

Структура определяет параметры преобразования системы координат.

\begin{MyTableThreeColAllCntr}{Структура CSTransform}{tbl:CSTransform}{|m{0.41\linewidth}|m{0.24\linewidth}|m{0.35\linewidth}|}{Элемент}{Тип}{Описание}
\hline ofs & \centering{\myreftosec{XYZ}} & Смещение \\
\hline scale & \centering{\myreftosec{XYZ}} & Коэффициенты масштабирования \\
\hline scaleOfs & \centering{\myreftosec{XYZ}} & Координаты центра масштабирования \\
\hline mirror & \centering{\myreftosec{int}} & Оси для зеркального отображения (MIRROR\_X | MIRROR\_Y | MIRROR\_Z) \\
\hline mirrorOfs & \centering{\myreftosec{XYZ}} & Координаты центра зеркального отображения \\

\hline rotVec & \centering{\myreftosec{XYZ}} & Компоненты вектора поворота  \\
\hline rotOfs & \centering{\myreftosec{XYZ}} & Координаты центра поворота \\

\hline rotAngle & \centering{\myreftosec{double}} & Угол поворота \\
\hline rot[9] & \centering{\myreftosec{double}} &   \\
\hline local & \centering{\myreftosec{XYZ}} &   \\

\hline Bias[32] & \centering{\myreftosec{double}} &   \\
\hline rot[9] & \centering{\myreftosec{double}} &   \\
\hline XYZ[6] & \centering{\myreftosec{double}} &   \\
\end{MyTableThreeColAllCntr}
% *******end subsection***************
%--------------------------------------------------------
% *******begin subsection***************
\subsubsection{\DbgSecSt{\StPart}{MoveMode}}
\index{Программный интерфейс ПЛК!Управление движением при фрезерной  обработке!Перечисление MoveMode}
\label{sec:MoveMode}

\begin{fHeader}
    Тип данных:            & \RightHandText{Перечисление MoveMode}\\
    Файл объявления:             & \RightHandText{include/motion/mill/motion.h} \\
\end{fHeader}

Перечисление определяет идентификаторы режимов движения.

\begin{MyTableTwoColAllCntr}{Перечисление MoveMode}{tbl:MoveMode}{|m{0.38\linewidth}|m{0.57\linewidth}|}{Идентификатор}{Описание}
\hline moveRapid &   Быстрые перемещения  \\
\hline moveLinear  &  Линейная интерполяция \\
\hline moveCircleCW  &  Круговая интерполяция по часовой стрелке \\
\hline moveCircleCCW &  Круговая интерполяция против часовой стрелки \\
\end{MyTableTwoColAllCntr}
% *******end subsection***************
%--------------------------------------------------------
% *******begin subsection***************
\subsubsection{\DbgSecSt{\StPart}{PositionMode}}
\index{Программный интерфейс ПЛК!Управление движением при фрезерной  обработке!Перечисление PositionMode}
\label{sec:PositionMode}

\begin{fHeader}
    Тип данных:            & \RightHandText{Перечисление PositionMode}\\
    Файл объявления:             & \RightHandText{include/motion/mill/motion.h} \\
\end{fHeader}

Перечисление определяет идентификаторы режимов позиционирования.

\begin{MyTableTwoColAllCntr}{Перечисление PositionMode}{tbl:PositionMode}{|m{0.38\linewidth}|m{0.57\linewidth}|}{Идентификатор}{Описание}
\hline posAbsolute &  В абсолютных координатах \\
\hline posIncremental  & В относительных координатах (в приращениях)\\
\end{MyTableTwoColAllCntr}
% *******end subsection***************
%--------------------------------------------------------
% *******begin subsection***************
\subsubsection{\DbgSecSt{\StPart}{FeedMode}}
\index{Программный интерфейс ПЛК!Управление движением при фрезерной  обработке!Перечисление FeedMode}
\label{sec:FeedMode}

\begin{fHeader}
    Тип данных:            & \RightHandText{Перечисление FeedMode}\\
    Файл объявления:             & \RightHandText{include/motion/mill/motion.h} \\
\end{fHeader}

Перечисление определяет идентификаторы режимов управления подачей.

\begin{MyTableTwoColAllCntr}{Перечисление FeedMode}{tbl:FeedMode}{|m{0.38\linewidth}|m{0.57\linewidth}|}{Идентификатор}{Описание}
\hline feedUnitsMin &  Минутная подача (асинхронная подача) \\
\hline feedUnitsRev & Оборотная подача (синхронная подача) \\
\hline feedInverse & Подача с обратно зависимым временем \\
\end{MyTableTwoColAllCntr}
% *******end subsection***************
%--------------------------------------------------------
% *******begin subsection***************
\subsubsection{\DbgSecSt{\StPart}{SpindleMode}}
\index{Программный интерфейс ПЛК!Управление движением при фрезерной  обработке!Перечисление SpindleMode}
\label{sec:SpindleMode}

\begin{fHeader}
    Тип данных:            & \RightHandText{Перечисление SpindleMode}\\
    Файл объявления:             & \RightHandText{include/motion/mill/motion.h} \\
\end{fHeader}

Перечисление определяет идентификаторы режимов управления скоростью шпинделя.

\begin{MyTableTwoColAllCntr}{Перечисление SpindleMode}{tbl:SpindleMode}{|m{0.38\linewidth}|m{0.57\linewidth}|}{Идентификатор}{Описание}
\hline spindleRevMin & Скорость в об/мин \\
\hline spindleUnitsMin & Постоянная скорость резания \\
\end{MyTableTwoColAllCntr}
% *******end subsection***************
%--------------------------------------------------------
% *******begin subsection***************
\subsubsection{\DbgSecSt{\StPart}{FeedOverMode}}
\index{Программный интерфейс ПЛК!Управление движением при фрезерной  обработке!Перечисление FeedOverMode}
\label{sec:FeedOverMode}

\begin{fHeader}
    Тип данных:            & \RightHandText{Перечисление FeedOverMode}\\
    Файл объявления:             & \RightHandText{include/motion/mill/motion.h} \\
\end{fHeader}

Перечисление определяет идентификаторы режимов коррекции величины подачи.

\begin{MyTableTwoColAllCntr}{Перечисление FeedOverMode}{tbl:FeedOverMode}{|m{0.38\linewidth}|m{0.57\linewidth}|}{Идентификатор}{Описание}
\hline feedPercent & Коррекция величины подачи в процентном отношении \\
\hline feedFixed100 & Фиксированная величина подачи 100\% \\
\end{MyTableTwoColAllCntr}
% *******end subsection***************
%--------------------------------------------------------
% *******begin subsection***************
\subsubsection{\DbgSecSt{\StPart}{SpindleOverMode}}
\index{Программный интерфейс ПЛК!Управление движением при фрезерной  обработке!Перечисление SpindleOverMode}
\label{sec:SpindleOverMode}

\begin{fHeader}
    Тип данных:            & \RightHandText{Перечисление SpindleOverMode}\\
    Файл объявления:             & \RightHandText{include/motion/mill/motion.h} \\
\end{fHeader}

Перечисление определяет идентификаторы режимов коррекции скорости шпинделя.

\begin{MyTableTwoColAllCntr}{Перечисление SpindleOverMode}{tbl:SpindleOverMode}{|m{0.38\linewidth}|m{0.57\linewidth}|}{Идентификатор}{Описание}
\hline spinPercent & Коррекция величины скорости в процентном отношении  \\
\hline spinFixed100 & Фиксированная величина скорости 100\% \\
\end{MyTableTwoColAllCntr}
% *******end subsection***************
%--------------------------------------------------------
% *******begin subsection***************
\subsubsection{\DbgSecSt{\StPart}{BlendMode}}
\index{Программный интерфейс ПЛК!Управление движением при фрезерной  обработке!Перечисление BlendMode}
\label{sec:BlendMode}

\begin{fHeader}
    Тип данных:            & \RightHandText{Перечисление BlendMode}\\
    Файл объявления:             & \RightHandText{include/motion/mill/motion.h} \\
\end{fHeader}

Перечисление определяет идентификаторы режимов сопряжения кадров (последовательных программных перемещений).

\begin{MyTableTwoColAllCntr}{Перечисление BlendMode}{tbl:BlendMode}{|m{0.38\linewidth}|m{0.57\linewidth}|}{Идентификатор}{Описание}
\hline blendExactStop & Режим точного останова \\
\hline blendCornerOverride & Режим угловой коррекции \\
\hline blendCutting & Режим резания \\
\hline blendTapping & Режим резьбонарезания \\
\end{MyTableTwoColAllCntr}
% *******end subsection***************
%--------------------------------------------------------
% *******begin subsection***************
\subsubsection{\DbgSecSt{\StPart}{ProgramState}}
\index{Программный интерфейс ПЛК!Управление движением при фрезерной  обработке!Перечисление ProgramState}
\label{sec:ProgramState}

\begin{fHeader}
    Тип данных:            & \RightHandText{Перечисление ProgramState}\\
    Файл объявления:             & \RightHandText{include/motion/mill/motion.h} \\
\end{fHeader}

Перечисление определяет идентификаторы состояния УП.

\begin{MyTableTwoColAllCntr}{Перечисление ProgramState}{tbl:ProgramState}{|m{0.38\linewidth}|m{0.57\linewidth}|}{Идентификатор}{Описание}
\hline progReset & УП готова к выполнению \\
\hline progStart & Начало выполнения УП  \\
\hline progRunning &  УП выполняется \\
\hline progHold & УП приостановлена \\
\hline progStop &  УП остановлена \\
\end{MyTableTwoColAllCntr}
% *******end subsection***************
%--------------------------------------------------------
% *******begin subsection***************
\subsubsection{\DbgSecSt{\StPart}{SpindleDirection}}
\index{Программный интерфейс ПЛК!Управление движением при фрезерной  обработке!Перечисление SpindleDirection}
\label{sec:SpindleDirection}

\begin{fHeader}
    Тип данных:            & \RightHandText{Перечисление SpindleDirection}\\
    Файл объявления:             & \RightHandText{include/motion/mill/motion.h} \\
\end{fHeader}

Перечисление определяет идентификаторы состояния движения шпинделя.

\begin{MyTableTwoColAllCntr}{Перечисление SpindleDirection}{tbl:SpindleDirection}{|m{0.38\linewidth}|m{0.57\linewidth}|}{Идентификатор}{Описание}
\hline spinStopped & Шпиндель остановлен \\
\hline spinForward & Шпиндель вращается в прямом направлении \\
\hline spinReverse & Шпиндель вращается в обратном направлении \\
\end{MyTableTwoColAllCntr}
% *******end subsection***************
%--------------------------------------------------------
% *******begin subsection***************
\subsubsection{\DbgSecSt{\StPart}{CycleReturnMode}}
\index{Программный интерфейс ПЛК!Управление движением при фрезерной  обработке!Перечисление CycleReturnMode}
\label{sec:CycleReturnMode}

\begin{fHeader}
    Тип данных:            & \RightHandText{Перечисление CycleReturnMode}\\
    Файл объявления:             & \RightHandText{include/motion/mill/motion.h} \\
\end{fHeader}

Перечисление определяет идентификаторы типов возврата инструмента при выполнении постоянных циклов.

\begin{MyTableTwoColAllCntr}{Перечисление CycleReturnMode}{tbl:CycleReturnMode}{|m{0.38\linewidth}|m{0.57\linewidth}|}{Идентификатор}{Описание}
\hline cycleReturnInitial & Возврат на исходный уровень \\
\hline cycleReturnR & Возврат на опорный уровень (уровень точки R) \\
\end{MyTableTwoColAllCntr}
% *******end subsection***************
%--------------------------------------------------------
% *******begin subsection***************
\subsubsection{\DbgSecSt{\StPart}{ProgramRuntime}}
\index{Программный интерфейс ПЛК!Управление движением при фрезерной  обработке!Структура ProgramRuntime}
\label{sec:ProgramRuntime}

\begin{fHeader}
    Тип данных:            & \RightHandText{Структура ProgramRuntime}\\
    Файл объявления:             & \RightHandText{include/motion/mill/motion.h} \\
\end{fHeader}

Структура определяет параметры и данные канала обработки.

\begin{MyTableThreeColAllCntr}{Структура ProgramRuntime}{tbl:ProgramRuntime}{|m{0.3\linewidth}|m{0.25\linewidth}|m{0.45\linewidth}|}{Элемент}{Тип}{Описание}
\hline cs & \centering{int} & Координатная система \\
\hline spindle & \centering{int} &  \\
\hline programState & \centering{int} & Идентификатор состояния УП (см. \myreftosec{ProgramState}) \\
\hline optionalStop & \centering{int} & Опциональный останов \\
\hline spindleDirection & \centering{int} & Идентификатор состояния движения шпинделя (см. \myreftosec{SpindleDirection}) \\

\hline baseWCS & \centering{\myreftosec{XYZ}} & Координаты центра системы координат заготовки  \\
\hline baseWCSStore & \centering{\myreftosec{XYZ}} &    \\
\hline baseWCSSync & \centering{\myreftosec{XYZ}} &    \\

\hline transform & \centering{\myreftosec{CSTransform}} & Параметры преобразования системы координат \\
\hline transformStore & \centering{\myreftosec{CSTransform}} & Параметры преобразования системы координат \\
\hline transformSync & \centering{\myreftosec{CSTransform}} & Параметры преобразования системы координат \\

\hline gcodes[ЧИСЛО\_G\_КОДОВ] & \centering{int} & Массив идентификаторов G-кодов \\
\hline mcodes[ЧИСЛО\_M\_КОДОВ] & \centering{int} & Массив идентификаторов М-кодов \\

\hline moveMode & \centering{int} & Идентификатор типа перемещения (см. \myreftosec{MoveMode}) \\
\hline posMode & \centering{int} & Идентификатор режима перемещения (см. \myreftosec{PositionMode}\\

\hline scaleUnits & \centering{double} & Единицы измерения перемещений (1.0 ~-- мм)\\

\hline blendMode & \centering{int} & Идентификатор режима сопряжения последовательных программных перемещений (см. \myreftosec{BlendMode}\\
\hline blendModeStore & \centering{int} &  \\

\hline F & \centering{double} & Величина заданной подачи \\
\hline Fact & \centering{double} & Величина фактической подачи \\
\hline feedMode & \centering{int} & Идентификатор режима управления подачей (см. \myreftosec{FeedMode}\\
\hline feedOverMode & \centering{int} & Идентификатор режима коррекции величины подачи (см. \myreftosec{FeedOverMode}\\
\hline feedOverride & \centering{double} & Величина коррекции подачи \\
\hline feedSlewRate & \centering{double} &   \\

\hline S & \centering{double} & Величина заданной скорости шпинделя \\
\hline Sact & \centering{double} & Величина фактической скорости шпинделя \\
\hline spinMode & \centering{int} & Идентификатор режима управления шпинделем (см. \myreftosec{SpindleMode}\\
\hline spinOverMode & \centering{int} & Идентификатор режима коррекции скорости шпинделя (см. \myreftosec{SpindleOverMode}\\
\hline spinOverride & \centering{double} & Величина коррекции скорости шпинделя \\
\hline spinSlewRate & \centering{double} &   \\
\hline Srun & \centering{double} &   \\

\hline T & \centering{int} & Номер инструмента (ячейки) \\
\hline TatM6 & \centering{int} &  \\
\hline D & \centering{int} & Номер корректора на радиус инструмента \\
\hline H & \centering{int} & Корректор корректора на длину инструмента \\
\hline Dvalue & \centering{double} &   \\
\hline Dwear & \centering{double} &   \\
\hline Hvalue & \centering{double} &   \\
\hline Hwear & \centering{double} &   \\

\hline tool & \centering{\myreftosec{Tool}} &  Параметры инструмента  \\
\hline toolWear & \centering{\myreftosec{Tool}} &    \\
\hline prevInterPoint & \centering{\myreftosec{Pos}} &    \\

\hline activeG17 & \centering{int} & Активна рабочая плоскость XY, продольная ось Z \\

\hline cycleReturnMode & \centering{int} & Идентификатор типа возврата инструмента при выполнении постоянных циклов (см. \myreftosec{CycleReturnMode} \\

\hline cycleZ & \centering{double} & Координата основания отверстия \\
\hline cycleR & \centering{double} & Координата опорного уровня (точки R)\\
\hline cycleP & \centering{double} & Длительность паузы в миллисекундах у основания отверстия \\
\hline cycleQ & \centering{double} & Глубина обработки для каждого прохода\\
\hline cycleI & \centering{double} & Величина сдвига инструмента по оси X \\
\hline cycleJ & \centering{double} & Величина сдвига инструмента по оси Y  \\
\hline cycleK & \centering{double} & Величина сдвига инструмента по оси Z  \\
\end{MyTableThreeColAllCntr}
% *******end subsection***************
%--------------------------------------------------------
% *******begin subsection***************
\subsubsection{\DbgSecSt{\StPart}{WorkCS}}
\index{Программный интерфейс ПЛК!Управление движением при фрезерной  обработке!Структура WorkCS}
\label{sec:WorkCS}

\begin{fHeader}
    Тип данных:            & \RightHandText{Структура WorkCS}\\
    Файл объявления:             & \RightHandText{include/motion/mill/motion.h} \\
\end{fHeader}

Структура определяет параметры системы координат заготовки (рабочей системы координат).

\begin{MyTableThreeColAllCntr}{Структура WorkCS}{tbl:WorkCS}{|m{0.3\linewidth}|m{0.25\linewidth}|m{0.45\linewidth}|}{Элемент}{Тип}{Описание}
\hline offset & \centering{\myreftosec{XYZ}} &   \\
\hline rot & \centering{\myreftosec{XYZ}} &   \\
\end{MyTableThreeColAllCntr}
% *******end subsection***************
%--------------------------------------------------------
% *******begin subsection***************
\subsection{\DbgSecSt{\StPart}{Функции}}

% *******begin subsection***************
\subsubsection{\DbgSecSt{\StPart}{initChannel}}
\index{Программный интерфейс ПЛК!Управление движением при фрезерной обработке!Функция initChannel}
\label{sec:initChannel}

\begin{pHeader}
    Синтаксис:      & \RightHandText{void initChannel(int chan, int cs, ProgramRuntime \&channel);}\\
   Аргумент(ы):    & \RightHandText{int chan ~-- номер канала,} \\    
       & \RightHandText{int cs ~-- номер координатной системы,} \\  
       & \RightHandText{\myreftosec{ProgramRuntime} \&channel ~-- параметры и данные канала} \\ 
%    Возвращаемое значение:       & \RightHandText{Нет} \\ 
    Файл объявления:             & \RightHandText{include/motion/mill/motion.h} \\       
\end{pHeader}

Функция инициализирует канал обработки параметрами по умолчанию. 

Является системной.
% *******end section*****************
%--------------------------------------------------------
% *******begin subsection***************
\subsubsection{\DbgSecSt{\StPart}{resetChannel}}
\index{Программный интерфейс ПЛК!Управление движением при фрезерной обработке!Функция resetChannel}
\label{sec:resetChannel}

\begin{pHeader}
    Синтаксис:      & \RightHandText{void resetChannel(int chan, int cs, ProgramRuntime \&channel);}\\
   Аргумент(ы):    & \RightHandText{int chan ~-- номер канала,} \\    
       & \RightHandText{int cs ~-- номер координатной системы,} \\  
       & \RightHandText{\myreftosec{ProgramRuntime} \&channel ~-- параметры и данные канала} \\ 
%    Возвращаемое значение:       & \RightHandText{Нет} \\ 
    Файл объявления:             & \RightHandText{include/motion/mill/motion.h} \\       
\end{pHeader}

Функция инициализирует канал обработки.

Является системной.
% *******end section*****************
%--------------------------------------------------------
% *******begin subsection***************
\subsubsection{\DbgSecSt{\StPart}{reapplyTransform}}
\index{Программный интерфейс ПЛК!Управление движением при фрезерной обработке!Функция reapplyTransform}
\label{sec:reapplyTransform}

\begin{pHeader}
    Синтаксис:      & \RightHandText{void reapplyTransform(int chan, int cs, ProgramRuntime \&channel);}\\
   Аргумент(ы):    & \RightHandText{int chan ~-- номер канала,} \\    
       & \RightHandText{int cs ~-- номер координатной системы,} \\  
       & \RightHandText{\myreftosec{ProgramRuntime} \&channel ~-- параметры и данные канала} \\ 
%    Возвращаемое значение:       & \RightHandText{Нет} \\ 
    Файл объявления:             & \RightHandText{include/motion/mill/motion.h} \\       
\end{pHeader}



Является системной.
% *******end section*****************
%--------------------------------------------------------
% *******begin subsection***************
\subsubsection{\DbgSecSt{\StPart}{csApplyTransform}}
\index{Программный интерфейс ПЛК!Управление движением при фрезерной обработке!Функция csApplyTransform}
\label{sec:csApplyTransform}

\begin{pHeader}
    Синтаксис:      & \RightHandText{void csApplyTransform(ProgramRuntime \&channel);}\\
   Аргумент(ы):   & \RightHandText{\myreftosec{ProgramRuntime} \&channel ~-- параметры и данные канала} \\ 
%    Возвращаемое значение:       & \RightHandText{Нет} \\ 
    Файл объявления:             & \RightHandText{include/motion/mill/motion.h} \\       
\end{pHeader}

Функция выполняет пространственные преобразования координатной системы.

Является системной.
% *******end section*****************
%--------------------------------------------------------
% *******begin subsection***************
\subsubsection{\DbgSecSt{\StPart}{csResetTransform}}
\index{Программный интерфейс ПЛК!Управление движением при фрезерной обработке!Функция csResetTransform}
\label{sec:csResetTransform}

\begin{pHeader}
    Синтаксис:      & \RightHandText{void csResetTransform(ProgramRuntime \&channel);}\\
   Аргумент(ы):   & \RightHandText{\myreftosec{ProgramRuntime} \&channel ~-- параметры и данные канала} \\ 
%    Возвращаемое значение:       & \RightHandText{Нет} \\ 
    Файл объявления:             & \RightHandText{include/motion/mill/motion.h} \\       
\end{pHeader}

Функция отменяет пространственные преобразования координатной системы.

Является системной.
% *******end section*****************
%--------------------------------------------------------
% *******begin subsection***************
\subsubsection{\DbgSecSt{\StPart}{csBase}}
\index{Программный интерфейс ПЛК!Управление движением при фрезерной обработке!Функция csBase}
\label{sec:csBase}

\begin{pHeader}
    Синтаксис:      & \RightHandText{void csBase(ProgramRuntime \&channel, const XYZ \&base);}\\
   Аргумент(ы):   & \RightHandText{\myreftosec{ProgramRuntime} \&channel ~-- параметры и данные канала, } \\ 
 & \RightHandText{const \myreftosec{XYZ} \&base ~-- базовое смещение системы координат} \\ 
  & \RightHandText{заготовки} \\
%    Возвращаемое значение:       & \RightHandText{Нет} \\ 
    Файл объявления:             & \RightHandText{include/motion/mill/motion.h} \\       
\end{pHeader}

Функция задаёт базовое смещение системы координат заготовки (G92).

Является системной.
% *******end section*****************
%--------------------------------------------------------
% *******begin subsection***************
\subsubsection{\DbgSecSt{\StPart}{csOffset}}
\index{Программный интерфейс ПЛК!Управление движением при фрезерной обработке!Функция csOffset}
\label{sec:csOffset}

\begin{pHeader}
    Синтаксис:      & \RightHandText{void csOffset(ProgramRuntime \&channel, const XYZ \&offset);}\\
   Аргумент(ы):   & \RightHandText{\myreftosec{ProgramRuntime} \&channel ~-- параметры и данные канала, } \\ 
   & \RightHandText{const \myreftosec{XYZ} \&offset ~-- смещение рабочей системы координат} \\ 
%    Возвращаемое значение:       & \RightHandText{Нет} \\ 
    Файл объявления:             & \RightHandText{include/motion/mill/motion.h} \\       
\end{pHeader}

Функция задаёт смещение рабочей системы координат (G54).

Является системной.
% *******end section*****************
%--------------------------------------------------------
% *******begin subsection***************
\subsubsection{\DbgSecSt{\StPart}{csScale}}
\index{Программный интерфейс ПЛК!Управление движением при фрезерной обработке!Функция csScale}
\label{sec:csScale}

\begin{pHeader}
    Синтаксис:      & \RightHandText{void csScale(ProgramRuntime \&channel, const XYZ \&center, }\\
    & \RightHandText{const XYZ \&scale);}\\
   Аргумент(ы):   & \RightHandText{\myreftosec{ProgramRuntime} \&channel ~-- параметры и данные канала, } \\ 
   & \RightHandText{const \myreftosec{XYZ} \&center ~-- координаты центра масштабирования,} \\ 
   & \RightHandText{const \myreftosec{XYZ} \&scale ~-- масштабы по осям} \\ 
%    Возвращаемое значение:       & \RightHandText{Нет} \\ 
    Файл объявления:             & \RightHandText{include/motion/mill/motion.h} \\       
\end{pHeader}

Функция задаёт масштабирование рабочей системы координат (G51).

Является системной.
% *******end section*****************
%--------------------------------------------------------
% *******begin subsection***************
\subsubsection{\DbgSecSt{\StPart}{csScaleReset}}
\index{Программный интерфейс ПЛК!Управление движением при фрезерной обработке!Функция csScaleReset}
\label{sec:csScaleReset}

\begin{pHeader}
    Синтаксис:      & \RightHandText{void csScaleReset(ProgramRuntime \&channel);}\\
   Аргумент(ы):   & \RightHandText{\myreftosec{ProgramRuntime} \&channel ~-- параметры и данные канала} \\ 
%    Возвращаемое значение:       & \RightHandText{Нет} \\ 
    Файл объявления:             & \RightHandText{include/motion/mill/motion.h} \\       
\end{pHeader}

Функция отменяет масштабирование рабочей системы координат.

Является системной.
% *******end section*****************
%--------------------------------------------------------
% *******begin subsection***************
\subsubsection{\DbgSecSt{\StPart}{csMirror}}
\index{Программный интерфейс ПЛК!Управление движением при фрезерной обработке!Функция csMirror}
\label{sec:csMirror}

\begin{pHeader}
    Синтаксис:      & \RightHandText{void csMirror(ProgramRuntime \&channel, const XYZ \&center, int mirror);}\\
   Аргумент(ы):   & \RightHandText{\myreftosec{ProgramRuntime} \&channel ~-- параметры и данные канала, } \\ 
   & \RightHandText{const \myreftosec{XYZ} \&center ~-- координаты центра зеркалирования,} \\ 
   & \RightHandText{int mirror ~-- оси для зеркалирования } \\ 
   & \RightHandText{(MIRROR\_X | MIRROR\_Y | MIRROR\_Z)} \\ 
%    Возвращаемое значение:       & \RightHandText{Нет} \\ 
    Файл объявления:             & \RightHandText{include/motion/mill/motion.h} \\       
\end{pHeader}

Функция задаёт зеркалирование рабочей системы координат (G51.1).

Является системной.
% *******end section*****************
%--------------------------------------------------------
% *******begin subsection***************
\subsubsection{\DbgSecSt{\StPart}{csMirrorReset}}
\index{Программный интерфейс ПЛК!Управление движением при фрезерной обработке!Функция csMirrorReset}
\label{sec:csMirrorReset}

\begin{pHeader}
    Синтаксис:      & \RightHandText{void csMirrorReset(ProgramRuntime \&channel);}\\
   Аргумент(ы):   & \RightHandText{\myreftosec{ProgramRuntime} \&channel ~-- параметры и данные канала} \\ 
%    Возвращаемое значение:       & \RightHandText{Нет} \\ 
    Файл объявления:             & \RightHandText{include/motion/mill/motion.h} \\       
\end{pHeader}

Функция отменяет зеркалирование рабочей системы координат.

Является системной.
% *******end section*****************
%--------------------------------------------------------
% *******begin subsection***************
\subsubsection{\DbgSecSt{\StPart}{csRotate}}
\index{Программный интерфейс ПЛК!Управление движением при фрезерной обработке!Функция csRotate}
\label{sec:csRotate}

\begin{pHeader}
    Синтаксис:      & \RightHandText{void csRotate(ProgramRuntime \&channel, const XYZ \&center, }\\
   & \RightHandText{const XYZ \&vector, double angle);}\\
   Аргумент(ы):   & \RightHandText{\myreftosec{ProgramRuntime} \&channel ~-- параметры и данные канала, } \\ 
   & \RightHandText{const \myreftosec{XYZ} \&center ~-- координаты центра поворота,} \\ 
   & \RightHandText{const \myreftosec{XYZ} \&vector ~-- вектор оси поворота,} \\
   & \RightHandText{double angle ~-- угол поворота} \\ 
%    Возвращаемое значение:       & \RightHandText{Нет} \\ 
    Файл объявления:             & \RightHandText{include/motion/mill/motion.h} \\       
\end{pHeader}

Функция задаёт поворот рабочей системы координат (G68).

Является системной.
% *******end section*****************
%--------------------------------------------------------
% *******begin subsection***************
\subsubsection{\DbgSecSt{\StPart}{csRotateReset}}
\index{Программный интерфейс ПЛК!Управление движением при фрезерной обработке!Функция csRotateReset}
\label{sec:csRotateReset}

\begin{pHeader}
    Синтаксис:      & \RightHandText{void csRotateReset(ProgramRuntime \&channel);}\\
   Аргумент(ы):   & \RightHandText{\myreftosec{ProgramRuntime} \&channel ~-- параметры и данные канала} \\ 
%    Возвращаемое значение:       & \RightHandText{Нет} \\ 
    Файл объявления:             & \RightHandText{include/motion/mill/motion.h} \\       
\end{pHeader}

Функция отменяет поворот рабочей системы координат.

Является системной.
% *******end section*****************
%--------------------------------------------------------
% *******begin subsection***************
\subsubsection{\DbgSecSt{\StPart}{csLocal}}
\index{Программный интерфейс ПЛК!Управление движением при фрезерной обработке!Функция csLocal}
\label{sec:csLocal}

\begin{pHeader}
    Синтаксис:      & \RightHandText{void csLocal(ProgramRuntime \&channel, const XYZ \&offset);}\\
   Аргумент(ы):   & \RightHandText{\myreftosec{ProgramRuntime} \&channel ~-- параметры и данные канала, } \\ 
   & \RightHandText{const \myreftosec{XYZ} \&offset ~-- смещение локалькой системы координат} \\ 
%    Возвращаемое значение:       & \RightHandText{Нет} \\ 
    Файл объявления:             & \RightHandText{include/motion/mill/motion.h} \\       
\end{pHeader}

Функция задаёт задание локальную систему координат (G52).

Является системной.
% *******end section*****************
%--------------------------------------------------------
% *******begin subsection***************
\subsubsection{\DbgSecSt{\StPart}{csFeedMode}}
\index{Программный интерфейс ПЛК!Управление движением при фрезерной обработке!Функция csFeedMode}
\label{sec:csFeedMode}

\begin{pHeader}
    Синтаксис:      & \RightHandText{void csFeedMode(ProgramRuntime \&channel, int mode);}\\
   Аргумент(ы):   & \RightHandText{\myreftosec{ProgramRuntime} \&channel ~-- параметры и данные канала, } \\ 
   & \RightHandText{int mode ~-- идентификатор режима управления подачей} \\ 
%    Возвращаемое значение:       & \RightHandText{Нет} \\ 
    Файл объявления:             & \RightHandText{include/motion/mill/motion.h} \\       
\end{pHeader}

Функция задаёт режим подачи, принимая в качестве аргумента идентификатор из перечисления \myreftosec{FeedMode}.

Является системной.
% *******end section*****************
%--------------------------------------------------------
% *******begin subsection***************
\subsubsection{\DbgSecSt{\StPart}{csFeedOverride}}
\index{Программный интерфейс ПЛК!Управление движением при фрезерной обработке!Функция csFeedOverride}
\label{sec:csFeedOverride}

\begin{pHeader}
    Синтаксис:      & \RightHandText{void csFeedOverride(ProgramRuntime \&channel, double value);}\\
   Аргумент(ы):   & \RightHandText{\myreftosec{ProgramRuntime} \&channel ~-- параметры и данные канала, } \\ 
   & \RightHandText{double value ~-- величина коррекции подачи} \\ 
%    Возвращаемое значение:       & \RightHandText{Нет} \\ 
    Файл объявления:             & \RightHandText{include/motion/mill/motion.h} \\       
\end{pHeader}

Функция задаёт корректор подачи, принимая в качестве аргумента величину коррекции подачи (1.0 = 100\%).

Является системной.
% *******end section*****************
%--------------------------------------------------------
% *******begin subsection***************
\subsubsection{\DbgSecSt{\StPart}{csFeedOverrideMode}}
\index{Программный интерфейс ПЛК!Управление движением при фрезерной обработке!Функция csFeedOverrideMode}
\label{sec:csFeedOverrideMode}

\begin{pHeader}
    Синтаксис:      & \RightHandText{void csFeedOverrideMode(ProgramRuntime \&channel, int mode);}\\
   Аргумент(ы):   & \RightHandText{\myreftosec{ProgramRuntime} \&channel ~-- параметры и данные канала, } \\ 
   & \RightHandText{int mode ~-- идентификатор режима коррекции величины подачи} \\ 
%    Возвращаемое значение:       & \RightHandText{Нет} \\ 
    Файл объявления:             & \RightHandText{include/motion/mill/motion.h} \\       
\end{pHeader}

Функция задаёт режим коррекции величины подачи, принимая в качестве аргумента идентификатор из перечисления \myreftosec{FeedOverMode}.

Является системной.
% *******end section*****************
%--------------------------------------------------------
% *******begin subsection***************
\subsubsection{\DbgSecSt{\StPart}{csSpindleMode}}
\index{Программный интерфейс ПЛК!Управление движением при фрезерной обработке!Функция csSpindleMode}
\label{sec:csSpindleMode}

\begin{pHeader}
    Синтаксис:      & \RightHandText{void csSpindleMode(ProgramRuntime \&channel, int mode);}\\
   Аргумент(ы):   & \RightHandText{\myreftosec{ProgramRuntime} \&channel ~-- параметры и данные канала, } \\ 
   & \RightHandText{int mode ~-- идентификатор режима управления скоростью  } \\ 
   & \RightHandText{ шпинделя} \\   
%    Возвращаемое значение:       & \RightHandText{Нет} \\ 
    Файл объявления:             & \RightHandText{include/motion/mill/motion.h} \\       
\end{pHeader}

Функция задаёт режим подачи, принимая в качестве аргумента идентификатор из перечисления \myreftosec{SpindleMode}.

Является системной.
% *******end section*****************
%--------------------------------------------------------
% *******begin subsection***************
\subsubsection{\DbgSecSt{\StPart}{csSpindleOverride}}
\index{Программный интерфейс ПЛК!Управление движением при фрезерной обработке!Функция csSpindleOverride}
\label{sec:csSpindleOverride}

\begin{pHeader}
    Синтаксис:      & \RightHandText{void csSpindleOverride(ProgramRuntime \&channel, double value);}\\
   Аргумент(ы):   & \RightHandText{\myreftosec{ProgramRuntime} \&channel ~-- параметры и данные канала, } \\ 
   & \RightHandText{double value ~-- величина коррекции скорости шпинделя} \\ 
%    Возвращаемое значение:       & \RightHandText{Нет} \\ 
    Файл объявления:             & \RightHandText{include/motion/mill/motion.h} \\       
\end{pHeader}

Функция задаёт корректор подачи, принимая в качестве аргумента величину коррекции скорости шпинделя (1.0 = 100\%).

Является системной.
% *******end section*****************
%--------------------------------------------------------
% *******begin subsection***************
\subsubsection{\DbgSecSt{\StPart}{csSpindleOverrideMode}}
\index{Программный интерфейс ПЛК!Управление движением при фрезерной обработке!Функция csSpindleOverrideMode}
\label{sec:csSpindleOverrideMode}

\begin{pHeader}
    Синтаксис:      & \RightHandText{void csSpindleOverrideMode(ProgramRuntime \&channel, int mode);}\\
   Аргумент(ы):   & \RightHandText{\myreftosec{ProgramRuntime} \&channel ~-- параметры и данные канала, } \\ 
   & \RightHandText{int mode ~-- идентификатор режима коррекции скорости шпинделя} \\ 
%    Возвращаемое значение:       & \RightHandText{Нет} \\ 
    Файл объявления:             & \RightHandText{include/motion/mill/motion.h} \\       
\end{pHeader}

Функция задаёт режим коррекции скорости шпинделя, принимая в качестве аргумента идентификатор из перечисления \myreftosec{SpindleOverMode}.

Является системной.
% *******end section*****************
%--------------------------------------------------------
% *******begin subsection***************
\subsubsection{\DbgSecSt{\StPart}{csBlendMode}}
\index{Программный интерфейс ПЛК!Управление движением при фрезерной обработке!Функция csBlendMode}
\label{sec:csBlendMode}

\begin{pHeader}
    Синтаксис:      & \RightHandText{void csBlendMode(ProgramRuntime \&channel, int mode);}\\
   Аргумент(ы):   & \RightHandText{\myreftosec{ProgramRuntime} \&channel ~-- параметры и данные канала, } \\ 
   & \RightHandText{int mode ~-- идентификатор режима сопряжения кадров} \\ 
%    Возвращаемое значение:       & \RightHandText{Нет} \\ 
    Файл объявления:             & \RightHandText{include/motion/mill/motion.h} \\       
\end{pHeader}

Функция задаёт режим сопряжения кадров, принимая в качестве аргумента идентификатор из перечисления \myreftosec{BlendMode}.

Является системной.
% *******end section*****************
%--------------------------------------------------------
% *******begin subsection***************
\subsubsection{\DbgSecSt{\StPart}{csMoveMode}}
\index{Программный интерфейс ПЛК!Управление движением при фрезерной обработке!Функция csMoveMode}
\label{sec:csMoveMode}

\begin{pHeader}
    Синтаксис:      & \RightHandText{void csMoveMode(ProgramRuntime \&channel, int mode);}\\
   Аргумент(ы):   & \RightHandText{\myreftosec{ProgramRuntime} \&channel ~-- параметры и данные канала, } \\ 
   & \RightHandText{int mode ~-- идентификатор режима движения} \\ 
%    Возвращаемое значение:       & \RightHandText{Нет} \\ 
    Файл объявления:             & \RightHandText{include/motion/mill/motion.h} \\       
\end{pHeader}

Функция задаёт режим движения, принимая в качестве аргумента идентификатор из перечисления \myreftosec{MoveMode}.

Является системной.
% *******end section*****************
%--------------------------------------------------------
% *******begin subsection***************
\subsubsection{\DbgSecSt{\StPart}{csPosMode}}
\index{Программный интерфейс ПЛК!Управление движением при фрезерной обработке!Функция csPosMode}
\label{sec:csPosMode}

\begin{pHeader}
    Синтаксис:      & \RightHandText{void csPosMode(ProgramRuntime \&channel, int mode);}\\
   Аргумент(ы):   & \RightHandText{\myreftosec{ProgramRuntime} \&channel ~-- параметры и данные канала, } \\ 
   & \RightHandText{int mode ~-- идентификатор режима позиционирования} \\ 
%    Возвращаемое значение:       & \RightHandText{Нет} \\ 
    Файл объявления:             & \RightHandText{include/motion/mill/motion.h} \\       
\end{pHeader}

Функция задаёт режим позиционирования, принимая в качестве аргумента идентификатор из перечисления \myreftosec{PositionMode}.

Является системной.
% *******end section*****************
%--------------------------------------------------------
% *******begin subsection***************
\subsubsection{\DbgSecSt{\StPart}{csSetUnits}}
\index{Программный интерфейс ПЛК!Управление движением при фрезерной обработке!Функция csSetUnits}
\label{sec:csSetUnits}

\begin{pHeader}
    Синтаксис:      & \RightHandText{void csSetUnits(ProgramRuntime \&channel, double units);}\\
   Аргумент(ы):   & \RightHandText{\myreftosec{ProgramRuntime} \&channel ~-- параметры и данные канала, } \\ 
   & \RightHandText{double units ~-- значение единиц измерения в долях мм} \\ 
%    Возвращаемое значение:       & \RightHandText{Нет} \\ 
    Файл объявления:             & \RightHandText{include/motion/mill/motion.h} \\       
\end{pHeader}

Функция задаёт единицы измерения перемещений, принимая в качестве аргумента значение единиц измерения в долях мм (1.0 = 1.0 мм).

Является системной.
% *******end section*****************
%--------------------------------------------------------
% *******begin subsection***************
\subsubsection{\DbgSecSt{\StPart}{csToolSelect}}
\index{Программный интерфейс ПЛК!Управление движением при фрезерной обработке!Функция csToolSelect}
\label{sec:csToolSelect}

\begin{pHeader}
    Синтаксис:      & \RightHandText{void csToolSelect(ProgramRuntime \&channel, Tool \&tool);}\\
   Аргумент(ы):   & \RightHandText{\myreftosec{ProgramRuntime} \&channel ~-- параметры и данные канала, } \\ 
   & \RightHandText{\myreftosec{Tool} \&tool ~-- параметры выбранного инструмента} \\ 
%    Возвращаемое значение:       & \RightHandText{Нет} \\ 
    Файл объявления:             & \RightHandText{include/motion/mill/motion.h} \\       
\end{pHeader}

Функция задаёт инструмент, параметры которого являются аргументом функции.

Является системной.
% *******end section*****************
%--------------------------------------------------------
% *******begin subsection***************
\subsubsection{\DbgSecSt{\StPart}{csToolReset}}
\index{Программный интерфейс ПЛК!Управление движением при фрезерной обработке!Функция csToolReset}
\label{sec:csToolReset}

\begin{pHeader}
    Синтаксис:      & \RightHandText{void csToolReset(ProgramRuntime \&channel);}\\
   Аргумент(ы):   & \RightHandText{\myreftosec{ProgramRuntime} \&channel ~-- параметры и данные канала} \\ 
%    Возвращаемое значение:       & \RightHandText{Нет} \\ 
    Файл объявления:             & \RightHandText{include/motion/mill/motion.h} \\       
\end{pHeader}

Функция отменяет выбор инструмента.

Является системной.
% *******end section*****************
%--------------------------------------------------------
% *******begin subsection***************
\subsubsection{\DbgSecSt{\StPart}{csToolWear}}
\index{Программный интерфейс ПЛК!Управление движением при фрезерной обработке!Функция csToolWear}
\label{sec:csToolWear}

\begin{pHeader}
    Синтаксис:      & \RightHandText{void csToolWear(ProgramRuntime \&channel, Tool \&tool);}\\
   Аргумент(ы):   & \RightHandText{\myreftosec{ProgramRuntime} \&channel ~-- параметры и данные канала, } \\ 
   & \RightHandText{\myreftosec{Tool} \&tool ~-- параметры выбранного инструмента} \\ 
%    Возвращаемое значение:       & \RightHandText{Нет} \\ 
    Файл объявления:             & \RightHandText{include/motion/mill/motion.h} \\       
\end{pHeader}



Является системной.
% *******end section*****************
%--------------------------------------------------------
% *******begin subsection***************
\subsubsection{\DbgSecSt{\StPart}{csToolWearReset}}
\index{Программный интерфейс ПЛК!Управление движением при фрезерной обработке!Функция csToolWearReset}
\label{sec:csToolWearReset}

\begin{pHeader}
    Синтаксис:      & \RightHandText{void csToolWearReset(ProgramRuntime \&channel);}\\
   Аргумент(ы):   & \RightHandText{\myreftosec{ProgramRuntime} \&channel ~-- параметры и данные канала} \\ 
%    Возвращаемое значение:       & \RightHandText{Нет} \\ 
    Файл объявления:             & \RightHandText{include/motion/mill/motion.h} \\       
\end{pHeader}



Является системной.
% *******end section*****************
%--------------------------------------------------------
% *******begin subsection***************
\subsubsection{\DbgSecSt{\StPart}{csToolXY}}
\index{Программный интерфейс ПЛК!Управление движением при фрезерной обработке!Функция csToolXY}
\label{sec:csToolXY}

\begin{pHeader}
    Синтаксис:      & \RightHandText{void csToolXY(ProgramRuntime \&channel, double X, double Y);}\\
   Аргумент(ы):   & \RightHandText{\myreftosec{ProgramRuntime} \&channel ~-- параметры и данные канала,} \\ 
   & \RightHandText{double X ~--  ,} \\    
   & \RightHandText{double Y ~--  } \\       
%    Возвращаемое значение:       & \RightHandText{Нет} \\ 
    Файл объявления:             & \RightHandText{include/motion/mill/motion.h} \\       
\end{pHeader}



Является системной.
% *******end section*****************
%--------------------------------------------------------
% *******begin subsection***************
\subsubsection{\DbgSecSt{\StPart}{csToolX}}
\index{Программный интерфейс ПЛК!Управление движением при фрезерной обработке!Функция csToolX}
\label{sec:csToolX}

\begin{pHeader}
    Синтаксис:      & \RightHandText{void csToolX(ProgramRuntime \&channel, double X);}\\
   Аргумент(ы):   & \RightHandText{\myreftosec{ProgramRuntime} \&channel ~-- параметры и данные канала,} \\ 
   & \RightHandText{double X ~--  } \\    
%    Возвращаемое значение:       & \RightHandText{Нет} \\ 
    Файл объявления:             & \RightHandText{include/motion/mill/motion.h} \\       
\end{pHeader}



Является системной.
% *******end section*****************
%--------------------------------------------------------
% *******begin subsection***************
\subsubsection{\DbgSecSt{\StPart}{csToolY}}
\index{Программный интерфейс ПЛК!Управление движением при фрезерной обработке!Функция csToolY}
\label{sec:csToolY}

\begin{pHeader}
    Синтаксис:      & \RightHandText{void csToolY(ProgramRuntime \&channel, double Y);}\\
   Аргумент(ы):   & \RightHandText{\myreftosec{ProgramRuntime} \&channel ~-- параметры и данные канала,} \\ 
   & \RightHandText{double Y ~--  } \\    
%    Возвращаемое значение:       & \RightHandText{Нет} \\ 
    Файл объявления:             & \RightHandText{include/motion/mill/motion.h} \\       
\end{pHeader}



Является системной.
% *******end section*****************
%--------------------------------------------------------
% *******begin subsection***************
\subsubsection{\DbgSecSt{\StPart}{csToolLength}}
\index{Программный интерфейс ПЛК!Управление движением при фрезерной обработке!Функция csToolLength}
\label{sec:csToolLength}

\begin{pHeader}
    Синтаксис:      & \RightHandText{void csToolLength(ProgramRuntime \&channel, double L);}\\
   Аргумент(ы):   & \RightHandText{\myreftosec{ProgramRuntime} \&channel ~-- параметры и данные канала,} \\ 
   & \RightHandText{double L ~--  длина инструмента} \\    
%    Возвращаемое значение:       & \RightHandText{Нет} \\ 
    Файл объявления:             & \RightHandText{include/motion/mill/motion.h} \\       
\end{pHeader}

Функция задаёт длину инструмента, значение которой является аргументом функции.

Является системной.
% *******end section*****************
%--------------------------------------------------------
% *******begin subsection***************
\subsubsection{\DbgSecSt{\StPart}{cycleReset}}
\index{Программный интерфейс ПЛК!Управление движением при фрезерной обработке!Функция cycleReset}
\label{sec:cycleReset}

\begin{pHeader}
    Синтаксис:      & \RightHandText{void cycleReset(ProgramRuntime \&channel);}\\
   Аргумент(ы):   & \RightHandText{\myreftosec{ProgramRuntime} \&channel ~-- параметры и данные канала} \\    
%    Возвращаемое значение:       & \RightHandText{Нет} \\ 
    Файл объявления:             & \RightHandText{include/motion/mill/motion.h} \\       
\end{pHeader}


Является системной.
% *******end section*****************
%--------------------------------------------------------
% *******begin subsection***************
\subsubsection{\DbgSecSt{\StPart}{stepBlock}}
\index{Программный интерфейс ПЛК!Управление движением при фрезерной обработке!Функция stepBlock}
\label{sec:stepBlock}

\begin{pHeader}
    Синтаксис:      & \RightHandText{void stepBlock(ProgramRuntime \&channel);}\\
   Аргумент(ы):   & \RightHandText{\myreftosec{ProgramRuntime} \&channel ~-- параметры и данные канала} \\    
%    Возвращаемое значение:       & \RightHandText{Нет} \\ 
    Файл объявления:             & \RightHandText{include/motion/mill/motion.h} \\       
\end{pHeader}


Является системной.
% *******end section*****************
%--------------------------------------------------------
% *******begin subsection***************
\subsubsection{\DbgSecSt{\StPart}{csSetGCode}}
\index{Программный интерфейс ПЛК!Управление движением при фрезерной обработке!Функция csSetGCode}
\label{sec:csSetGCode}

\begin{pHeader}
    Синтаксис:      & \RightHandText{void csSetGCode(ProgramRuntime \&channel, int group, int value);}\\
   Аргумент(ы):   & \RightHandText{\myreftosec{ProgramRuntime} \&channel ~-- параметры и данные канала,} \\    длина инструмента
   & \RightHandText{int group ~--  идентификатор группы,} \\    
   & \RightHandText{int value ~--  номер G-кода в группе} \\          
%    Возвращаемое значение:       & \RightHandText{Нет} \\ 
    Файл объявления:             & \RightHandText{include/motion/mill/motion.h} \\       
\end{pHeader}


Является системной.
% *******end section*****************
%--------------------------------------------------------
% *******begin subsection***************
\subsubsection{\DbgSecSt{\StPart}{csSetMCode}}
\index{Программный интерфейс ПЛК!Управление движением при фрезерной обработке!Функция csSetMCode}
\label{sec:csSetMCode}

\begin{pHeader}
    Синтаксис:      & \RightHandText{void csSetMCode(ProgramRuntime \&channel, int group, int value);}\\
   Аргумент(ы):   & \RightHandText{\myreftosec{ProgramRuntime} \&channel ~-- параметры и данные канала,} \\    длина инструмента
   & \RightHandText{int group ~--  идентификатор группы,} \\    
   & \RightHandText{int value ~--  номер G-кода в группе} \\          
%    Возвращаемое значение:       & \RightHandText{Нет} \\ 
    Файл объявления:             & \RightHandText{include/motion/mill/motion.h} \\       
\end{pHeader}


Является системной.
% *******end section*****************

%--------------------------------------------------------
\index{Программный интерфейс ПЛК|)}

\clearpage
       % API: управление движением - фрезерная обработка
	\etocsettocdepth.toc {section}

\definecolor{title_color}{HTML}{CDCDB4} %A98040

\lstset{basicstyle=\footnotesize, keywordstyle=\color{indigo_dye}\bfseries, stringstyle=\ttfamily, language=C, numbers=left, numberstyle=\tiny, stepnumber=1, numbersep=5pt}

\begin{comment}
\lstset{% general command to set parameter(s)
basicstyle=\small, % print whole listing small
keywordstyle=\color{black}\bfseries\underbar,
% underlined bold black keywords
identifierstyle=, % nothing happens
commentstyle=\color{white}, % white comments
stringstyle=\ttfamily, % typewriter type for strings
showstringspaces=false} % no special string spaces
\end{comment}


\chapterimage{chapter_head_0} 
\chapter{\DbgSecSt{\StPart}{Реализация программ ПЛК}}
\label{sec:PLCs}
\index{Реализация программ ПЛК|(}

%--------------------------------------------------------
% *******begin section***************

\section{\DbgSecSt{\StPart}{Таймеры}}
\index{Реализация программ ПЛК!Таймеры}

% *******begin subsection***************
\subsection{\DbgSecSt{\StPart}{Таймер однократного запуска}}
\index{Реализация программ ПЛК!Таймеры!Таймер однократного запуска}

Таймер однократного запуска (не периодический) ~-- таймер, выходное значение которого становится равным 1 по истечении заданного интервала и не меняется до повторного запуска (таймер, отмеряющий заданный интервал времени с момента запуска). \killoverfullbefore

Для использования таймера однократного запуска необходимо объявить переменную типа \myreftosec{Timer}. 

Запуск таймера осуществляется вызовом макроса \myreftosec{timerStart}, аргументами которого являются переменная типа \myreftosec{Timer} и величина интервала срабатывания в периодах сервоцикла.

Для проверки срабатывания таймера предназначен макрос \myreftosec{timerTimeout}, который возвращает значение, отличное от 0, если истёк заданный интервал срабатывания.

Перезапуск таймера осуществляется повторным вызовом макроса \myreftosec{timerStart}.

Временная диаграмма таймера представлена на рис. ~\ref{fig:Timer_1}.

\DrawPictEpsFromSvg[0.75\textwidth]{./Pictures/svg/Timer_1}{Таймер однократного запуска}{Timer_1}

Листинг ~\ref{lst:automat_mt} на стр. \pageref{lst:automat_mt} иллюстрирует использование таймера однократного запуска.

% *******end subsection***************

% *******begin subsection***************
\subsection{\DbgSecSt{\StPart}{Периодический таймер}}
\index{Реализация программ ПЛК!Таймеры!Периодический таймер}

Периодический (импульсный) таймер ~-- таймер, выходное значение которого периодически переключается с 0 на 1 и обратно через интервал, равный половине заданного периода таймера.

Функция \myreftosec{timerSc} возвращает 1, если с момента переключения таймера с 1 на 0 истёк интервал, больший или равный половине периода таймера, и 0 в противном случае. 

Временная диаграмма таймера представлена на рис. ~\ref{fig:Timer_1}.

\DrawPictEpsFromSvg[0.75\textwidth]{./Pictures/svg/Timer_2}{Периодический таймер}{Timer_2}

Периодический таймер используется для организации равных промежутков времени, например, для мигающей индикации или разного рода фильтров.

Листинг ~\ref{lst:p_timer} показывает применение периодического таймера для индикации пульта оператора.\killoverfullbefore 

Переменная \texttt{homeLed} (строка 5) определяет состояние индикатора реферирования осей. Если не выполнен выезд в 0 или не выполнено позиционирование всех осей при включении станка, то переменной \texttt{homeLed} будет присваиваться периодически 0 или 1. Значение \texttt{homeLed} записывается в \texttt{mt.PultOut.modeHome} (строка 19) и индикатор реферирования осей пульта оператора будет мигать. \killoverfullbefore \BL

\IncludeListing{./listings/p_timer2.c}{Применение периодического таймера \newline}{lst:p_timer}

% *******end subsection***************

% *******end section*****************
%--------------------------------------------------------

% *******begin section***************

\section{\DbgSecSt{\StPart}{Входы/выходы}}
\index{Реализация программ ПЛК!Входы/выходы}

%\label{MTInputs}
%\label{MTOutputs}
%\label{PultInputs}
%\label{PultOutputs}
%\label{PortablePultInputs}

Обращение ко входам и выходам осуществляется через 4-х байтовые переменные Servo[i].IO[j].DataIn[k] и Servo[i].IO[j].DataOut[k] соответственно, где i ~-- номер платы управления, j ~-- номер порта платы управления, k ~-- номер регистра выходных/выходных данных.

Переменные Servo[i].IO[j].DataIn[k] представляют собой набор 1-но битных полей, каждое из которых содержит состояние отдельного входа.

Переменные Servo[i].IO[j].DataOut[k] представляют собой набор 1-но битных полей, в каждое из которых записывается состояние отдельного выхода.\killoverfullbefore 

Для работы со входами и выходами пользователем объявляются \hypertarget{IO_union}{объединения}:
\begin{itemize}
\item MTInputs ~-- выходные сигналы электрооборудования станка (путь по умолчанию: include\textbackslash platform\textbackslash имя\_проекта\textbackslash stanok.h); \killoverfullbefore
\item MTOutputs ~-- входные сигналы электрооборудования станка (путь по умолчанию: include\textbackslash platform\textbackslash имя\_проекта\textbackslash stanok.h);\killoverfullbefore
\item PultInputs ~-- выходные сигналы пульта оператора (путь по умолчанию: include\textbackslash platform\textbackslash имя\_проекта\textbackslash operator\_pult.h);  \killoverfullbefore
\item PultOutputs ~-- входные сигналы пульта оператора (путь по умолчанию: include\textbackslash platform\textbackslash имя\_проекта\textbackslash operator\_pult.h); \killoverfullbefore
\item PortablePultInputs ~-- выходные сигналы переносного пульта (путь по умолчанию: include\textbackslash platform\textbackslash имя\_проекта\textbackslash portable\_pult.h). \killoverfullbefore \BL
\end{itemize} 

Для указания функционального назначения отдельного входа или выхода следует определить имена идентификаторов соответствующего битового поля в объявлении объединений.

Листинг ~\ref{lst:IO_unions} показывает пример определения имён идентификаторов в объявлении объединений \texttt{MTInputs} и \texttt{MTOutputs}. \killoverfullbefore \BL

\IncludeListing{./listings/IO_unions2.c}{Пример определения имён идентификаторов \newline}{lst:IO_unions}

\begin{comment}
\begin{pExample}
\IncludeLstWithoutBorder{./listings/NullSample.pas}{~}{lst:IO-ex-1}
              
\begin{tabular}{l l}

union MTInputs \{ &  \\
\quad    struct \{ &  \\
\quad \textcolor{exComm}{// Первая плата входов} &   \\
\quad unsigned overloadPumpA:1; & \textcolor{exComm}{//Вход 0 - перегрузка насоса СОЖ А} \\
\quad \vdots &   \\
\quad unsigned onFanSPND:1; & \textcolor{exComm}{//Вход 4 - вентилятор шпинделя включён}\\
\quad \vdots &   \\
\quad unsigned mtOn:1; &  \textcolor{exComm}{//Вход 31 - станок включён}\\
\quad \textcolor{exComm}{// Вторая плата входов} &   \\
\quad lubeLevelLow:1; & \textcolor{exComm}{//Вход 0 - низкий уровень масла} \\
\quad \vdots &   \\
\quad accessWithKey:1; & \textcolor{exComm}{//Вход 10 - доступ с ключом}\\
\quad \vdots &   \\
\quad unsigned unclampingAxisC:1; &  \textcolor{exComm}{//Вход 31 - ось С разжата}\\
\quad \}; & \\
\quad int Inputs[2];  & \\
\}; & \\
  & \\
union MTOutputs \{ &  \\
\quad    struct \{ &  \\
\quad \textcolor{exComm}{// Первая плата реле} &   \\
\quad unsigned clearCoolantOn:1; & \textcolor{exComm}{//Выход 0 - включение очистки СОЖ от масла} \\
\quad \vdots &   \\
\quad unsigned workpieceBlast:1; & \textcolor{exComm}{//Выход 14 - обдув рабочей зоны}\\
\quad \vdots &   \\
\quad operatorDoorOpen:1; &  \textcolor{exComm}{//Выход 23 - открытие двери оператора}\\
\quad \textcolor{exComm}{// Вторая плата реле} &   \\
\quad clampingAxisA:1; & \textcolor{exComm}{//Выход 0 - зажим оси А} \\
\quad \vdots &   \\
\quad pumpB:1; & \textcolor{exComm}{//Выход 10 - включение насоса B подачи СОЖ}\\
\quad \vdots &   \\
\quad unsigned spindleChiller:1; &  \textcolor{exComm}{//Выход 23 - включение охлаждения шпинделя}\\
\quad \}; & \\
\quad int Outputs[2];  & \\
\}; & \\

\end{tabular}

\end{pExample}  

\end{comment}


Плата входов имеет возможность подключения 32 входных сигналов, плата выходов имеет возможность вывода до 24 выходных сигналов.

Перечисленные выше объединения являются полями структуры \myreftosec{MTDesc}.

Пользователь должен объявить переменную \texttt{mt} типа \myreftosec{MTDesc} для возможности работы со входами и выходами.

% *******begin subsection***************
%\subsection{\DbgSecSt{\StPart}{Входы}}

Перед чтением состояний входов необходимо проверить корректность полученных данных: 1-й бит переменной \texttt{Servo[i].IO[j].Status} должен быть установлен (равен 1).

Листинг ~\ref{lst:IO} показывает обращение ко входам и выходам. В данном примере пульт оператора подключён к порту №0, плата входов/выходов для управления электроавтоматикой станка ~-- к порту №1. \killoverfullbefore \BL

\IncludeListing{./listings/IO2.c}{Обращение ко входам и выходам \newline}{lst:IO}

\begin{comment}
\begin{pExample}
\IncludeLstWithoutBorder{./listings/NullSample.pas}{~}{lst:IO-ex-2}
              
\begin{tabular}{l l}
MTDesc mt; &  \\

\vdots &  \\

void readInputs() \{ & \textcolor{exComm}{// Чтение входов} \\
\quad  if (Servo[0].IO[0].Status \& 1) \{ & \textcolor{exComm}{// Проверка корректности} \\
\quad   & \textcolor{exComm}{// данных} \\
\qquad mt.PultIn.PultBtn[0] = Servo[0].IO[0].DataIn[0]; & \textcolor{exComm}{// } \\
\qquad mt.PultIn.PultBtn[1] = Servo[0].IO[0].DataIn[1]; & \textcolor{exComm}{// }\\
\qquad mt.PultIn.PultBtn[2] = Servo[0].IO[0].DataIn[2]; &   \\
\qquad mt.PultIn.PultBtn[3] = Servo[0].IO[0].DataIn[3]; &  \textcolor{exComm}{// }\\
\qquad countErrorLinkOperatorPult = 0; &  \textcolor{exComm}{// Обнуление числа ошибок} \\
\qquad  &  \textcolor{exComm}{//соединения с пультом оператора} \\
\quad \} &   \\
\quad else \{ &   \\
\qquad countErrorLinkOperatorPult++; &   \\
\qquad if (countErrorLinkOperatorPult >= 100) & \textcolor{exComm}{// }\\
\qquad errorSet(systemErrors.machine.linkOperatorPult); &   \\
\quad \} & \\
\quad  if (Servo[0].IO[1].Status \& 1) \{ & \textcolor{exComm}{// Проверка корректности} \\
\quad  & \textcolor{exComm}{// данных} \\
\qquad mt.IN.Inputs[0] = Servo[0].IO[1].DataIn[0]; & \textcolor{exComm}{// } \\
\qquad mt.IN.Inputs[1] = Servo[0].IO[1].DataIn[1]; & \textcolor{exComm}{// }\\
\qquad countErrorLinkIntIO = 0; & \textcolor{exComm}{// Обнуление числа ошибок}  \\
\qquad countErrorLinkIntIO = 0; & \textcolor{exComm}{// соединения с платой входов}  \\
\quad \} &   \\
\quad else \{ &   \\
\qquad countErrorLinkIntIO++; &   \\
\qquad if (countErrorLinkIntIO >= 100) & \textcolor{exComm}{// }\\
\qquad errorSet(systemErrors.machine.linkIntIO); &   \\
\quad \} & \\
\}; & \\
\vdots &  \\

void writeOutputs() \{ & \textcolor{exComm}{// Запись выходов} \\
\quad Servo[0].IO[0].DataOut[0] = mt.PultOut.PultLed[0]; & \textcolor{exComm}{// } \\
\quad Servo[0].IO[0].DataOut[1] = mt.PultOut.PultLed[1]; & \textcolor{exComm}{// } \\
\quad Servo[0].IO[0].DataOut[2] = mt.PultOut.PultLed[2]; & \textcolor{exComm}{// } \\
\quad Servo[0].IO[1].DataOut[0] = mt.OUT.Outputs[0]; & \textcolor{exComm}{// } \\
\} &  \\
\vdots &  \\
\end{tabular}

\end{pExample}  
\end{comment}


% *******end subsection***************

% *******begin subsection***************
%\subsection{\DbgSecSt{\StPart}{Выходы}}



% *******end subsection***************

% *******end section*****************
%--------------------------------------------------------

% *******begin section***************

\section{\DbgSecSt{\StPart}{Программирование алгоритмов управления}}
\index{Реализация программ ПЛК!Программирование алгоритмов управления}

Для программирования алгоритмов управления используются конечные автоматы. 

Конечные автоматы ~-- конструкции, которые описываются ограниченным набором возможных состояний, набором сигналов (событий) и условиями переходов из одного состояния в другое. Последующее состояние автомата определяется текущим состоянием и входными сигналами.

Листинг ~\ref{lst:automat_mt} показывает фрагмент реализации конечного автомата включения/выключения станка с помощью оператора множественного выбора \texttt{switch-case}. Полностью автомат приведён в \textbf{ПРИЛОЖЕНИИ 1} в листинге <<Программа включения/выключения станка>> на стр. \pageref{MtOnOff}.\killoverfullbefore \BL


\IncludeListing{./listings/automat_mt.c}{Фрагмент реализации конечного автомата управления станком \newline}{lst:automat_mt}


\begin{comment}
\begin{pExample}
\IncludeLstWithoutBorder{./listings/NullSample.pas}{~}{lst:automat-ex-1}
              
\begin{tabular}{l l}

\quad switch (mt.State) \{ &  \\

\quad case mtNotReady: \{ & \textcolor{exComm}{// Ожидание включения} \\
\qquad if (CNC.request == mtcncReset) \{ mtReset(); \} & \textcolor{exComm}{// } \\
\qquad if (mtIsOn() \&\& !mt.ncNotReadyReq) mt.State=mtStartOn; & \textcolor{exComm}{// }  \\
\qquad break;&   \\
\quad \} &   \\

\quad case mtStartOn: \{ & \textcolor{exComm}{// Начало включения} \\
\qquad if (CNC.request == mtcncReset) \{ mtReset(); \} & \textcolor{exComm}{// } \\
\qquad if (mt.ncNotReadyReq) \{ mtAbortRequest(); break; \} & \textcolor{exComm}{// } \\
\qquad mt.State = mtDriveOn;
\qquad if (!axesPhaseRefComplete() || !spinsPhaseRefComplete()) mt.State = mtPhaseRef;
\qquad break; &   \\
\quad \} &   \\

\quad case mtPhaseRef: \{ & \textcolor{exComm}{// Фазировка} \\
\qquad if (CNC.request == mtcncReset) \{ mtReset(); \} & \textcolor{exComm}{// } \\
\qquad if (mt.ncNotReadyReq) \{ mtAbortRequest(); break; \} & \textcolor{exComm}{// } \\
\qquad if (axesPhaseRef() || spinsPhaseRef()) \{ &   \\
\qquad \quad mt.State = mtWaitPhaseRef; &   \\
\qquad \quad timerStart(mt.timerState, MT\_TIME\_DRIVE\_PHASE\_REF); &   \\
\qquad \} else \{ &   \\
\qquad \quad mt.State = mtDriveOn;
\qquad \}
\qquad break;&   \\
\quad \} &   \\

%\quad \vdots &   \\

\quad case default: \{ &   \\
\qquad mt.OUT.pumpA = 0; & \textcolor{exComm}{//Насос А выключить}  \\
\qquad mt.OUT.pumpB = 0; & \textcolor{exComm}{//Насос В выключить}  \\
\qquad break;&   \\
\quad \} &   \\

\} & \\
\end{tabular}

\end{pExample} 
\end{comment}

\begin{comment}
\begin{lstlisting}[label=some-code,caption={Это крутой исходный код}]
 // управление сигналом готовности системы
    switch (mt.State) {
    // ждём вход от главного пускателя
    case mtNotReady: {
        if (CNC.request == mtcncReset) { mtReset(); }
        if (mtIsOn() && !mt.ncNotReadyReq) mt.State=mtStartOn;
        // Обработка нажатия кнопки Reset
        //if (MT.PultIn.modeReset) resetCNC();
        break;
    }
        // начало включения
    case mtStartOn: {
        if (CNC.request == mtcncReset) { mtReset(); }
        if (mt.ncNotReadyReq) { mtAbortRequest(); break; }
        mt.State = mtDriveOn;
        if (!axesPhaseRefComplete() || !spinsPhaseRefComplete()) mt.State = mtPhaseRef;
        break;
    }
        // фазировка
    case mtPhaseRef: {
        if (CNC.request == mtcncReset) { mtReset(); }
        if (mt.ncNotReadyReq) { mtAbortRequest(); break; }
        if (axesPhaseRef() || spinsPhaseRef()) {
            mt.State = mtWaitPhaseRef;
            timerStart(mt.timerState,
                       MT_TIME_DRIVE_PHASE_REF);
        } else {
            // уже выполнено
            mt.State = mtDriveOn;
        }
        break;
    }
        // ожидание окончания фазировки
    case mtWaitPhaseRef: {
        if (CNC.request == mtcncReset) { mtReset(); }
        if (mt.ncNotReadyReq) { mtAbortRequest(); break; }
        if (timerTimeout(mt.timerState)) {
            errorSet(systemErrors.channel[0].phaseRefTimeout);
            break;
        }
        if (axesPhaseRefComplete() && spinsPhaseRefComplete()) {
            mt.State=mtDriveOn;
        }
        break;
    }
        // включение приводов
    case mtDriveOn: {
        if (CNC.request == mtcncReset) { mtReset(); }
        if (mt.ncNotReadyReq) { mtAbortRequest(); break; }
        axesActivate();
        //spinsActivate();
        timerStart(mt.timerState,
                   MT_TIME_DRIVE_ON);
        mt.State=mtWaitDriveOn;
        break;
    }
        // ожидание включения приводов
    case mtWaitDriveOn: {
        if (CNC.request == mtcncReset) { mtReset(); }
        if (mt.ncNotReadyReq) { mtAbortRequest(); break; }
        if (timerTimeout(mt.timerState)) {
            errorSet(systemErrors.channel[0].driveOnTimeout);
            break;
        }
        if (axesActive()/* && spinsActive()*/) {
            mt.State=mtOthersMotorOn;
        }
        break;
    }
\end{lstlisting}
\end{comment}

\begin{comment}
\begin{pExample}
\IncludeLstWithoutBorder{./listings/NullSample.pas}{~}{lst:automat-ex-1}
              
\begin{tabular}{l l}

enum CoolantPump \{ & \textcolor{exComm}{//Перечисление ~-- идентификаторы насосов СОЖ}  \\
\quad    coolNone = 0,  &  \\
\quad    coolA = 1,   &  \\
\quad    coolB,  &  \\
\quad    coolAB &  \\
\}; &  \\

\quad \vdots &   \\

void coolantOn (int pump) \{ &  \\
\quad actPump = pump; &  \\
\quad switch (pump) \{ &  \\

\quad case coolA: \{ & \textcolor{exComm}{//actPump = A;} \\
\qquad mt.OUT.pumpA = 1; & \textcolor{exComm}{//Насос А включить} \\
\qquad mt.OUT.pumpB = 0; & \textcolor{exComm}{//Насос В выключить}  \\
\qquad break;&   \\
\quad \} &   \\

\quad case coolB: \{ & \textcolor{exComm}{//actPump = B;} \\
\qquad mt.OUT.pumpA = 0; & \textcolor{exComm}{//Насос А выключить} \\
\qquad mt.OUT.pumpB = 1; & \textcolor{exComm}{//Насос В включить} \\
\qquad break;&   \\
\quad \} &   \\

\quad case coolAB: \{ & \textcolor{exComm}{//actPump = AB;} \\
\qquad mt.OUT.pumpA = 1; & \textcolor{exComm}{//Насос А включить} \\
\qquad mt.OUT.pumpB = 1; & \textcolor{exComm}{//Насос В включить} \\
\qquad break;&   \\
\quad \} &   \\

%\quad \vdots &   \\

\quad case default: \{ &   \\
\qquad mt.OUT.pumpA = 0; & \textcolor{exComm}{//Насос А выключить}  \\
\qquad mt.OUT.pumpB = 0; & \textcolor{exComm}{//Насос В выключить}  \\
\qquad break;&   \\
\quad \} &   \\

\} & \\
\end{tabular}

\end{pExample} 
\end{comment}

% *******end section*****************
%--------------------------------------------------------
% *******begin section***************
\section{\DbgSecSt{\StPart}{Обработка аварийных ситуаций и ошибок электрооборудования станка}}
\index{Реализация программ ПЛК!Обработка аварийных ситуаций и ошибок электрооборудования станка}

Обработка аварийных ситуаций и ошибок электрооборудования станка выполняется с помощью следующих программных средств:\killoverfullbefore
\begin{itemize}
\item объединение \texttt{MachineErrors} ~--  список аварийных ситуаций и ошибок;
\item макрос \texttt{errorSet} ~-- установка ошибки (соответствующему битовому полю объединения \texttt{MachineErrors} присваивается 1); \killoverfullbefore
\item макрос \texttt{DEFINE\_ERROR} ~--  создание и инициализация переменной типа \myreftosec{ErrorDescription}; \killoverfullbefore
\item пользовательская функция \myreftosec{errorsMachineScan} ~-- набор вызовов макроса \texttt{errorScanSet}; \killoverfullbefore
\item пользовательская функция \myreftosec{errorsMachineReaction} ~-- набор вызовов функции \myreftosec{errorReaction}. \killoverfullbefore \BL
\end{itemize}

\hypertarget{Machine_Errors}{В объединении} \texttt{MachineErrors} (путь по умолчанию: include\textbackslash platform\textbackslash имя\_проекта\textbackslash machine\_error.h) пользователем определяются аварийные ситуации и ошибки электрооборудования станка в виде битовых полей. Оно является полем структуры \myreftosec{Errors}. \killoverfullbefore

Листинг ~\ref{lst:machine_errors_union} показывает пример объявления объединения \texttt{MachineErrors}. \killoverfullbefore \BL

\IncludeListing{./listings/machine_errors_union2.c}{Пример объявления объединения \texttt{MachineErrors} \newline}{lst:machine_errors_union}

\BL

Листинг ~\ref{lst:coolant} показывает пример использования макроса \texttt{errorSet} в функции контроля СОЖ. 

Если насос А включен и произошла его перегрузка (проверка соответствующего выхода и входа в строке 7), то выключается подача СОЖ (строка 8) и выставляется соответствующий бит ошибки (строка 9).\killoverfullbefore

Если уровень СОЖ высокий (проверка соответствующего входа в строке 13) и станок включен (строка 14), то выключается подача СОЖ (строка 15) и выставляется соответствующий бит ошибки (строка 17).\killoverfullbefore

Системная переменная \texttt{systemErrors} имеет тип \myreftosec{Errors}. \killoverfullbefore \BL

\IncludeListing{./listings/coolant.c}{Пример использования макроса \texttt{errorSet} \newline}{lst:coolant}

Макрос \texttt{DEFINE\_ERROR(name, code, react, cl)} создаёт переменную типа \myreftosec{ErrorDescription} с именем descError\texttt{\{name\}} и инициализирует её поля аргументами descError\texttt{\{name\}}.id=code, descError\texttt{\{name\}}.reaction=react, descError\texttt{\{name\}}.clear=cl. \killoverfullbefore

Поле descError\texttt{\{name\}}.id ~-- приоритет ошибки. 

Поле descError\texttt{\{name\}}.reaction ~-- идентификаторы перечисления \myreftosec{ErrorReaction}, определяющие реакцию на ошибку. 

Поле descError\texttt{\{name\}}.clear ~-- идентификаторы перечисления \myreftosec{ErrorClear}, определяющие тип сброса ошибки.

Листинг ~\ref{lst:define_error} показывает вызовы макроса \texttt{DEFINE\_ERROR}, которые создают переменные descErrorMachineEmergencyStop и descErrorMachineLubeError. \killoverfullbefore \BL

\IncludeListing{./listings/define_error.c}{Пример вызова макроса \texttt{DEFINE\_ERROR} \newline}{lst:define_error}

Функция \myreftosec{errorsMachineScan} должна быть реализована пользователем. В ней
вызывается макрос \myreftosec{errorScanSet} для ошибок, 
%вызывается макрос \texttt{errorScanSet(error, input, desc, request)} для ошибок,
определенных в объединении \texttt{MachineErrors} и макросом \texttt{DEFINE\_ERROR}. \killoverfullbefore

Агрументы \texttt{errorScanSet}:
\begin{itemize}
\item error ~-- битовое поле объединения \texttt{MachineErrors}; \killoverfullbefore
\item input ~-- значение, которое возвращает функция контроля соответствующего параметра (состояние соответствующего входа); \killoverfullbefore
\item desc ~-- переменная типа \myreftosec{ErrorDescription}; \killoverfullbefore
\item request ~-- переменная типа \myreftosec{ErrorClear}. \killoverfullbefore \BL
\end{itemize}

Функция \myreftosec{errorsMachineReaction} должна быть реализована пользователем. В ней вызывается функция \myreftosec{errorReaction} для ошибок, определенных в объединении \texttt{MachineErrors} и макросом \texttt{DEFINE\_ERROR}. \killoverfullbefore

Листинг ~\ref{lst:machine_errors} показывает пример реализации файла обработки аварийных ситуаций и ошибок электрооборудования станка (путь по умолчанию: include\textbackslash platform\textbackslash имя\_проекта\textbackslash machine\_error.cfg). \killoverfullbefore \BL

\IncludeListing{./listings/machine_errors2.c}{Пример обработки аварийных ситуаций и ошибок электрооборудования станка \newline}{lst:machine_errors}

\BL

Листинг ~\ref{lst:lube} показывает пример реализации функции контроля смазки направляющих \texttt{hasLubeError()}, которая вызывается в строке 21 листинга ~\ref{lst:machine_errors}. \killoverfullbefore \BL

\IncludeListing{./listings/lube.c}{Пример реализации функции контроля смазки направляющих \newline}{lst:lube}

% *******end section*****************
%--------------------------------------------------------
\index{Реализация программ ПЛК|)}

\clearpage                   % Примеры ПЛК программ
    \renewcommand\thefigure{\arabic{figure}} 

\fancyfoot[CE,CO]{\normalsize\thepage}
\fancyhead[RO]{\normalsize ПРИЛОЖЕНИЕ 1. Примеры программ ПЛК}% odd page header and number to right top
\fancyhead[RE]{\normalsize ПРИЛОЖЕНИЕ 1. Примеры программ ПЛК}%Even page header and%number at left top
%\fancyfoot[L,R,C]{}
\renewcommand{\headrulewidth}{0.4pt}% disable the underline of the header part

\phantomsection

\lstset{basicstyle=\footnotesize, keywordstyle=\color{indigo_dye}\bfseries, stringstyle=\ttfamily, language=C, numbers=left, numberstyle=\tiny, stepnumber=1, numbersep=5pt, firstnumber=1}

\etocsettocdepth.toc {chapter}

\chapterimage{chapter_head_0} % Chapter heading image
\chapter*{ПРИЛОЖЕНИЕ 1}
\addcontentsline{toc}{chapter}{ПРИЛОЖЕНИЕ 1. Примеры программ ПЛК}
\label{sec:appendix1}

\begin{center}
 \textcolor{blue}{\textbf{\LARGE{Примеры программ ПЛК}}}
\end{center}
\BL
\begin{center}
\textbf{\large{Управление толчковыми перемещениям }}
\end{center}
\BL

\label{MotionJog}

%Листинг ~\ref{lst:stanok} показывает пример программы управления станком. \killoverfullbefore \BL

%\IncludeListing{./listings/stanok4.c}{Программа управления станком \newline}{lst:stanok}

\IncludeLstWithoutCaption{./listings/jog.c}

\clearpage

\begin{center}
\textbf{\large{Задание программы движения}}
\end{center}
\BL

\label{MotionNC}

\IncludeLstWithoutCaption{./listings/stanok11.c}

\lstset{basicstyle=\footnotesize, keywordstyle=\color{indigo_dye}\bfseries, stringstyle=\ttfamily, language=C, numbers=left, numberstyle=\tiny, stepnumber=1, numbersep=5pt, firstnumber=168}

\IncludeLstWithoutCaption{./listings/stanok12.c}
%\IncludeLstWithoutCaption{./listings/stend_1.c}
%\IncludeLstWithoutCaption{./listings/stend_2.c}
\BL

\clearpage 

\begin{center}
\textbf{\large{Программа включения/выключения станка}}
\end{center}
\BL

\label{MtOnOff}

\lstset{basicstyle=\footnotesize, keywordstyle=\color{indigo_dye}\bfseries, stringstyle=\ttfamily, language=C, numbers=left, numberstyle=\tiny, stepnumber=1, numbersep=5pt, firstnumber=1}

\IncludeLstWithoutCaption{./listings/stanok41.c}

\lstset{basicstyle=\footnotesize, keywordstyle=\color{indigo_dye}\bfseries, stringstyle=\ttfamily, language=C, numbers=left, numberstyle=\tiny, stepnumber=1, numbersep=5pt, firstnumber=126}

\IncludeLstWithoutCaption{./listings/stanok42.c}
%\IncludeLstWithoutCaption{./listings/stanok4_1.c}

%\IncludeLstWithoutCaption{./listings/stanok4_2.c}

\newpage
             % Приложение 1
%    \input{Input_files/appendix_2}            % Приложение 2
    \fancyhead[RO,RE]{\normalsize Предметный указатель}% odd page header and number to right top
\renewcommand{\headrulewidth}{0.4pt}% disable the underline of the header part
\fancyfoot[CE,CO]{\normalsize\thepage}

\phantomsection

\chapterimage{chapter_head_0} % Chapter heading image

\setlength{\columnsep}{0.75cm}
%\addcontentsline{toc}{chapter}{\textcolor{ocre}{Предметный указатель}}
\addcontentsline{toc}{chapter}{Предметный указатель}

\printindex
                  % Предметный указатель

\end{document}